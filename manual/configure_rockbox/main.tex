\chapter{Configuring Rockbox}

\section{Sound Settings}
{\centering\itshape
  [Warning: Image ignored] % Unhandled or unsupported graphics:
%\includegraphics[width=4.15cm,height=2.371cm]{images/rockbox-manual-img32.png}
 \textmd{  }  [Warning: Image ignored]
% Unhandled or unsupported graphics:
%\includegraphics[width=4.15cm,height=1.951cm]{images/rockbox-manual-img33.png}
 \newline
Recorder sound settings  Player sound settings  
\par}

This menu offers a selection of sound properties you may change to
improve your sound experience.

\begin{itemize}
\item \textbf{Volume}

The sound volume your music is played at. Although settable range is
0{}-100\%, many units don't produce audible output
below 40\%.  On Recorders, volume settings above 92\% will cause
distortion (clipping) and are not recommended.

\item \textbf{Bass}
This emphasises or suppresses the lower
(bass) sounds in the track.  0 means that bass sounds are unaltered
(flat response).

\item \textbf{Treble}
This emphasises or suppresses the higher
(treble) sounds in the track.  0 means that treble sounds are unaltered
(flat response).

\item \textbf{Balance}
How much of the volume is generated by the left or right channel of the
sound.  The default, 0, means that the left and right outputs are equal
in volume.  Negative numbers increase the volume of the left channel
relative to the right, positive numbers increase the volume of the
right channel relative to the left.

\item \textbf{Channels}

This option controls the on{}-board mixing
facilities of the Jukebox.  A stereo audio signal consists of two
channels, left and right.  Available options are

\begin{itemize}
\item \textbf{Mono Left: }Plays the left channel in both stereo channels.
\item \textbf{Mono Right:} Plays the right channel in both stereo channels.
\item \textbf{Mono:} Mix both channels down to mono and send the mixed signal
back to both.
\item \textbf{Stereo:} Do not mix the signal
\item \textbf{Stereo Narrow: }Mixes small amounts of the opposite channel into
the left and right channels, thus making the sound seem closer
together.
\item \textbf{Stereo Wide:} Elements of one channel that are present in the
opposite channel are removed from the latter.  This results in the
sound seeming further apart.
\item \textbf{Karaoke:} Removes all sound that is the same in both channels. 
Since most vocals are recorded in this way to make the artist sound
central, this often (but not always) has the effect of removing the
voice track from a song.
\end{itemize}

\item \textbf{Loudness (Recorder only)}
Loudness is an effect which emphasises bass and treble.  This makes the
track seem louder by amplifying the frequencies that the human ear
finds hard to hear.  Frequencies in the vocal range are unaffected,
since the human ear picks these up very easily.

\item \textbf{Auto Volume (Recorder only)}
Auto volume is a feature that automatically lowers the volume on loud
parts, and then slowly restores the volume to the previous level over a
time interval. That time interval is configurable here.  Short values
like 20ms are useful for ensuring a constant volume for in car use and
other applications where background noise makes a constant loudness
desirable.  A longer timeout means that the change in volume back to
the previous level will be smoother, so there will be less sharp
changes in volume level.

\item \textbf{Super Bass (Recorder Only)}
This setting changes the threshold at which bass frequencies are
affected by the \textbf{Loudness} setting, making the sound of drums
and bass guitar louder in comparison to the rest of the track.  This
setting only has an effect if \textbf{Loudness} is set to a value
larger than 0dB.

\item \textbf{MDB {}- Micronas Dynamic Bass (Recorder Only)}
The rest of the parameters on this menu relate to the Micronas Dynamic
Bass (MDB) function.  This is designed to enable the user to hear bass
notes that the headphones and/or speakers are not capable of
reproducing.  Every tone has a fundamental frequency (the ``main tone'') and also several harmonics, which are related to that tone.  The human brain has a
mechanism whereby it can actually infer the presence of bass notes from
the higher harmonics that they would generate.\\

The practical upshot of this is that MDB produces a more authentic
sounding bass by tricking the brain in believing it's
hearing tones that the headphones or speakers aren't
capable of reproducing.  Try it and see what you think.\\

The MDB parameters are as follows.

\begin{itemize}
\item \textbf{MDB enable: } This turns the MDB feature on or off.  For many
users this will be the only setting they need, since Rockbox picks
sensible defaults for the other parameters.  MDB is turned off by
default.
\item \textbf{MDB strength:} How loud the harmonics generated by the MDB will
be.
\item \textbf{MDB Harmonics}: The percentage of the low notes that is
converted into harmonics.  If low notes are causing speaker distortion,
this can be set to 100\% to eliminate the fundamental completely and
only produce harmonics in the signal.  If set to 0\% this is the same
as turning the MDB feature off.
\item \textbf{MDB Centre Frequency: }The cutoff frequency of your headphones or speakers.  This is usually given in the specification for the headphones/speakers.
\item \textbf{MDB shape: }It is recommended that this parameter be set to 1.5
times the centre frequency.\\

This is the frequency up to which harmonics are generated.  Some of the
lower fundamentals near the cut{}-off range
will have their lower harmonics cut off, since they will be below the
range of the speakers. Fundamentals between the
cut{}-off frequency and the lower frequency will have their harmonics proportionally boosted to compensate and restore the 'loudness' of these
notes.\\

For most users, the defaults should provide an improvement in sound
quality and can be safely left as they are.  For reference, the
defaults Rockbox uses are:


\begin{table}[h!]
  \begin{center}
    \begin{tabular}{@{}lc@{}}\toprule
      Setting & Value \\\midrule
      MDB Strength & 50dB \\
      MDB Harmonics & 48\% \\
      MDB Centre Frequency & 60Hz \\
      MDB Shape & 90Hz \\\bottomrule
    \end{tabular}
  \end{center}
\end{table}

\end{itemize}
\end{itemize}

\section{\label{ref:GeneralSettings}General Settings}
{\centering\mdseries\itshape
  [Warning: Image ignored] % Unhandled or unsupported graphics:
%\includegraphics[width=3.822cm,height=2.184cm]{images/rockbox-manual-img34.png}
     [Warning: Image ignored] % Unhandled or unsupported graphics:
%\includegraphics[width=4.667cm,height=1.963cm]{images/rockbox-manual-img35.png}
 \newline
Recorder general settings  Player general settings  
\par}

\subsubsection{\label{ref:PlaybackOptions}Playback Options}
This menu is for configuring settings related to MP3 playback

\begin{itemize}
\item \textbf{Shuffle}
Select shuffle ON/OFF. This alters how Rockbox will select which next
song to play. 
\item \textbf{Repeat}
Repeat modes are Off/One/All. ``Off'' means no
repeat. ``One'' means repeat one track over
and over. ``All'' means repeat playlist/directory. 
\item \item{Play Selected First }
This setting controls what happens when you press PLAY on a file in a
directory and shuffle mode is on. If this setting is Yes, the file you
selected will be played first. If this setting is No, a random file in
the directory will be played first. 
\item \textbf{Resume}
Sets whether Rockbox will resume playing at the point where you shut
off. Options are: Ask/Yes/No/Ask once.
``Ask'' means it will ask at boot time. ``Yes'' means it will unconditionally try to resume. ``No'' means it will not resume. ``Ask once'' will erase the resume info if you answer no, and thus not ask you again.
\item \textbf{FFwd / Rewind}
Two options are available at this point

\begin{itemize}
\item \textbf{FF/RW Min Step}
The smallest step, in seconds, you want to fast forward or rewind in a
track.
\item \textbf{FF/RW Accel}
How fast you want search (ffwd/rew) to accelerate when you hold
down the button. ``Off'' means no acceleration. ``2x/1s'' means double the
search speed once every second the button is held. ``2x/5s'' means double the search speed once every 5 seconds the button is held.
\end{itemize}

\item \textbf{Anti{}-skip Buffer}
This setting is really ``extra anti{}-skip''. It lets you set
a timer for how many seconds earlier than normally necessary the disk
should spin up and start reading data. You don't need
this unless you shake and bump the unit a lot. Spinning up the disk
earlier than necessary naturally drains the batteries a little extra. 
Most users will not need this setting.

\item \textbf{Fade on Stop/Pause}
This setting enables and disables a fade effect when you pause
or stop playing a song. Fade is a progressive increase or reduction of
volume, from your set volume to 0, and vice versa.

\item \textbf{ID3 tag priority}
ID3 tags in an MP3 file contain information about the artist, title,
album etc. of the track.  This option controls whether Rockbox uses the information from ID3v2 tags in preference to that from ID3v1 tags when both types of tag are present.
\end{itemize}

\subsection{File View}
This menu deals with options relating to how the file browser displays
files

\begin{itemize}
\item \textbf{Sort Case Sensitive}
If this option is enabled files that start with a
lower case letter will appear after the files that start with an upper case letter have all been listed  If disabled, then case will be ignored when sorting files.
\item \textbf{Sort Directories}
This option controls how Rockbox sorts folders.  The default is to sort
them alphabetically.  ``By date'' sorts them with the oldest folder
first.  ``By newest date'' sorts them with the newest folder first.

\item \textbf{Sort Files}
This option controls how Rockbox sorts files.  In addition to the
options for directory sorting above, there is a ``By type'' option
which sorts files alphabetically by their type (such as .mp3) then
alphabetically within each type.

\item \textbf{\label{ref:ShowFiles}Show Files}
Controls which files are displayed in the directory browser:

\begin{itemize}
\item \textbf{Music: }
Only directories, .mp3, .mp2, .mpa and .m3u files
are shown. Extensions are strippe'd. Files anddirectories starting with . Or with the ``hidden'' flag set are hidden.
\item \textbf{Playlists:} 
Only shows directories and playlists, for
simplified navigation.
\item \textbf{Supported:} 
All directories and files Rockbox understands (see page \pageref{ref:Supportedfileformats}) are shown. Files and directories starting with . or with the
``hidden'' flag set are hidden.
\item \textbf{All:}
All files and directories are shown. Extensions are shown. No files or
directories arehidden
\end{itemize}

\item \textbf{Follow Playlist}
If Follow Playlist is set to ``Yes'', you will find yourself in the same
directory as the currently playing file if you go to the Directory
Browser from the WPS. If set to ``No'', you will stay in the same directory as you were last in.

\item \textbf{Show Icons}
This indicates whether Rockbox will display an icon representing what
type a file is on the left of the file in the browser.  For details of
these icons, please see page \pageref{ref:Supportedfileformats}.
\end{itemize}

\subsection{\label{ref:Displayoptions}Display Options}

\begin{itemize}
\item \textbf{Browse fonts}
Browse the fonts that reside in your \textbf{/.rockbox} directory.
Selecting one will activate it.  See page \pageref{ref:Loadingfonts} for further details about fonts.

\item \textbf{Browse WPS files}
Opens the file browser in the \textbf{/.rockbox} directory and displays
all .wps files. Selecting one will activate it, stop will exit back to
the menu.\\

For further information about the WPS see page \pageref{ref:WPS}. For
information  about editing a .wps file see page \pageref{ref:ConfiguringtheWPS}.

\item \textbf{LCD Settings}

%\begin{itemize}
This submenu contains settings that relate to the display of the
Jukebox.
\item \textbf{Backlight:} 
How long the backlight shines after a key
press. Set to OFF to never light it, set to ON to never shut it off or
set a preferred timeout period.
\item \textbf{Backlight on WhenPlugged:}
This option turns the backlight on constantly while the charger cable is connected.
\item \textbf{Caption Backlight:} This option turns the backlight on for
25 seconds either side of the start of a new track so that the display
can be read to see song information.
\item \textbf{Contrast:} Changes the contrast of your LCD display.
Warning: Setting the contrast too dark or too light can make it hard to
find this menu option again!
\item \textbf{LCD Mode} (Recorder only): This setting lets you invert
the whole screen, so now you get a black background and green text
graphics.
\item \textbf{Upside Down: }Displays the screen so that the top of the
display is nearest the buttons.  This is sometimes useful when carrying
the Recorder in a pocket for easy access to the headphone socket.
\item \textbf{Line Selector: }Select this option to have a bar of
inverted text (``Bar'' option) mark the current line in the File
Browser rather than the default arrow to the left (``Pointer'' option).
 This gives slightly more room for filenames.
%\end{itemize}

\item \textbf{Scrolling}
This feature controls how text will scroll in Rockbox. You can configure
the following parameters:

\begin{itemize}
\item \textbf{Scroll Speed:} 
Controls how many times per second the scrolling text moves a step.
\item \textbf{Scroll StartDelay:} 
Controls how many milliseconds Rockbox should wait before a new text begins scrolling.
\item \textbf{Scroll Step Size:}
Controls how many pixels the text scroll should move for each step. (Recorder/Ondio only)
\item \textbf{Bidirectional Scroll Limit: }
Rockbox has two different scroll methods,  always scrolling the text to the left, and when the line has ended, beginning again at the start, or moving to the
left until you can read the end of the line, and scroll right until you
see the beginning again. Rockbox chooses which method it should use,
depending of how much it has to scroll left. This setting lets you tell
Rockbox where that limit is, expressed in percentage of line length.
\end{itemize}

\item \textbf{Status/Scrollbar (Recorder only)}
Settings related to on screen status display and the scrollbar.

\begin{itemize}
\item \textbf{Scroll Bar: }Enables or disables the scroll bar at the
left.
\item \textbf{Status Bar: }Enables or disables the status bar
at the upper side.
\item \textbf{Button Bar:} Enables or disables the button bar prompts
for the F keys at the bottom of the screen.
\item \textbf{Volume Display:} Controls whether the volume is displayed
as a graphic or a numerical percentage value on the Status Bar.
\item \textbf{Battery Display: }Controls whether the battery charge
status is displayed as a graphic or numerical percentage value on the
Status Bar.
\end{itemize}

\item \textbf{Peak Meter (Recorder only) }
The peak meter can be configured with a number of parameters.  (For a description of the peak meter see page \pageref{ref:Peakmeter}.)

\begin{itemize}
\item \textbf{Peak Release:}
This determines how fast the bar shrinks when the music becomes softer.
Lower values make the peak meter look smoother.
\item \textbf{Peak Hold Time:} 
Specifies the time after which the peak indicator will reset. If you set this value e.g. to 5s then the peak indicator displays the loudest volume value
that occurred within the last 5 seconds. Big values are good if you
want to find the peak level of a song, which might be of interest when
copying music from the jukebox via the analogue output to some other
recording device.
\item \textbf{Clip Hold Time:}
How long the clipping indicator will be visible after clipping was detected
\item \textbf{Performance:}
In high performance mode, the peak meter is updated as often as possible. This reduces the chance of missing a peak value, making the peak meter more precise. In energy save mode the peak meter is updated just often enough to look fluid.
This reduces the load on the CPU and thus saves a little bit of energy. If you crave every second of runtime for your jukebox or simply use the peak meter as a screen effect, the use of energy save mode is recommended. If you want to use
the peak meter as a measuring instrument you'll want to use high performance mode.
\item \textbf{Scale:}
Select whether the peak meter displays linear or logarithmic values. In
``dB'' (decibel) scale the volume values are scaled logarithmically.
This very similar to the perception of loudness. The volume meters of
digital audio devices usually are scaled this way. If you are
interested in the power level that is applied to your headphones you
should choose ``linear'' display. Unfortunately this value
doesn't have real units like volts or watts since that
depends on the phones. So they can only be displayed as percentage
values.
\item \textbf{Minimum and maximum range:} These two options define the
full value range that the peak meter displays. Recommended values for
dbFs are {}-40 for min. and 0 for maximum. For linear display, use 0
and 100\%. Note that {}-40 dbFs is approximately 1\% in linear value,
but if you change the minimum setting in linear mode slightly and then change to dbFs there will be a large change. You can use these values for
'zooming' into the peak meter.
\end{itemize}
\end{itemize}

\subsubsection{\label{ref:SystemOptions}System Options}

\begin{itemize}
\item \textbf{Battery}
Options relating to the batteries in the Jukebox unit.  
\begin{itemize}
\item \textbf{Battery Capacity} can be used to tell the Jukebox what
capacity (in mAh) of battery is being used inside it.  The default is
1500mAh for NiMH battery based units, and 2300mAh for LiOn battery
based units, which is the capacity value for the standard batteries
shipped with these units. This value is used for calculating remaining
battery life.
\item \textbf{Deep discharge (Non{}-FM recorder only)}
Set this to ON if you intend to keep your charger connected for a long
period of time. It lets the batteries go down to 10\% before starting
to charge again. Setting this to OFF will cause the charging to restart
on 95\%.
\item \textbf{Trickle Charge (Non{}-FM recorder only)}
The Jukebox cannot be turned off while the charger is connected.
Therefore, trickle charge is needed to keep the batteries full after
charging has completed. For more in depth information about charging
see Battery FAQ in your \textbf{/.rockbox/docs }directory.
\end{itemize}

\item \textbf{Disk}
Options relating to the hard disk.  

\begin{itemize}
\item \textbf{DiskSpindown:}
Rockbox has a timer that makes it spin down the hard disk after being idle for acertain time. You can modify this timeout here. This idle time is only
affected by user activity, like navigating through file browser. When
the hard disk spins up to fill mp3 buffer, it automatically spins down
afterwards.
\item \textbf{Disk Poweroff:}(non v2/FM{}-recorder only)
Whether the disk is powered OFF or only set to ``sleep'' when spun
down. Power off uses less power but takes longer to spin{}-up.
\end{itemize}

\item \textbf{Time and Date (Recorder Only)}
Time related menu options.

\begin{itemize}
\item \textbf{Set Time/Date: }
Set current time and date.
\item \textbf{Time Format: }
Choose 12 or 24 Hour clock. 
\end{itemize}

\item \textbf{\label{ref:idlepoweroff}Idle Poweroff}
Rockbox can be configured to turn off power after the unit has been idle
for a defined number of minutes. The unit is idle when playback is
stopped or paused. It is not idle while the USB or charger is
connected, or while recording.

\item \textbf{Sleep Timer}
This option lets you power off your jukebox after playing for a given
time. This setting is reset on boot.  Using this option disables the
\textbf{Wake up alarm} (see below).

\item \textbf{Wake up alarm (Recorder v2/FM only)}
This option turns the Jukebox off and then starts it up again at the
specified time. This is most useful when combined with the Resume
setting in the Playback options set to ``Yes'', so that the Jukebox
wakes up and immediately  starts playing music. Use LEFT and RIGHT to
adjust the minutes setting, UP and DOWN to adjust the HOURS.  PLAY
confirms the alarm and shuts your Jukebox down, and STOP cancels
setting an alarm.  If the Jukebox is turned on again before the alarm
occurs the alarm will be canceled.  Using this option disables the \textbf{Sleep Timer} (see above).

\item \textbf{Limits}
This submenu relates to limits in the Rockbox operating system.

\begin{itemize}
\item \textbf{Max files in dir browser: }Configurable limit of files in
the directory browser (file buffer size). You can configure the size to
be between 50 and 10000 files in steps of 50 files. The default is 400,
higher values will shorten the music buffer.\\

Note: the device must be rebooted for settings to take effect! 
\item \textbf{Max playlist size: }Option to configure the maximum size
of a playlist. The playlist size can be between 1000 and 20000 files in
steps of 1000.  By default it is 10000.  Higher values will shorten the
music buffer.\\

Note: the device must be rebooted for settings to take effect! 
\end{itemize}

\item \textbf{Car Adapter Mode}
This option turns on and off the car ignition auto stop
function. 

When using the Jukebox in a car, car adapter mode automatically stops
playback on the Jukebox when power (i.e. from cigarette lighter power
adapter) to the external DC in jack is turned off. 

When the external power off condition is detected, the Car Adapter Mode
function only pauses the playback. In order to shut down the Jukebox
completely the \textbf{Idle Poweroff} function (see above) must also be
set.

If power to the DC in jack is turned back on before the \textbf{Idle
Poweroff}  function has shut the Jukebox off, playback will be resumed
5 seconds after the power is applied. This delay is to allow for the
time while the car engine is being started. Once the Jukebox is shut
off either manually, or automatically with the \textbf{Idle Poweroff
}function, it must be powered up manually to resume playback.

\item \textbf{Line In (Player only)}
This option activates the line in port on Jukebox Player, which
is off by default.

This is useful for such applications as:
\begin{itemize}
\item Game boy {}-{\textgreater} Jukebox {}-{\textgreater} human
\item laptop {}-{\textgreater} Jukebox {}-{\textgreater}human
\item LAN party computer {}-{\textgreater} Jukebox {}-{\textgreater} human 
\end{itemize}

\item \textbf{Manage settings}
This submenu deals with loading and saving settings.

\begin{itemize}
\item \textbf{Browse .cfg Files: }
This displays a list of configuration
(.cfg) files stored in the \textbf{/.rockbox} system directory.  This
is useful if the Jukebox is plugged into more than one different output
device (e.g. headphones, computer, car stereo, hi{}-fi) so that a settings file can be maintained for each.
\item \textbf{Browse Firmwares:} This displays a list of firmware (.mod
for Players and .ajz for Recorders) file in the \textbf{/.rockbox} system directory. Playing a firmware file loads it into memory.  Thus it is possible to
run the original Archos firmware or a different version of Rockbox from
here assuming that you have the right files installed on your disk.
\item \textbf{Reset Settings: }This wipes the saved settings in the
Jukebox and resets all settings to their default values.
\item \textbf{Write .cfg file: }Saves the current settings into a .cfg
file for later use with \textbf{Browse .cfg Files} above.
\end{itemize}

\end{itemize}

\subsubsection{\label{ref:Bookmarkconfigactual}Bookmarking}

\begin{itemize}
\item \textbf{Bookmark on Stop}
Write a bookmark to the disk whenever the stop key is pressed.  If
playback is stopped it can be resumed easily at a later time. The
\textbf{Resume} function remembers your position in the most
recently accessed track regardless of this setting.
\item \textbf{Load Last Bookmark}
When this is on, Rockbox automatically returns to the position of the
last bookmark within a file when it is played.  If set to Ask, Rockbox
will ask the user whether they want to start from the beginning or the
bookmark.  When set to no, playback always starts from the beginning
and the Bookmark file must be played or \textbf{Load Bookmark} selected
from the \textbf{Bookmarks} submenu of the Main Menu while the file is
playing.
\item \textbf{Maintain a list of Recently Used Bookmarks}
If this option is turned on, Rockbox will store a list of Bookmarks that
have been accessed recently.  This is then accessible from the
\textbf{Recent Bookmarks} option of the \textbf{Bookmarks} submenu of
the Main Menu.
\end{itemize}

\subsection{\label{ref:Language}Language}
This setting controls the language of the Rockbox user interface.
Selecting  a language will activate it. The language files must be in
the \textbf{/.rockbox/lang/} directory. 

See page \pageref{ref:Loadinglanguages} for further details about
languages.

\subsection{Voice}

\begin{itemize}
\item \textbf{Voice Menus}
This option turns on the Voice User Interface, which will read out menu items and settings as they are selected by the cursor.  In order for this to work, a voice file must be present in the \textbf{/.rockbox/lang/} directory on the recorder.  Voice files are large (1.5MB) and are not shipped with Rockbox by
default.

The voice file is the name of the language for which it is made,
followed by the extension .voice.  So for English, the file name would
be \textbf{english.voice}. 

This option is on by default.  It will do nothing unless the appropriate
.voice file is installed in the correct place on the Jukebox.

\begin{itemize}
\item \textbf{Limitations}
\begin{itemize}
\item Setting the Sound Option \textbf{Channels} to ``karaoke'' may
disable voice menus.  
\item Plugins and the wake up alarm do not support voice features.
\end{itemize}

\item \textbf{Voice Directories}
This option turns on the speaking of directory names.  The Jukebox is
not powerful enough to produce these voices in real time, so a number of options are available.

\begin{itemize}
\item \textbf{.talk mp3 clip: }
Use special pre{}-recorded MP3 files (\textbf{\_dirname.talk}) in each directory.  These must be generated in advance, and are typically produced synthetically using a text to speech engine on a PC.  If no such file exists, the output is as for the ``numbers'' option below.
\item \textbf{Spell: }
Speak the directory name by spelling it out letter
by letter.  Support is provided only for the most common letters and
punctuation.
\item \textbf{Numbers: }
Each directory is assigned a number based upon its position in the file list.  They are then announced as ``Directory 1'', ``Directory 2'' etc.
\item \textbf{Off: }
No attempt will be made to speak directory names.
\end{itemize}

\item \textbf{Voice Filenames}
This option turns on the speaking of directory names.  The options
provided are ``Spell'', ``Numbers'', and ``Off'' which function the same as for \textbf{Voice Directories} and ``.talk mp3 clip'', which functions as above except that the files are named with the same name as the music file (e.g. \textbf{Punkadiddle.mp3 } would require a file called \textbf{Punkadiddle.mp3.talk}).
\end{itemize}
\end{itemize}
See
\url{http://www.rockbox.org/twiki/bin/view/Main/VoiceHowto} for more details on configuring speech support in Rockbox.


\subsubsection{Status Bar}
\begin{tabularx}{\textwidth}{lX}\toprule
\textbf{Tag} & \textbf{Description}\\\midrule
\%we & Status Bar Enabled\\
\%wd & Status Bar Disabled\\\bottomrule
\end{tabularx}
\newline
\newline
These tags override the player setting for the display of the status bar, they must be on their own line.

\subsubsection{ID3 Info}

  \begin{tabularx}{\textwidth}{lX}\toprule
    \textbf{Tag} & \textbf{Description}\\\midrule
    \%ia & ID3 Artist\\
    \%ic & ID3 Composer\\
    \%id & ID3 Album Name\\
    \%ig & ID3 Genre Name\\
    \%in & ID3 Track Number\\
    \%it & ID3 Track Title\\
    \%iv & ID3 Version (1.0, 1.1, 2.2, 2.3, 2.4 or empty if no id3 tag)\\
    \%iy & ID3 Year\\\bottomrule
  \end{tabularx}
\newline
\newline
Remember that this information is not always available, so use the conditionals to show alternate information in preference to assuming.

\subsubsection{Power Related Information}

  \begin{tabularx}{\textwidth}{lX}\toprule
    \textbf{Tag} & \textbf{Description}\\\midrule
    \%bl & Show numeric battery level in percent.\\
         & Can also be used in a conditional: \%?bl{\textless}0{\textbar}1{\textbar}2{\textbar}3{\textbar}4{\textgreater}\\
    \%bv & Show the battery level in volts\\
    \%bt & Show estimated battery time left\\
    \%bp & "p" if the charger is connected \\
         & (only on targets that can charge batteries)\\
    \%bc & "c" if the unit is currently charging the battery\\
         & (only on targets that have software charge control or monitoring)\\
    \%bs & Sleep timer. Shows the remaining time if the sleeptimer is set\\
  \bottomrule
  \end{tabularx}

\subsubsection{File Info}

  \begin{tabularx}{\textwidth}{lX}\toprule
    \textbf{Tag} & \textbf{Description}\\\midrule
    \%fb & File Bitrate (in kbps)\\
    \%fc & File Codec (e.g. "MP3" or "FLAC")\\
         & This tag can also be used in a conditional tag,\\
         & \%?fc{\textless}mp1{\textbar}mp2{\textbar}mp3{\textbar}wav{\textbar}vorbis{\textbar}flac{\textbar}mpc{\textbar}a52{\textbar}wavpack{\textbar}unknown{\textgreater} %
           The codec order is as follows: MP1, MP2, MP3, WAV, Ogg Vorbis (OGG),%
           FLAC, MPC, AC3, WavPack (WV), ALAC, AAC, Shorten (SHN), AIFF\\
    \%ff & File Frequency (in Hz)\\
    \%fm & File Name\\
    \%fn & File Name (without extension)\\
    \%fp & File Path\\
    \%fs & File Size (In Kilobytes)\\
    \%fv & "(avg)" if variable bit rate or "" if constant bit rate\\
    \%d1 & First directory from end of file path.\\
    \%d2 & Second directory from end of file path.\\
    \%d3 & Third directory from end of file path.\\\bottomrule
  \end{tabularx}
\newline
\newline
Example for the the \%dN commands: If the path is "/Rock/Kent/Isola/11 - 747.mp3", \%d1 is "Isola", \%d2 is "Kent"... You get the picture.

\subsubsection{Playlist/Song Info}

  \begin{tabularx}{\textwidth}{lX}\toprule
    \textbf{Tag} & \textbf{Description}\\\midrule
    \%pb & Progress Bar\\
    \opt{player}{
          & This will display a 1 character "cup" %
            that empties as the time progresses.}
    \opt{recorder,recorderv2fm,h1xx,h300,ipodcolor,ipodnano}{
         & This will replace the entire line with a progress bar. \\
         & You can set the height, position and width of the progressbar %
           (in pixels): \%pb{\textbar}height{\textbar}leftpos{\textbar}rightpos{\textbar}} \\
    \%pf & Player: Full-line progress bar + time display\\
    \%pc & Current Time In Song\\
    \%pe & Total Number of Playlist Entries\\
    \%pm & Peak Meter (Recorder only) The entire line is used as volume peak meter.\\
    \%pn & Playlist Name (Without path or extension)\\
    \%pp & Playlist Position\\
    \%pr & Remaining Time In Song\\
    \%ps & Shuffle. Shows 's' if shuffle mode is enabled.\\
    \%pt & Total Track Time\\
    \%pv & Current volume. Can also be used in a conditional: \\
         & \%?pv{\textless}0{\textbar}1{\textbar}2{\textbar}3{\textbar}4{\textbar}5{\textbar}6{\textbar}7{\textbar}8{\textbar}9{\textbar}10{\textgreater}\\\bottomrule
  \end{tabularx}

\subsubsection{Runtime Database}

  \begin{tabularx}{\textwidth}{lX}\toprule
    \textbf{Tag} & \textbf{Description}\\\midrule
    \%rp & Song playcount\\
    \%rr & Song rating (0-10). This tag can also be used in a conditional tag, %
           \%?rr{\textless}0{\textbar}1{\textbar}2{\textbar}3{\textbar}4{\textbar}5{\textbar}6{\textbar}7{\textbar}8{\textbar}9{\textbar}10{\textgreater}\\\bottomrule
  \end{tabularx}

\opt{h1xx,h300}{
\subsubsection{Hold Switches}

  \begin{tabularx}{\textwidth}{lX}\toprule
    \textbf{Tag} & \textbf{Description}\\\midrule
    \%mh & "h" if the main unit hold switch is on\\
    \%mr & "r" if the remote hold switch is on\\\bottomrule
  \end{tabularx}
}

\subsubsection{Virtual LED}

  \begin{tabularx}{\textwidth}{lX}\toprule
    \textbf{Tag} & \textbf{Description}\\\midrule
    \%lh & "h" if there is hard disk activity\\\bottomrule
  \end{tabularx}

\subsubsection{Repeat Mode}

  \begin{tabularx}{\textwidth}{lX}\toprule
    \textbf{Tag} & \textbf{Description}\\\midrule
    \%mm & Repeat mode, 0-4, in the order: Off, All, One, Shuffle\opt{player,recorder,recorderv2fm}{, A-B}\\\bottomrule
  \end{tabularx}
\newline
\newline
Example: \%?mm{\textless}Off{\textbar}All{\textbar}One{\textbar}Shuffle{\textbar}A-B{\textgreater}

\subsubsection{Playback Mode Tags}

  \begin{tabularx}{\textwidth}{lX}\toprule
    \textbf{Tag} & \textbf{Description}\\\midrule
    \%mp & Play status, 0-4, in the order: Stop, Play, Pause, Fast forward, Rewind\\\bottomrule
  \end{tabularx}
\newline
\newline
Example: \%?mp{\textless}Stop{\textbar}Play{\textbar}Pause{\textbar}Ffwd{\textbar}Rew{\textgreater}

\subsubsection{Images}

  \begin{tabularx}{\textwidth}{lX}\toprule
    \textbf{Tag} & \textbf{Description}\\\midrule
    \%X{\textbar}filename.bmp{\textbar} & Load and set a backdrop image for the WPS. %
                        This image must be exactly the same size as your LCD.\\
    \%P{\textbar}filename.bmp{\textbar} & Load a Progress bar image for the WPS. %
                        Use \%pb tag to show the progress bar\\
    \%x{\textbar}n{\textbar}filename{\textbar}x{\textbar}y{\textbar} & Load and display an image\\
                        & n = image ID (a-z and A-Z)\\
                        & filename = filename (relative to /.rockbox/ and including .bmp)\\
                        & x = x coordinate\\
                        & y = y coordinate.\\
    \%xl{\textbar}n{\textbar}filename{\textbar}x{\textbar}y{\textbar} & Preload an image for later display\\
                         & n = image ID (a-z and A-Z)\\
                         & filename = filename (relative to /.rockbox/ and including .bmp)\\
                         & x = x coordinate\\
                         & y = y coordinate.\\
    \%xdn & Display a preloaded image\\

          & n = image ID (a-z and A-Z)\\\bottomrule
  \end{tabularx}
\newline
\newline
Example: image /.rockbox/bg.bmp with ID "a" at 37, 109 would be:\\
\%x{\textbar}a{\textbar}bg.bmp{\textbar}37{\textbar}109{\textbar}

\begin{itemize}
\item \textbf{Note:} The images must be in a rockbox compatible format (1 bit per pixel BMP)
\item \textbf{Note:} The image tag must be on its own line
\item \textbf{Note:} The ID is case sensitive, giving 52 different ID's
\item \textbf{Note:} The size of the LCD screen for each player varies. See table below for appropriate sizes of each device. The x and y coordinates must repect each of the players' limits.
\end{itemize}

\subsubsection{Alignment}

  \begin{tabularx}{\textwidth}{lX}\toprule
    \textbf{Tag} & \textbf{Description}\\\midrule
    \%al & Text is left aligned\\
    \%ac & Text is center aligned\\
    \%ar & Text is right aligned\\\bottomrule
  \end{tabularx}
\newline
\newline
All alignment tags may be present in one line, but they need to be in the order left - center - right. If the aligned texts overlap, they are merged.

\subsubsection{Conditional Tags}

\textbf{If/else}\\

Syntax: \%?xx{\textless}true{\textbar}false{\textgreater}\\

\textbf{Enumerations}\\

Syntax: \%?xx{\textless}alt1{\textbar}alt2{\textbar}alt3{\textbar}...{\textbar}else{\textgreater}\\

\subsection{Other Tags}
\begin{tabularx}{\textwidth}{lX}\toprule
\textbf{Tag} & \textbf{Description}\\\midrule
  \%\%          & Display a '\%'\\
  \%{\textless} & Display a '{\textless}'\\
  \%{\textbar}  & Display a '{\textbar}'\\
  \%{\textgreater} & Display a '{\textgreater}'\\
  \%;           & Display a ';'\\
  \%s           & Indicate that the line should scroll. Can occur anywhere in a line\\
                & (given that the text is displayed; see conditionals above). \\
                & You can specify up to 10 scrolling lines. Scrolling lines can not \\
                & contain dynamic content such as timers, peak meters or progress bars.\\\bottomrule

\end{tabularx}

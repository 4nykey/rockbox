\chapter{Plugins}
\newpage
Plugins are little programs that Rockbox can load and run. Plugins have
the file extension .rock.  Most of them can be started from the main
menu if you put them in the \textbf{/.rockbox/rocks} directory. Press
PLAY on them to start them. 

Viewer plugins get started automatically by
``playing'' an associated file (i.e. text
files, chip8 games), or from the ``Open with'' option on the File menu.

Plugins listed here have the platforms they run under (Player for
Jukebox players, Recorder for Jukebox recorders including Jukebox FM,
and Ondio for the Ondio SP and FM).  If no platforms are listed then
the plugin runs on all Rockbox platforms.\\

\textbf{The plugin loader}
Only one plugin can be loaded at a time. Plugins run in the GUI thread
and have exclusive control over the user interface. This means you
cannot switch back and forth between a plugin and Rockbox. A plugin is
loaded, run and then exited, which returns control to Rockbox.  Music
will carry on playing whilst plugins are being run.

\section{Games}
See also the Chip{}-8 emulator on page \pageref{ref:Chip8emulator}.

\opt{recorder,recorderv2fm,ondio,h1xx,h300}{% $Id$ %
\subsection{Flipit}
\screenshot{plugins/images/ss-flipit}{Flipit}{img:flipit}
Flipping the colour of the token under the cursor also flips the tokens
above, below, left and right of the cursor.  The aim is to end up with
a screen containing tokens of only one colour.

\begin{table}
\begin{btnmap}{}{}
\opt{PLAYER_PAD}{\ButtonOn{} / \ButtonMenu{} / \ButtonLeft{} / \ButtonRight}
\opt{RECORDER_PAD,ONDIO_PAD,IRIVER_H100_PAD,IRIVER_H300_PAD,IAUDIO_X5_PAD,SANSA_E200_PAD,SANSA_C200_PAD,GIGABEAT_PAD,MROBE100_PAD}
    {\ButtonUp{} / \ButtonDown{} / \ButtonLeft{} / \ButtonRight}
\opt{IPOD_4G_PAD,IPOD_3G_PAD}{\ButtonMenu{} / \ButtonPlay{} / \ButtonLeft{} / \ButtonRight}
\opt{IRIVER_H10_PAD}{\ButtonScrollUp{} / \ButtonScrollDown{} / \ButtonLeft{} / \ButtonRight}
    & Move the cursor \\
\opt{PLAYER_PAD,RECORDER_PAD}{\ButtonPlay}
\opt{ONDIO_PAD}{\ButtonMenu}
\opt{IRIVER_H100_PAD,IRIVER_H300_PAD,IPOD_4G_PAD,IPOD_3G_PAD,IAUDIO_X5_PAD,SANSA_E200_PAD,SANSA_C200_PAD,GIGABEAT_PAD,MROBE100_PAD}
    {\ButtonSelect}
\opt{IRIVER_H10_PAD}{\ButtonRew}
    & Flip \\
\opt{PLAYER_PAD}{\ButtonOn+\ButtonLeft}
\opt{RECORDER_PAD}{\ButtonFOne}
\opt{ONDIO_PAD}{\ButtonMenu+\ButtonLeft}
\opt{IRIVER_H100_PAD,IRIVER_H300_PAD}{\ButtonMode}
\opt{IPOD_4G_PAD,IPOD_3G_PAD}{\ButtonSelect+\ButtonLeft}
\opt{IAUDIO_X5_PAD,IRIVER_H10_PAD}{\ButtonPlay+\ButtonLeft}
\opt{SANSA_E200_PAD,SANSA_C200_PAD}{\ButtonRec+\ButtonLeft}
\opt{GIGABEAT_PAD,MROBE100_PAD}{\ButtonMenu}
    & Shuffle \\
\opt{PLAYER_PAD}{\ButtonOn+\ButtonRight}
\opt{RECORDER_PAD}{\ButtonFTwo}
\opt{ONDIO_PAD}{\ButtonMenu+\ButtonUp}
\opt{IRIVER_H100_PAD,IRIVER_H300_PAD}{\ButtonOn}
\opt{IPOD_4G_PAD,IPOD_3G_PAD}{\ButtonSelect+\ButtonPlay}
\opt{IAUDIO_X5_PAD,IRIVER_H10_PAD}{\ButtonPlay+\ButtonRight}
\opt{SANSA_E200_PAD,SANSA_C200_PAD}{\ButtonRec+\ButtonRight}
\opt{GIGABEAT_PAD}{\ButtonVolUp}
\opt{MROBE100_PAD}{\ButtonMenu}
    & Solve \\
\opt{PLAYER_PAD}{\ButtonOn+\ButtonPlay}
\opt{RECORDER_PAD}{\ButtonFThree}
\opt{ONDIO_PAD}{\ButtonMenu+\ButtonRight}
\opt{IRIVER_H100_PAD,IRIVER_H300_PAD}{\ButtonRec}
\opt{IPOD_4G_PAD,IPOD_3G_PAD}{\ButtonSelect+\ButtonRight}
\opt{IAUDIO_X5_PAD}{\ButtonPlay+\ButtonUp}
\opt{IRIVER_H10_PAD}{\ButtonPlay+\ButtonScrollUp}
\opt{SANSA_E200_PAD,SANSA_C200_PAD}{\ButtonRec+\ButtonSelect}
\opt{GIGABEAT_PAD}{\ButtonVolDown}
\opt{MROBE100_PAD}{\ButtonDisplay}
    & Solve step by step \\
\opt{PLAYER_PAD}{\ButtonStop}
\opt{RECORDER_PAD,ONDIO_PAD,IRIVER_H100_PAD,IRIVER_H300_PAD}{\ButtonOff}
\opt{IPOD_4G_PAD,IPOD_3G_PAD}{\ButtonSelect+\ButtonMenu}
\opt{IAUDIO_X5_PAD,IRIVER_H10_PAD,SANSA_E200_PAD,GIGABEAT_PAD,MROBE100_PAD,SANSA_C200_PAD}{\ButtonPower}
    & Quit the game \\
\end{btnmap}
\end{table}
}

\opt{player}{% $Id$ %
\subsection{Jackpot}
\screenshot{plugins/images/ss-jackpot}{Jackpot}{img:Jackpot}
This is a jackpot slot machine game. At the beginning of the game you
have 20\$.  Payouts are given when three matching symbols come up.

\begin{btnmap}
    \opt{PLAYER_PAD,RECORDER_PAD}{\ButtonPlay}
    \opt{IRIVER_H100_PAD,IAUDIO_X5_PAD,GIGABEAT_PAD,GIGABEAT_S_PAD,%
      MROBE100_PAD,SANSA_CLIP_PAD,M200,IPOD_3G_PAD,IPOD_4G_PAD,%
      SANSA_C200_PAD,SANSA_E200_PAD,SANSA_FUZE_PAD}{\ButtonSelect}
    \opt{ONDIO_PAD}{\ButtonUp}
    \opt{IRIVER_H10_PAD}{\ButtonRew}
    \opt{IAUDIO_M3_PAD}{}
    \opt{COWON_D2_PAD}{\ButtonPlus{} or \TouchTopMiddle}
    \opt{PBELL_VIBE500_PAD}{\ButtonOK}
      \opt{HAVEREMOTEKEYMAP}{& 
          \opt{IRIVER_RC_H100_PAD}{\ButtonRCSelect}
          \opt{IAUDIO_RC_PAD}{\ButtonRCMode}
        }
    & Play \\
    \opt{PLAYER_PAD}{\ButtonStop}
    \opt{IRIVER_H100_PAD,RECORDER_PAD,ONDIO_PAD}{\ButtonOff}
    \opt{IPOD_3G_PAD,IPOD_4G_PAD}{\ButtonMenu}
    \opt{IAUDIO_X5_PAD,GIGABEAT_PAD,SANSA_E200_PAD,SANSA_C200_PAD,%
      SANSA_CLIP_PAD,IRIVER_H10_PAD,MROBE100_PAD}{\ButtonPower}
    \opt{GIGABEAT_S_PAD}{\ButtonBack}
    \opt{SANSA_FUZE_PAD}{\ButtonHome}
    \opt{IAUDIO_M3_PAD}{}
    \opt{COWON_D2_PAD}{\ButtonPower{} or \TouchBottomRight}
    \opt{PBELL_VIBE500_PAD}{\ButtonRec}
      \opt{HAVEREMOTEKEYMAP}{& 
          \opt{IRIVER_RC_H100_PAD}{\ButtonRCStop}
          \opt{IAUDIO_RC_PAD}{\ButtonRCRec}
        }
    & Exit the game \\
\end{btnmap}
}

\opt{recorder,recorderv2fm,ondio,h1xx,h300}{\subsection{Minesweeper}
\begin{figure}[h!]
\begin{center}
\includegraphics[width=4cm]{plugins/images/ss-minesweeper-\genericimg.png}
\end{center}
\caption{Minesweeper plugin}
\end{figure}

The classic game of minesweeper.  Use the UP and DOWN keys to select the
required percentage of mines to set the difficulty then press the MENU
key to begin.

The aim of the game is to uncover all of the squares on the board.  If a
mine is uncovered then the game is over.  If a mine is not uncovered,
then the number of mines adjacent to the current square is revealed. 
The aim is to use the information you are given to work out where the
mines are and avoid them.  When the player is certain that they know
the location of a mine, it can be tagged to avoid accidentally
``stepping'' on it.

\begin{table}[h!]
\begin{tabular}{@{}ll@{}}\toprule
\textbf{Key} & \textbf{Action} \\\midrule
UP/DOWN/LEFT/RIGHT & Move the cursor across the minefield \\
PLAY / F1 & Toggle flag on / off \\
MENU / F2 & Reveal the contents of the current square \\
STOP & Exit the game \\\bottomrule
\end{tabular}
\end{table}
}

\opt{player}{% $Id$ %
\subsection{Nim}
\screenshot{plugins/images/ss-nim}{Nim}{img:Nim}
Rules of Nim: There are 21 matches. Two players (you and the Jukebox)
alternately pick a certain number of matches and the one who takes the
last match loses.  You can take up to twice as many matches as the
Jukebox selected, and vice versa. 

\begin{table}
\begin{btnmap}{}{}
\ButtonLeft  & Decrease the number of matches \\
\ButtonRight & Increase the number of matches \\
\ButtonPlay   & Remove the number of matches you have selected \\
\ButtonStop  & Exit the game \\
\end{btnmap}
\end{table}

}

\opt{recorder,recorderv2fm,ondio,h1xx,h300}{\subsection{Rockblox}
\screenshot{plugins/images/ss-rockblox-\genericimg.png}{Rockblox plugin}%
{img:rockblox}

This well{}-known game will probably be familiar. The aim of the game is
to complete rows with the given pieces (blocks). Pieces can be rotated
to make them fit into the rows.  Once you complete a row, it gets
cleared, but if the blocks reach the top row then you lose.

The controls for this game (with the Jukebox turned so that the buttons
are to the right of the screen) are:

\begin{table}[h!]
\begin{center}
\begin{tabular}{@{}ll@{}}\toprule
\textbf{Key} & \textbf{Action} \\\midrule
UP & Rotate piece \\
LEFT/RIGHT & Move piece to the left/right \\
DOWN & Move faster the piece downwards \\
OFF & Exit Rockblox\\\bottomrule
\end{tabular}
\end{center}
\end{table}
}

\opt{recorder,recorderv2fm,ondio,h1xx,h300}{\subsection{Sliding Puzzle}
\screenshot{plugins/images/ss-sliding}{Sliding puzzle}{img:slidingpuzzle}

The classic sliding puzzle game.  Rearrange the pieces so that you can
see the whole picture, or switch to number tiles if you like it a little easier

Key controls:

\begin{table}
  \begin{btnmap}{}{}
  \opt{RECORDER_PAD,ONDIO_PAD,h100,h300,x5,SANSA_E200_PAD,SANSA_C200_PAD%
       ,GIGABEAT_PAD,GIGABEAT_S_PAD,MROBE100_PAD}
    {\ButtonLeft, \ButtonRight, \ButtonUp\ and \ButtonDown}
    \opt{IRIVER_RC_H100_PAD}{&} 
  \opt{IPOD_4G_PAD,IPOD_3G_PAD}
    {\ButtonLeft{} / \ButtonRight{} / \ButtonMenu{} / \ButtonPlay}
  \opt{IRIVER_H10_PAD}
    {\ButtonLeft{} / \ButtonRight{} / \ButtonScrollUp{} / \ButtonScrollDown}
  & Move Tile \\
  %
  \opt{RECORDER_PAD}{\ButtonFOne}
  \opt{ONDIO_PAD}{Long \ButtonMenu}
  \opt{h100,h300,GIGABEAT_PAD,GIGABEAT_S_PAD,MROBE100_PAD}{\ButtonSelect} 
    \opt{IRIVER_RC_H100_PAD}{&} 
  \opt{x5,SANSA_E200_PAD,SANSA_C200_PAD}{\ButtonRec}
  \opt{IPOD_4G_PAD,IPOD_3G_PAD}{\ButtonSelect+\ButtonLeft}
  \opt{IRIVER_H10_PAD}{\ButtonRew}
  & Shuffle \\
  %
  \opt{RECORDER_PAD}{\ButtonFTwo}
  \opt{ONDIO_PAD}{\ButtonMenu}
  \opt{h100,h300}{\ButtonOn}
    \opt{IRIVER_RC_H100_PAD}{&} 
  \opt{x5,IRIVER_H10_PAD}{\ButtonPlay}
  \opt{IPOD_4G_PAD,IPOD_3G_PAD}{\ButtonSelect+\ButtonRight}
  \opt{SANSA_E200_PAD,SANSA_C200_PAD}{\ButtonSelect}
  \opt{GIGABEAT_PAD}{\ButtonA}
  \opt{MROBE100_PAD}{\ButtonDisplay}
  \opt{GIGABEAT_S_PAD}{\ButtonMenu}
  & Change between picture and numbered tiles \\
  %
  \opt{RECORDER_PAD,ONDIO_PAD,IRIVER_H100_PAD,IRIVER_H300_PAD}{\ButtonOff}
    \opt{IRIVER_RC_H100_PAD}{&} 
  \opt{x5,IRIVER_H10_PAD,SANSA_E200_PAD,SANSA_C200_PAD,GIGABEAT_PAD,MROBE100_PAD}
    {\ButtonPower}
  \opt{IPOD_4G_PAD,IPOD_3G_PAD}{\ButtonSelect+\ButtonMenu}
  \opt{GIGABEAT_S_PAD}{\ButtonBack}
  & Stop the game \\
  \end{btnmap}
\end{table}
}

\opt{recorder,recorderv2fm,ondio}{\subsection{Snake}
\screenshot{plugins/images/ss-snake}{Snake}{fig:snake}

This is the popular snake game. The aim is to grow your snake as large
as possible by eating the dots that appear on the screen. The game will
end when the snake touches either the borders of the screen or itself.

\begin{btnmap}
     \opt{RECORDER_PAD,IAUDIO_X5_PAD,ONDIO_PAD,IRIVER_H100_PAD,IRIVER_H300_PAD%
      ,SANSA_E200_PAD,SANSA_C200_PAD,GIGABEAT_PAD,GIGABEAT_S_PAD,MROBE100_PAD%
      ,SANSA_FUZE_PAD,PBELL_VIBE500_PAD}
      {\ButtonUp/\ButtonDown}
     \opt{IPOD_3G_PAD,IPOD_4G_PAD}{\ButtonMenu/\ButtonPlay}
     \opt{IRIVER_H10_PAD,MPIO_HD300_PAD}{\ButtonScrollUp/\ButtonScrollDown}
    \opt{COWON_D2_PAD}{\TouchTopMiddle/\TouchBottomMiddle}
       \opt{HAVEREMOTEKEYMAP}{& }
     & Change levels (1 is slowest, 9 is fastest)\\
    %
    \opt{RECORDER_PAD,IAUDIO_X5_PAD,IRIVER_H10_PAD,PBELL_VIBE500_PAD}{\ButtonPlay}
    \opt{IPOD_4G_PAD,IPOD_3G_PAD,SANSA_E200_PAD,SANSA_C200_PAD,GIGABEAT_PAD%
      ,GIGABEAT_S_PAD,MROBE100_PAD,SANSA_FUZE_PAD}
      {\ButtonSelect}
    \opt{IRIVER_H100_PAD,IRIVER_H300_PAD}{\ButtonOn}
    \opt{ONDIO_PAD}{\ButtonMenu}
    \opt{COWON_D2_PAD}{\TouchCenter}
    \opt{MPIO_HD300_PAD}{\ButtonPlay}
       \opt{HAVEREMOTEKEYMAP}{& }
     & Toggle Play/Pause\\
    %
\end{btnmap}
}

\opt{recorder,recorderv2fm,ondio,h1xx,h300}{\subsection{Snake 2}
\screenshot{plugins/images/ss-snake2-\genericimg.png}%
{Snake 2 {--} The Snake Strikes Back}{img:snake2}

Another version of the Snake game. Move the snake around, and eat the
apples that pop up on the screen. Each time an apple is eaten, the
snake gets longer. The game ends when the snake hits a wall, or runs
into itself. 

The controls are:

\begin{table}[h!]
\begin{center}
\begin{tabular}{@{}ll@{}}\toprule
\textbf{Key} & \textbf{Action} \\\midrule
UP/DOWN & (in menu) Set game speed \\
F1 & (in menu) Change starting maze \\
F3 & (in menu) Select game type (A or B) \\
UP/DOWN/LEFT/RIGHT & Steer the snake \\
PLAY & Pause the game \\
STOP & Exit the game \\\bottomrule
\end{tabular}
\end{center}
\end{table}
In game A, the maze stays the same, in Game B
after an increasing number of apples eaten the maze is replaced by a
new one.

}

\opt{recorder,recorderv2fm,ondio,h1xx,h300}{\subsection{Sokoban}
\screenshot{plugins/images/ss-sokoban}{Sokoban}{fig:sokoban}

The object of the game is to push boxes into their correct position in a
crowded warehouse with a minimal number of pushes and moves. The boxes
can only be pushed, never pulled, and only one can be pushed at a time.

Sokoban may be used as a viewer for viewing saved solutions and playing
external level sets with the \fname{.sok} extension. Level sets should be in
the standard Sokoban text format or RLE (Run Length Encoded). For more
information about the level format, see
\url{http://sokobano.de/wiki/index.php?title=Level_format}

\begin{btnmap}
\multicolumn{2}{c}{\textbf{In game}} \\
\hline
\opt{RECORDER_PAD,ARCHOS_AV300_PAD,ONDIO_PAD,IRIVER_H100_PAD,IRIVER_H300_PAD%
    ,IAUDIO_X5_PAD,GIGABEAT_PAD,GIGABEAT_S_PAD,MROBE100_PAD,SANSA_E200_PAD%
    ,SANSA_FUZE_PAD,SANSA_C200_PAD,PBELL_VIBE500_PAD,SANSA_FUZEPLUS_PAD}
    {\ButtonUp, \ButtonDown, }%
\opt{IPOD_4G_PAD,IPOD_3G_PAD}{\ButtonMenu, \ButtonPlay, }%
\opt{IRIVER_H10_PAD}{\ButtonScrollUp, \ButtonScrollDown, }%
\opt{RECORDER_PAD,ARCHOS_AV300_PAD,ONDIO_PAD,IRIVER_H100_PAD,IRIVER_H300_PAD%
    ,IAUDIO_X5_PAD,GIGABEAT_PAD,GIGABEAT_S_PAD,MROBE100_PAD,SANSA_E200_PAD%
    ,SANSA_FUZE_PAD,SANSA_C200_PAD,IPOD_4G_PAD,IPOD_3G_PAD,IRIVER_H10_PAD,PBELL_VIBE500_PAD%
    ,SANSA_FUZEPLUS_PAD}
    {\ButtonLeft, \ButtonRight}
\opt{COWON_D2_PAD}
    {\TouchTopMiddle, \TouchBottomMiddle, \TouchMidLeft, \TouchMidRight}
\opt{MPIO_HD300_PAD}{\ButtonScrollUp, \ButtonScrollDown, \ButtonRew, \ButtonFF}
       \opt{HAVEREMOTEKEYMAP}{& }
    & Move the ``sokoban'' up, down, left, or right\\
\opt{RECORDER_PAD,ARCHOS_AV300_PAD,ONDIO_PAD,IRIVER_H100_PAD,IRIVER_H300_PAD}{\ButtonOff}
\opt{IPOD_4G_PAD,IPOD_3G_PAD}{\ButtonSelect+\ButtonMenu}
\opt{IAUDIO_X5_PAD,IRIVER_H10_PAD,GIGABEAT_PAD,MROBE100_PAD,SANSA_E200_PAD,SANSA_C200_PAD%
    ,SANSA_FUZEPLUS_PAD}{\ButtonPower}
\opt{SANSA_FUZE_PAD}{Long \ButtonHome}
\opt{GIGABEAT_S_PAD,COWON_D2_PAD}{\ButtonMenu}
\opt{MPIO_HD300_PAD}{Long \ButtonMenu}
\opt{PBELL_VIBE500_PAD}{\ButtonRec}
       \opt{HAVEREMOTEKEYMAP}{&
          \opt{IRIVER_RC_H100_PAD}{\ButtonRCStop}
         }
    & Menu \\
\nopt{IAUDIO_X5_PAD}{
    \opt{RECORDER_PAD,ARCHOS_AV300_PAD}{\ButtonFOne}
    \opt{ONDIO_PAD}{\ButtonMenu+\ButtonLeft}
    \opt{IRIVER_H100_PAD,IRIVER_H300_PAD}{\ButtonOn+\ButtonDown}
    \opt{IPOD_4G_PAD,IPOD_3G_PAD}{\ButtonSelect+\ButtonLeft}
    \opt{IRIVER_H10_PAD}{\ButtonPlay+\ButtonScrollDown}
    \opt{GIGABEAT_PAD,SANSA_C200_PAD,SANSA_FUZEPLUS_PAD}{\ButtonVolDown}
    \opt{MROBE10_PAD}{\ButtonDisplay}
    \opt{SANSA_E200_PAD,SANSA_FUZE_PAD}{\ButtonSelect+\ButtonDown}
    \opt{GIGABEAT_S_PAD}{\ButtonPrev}
    \opt{COWON_D2_PAD}{\ButtonMinus}
    \opt{PBELL_VIBE500_PAD}{\ButtonOK+\ButtonLeft}
    \opt{MPIO_HD300_PAD}{\ButtonPlay + \ButtonRew}
       \opt{HAVEREMOTEKEYMAP}{& }
    & Back to previous level \\
}
\nopt{IPOD_4G_PAD,IPOD_3G_PAD}{
    \opt{RECORDER_PAD,ARCHOS_AV300_PAD}{\ButtonFTwo}
    \opt{ONDIO_PAD}{\ButtonMenu+\ButtonUp}
    \opt{IRIVER_H100_PAD,IRIVER_H300_PAD}{\ButtonOn}
    \opt{IAUDIO_X5_PAD}{\ButtonRec}
    \opt{IRIVER_H10_PAD}{\ButtonPlay+\ButtonRight}
    \opt{GIGABEAT_PAD,MROBE100_PAD}{\ButtonMenu}
    \opt{SANSA_E200_PAD,SANSA_FUZE_PAD,SANSA_C200_PAD}{\ButtonSelect+\ButtonRight}
    \opt{GIGABEAT_S_PAD}{\ButtonPlay}
    \opt{COWON_D2_PAD}{\TouchTopRight}
    \opt{PBELL_VIBE500_PAD}{\ButtonOK+\ButtonCancel}
    \opt{MPIO_HD300_PAD}{\ButtonPlay + \ButtonEnter}
    \opt{SANSA_FUZEPLUS_PAD}{\ButtonBack}
       \opt{HAVEREMOTEKEYMAP}{& }
    & Restart level \\
}
\nopt{IAUDIO_X5_PAD}{
    \opt{RECORDER_PAD,ARCHOS_AV300_PAD}{\ButtonFThree}
    \opt{ONDIO_PAD}{\ButtonMenu+\ButtonRight}
    \opt{IRIVER_H100_PAD,IRIVER_H300_PAD}{\ButtonOn+\ButtonUp}
    \opt{IPOD_4G_PAD,IPOD_3G_PAD}{\ButtonSelect+\ButtonRight}
    \opt{IRIVER_H10_PAD}{\ButtonPlay+\ButtonScrollUp}
    \opt{GIGABEAT_PAD,SANSA_C200_PAD,SANSA_FUZEPLUS_PAD}{\ButtonVolUp}
    \opt{GIGABEAT_S_PAD}{\ButtonNext}
    \opt{MROBE100_PAD}{\ButtonPlay}
    \opt{SANSA_E200_PAD,SANSA_FUZE_PAD}{\ButtonSelect+\ButtonUp}
    \opt{COWON_D2_PAD}{\ButtonPlus}
    \opt{PBELL_VIBE500_PAD}{\ButtonOK+\ButtonRight}
    \opt{MPIO_HD300_PAD}{\ButtonPlay + \ButtonFF}
       \opt{HAVEREMOTEKEYMAP}{& }
    & Go to next level \\
}
\opt{RECORDER_PAD,ARCHOS_AV300_PAD}{\ButtonOn}
\opt{ONDIO_PAD}{\ButtonMenu}
\opt{IRIVER_H100_PAD,IRIVER_H300_PAD}{\ButtonRec}
\opt{IPOD_4G_PAD,IPOD_3G_PAD,IAUDIO_X5_PAD,GIGABEAT_PAD,MROBE100_PAD%
     ,SANSA_E200_PAD,SANSA_FUZE_PAD,SANSA_C200_PAD}{\ButtonSelect}
\opt{IRIVER_H10_PAD}{\ButtonRew}
\opt{GIGABEAT_S_PAD}{\ButtonVolUp}
\opt{COWON_D2_PAD}{\TouchTopRight}
\opt{SANSA_FUZEPLUS_PAD}{\ButtonBottomLeft}
\opt{PBELL_VIBE500_PAD}{\ButtonCancel}
\opt{MPIO_HD300_PAD}{\ButtonRec}
       \opt{HAVEREMOTEKEYMAP}{& }
    & Undo last movement \\

\opt{RECORDER_PAD,ARCHOS_AV300_PAD,IAUDIO_X5_PAD}{\ButtonPlay}
\opt{ONDIO_PAD}{\ButtonMenu+\ButtonDown}
\opt{IRIVER_H100_PAD,IRIVER_H300_PAD}{\ButtonMode}
\opt{IPOD_4G_PAD,IPOD_3G_PAD}{\ButtonSelect+\ButtonPlay}
\opt{IRIVER_H10_PAD}{\ButtonFF}
\opt{GIGABEAT_PAD}{\ButtonA}
\opt{GIGABEAT_S_PAD}{\ButtonVolDown}
\opt{MROBE100_PAD}{\ButtonDisplay}
\opt{SANSA_E200_PAD,SANSA_C200_PAD}{\ButtonRec}
\opt{SANSA_FUZE_PAD}{\ButtonSelect+\ButtonLeft}
\opt{COWON_D2_PAD}{\TouchBottomLeft}
\opt{SANSA_FUZEPLUS_PAD}{\ButtonBottomRight}
\opt{PBELL_VIBE500_PAD}{\ButtonOK}
\opt{MPIO_HD300_PAD}{\ButtonPlay}
       \opt{HAVEREMOTEKEYMAP}{& }
    & Redo previously undone move \\
\hline
\multicolumn{2}{c}{\textbf{Solution playback}} \\
\hline
\opt{RECORDER_PAD,ARCHOS_AV300_PAD,IAUDIO_X5_PAD,IRIVER_H10_PAD%
    ,MPIO_HD300_PAD,SANSA_FUZEPLUS_PAD}{\ButtonPlay}
\opt{ONDIO_PAD}{\ButtonMenu}
\opt{IRIVER_H100_PAD,IRIVER_H300_PAD}{\ButtonOn}
\opt{IPOD_4G_PAD,IPOD_3G_PAD,GIGABEAT_PAD,GIGABEAT_S_PAD,MROBE100_PAD%
      ,SANSA_E200_PAD,SANSA_FUZE_PAD,SANSA_C200_PAD}{\ButtonSelect}
\opt{COWON_D2_PAD}{\TouchCenter}
       \opt{HAVEREMOTEKEYMAP}{& }
    & Pause/resume \\
\opt{RECORDER_PAD,ARCHOS_AV300_PAD,ONDIO_PAD,IRIVER_H100_PAD,IRIVER_H300_PAD%
    ,IAUDIO_X5_PAD,GIGABEAT_PAD,GIGABEAT_S_PAD,MROBE100_PAD,SANSA_E200_PAD%
    ,SANSA_FUZE_PAD,SANSA_C200_PAD,SANSA_FUZEPLUS_PAD}
    {\ButtonUp/\ButtonDown}
\opt{IPOD_4G_PAD,IPOD_3G_PAD}{\ButtonMenu/\ButtonPlay}
\opt{IRIVER_H10_PAD,MPIO_HD300_PAD}{\ButtonScrollUp/\ButtonScrollDown}
\opt{COWON_D2_PAD}{\TouchTopMiddle/\TouchBottomMiddle}
       \opt{HAVEREMOTEKEYMAP}{& }
    & Increase/decrease playback speed \\
\opt{RECORDER_PAD,ARCHOS_AV300_PAD,IAUDIO_X5_PAD,IRIVER_H10_PAD,ONDIO_PAD%
    ,IRIVER_H100_PAD,IRIVER_H300_PAD,IPOD_4G_PAD,IPOD_3G_PAD,GIGABEAT_PAD%
    ,GIGABEAT_S_PAD,MROBE100_PAD,SANSA_E200_PAD,SANSA_FUZE_PAD,SANSA_C200_PAD%
    ,SANSA_FUZEPLUS_PAD}
    {\ButtonLeft/\ButtonRight}
\opt{COWON_D2_PAD}{\TouchMidLeft/\TouchMidRight}
\opt{MPIO_HD300_PAD}{\ButtonRew/\ButtonFF}
       \opt{HAVEREMOTEKEYMAP}{& }
    & Go backward/forward (while paused) \\
\opt{RECORDER_PAD,ARCHOS_AV300_PAD,ONDIO_PAD,IRIVER_H100_PAD,IRIVER_H300_PAD}{\ButtonOff}
\opt{IPOD_4G_PAD,IPOD_3G_PAD}{\ButtonSelect+\ButtonMenu}
\opt{IAUDIO_X5_PAD,IRIVER_H10_PAD,GIGABEAT_PAD,MROBE100_PAD,SANSA_E200_PAD,SANSA_C200_PAD%
    ,SANSA_FUZEPLUS_PAD}{\ButtonPower}
\opt{SANSA_FUZE_PAD}{Long \ButtonHome}
\opt{GIGABEAT_S_PAD,COWON_D2_PAD}{\ButtonMenu}
\opt{MPIO_HD300_PAD}{Long \ButtonMenu}
       \opt{HAVEREMOTEKEYMAP}{& }
    & Quit \\
\end{btnmap}

Some places where can you can find level sets:
\begin{itemize}
\item \url{http://www.sourcecode.se/sokoban/levels.php}
\item \url{http://sokobano.de/en/levels.php}
\end{itemize}
Note that some level sets may contain levels that are too large for this
version of Sokoban and are unplayable as a result.
}

\opt{recorder,recorderv2fm,ondio,h1xx,h300}{\subsection{Solitaire}
\screenshot{plugins/images/ss-solitaire}{Klondike solitaire}{fig:solitaire}

This is the classic Klondike solitaire game for Rockbox.
This is probably the best-known solitaire in the world. Many people 
do not even realize that other games exist. Though the name may not 
be familiar, the game itself certainly is. This is due in no small 
part to Microsoft's inclusion of the the game in every version of 
Windows. Though popular, the odds of winning are rather low, perhaps 
one in thirty hands.

For the full set of rules to the game, and other interesting information
visit
\url{http://www.solitairecentral.com/rules/klondike.html}

\begin{table}
  \begin{btnmap}{}{}
    \opt{RECORDER_PAD,ONDIO_PAD,IRIVER_H100_PAD,IRIVER_H300_PAD,IAUDIO_X5_PAD%
        ,GIGABEAT_PAD,GIGABEAT_S_PAD,MROBE100_PAD,SANSA_C200_PAD,PBELL_VIBE500_PAD}
      {\ButtonUp{} / \ButtonDown}
    \opt{IPOD_4G_PAD,IPOD_3G_PAD,SANSA_E200_PAD,SANSA_FUZE_PAD}{\ButtonScrollFwd{} / \ButtonScrollBack}
    \opt{IRIVER_H10_PAD}{\ButtonScrollUp{} / \ButtonScrollDown}
    \opt{RECORDER_PAD,ONDIO_PAD,IRIVER_H100_PAD,IRIVER_H300_PAD,IAUDIO_X5_PAD%
        ,GIGABEAT_PAD,GIGABEAT_S_PAD,MROBE100_PAD,SANSA_C200_PAD,IPOD_4G_PAD%
        ,IPOD_3G_PAD,SANSA_E200_PAD,SANSA_FUZE_PAD,IRIVER_H10_PAD,PBELL_VIBE500_PAD}
      {/ \ButtonLeft{} / \ButtonRight}
    \opt{COWON_D2_PAD}{\TouchTopMiddle{} / \TouchBottomMiddle{} / \TouchMidLeft{} / \TouchMidRight}
       \opt{HAVEREMOTEKEYMAP}{& }
      & Move Cursor around.\\
    %
    \opt{RECORDER_PAD}{\ButtonOn}
    \opt{ONDIO_PAD}{\ButtonMenu}
    \opt{IRIVER_H100_PAD,IRIVER_H300_PAD,IPOD_4G_PAD,IPOD_3G_PAD,IAUDIO_X5_PAD%
      ,SANSA_E200_PAD,SANSA_FUZE_PAD,SANSA_C200_PAD,GIGABEAT_PAD,GIGABEAT_S_PAD%
      ,MROBE100_PAD}
      {\ButtonSelect}
    \opt{IRIVER_H10_PAD}{\ButtonPlay}
    \opt{COWON_D2_PAD}{\TouchCenter}
    \opt{PBELL_VIBE500_PAD}{\ButtonOK}
       \opt{HAVEREMOTEKEYMAP}{& }
      & Select cards, move cards, reveal hidden cards...\\
    %
    \opt{RECORDER_PAD}{\ButtonFOne}
    \opt{ONDIO_PAD}{Long \ButtonMenu}
    \opt{IRIVER_H100_PAD,IRIVER_H300_PAD}{\ButtonMode}
    \opt{IPOD_4G_PAD,IPOD_3G_PAD,GIGABEAT_PAD,GIGABEAT_S_PAD,MROBE100_PAD}
      {\ButtonMenu}
    \opt{IAUDIO_X5_PAD}{\ButtonPlay}
    \opt{IRIVER_H10_PAD}{Long \ButtonLeft}
    \opt{SANSA_E200_PAD}{\ButtonRec}
    \opt{SANSA_FUZE_PAD}{\ButtonHome}
    \opt{SANSA_C200_PAD}{\ButtonVolDown}
    \opt{COWON_D2_PAD}{\TouchTopLeft}
    \opt{PBELL_VIBE500_PAD}{\ButtonMenu}
       \opt{HAVEREMOTEKEYMAP}{& }
      & If a card was selected -- unselect it, else\\
       \opt{HAVEREMOTEKEYMAP}{& }
      & Draw 3 new cards from the remains stack\\
    %
    \opt{RECORDER_PAD,IPOD_4G_PAD,IPOD_3G_PAD}{\ButtonPlay}
    \opt{ONDIO_PAD}{Long \ButtonDown}
    \opt{IRIVER_H100_PAD,IRIVER_H300_PAD}{\ButtonOn{} + \ButtonLeft}
    \opt{IAUDIO_X5_PAD}{Long \ButtonPlay}
    \opt{IRIVER_H10_PAD}{\ButtonFF}
    \opt{SANSA_E200_PAD,SANSA_FUZE_PAD}{\ButtonLeft}
    \opt{SANSA_C200_PAD}{\ButtonRec}
    \opt{GIGABEAT_PAD}{\ButtonA{} + \ButtonLeft}
    \opt{GIGABEAT_S_PAD}{\ButtonSelect{} + \ButtonLeft}
    \opt{MROBE100_PAD}{\ButtonDisplay{} + \ButtonLeft}
    \opt{COWON_D2_PAD}{\TouchTopRight}
    \opt{PBELL_VIBE500_PAD}{\ButtonCancel}
       \opt{HAVEREMOTEKEYMAP}{& }
      & Put the card from the top of the remains stack on top of the cursor\\
    %
    \opt{RECORDER_PAD}{\ButtonFTwo}
    \opt{ONDIO_PAD}{Long \ButtonUp}
    \opt{IRIVER_H100_PAD,IRIVER_H300_PAD,GIGABEAT_PAD,GIGABEAT_S_PAD%
      ,MROBE100_PAD,SANSA_C200_PAD}{Long \ButtonSelect}
    \opt{IPOD_4G_PAD,IPOD_3G_PAD}{Long \ButtonMenu}
    \opt{IAUDIO_X5_PAD}{Long \ButtonSelect}
    \opt{IRIVER_H10_PAD}{\ButtonRew}
    \opt{SANSA_E200_PAD}{\ButtonRec{} + \ButtonRight}
    \opt{SANSA_FUZE_PAD}{\ButtonRight}
    \opt{COWON_D2_PAD}{\TouchBottomLeft}
    \opt{PBELL_VIBE500_PAD}{\ButtonPlay}
       \opt{HAVEREMOTEKEYMAP}{& }
      & Put the card under the cursor on one of the 4 final colour stacks.\\
    %
    \opt{RECORDER_PAD}{\ButtonFThree}
    \opt{IRIVER_H10_PAD,ONDIO_PAD,IPOD_4G_PAD,IPOD_3G_PAD}{Long \ButtonRight}
    \opt{IRIVER_H100_PAD,IRIVER_H300_PAD}{\ButtonOn{} + \ButtonRight}
    \opt{IAUDIO_X5_PAD}{\ButtonRec}
    \opt{SANSA_E200_PAD}{\ButtonRight}
    \opt{SANSA_FUZE_PAD}{Long \ButtonLeft}
    \opt{GIGABEAT_PAD}{\ButtonA{} + \ButtonRight}
    \opt{GIGABEAT_S_PAD}{\ButtonSelect{} + \ButtonRight}
    \opt{MROBE100_PAD}{\ButtonDisplay{} + \ButtonRight}
    \opt{SANSA_C200_PAD}{\ButtonVolUp}
    \opt{COWON_D2_PAD}{\TouchBottomRight}
    \opt{PBELL_VIBE500_PAD}{Long \ButtonPlay}
       \opt{HAVEREMOTEKEYMAP}{& }
      & Put the card on top of the remains stack on one of the final colour stacks.\\
    %
    \opt{RECORDER_PAD,ONDIO_PAD,IRIVER_H300_PAD,IRIVER_H100_PAD}{\ButtonOff}
    \opt{IPOD_4G_PAD,IPOD_3G_PAD}{\ButtonMenu{} + \ButtonSelect}
    \opt{IAUDIO_X5_PAD,IRIVER_H10_PAD,SANSA_E200_PAD,SANSA_C200_PAD%
        ,GIGABEAT_PAD,MROBE100_PAD,COWON_D2_PAD}{\ButtonPower}
    \opt{SANSA_FUZE_PAD}{Long \ButtonHome}
    \opt{GIGABEAT_S_PAD}{\ButtonBack}
    \opt{PBELL_VIBE500_PAD}{\ButtonRec}
       \opt{HAVEREMOTEKEYMAP}{& 
          \opt{IRIVER_RC_H100_PAD}{\ButtonRCStop}
        }
      & Show menu\\
   \end{btnmap}
  \end{table}
}

\opt{recorder,recorderv2fm,ondio}{\subsection{Star}
\screenshot{plugins/images/ss-star}{Star game}{fig:star}

This is a puzzle game.  It is actually a rewrite of Star, a game written
by CDK designed for the hp48 calculator.

Rules: Take all of the ``o''s to go to the
next level.  You can switch control between the filled circle,
which can take ``o''s, and the filled square, which is used as a mobile
wall to allow your filled circle to get to places on the screen it
could not otherwise reach. The block cannot take ``o''s.

\begin{btnmap}
    \opt{RECORDER_PAD,ONDIO_PAD,IRIVER_H100_PAD,IRIVER_H300_PAD,IAUDIO_X5_PAD%
        ,SANSA_E200_PAD,SANSA_FUZE_PAD,SANSA_C200_PAD,SANSA_CLIP_PAD,GIGABEAT_PAD%
        ,GIGABEAT_S_PAD,MROBE100_PAD,IPOD_4G_PAD,IPOD_3G_PAD,IRIVER_H10_PAD%
        ,PBELL_VIBE500_PAD,SANSA_FUZEPLUS_PAD}
        {\ButtonLeft}
    \opt{COWON_D2_PAD}{\TouchMidLeft}
    \opt{MPIO_HD300_PAD}{\ButtonRew}
      \opt{HAVEREMOTEKEYMAP}{& }
        & Move Left\\
    \opt{RECORDER_PAD,ONDIO_PAD,IRIVER_H100_PAD,IRIVER_H300_PAD,IAUDIO_X5_PAD%
        ,SANSA_E200_PAD,SANSA_FUZE_PAD,SANSA_C200_PAD,SANSA_CLIP_PAD,GIGABEAT_PAD%
        ,GIGABEAT_S_PAD,MROBE100_PAD,IPOD_4G_PAD,IPOD_3G_PAD,IRIVER_H10_PAD%
        ,PBELL_VIBE500_PAD,SANSA_FUZEPLUS_PAD}
        {\ButtonRight}
    \opt{MPIO_HD300_PAD}{\ButtonFF}
    \opt{COWON_D2_PAD}{\TouchMidRight}
      \opt{HAVEREMOTEKEYMAP}{& }
        & Move Right\\
    \opt{RECORDER_PAD,ONDIO_PAD,IRIVER_H100_PAD,IRIVER_H300_PAD,IAUDIO_X5_PAD%
        ,SANSA_E200_PAD,SANSA_FUZE_PAD,SANSA_C200_PAD,SANSA_CLIP_PAD,GIGABEAT_PAD%
        ,GIGABEAT_S_PAD,MROBE100_PAD,PBELL_VIBE500_PAD,SANSA_FUZEPLUS_PAD}
        {\ButtonUp}
    \opt{IPOD_4G_PAD,IPOD_3G_PAD}{\ButtonMenu}
    \opt{IRIVER_H10_PAD}{\ButtonScrollUp}
    \opt{COWON_D2_PAD}{\TouchTopMiddle}
    \opt{MPIO_HD300_PAD}{\ButtonScrollUp}
      \opt{HAVEREMOTEKEYMAP}{& }
        & Move Up\\
    \opt{RECORDER_PAD,ONDIO_PAD,IRIVER_H100_PAD,IRIVER_H300_PAD,IAUDIO_X5_PAD%
        ,SANSA_E200_PAD,SANSA_FUZE_PAD,SANSA_C200_PAD,SANSA_CLIP_PAD,GIGABEAT_PAD%
        ,GIGABEAT_S_PAD,MROBE100_PAD,PBELL_VIBE500_PAD,SANSA_FUZEPLUS_PAD}
        {\ButtonDown}
    \opt{IPOD_4G_PAD,IPOD_3G_PAD}{\ButtonPlay}
    \opt{IRIVER_H10_PAD}{\ButtonScrollDown}
    \opt{COWON_D2_PAD}{\TouchBottomMiddle}
    \opt{MPIO_HD300_PAD}{\ButtonScrollDown}
      \opt{HAVEREMOTEKEYMAP}{& }
        & Move Down\\
    \opt{RECORDER_PAD}{\ButtonOn}
    \opt{ONDIO_PAD}{\ButtonMenu}
    \opt{IRIVER_H100_PAD,IRIVER_H300_PAD}{\ButtonMode}
    \opt{IPOD_4G_PAD,IPOD_3G_PAD,IAUDIO_X5_PAD,SANSA_E200_PAD,SANSA_FUZE_PAD%
        ,SANSA_C200_PAD,SANSA_CLIP_PAD,GIGABEAT_PAD,GIGABEAT_S_PAD,MROBE100_PAD}{\ButtonSelect}
    \opt{IRIVER_H10_PAD}{\ButtonRew}
    \opt{COWON_D2_PAD}{\TouchCenter}
    \opt{PBELL_VIBE500_PAD,SANSA_FUZEPLUS_PAD}{\ButtonPlay}
    \opt{MPIO_HD300_PAD}{\ButtonEnter}
      \opt{HAVEREMOTEKEYMAP}{& }
        & Switch between circle and square\\
    \opt{RECORDER_PAD}{\ButtonFOne}
    \opt{ONDIO_PAD}{\ButtonMenu+\ButtonLeft}
    \opt{IRIVER_H100_PAD,IRIVER_H300_PAD}{\ButtonMode+\ButtonLeft}
    \opt{IPOD_4G_PAD,IPOD_3G_PAD,SANSA_E200_PAD,SANSA_FUZE_PAD,SANSA_C200_PAD,SANSA_CLIP_PAD}{\ButtonSelect+\ButtonLeft}
    \opt{IAUDIO_X5_PAD}{\ButtonPlay+\ButtonDown}
    \opt{IRIVER_H10_PAD}{\ButtonPlay+\ButtonScrollDown}
    \opt{GIGABEAT_PAD,GIGABEAT_S_PAD,SANSA_FUZEPLUS_PAD}{\ButtonVolDown}
    \opt{MROBE100_PAD}{\ButtonMenu}
    \opt{COWON_D2_PAD}{\TouchBottomLeft}
    \opt{PBELL_VIBE500_PAD}{\ButtonCancel}
    \opt{MPIO_HD300_PAD}{\ButtonPlay + \ButtonRew}
      \opt{HAVEREMOTEKEYMAP}{& }
        & Previous level\\
    \opt{RECORDER_PAD}{\ButtonFTwo}
    \opt{ONDIO_PAD}{\ButtonMenu+\ButtonUp}
    \opt{IRIVER_H100_PAD,IRIVER_H300_PAD}{\ButtonMode+\ButtonUp}
    \opt{IPOD_4G_PAD,IPOD_3G_PAD}{\ButtonSelect+\ButtonPlay}
    \opt{IAUDIO_X5_PAD,IRIVER_H10_PAD}{\ButtonPlay+\ButtonRight}
    \opt{SANSA_E200_PAD,SANSA_FUZE_PAD,SANSA_C200_PAD,SANSA_CLIP_PAD}{\ButtonSelect+\ButtonDown}
    \opt{GIGABEAT_PAD}{\ButtonA}
    \opt{GIGABEAT_S_PAD,PBELL_VIBE500_PAD}{\ButtonMenu}
    \opt{MROBE100_PAD}{\ButtonDisplay}
    \opt{COWON_D2_PAD}{\TouchBottomRight}
    \opt{MPIO_HD300_PAD}{Long \ButtonPlay}
    \opt{SANSA_FUZEPLUS_PAD}{Long \ButtonBack}
      \opt{HAVEREMOTEKEYMAP}{& }
        & Reset level \\
    \opt{RECORDER_PAD}{\ButtonFThree}
    \opt{ONDIO_PAD}{\ButtonMenu+\ButtonRight}
    \opt{IRIVER_H100_PAD,IRIVER_H300_PAD}{\ButtonMode+\ButtonRight}
    \opt{IPOD_4G_PAD,IPOD_3G_PAD,SANSA_E200_PAD,SANSA_FUZE_PAD,SANSA_C200_PAD,SANSA_CLIP_PAD}{\ButtonSelect+\ButtonRight}
    \opt{IAUDIO_X5_PAD}{\ButtonPlay+\ButtonRight}
    \opt{IRIVER_H10_PAD}{\ButtonPlay+\ButtonScrollUp}
    \opt{GIGABEAT_PAD,GIGABEAT_S_PAD,SANSA_FUZEPLUS_PAD}{\ButtonVolUp}
    \opt{MROBE100_PAD}{\ButtonPlay}
    \opt{COWON_D2_PAD}{\TouchTopLeft}
    \opt{PBELL_VIBE500_PAD}{\ButtonOK}
    \opt{MPIO_HD300_PAD}{\ButtonPlay + \ButtonFF}
      \opt{HAVEREMOTEKEYMAP}{& }
        & Next level \\
    \opt{RECORDER_PAD,ONDIO_PAD,IRIVER_H100_PAD,IRIVER_H300_PAD}{\ButtonOff}
    \opt{IPOD_4G_PAD,IPOD_3G_PAD}{\ButtonSelect+\ButtonMenu}
    \opt{IAUDIO_X5_PAD,IRIVER_H10_PAD,SANSA_E200_PAD,SANSA_C200_PAD,SANSA_CLIP_PAD%
        ,GIGABEAT_PAD,COWON_D2_PAD,SANSA_FUZEPLUS_PAD}{\ButtonPower}
    \opt{SANSA_FUZE_PAD}{Long \ButtonHome}
    \opt{GIGABEAT_S_PAD}{\ButtonBack}
    \opt{PBELL_VIBE500_PAD}{\ButtonRec}
    \opt{MPIO_HD300_PAD}{Long \ButtonMenu}
      \opt{HAVEREMOTEKEYMAP}{& 
          \opt{IRIVER_RC_H100_PAD}{\ButtonRCStop}
       }
        & Exit the game \\
\end{btnmap}
}

\opt{recorder,recorderv2fm}{\subsection{Wormlet}
\screenshot{plugins/images/ss-wormlet}{Wormlet game}{img:wormlet}
Wormlet is a \opt{RECORDER_PAD}{multi{}-user }multi{}-worm game on a multi{}-threaded
multi{}-functional Rockbox console. You navigate a hungry little worm.
Help your worm to find food and to avoid poisoned argh{}-tiles. The
goal is to turn your tiny worm into a big worm for as long as possible.

\opt{RECORDER_PAD}{
For 2{}-player games a remote control is not necessary but recommended.
If you try to hold the \dap\ in the four hands of two players
you'll find out why. Games with three players are only
possible using a remote control.\\}


%The following table is only for the recorder version of the game, since the
%other versions do not support either multi player or different control modes.
%It is however prepared for the other targets should they ever support these
%features. Also some other parts of the text is "opted" out for these targets.

{\bfseries
Game controls:}

\opt{RECORDER_PAD}{
\renewcommand{\arraystretch}{1.8}
\begin{rbtabular}{\textwidth}{c X p{3cm} p{3cm} p{3cm}}%
{\textbf{Players} & \textbf{Modes} & \textbf{Player 1} & \textbf{Player 2}
                        & \textbf{Player 3}}{}{}
%
0 & Out of control & \multicolumn{3}{p{9cm}}{With no player taking part in the
    game all worms are out of control and steered by artificial stupidity.}\\
%
\multirow{2}{*}{1} & 2 key control & on \dap\ \ButtonLeft: turn left
                        \ButtonRight: turn right & {}- & {}-\\
                    & 4 key control & on \dap\ \ButtonLeft: turn  left
                        \opt{RECORDER_PAD,ONDIO_PAD,IRIVER_H100_PAD,IRIVER_H300_PAD,IAUDIO_X5_PAD}
                            {\ButtonUp}\opt{IPOD_4G_PAD}{\ButtonMenu}: turn up
                            \ButtonRight: turn right
                            \opt{RECORDER_PAD,ONDIO_PAD,IRIVER_H100_PAD,IRIVER_H300_PAD,IAUDIO_X5_PAD}
                            {\ButtonDown}\opt{IPOD_4G_PAD}{\ButtonPlay}: turn down & {}- & {}- \\
%
\multirow{2}{*}{2} & Remote control & on \dap\ \ButtonLeft: turn left
                        \ButtonRight: turn right & on remote control VOL DOWN:
                        turn left VOL UP: turn right & {}- \\
                    & No remote control & on \dap\ \ButtonLeft: turn left
                        \ButtonRight: turn right & on \dap\ \ButtonFTwo: turn
                        left \ButtonFThree: turn right & {}- \\
3 & Remote control & on \dap\ \ButtonLeft: turn left \ButtonRight: turn right
                        & on remote control VOL DOWN: turn left VOL UP: turn
                        right & on \dap\ \ButtonFTwo: turn left \ButtonFThree:
                        turn right \\
\end{rbtabular}
\renewcommand{\arraystretch}{1.0}
}

\nopt{RECORDER_PAD}{
\begin{table}
    \begin{btnmap}{}{}
        \opt{ONDIO_PAD,IRIVER_H100_PAD,IRIVER_H300_PAD,IAUDIO_X5_PAD%
            ,SANSA_E200_PAD,SANSA_FUZE_PAD,SANSA_C200_PAD,GIGABEAT_PAD%
            ,GIGABEAT_S_PAD,MROBE100_PAD,IPOD_3G_PAD,IPOD_4G_PAD,IRIVER_H10_PAD}
        {\ButtonLeft}
        \opt{COWON_D2_PAD}{\TouchMidLeft}
       \opt{HAVEREMOTEKEYMAP}{& }
        & Turn left\\
        \opt{ONDIO_PAD,IRIVER_H100_PAD,IRIVER_H300_PAD,IAUDIO_X5_PAD%
            ,SANSA_E200_PAD,SANSA_FUZE_PAD,SANSA_C200_PAD,GIGABEAT_PAD%
            ,GIGABEAT_S_PAD,MROBE100_PAD,IPOD_3G_PAD,IPOD_4G_PAD,IRIVER_H10_PAD}
        {\ButtonRight}
        \opt{COWON_D2_PAD}{\TouchMidRight}
             \opt{HAVEREMOTEKEYMAP}{& }
        & Turn right\\
        \opt{ONDIO_PAD,IRIVER_H100_PAD,IRIVER_H300_PAD,IAUDIO_X5_PAD%
            ,SANSA_E200_PAD,SANSA_FUZE_PAD,SANSA_C200_PAD,GIGABEAT_PAD%
            ,GIGABEAT_S_PAD,MROBE100_PAD}{\ButtonUp}
        \opt{IPOD_3G_PAD,IPOD_4G_PAD}{\ButtonMenu}
        \opt{IRIVER_H10_PAD}{\ButtonScrollUp}
        \opt{COWON_D2_PAD}{\TouchTopMiddle}
            \opt{HAVEREMOTEKEYMAP}{& }
        & Turn Up\\
        \opt{ONDIO_PAD,IRIVER_H100_PAD,IRIVER_H300_PAD,IAUDIO_X5_PAD%
            ,SANSA_E200_PAD,SANSA_FUZE_PAD,SANSA_C200_PAD,GIGABEAT_PAD%
            ,GIGABEAT_S_PAD,MROBE100_PAD}{\ButtonDown}
        \opt{IPOD_3G_PAD,IPOD_4G_PAD}{\ButtonPlay}
        \opt{IRIVER_H10_PAD}{\ButtonScrollDown}
        \opt{COWON_D2_PAD}{\TouchBottomMiddle}
           \opt{HAVEREMOTEKEYMAP}{& }
        & Turn Down\\
    \end{btnmap}
\end{table}
}

\subsubsection{The game}
Use the control keys of your worm to navigate around obstacles and find
food. Worms do not stop moving except when dead. Dead worms are no fun.
Be careful as your worm will try to eat anything that you steer it
across. It won't distinguish whether it is edible or not.

\begin{description}
\item[Food.]
The small square hollow pieces are food. Move the worm over a food tile
to eat it. After eating the worm grows. Each time a piece of food has
been eaten a new piece of food will pop up somewhere. Unfortunately for
each new piece of food that appears two new ``argh'' pieces will
appear, too.
\item[Argh.]
An ``argh'' is a black square poisoned piece {}- slightly bigger than
food {}- that makes a worm say ``Argh!'' when
run into.  A worm that eats an ``argh'' is dead. Thus eating an
``argh'' must be avoided under any circumstances. ``Arghs'' have the
annoying tendency to accumulate. 
\item[Worms.]
Thou shall not eat worms. Neither other worms nor thyself. Eating worms
is blasphemous cannibalism, not healthy and causes instant
death. And it doesn't help anyway: the other worm
isn't hurt by the bite. It will go on creeping happily
and eat all the food you left on the table. 
\item[Walls.]
Don't crash into the walls. Walls are not edible.
Crashing a worm against a wall causes it a headache it
doesn't survive. 
\item[Game over.]
The game is over when all worms are dead. The longest worm wins the
game. 
\item [Pause the game.]
Press
\opt{RECORDER_PAD,IAUDIO_X5_PAD}{\ButtonPlay}%
\opt{ONDIO_PAD}{\ButtonMenu}%
\opt{IRIVER_H100_PAD,IRIVER_H300_PAD,IPOD_4G_PAD,SANSA_E200_PAD,SANSA_FUZE_PAD%
  ,GIGABEAT_PAD,GIGABEAT_S_PAD}{\ButtonSelect}
\opt{COWON_D2_PAD}{\TouchCenter}
to pause the game. Press it again to resume the game.

\item[Stop the game.]
There are two ways to stop a running game.

\begin{itemize}
\item If you want to quit Wormlet entirely simply hit
\opt{RECORDER_PAD,ONDIO_PAD,IRIVER_H100_PAD,IRIVER_H300_PAD}{\ButtonOff}%
\opt{IPOD_4G_PAD}{\ButtonMenu+\ButtonSelect}%
\opt{IAUDIO_X5_PAD,SANSA_E200_PAD,GIGABEAT_PAD}{\ButtonPower}%
\opt{SANSA_FUZE_PAD}{Long \ButtonHome}
\opt{GIGABEAT_S_PAD}{\ButtonBack}.
The game will stop immediately and you will return to the game menu. 
\item If you want to stop the game and still see the screen hit 
\opt{RECORDER_PAD,IRIVER_H100_PAD,IRIVER_H300_PAD}{\ButtonOn}%
\opt{ONDIO_PAD}{\ButtonOff+\ButtonMenu}%
\opt{IPOD_4G_PAD}{\ButtonSelect+\ButtonPlay}%
\opt{IAUDIO_X5_PAD,SANSA_E200_PAD}{\ButtonRec}%
\opt{SANSA_FUZE_PAD}{\ButtonSelect+\ButtonUp}%
\opt{GIGABEAT_PAD}{\ButtonA}%
\opt{GIGABEAT_S_PAD}{\ButtonMenu}.
This freezes the game. If you hit
\opt{RECORDER_PAD,IRIVER_H100_PAD,IRIVER_H300_PAD}{\ButtonOn}%
\opt{ONDIO_PAD}{\ButtonOff+\ButtonMenu}%
\opt{IPOD_4G_PAD}{\ButtonSelect+\ButtonPlay}%
\opt{IAUDIO_X5_PAD,SANSA_E200_PAD}{\ButtonRec}%
\opt{SANSA_FUZE_PAD}{\ButtonSelect+\ButtonUp}%
\opt{GIGABEAT_PAD}{\ButtonA}%
\opt{GIGABEAT_S_PAD}{\ButtonMenu}
button again a new game starts with the same configuration. To return to the
games menu you can hit
\opt{RECORDER_PAD,ONDIO_PAD,IRIVER_H100_PAD,IRIVER_H300_PAD}{\ButtonOff}%
\opt{IPOD_4G_PAD}{\ButtonMenu+\ButtonSelect}%
\opt{IAUDIO_X5_PAD,SANSA_E200_PAD,GIGABEAT_PAD}{\ButtonPower}%
\opt{SANSA_FUZE_PAD}{Long \ButtonHome}
\opt{GIGABEAT_S_PAD}{\ButtonBack}. A stopped game can not be resumed. 
\end{itemize}
\end{description}

\subsubsection{The scoreboard}
On the right side of the game field is the score board. For each worm it
displays its status and its length. The top most entry displays the
state of worm 1, the second worm 2 and the third worm 3. When a worm
dies its entry on the score board turns black.

\begin{description}
\item[Len:]
Here the current length of the worm is displayed. When a worm is eating
food it grows by one pixel for each step it moves. 

\item[Hungry:]
That's the normal state of a worm. Worms are always
hungry and want to eat. It is good to have a hungry
worm since it means that your worm is alive. But it is
better to get your worm growing. 

\item[Growing:]
When a worm has eaten a piece of food it starts growing. For each step
it moves over food it can grow by one pixel. One piece of food lasts
for 7 steps. After your worm has moved 7 steps the food is used up. If
another piece of food is eaten while growing it will increase the size
of the worm for another 7 steps. 

\item[Crashed:]
This indicates that a worm has crashed against a wall.

\item[Argh:]
If the score board entry displays ``Argh!'' it
means the worm is dead because it tried to eat an ``argh''. Until we
can make the worm say ``Argh!'' it is your job to say ``Argh!'' aloud.

\item[Wormed:]
The worm tried to eat another worm or even itself.
That's why it is dead now.  Making traps for other players with a worm
is a good way to get them out of the game.
\end{description}

\subsubsection{Hints}

\begin{itemize}

\item Initially you will be busy with controlling your worm. Try to
avoid other worms and crawl far away from them. Wait until they curl up
themselves and collect the food afterwards. Don't worry if the other
worms grow longer than yours {}- you can catch up after they've died. 

\item When you are more experienced watch the tactics of other worms.
Those worms controlled by artificial stupidity head straight for the
nearest piece of food. Let the other worm have its next piece of food
and head for the food it would probably want next. Try to put yourself
between the opponent and that food. From now on you can 'control' the
other worm by blocking it. You could trap it by making a 1 pixel wide
U{}-turn. You also could move from food to food and make sure you keep
between your opponent and the food. So you can always reach it before
your opponent. 

\opt{RECORDER_PAD}{
\item While playing the game the \dap\ can still play music. For
single player game use any music you like. For berserk games with 2 players use
hard rock and for 3 player games use heavy metal or X{}-Phobie
(\url{http://www.x-phobie.de/}).
Play fair and don't kick your opponent in the toe or
poke him in the eye. That would be bad manners.
}
\end{itemize}
}

\section{Demos}

\opt{recorder,recorderv2fm,ondio,h1xx,h300}{\subsection{Bounce}
\screenshot{plugins/images/ss-bounce}{Bounce}{img:bounce}
This demo is of the word ``Rockbox'' bouncing across the screen.
\opt{rtc}{There is also an analogue clock in the background.}
In \setting{Scroll mode} the bouncing text is replaced by a different one
scrolling from right to left.

\begin{btnmap}
\opt{RECORDER_PAD,ONDIO_PAD,IRIVER_H100_PAD,IRIVER_H300_PAD,IAUDIO_X5_PAD%
    ,SANSA_C200_PAD,GIGABEAT_PAD,GIGABEAT_S_PAD,MROBE100_PAD,PBELL_VIBE500_PAD}
    {\ButtonUp\ /\ \ButtonDown}
\opt{IPOD_4G_PAD,IPOD_3G_PAD,SANSA_E200_PAD,SANSA_FUZE_PAD}%
    {\ButtonScrollBack\ /\ \ButtonScrollFwd}
\opt{IRIVER_H10_PAD,MPIO_HD300_PAD}{\ButtonScrollDown\ /\ \ButtonScrollUp}
\opt{COWON_D2_PAD}{\TouchBottomMiddle{} / \TouchTopMiddle}
\opt{MPIO_HD200_PAD}{\ButtonRew / \ButtonFF}
       \opt{HAVEREMOTEKEYMAP}{& }
& Moves to next/previous option\\
\opt{RECORDER_PAD,ONDIO_PAD,IRIVER_H100_PAD,IRIVER_H300_PAD,IAUDIO_X5_PAD%
    ,SANSA_C200_PAD,GIGABEAT_PAD,GIGABEAT_S_PAD,MROBE100_PAD,IPOD_4G_PAD%
    ,IPOD_3G_PAD,SANSA_E200_PAD,IRIVER_H10_PAD,SANSA_FUZE_PAD,PBELL_VIBE500_PAD}
    {\ButtonRight{} / \ButtonLeft}
\opt{COWON_D2_PAD}{\TouchMidRight{} / \TouchMidLeft}
\opt{MPIO_HD200_PAD}{\ButtonVolDown / \ButtonVolUp}
\opt{MPIO_HD300_PAD}{\ButtonRew / \ButtonFF}
       \opt{HAVEREMOTEKEYMAP}{& }
& Increases/decreases option value\\
\opt{RECORDER_PAD}{\ButtonOn}
\opt{ONDIO_PAD}{\ButtonMenu}
\opt{IRIVER_H100_PAD,IRIVER_H300_PAD,IPOD_4G_PAD,IPOD_3G_PAD,SANSA_E200_PAD%
    ,SANSA_C200_PAD,SANSA_FUZE_PAD}
    {\ButtonSelect}
\opt{IAUDIO_X5_PAD,IRIVER_H10_PAD,MPIO_HD300_PAD}{\ButtonPlay}
\opt{GIGABEAT_PAD}{\ButtonA}
\opt{GIGABEAT_S_PAD,COWON_D2_PAD,PBELL_VIBE500_PAD}{\ButtonMenu}
\opt{MROBE100_PAD}{\ButtonDisplay}
\opt{MPIO_HD200_PAD}{\ButtonFunc}
       \opt{HAVEREMOTEKEYMAP}{& }
& Toggles Scroll mode\\
\opt{RECORDER_PAD,ONDIO_PAD,IRIVER_H100_PAD,IRIVER_H300_PAD}{\ButtonOff}
\opt{IPOD_4G_PAD,IPOD_3G_PAD}{\ButtonMenu}
\opt{IAUDIO_X5_PAD,IRIVER_H10_PAD,SANSA_E200_PAD,SANSA_C200_PAD,GIGABEAT_PAD%
    ,COWON_D2_PAD}{\ButtonPower}
\opt{SANSA_FUZE_PAD}{Long \ButtonHome}
\opt{GIGABEAT_S_PAD}{\ButtonBack}
\opt{PBELL_VIBE500_PAD}{\ButtonRec}
\opt{MPIO_HD200_PAD}{\ButtonRec + \ButtonPlay}
\opt{MPIO_HD300_PAD}{Long \ButtonMenu}
       \opt{HAVEREMOTEKEYMAP}{& }
& Exits bounce demo\\
\end{btnmap}

Available options are:

\begin{description}
\item[Xdist/Ydist.] The distance to X axis and Y axis
respectively
\item[Xadd/Yadd.]How fast the code moves on the sine curve on
each axis
\item[Xsane/Ysane.] Changes the appearance of the bouncing.
\end{description}
}

\opt{recorder,recorderv2fm,ondio,h1xx,h300}{\subsection{Cube}
\screenshot{plugins/images/ss-cube}{Cube}{img:cube}
This is a rotating cube screen saver in 3D.
\begin{table}
    \begin{btnmap}{}{}
    \opt{PLAYER_PAD,RECORDER_PAD}{\ButtonOn}
    \opt{ONDIO_PAD}{\ButtonMenu+\ButtonRight}
    \opt{IAUDIO_X5_PAD,SANSA_E200_PAD,SANSA_FUZE_PAD,SANSA_C200_PAD%
        ,IRIVER_H100_PAD,IRIVER_H300_PAD,GIGABEAT_S_PAD}{\ButtonSelect}
    \opt{IPOD_4G_PAD,IPOD_3G_PAD}{\ButtonSelect+\ButtonPlay}
    \opt{IRIVER_H10_PAD}{\ButtonFF}
    \opt{GIGABEAT_PAD}{\ButtonA}
    \opt{MROBE100_PAD}{\ButtonDisplay}
    \opt{COWON_D2_PAD}{\TouchBottomRight}
    \opt{PBELL_VIBE500_PAD}{\ButtonOK}
       \opt{HAVEREMOTEKEYMAP}{& }
        & Display at maximum frame rate\\
    \opt{PLAYER_PAD,RECORDER_PAD,IPOD_4G_PAD,IPOD_3G_PAD,IAUDIO_X5_PAD%
        ,IRIVER_H10_PAD,GIGABEAT_S_PAD,PBELL_VIBE500_PAD}{\ButtonPlay}
    \opt{ONDIO_PAD}{\ButtonMenu+\ButtonLeft}
    \opt{IRIVER_H100_PAD,IRIVER_H300_PAD}{\ButtonOn}
    \opt{GIGABEAT_PAD}{\ButtonSelect}
    \opt{SANSA_E200_PAD,SANSA_FUZE_PAD,SANSA_C200_PAD,MROBE100_PAD}{\ButtonUp}
    \opt{COWON_D2_PAD}{\TouchCenter}
       \opt{HAVEREMOTEKEYMAP}{& }
        & Pause\\
    \opt{PLAYER_PAD,ONDIO_PAD,GIGABEAT_PAD,GIGABEAT_S_PAD,MROBE100_PAD,PBELL_VIBE500_PAD}{\ButtonMenu}
    \opt{RECORDER_PAD}{\ButtonFThree}
    \opt{IRIVER_H100_PAD,IRIVER_H300_PAD}{\ButtonMode}
    \opt{IPOD_4G_PAD,IPOD_3G_PAD}{\ButtonSelect+\ButtonMenu}
    \opt{IAUDIO_X5_PAD}{\ButtonSelect}
    \opt{IRIVER_H10_PAD}{\ButtonRew}
    \opt{SANSA_E200_PAD,SANSA_FUZE_PAD,SANSA_C200_PAD}{\ButtonDown}
    \opt{COWON_D2_PAD}{\TouchTopRight}
       \opt{HAVEREMOTEKEYMAP}{& }
        & Cycle draw mode\\
    \opt{ONDIO_PAD,GIGABEAT_PAD,GIGABEAT_S_PAD,MROBE100_PAD,RECORDER_PAD%
        ,IRIVER_H100_PAD,IRIVER_H300_PAD,IPOD_4G_PAD,IPOD_3G_PAD,IAUDIO_X5_PAD%
        ,IRIVER_H10_PAD,SANSA_E200_PAD,SANSA_FUZE_PAD,SANSA_C200_PAD}
        {\ButtonRight{} / \ButtonLeft}
    \opt{PLAYER_PAD}{\ButtonOn+\ButtonRight{} / \ButtonOn+\ButtonLeft}
    \opt{COWON_D2_PAD}{\TouchMidRight{} / \TouchMidLeft}
    \opt{PBELL_VIBE500_PAD}{\ButtonPrev{} / \ButtonNext}
       \opt{HAVEREMOTEKEYMAP}{& }
        & Select axis to adjust\\
    \opt{PLAYER_PAD}{\ButtonRight{} / \ButtonLeft}
    \opt{RECORDER_PAD,ONDIO_PAD,IRIVER_H100_PAD,IRIVER_H300_PAD,IAUDIO_X5_PAD%
        ,GIGABEAT_PAD,GIGABEAT_S_PAD,MROBE100_PAD,PBELL_VIBE500_PAD}{\ButtonUp{} / \ButtonDown}
    \opt{IPOD_4G_PAD,IPOD_3G_PAD,SANSA_E200_PAD,SANSA_FUZE_PAD}{\ButtonScrollFwd{} / \ButtonScrollBack}
    \opt{IRIVER_H10_PAD}{\ButtonScrollUp{} / \ButtonScrollDown}
    \opt{SANSA_C200_PAD}{\ButtonVolDown{} / \ButtonVolUp}
    \opt{COWON_D2_PAD}{\TouchTopMiddle{} / \TouchBottomMiddle}
       \opt{HAVEREMOTEKEYMAP}{& }
        & Change speed/angle (speed can not be changed while paused)\\
    \opt{PLAYER_PAD}{\ButtonStop}
    \opt{RECORDER_PAD,ONDIO_PAD,IRIVER_H100_PAD,IRIVER_H300_PAD}{\ButtonOff}
    \opt{IPOD_4G_PAD,IPOD_3G_PAD}{\ButtonMenu}
    \opt{IAUDIO_X5_PAD,IRIVER_H10_PAD,SANSA_E200_PAD,SANSA_C200_PAD,GIGABEAT_PAD%
        ,MROBE100_PAD,COWON_D2_PAD}{\ButtonPower}
    \opt{SANSA_FUZE_PAD}{Long \ButtonHome}
    \opt{GIGABEAT_S_PAD}{\ButtonBack}
    \opt{PBELL_VIBE500_PAD}{\ButtonRec}
       \opt{HAVEREMOTEKEYMAP}{& 
          \opt{IRIVER_RC_H100_PAD}{\ButtonRCStop}
       }
        & Quit\\
    \end{btnmap}
\end{table}
}

\opt{recorder,recorderv2fm,ondio,h1xx,h300}{\subsection{Grayscale}
{\centering\itshape
  [Warning: Image ignored] % Unhandled or unsupported graphics:
%\includegraphics[width=4.359cm,height=2.492cm]{images/rockbox-manual-img50.png}
 \newline
Grayscale
\par}

This is a demonstration of the Rockbox grayscale engine which supports grayscalegraphics on the Jukebox.  Press OFF to quit the demo.


}

{\subsection{Hello World}
{\centering\itshape
  [Warning: Image ignored] % Unhandled or unsupported graphics:
%\includegraphics[width=4.688cm,height=1.849cm]{images/rockbox-manual-img51.png}
 \newline
Hello world!
\par}

This is a plugin demo for hackers. Every programmer's
first program is the hello world{}-program
which does nothing except displaying ``Hello
world!'' on the screen.


}

\opt{recorder,recorderv2fm,ondio,h1xx,h300}{\subsection{Mandelbrot}
\screenshot{plugins/images/ss-mandelbrot}{Mandelbrot}{img:mandelbrot}
This demonstration draws fractal images from the Mandelbrot set
\opt{archos,iriverh100}{using the greyscale engine}.
\begin{table}
  \begin{btnmap}{}{}
    Direction keys
    \opt{HAVEREMOTEKEYMAP}{&}
    & Move about the image\\
    %
    \opt{RECORDER_PAD,IRIVER_H10_PAD}{\ButtonPlay}
    \opt{ONDIO_PAD}{\ButtonMenu\ / \ButtonMenu+\ButtonUp}
    \opt{IRIVER_H100_PAD,IRIVER_H300_PAD,IAUDIO_X5_PAD,GIGABEAT_PAD,MROBE100_PAD}{\ButtonSelect}
    \opt{IPOD_4G_PAD,IPOD_3G_PAD,SANSA_E200_PAD,SANSA_FUZE_PAD}{\ButtonScrollFwd}
    \opt{SANSA_C200_PAD,GIGABEAT_S_PAD}{\ButtonVolUp}
       \opt{HAVEREMOTEKEYMAP}{& }
    & Zoom in\\
    %
    \opt{RECORDER_PAD}{\ButtonOn}
    \opt{ONDIO_PAD}{\ButtonMenu+\ButtonDown}
    \opt{IRIVER_H100_PAD,IRIVER_H300_PAD}{\ButtonMode}
    \opt{IPOD_4G_PAD,IPOD_3G_PAD,SANSA_E200_PAD,SANSA_FUZE_PAD}{\ButtonScrollBack}
    \opt{IAUDIO_X5_PAD,GIGABEAT_PAD,MROBE100_PAD}{Long \ButtonSelect}
    \opt{IRIVER_H10_PAD}{Long \ButtonPlay}
    \opt{SANSA_C200_PAD,GIGABEAT_S_PAD}{\ButtonVolDown}
       \opt{HAVEREMOTEKEYMAP}{& }
    & Zoom out\\
    %
    \opt{RECORDER_PAD}{\ButtonFOne}
    \opt{ONDIO_PAD}{\ButtonMenu+\ButtonLeft}
    \opt{IRIVER_H100_PAD,IRIVER_H300_PAD}{\ButtonOn+\ButtonLeft}
    \opt{IPOD_4G_PAD,IPOD_3G_PAD,SANSA_E200_PAD,SANSA_FUZE_PAD,SANSA_C200_PAD}{\ButtonSelect+\ButtonLeft}
    \opt{IAUDIO_X5_PAD}{Long \ButtonPlay}
    \opt{IRIVER_H10_PAD}{\ButtonRew}
    \opt{GIGABEAT_PAD}{\ButtonVolDown}
    \opt{GIGABEAT_S_PAD}{\ButtonNext}
    \opt{MROBE100_PAD}{\ButtonPlay}
       \opt{HAVEREMOTEKEYMAP}{& }
    & Decrease iteration depth (less detail)\\
    %
    \opt{RECORDER_PAD}{\ButtonFTwo}
    \opt{ONDIO_PAD}{\ButtonMenu+\ButtonRight}
    \opt{IRIVER_H100_PAD,IRIVER_H300_PAD}{\ButtonOn+\ButtonRight}
    \opt{IPOD_4G_PAD,IPOD_3G_PAD,SANSA_E200_PAD,SANSA_FUZE_PAD,SANSA_C200_PAD}{\ButtonSelect+\ButtonRight}
    \opt{IAUDIO_X5_PAD}{\ButtonPlay}
    \opt{IRIVER_H10_PAD}{\ButtonFF}
    \opt{GIGABEAT_PAD}{\ButtonVolUp}
    \opt{GIGABEAT_S_PAD}{\ButtonPrev}
    \opt{MROBE100_PAD}{\ButtonMenu}
       \opt{HAVEREMOTEKEYMAP}{& }
    & Increase iteration depth (more detail)\\
    %
    \opt{RECORDER_PAD}{\ButtonFThree}
    \opt{ONDIO_PAD}{\ButtonMenu+\ButtonOff}
    \opt{IRIVER_H100_PAD,IRIVER_H300_PAD,IAUDIO_X5_PAD,SANSA_E200_PAD,SANSA_C200_PAD}{\ButtonRec}
    \opt{SANSA_FUZE_PAD}{Long \ButtonSelect}
    \opt{IPOD_4G_PAD,IPOD_3G_PAD}{\ButtonSelect+\ButtonPlay}
    \opt{IRIVER_H10_PAD}{\ButtonPlay + \ButtonRew}
    \opt{GIGABEAT_PAD}{\ButtonA}
    \opt{GIGABEAT_S_PAD}{\ButtonMenu}
    \opt{MROBE100_PAD}{\ButtonDisplay}
       \opt{HAVEREMOTEKEYMAP}{& }
    & Reset and return to the default image\\
    %
    \opt{RECORDER_PAD,ONDIO_PAD,IRIVER_H100_PAD,IRIVER_H300_PAD}{\ButtonOff}
    \opt{IPOD_4G_PAD,IPOD_3G_PAD}{\ButtonSelect+\ButtonMenu}
    \opt{IAUDIO_X5_PAD,IRIVER_H10_PAD,SANSA_E200_PAD,SANSA_C200_PAD,GIGABEAT_PAD,MROBE100_PAD}{\ButtonPower}
    \opt{SANSA_FUZE_PAD}{Long \ButtonHome}
    \opt{GIGABEAT_S_PAD}{\ButtonBack}
       \opt{HAVEREMOTEKEYMAP}{& 
          \opt{IRIVER_RC_H100_PAD}{\ButtonRCStop}
       }
    & Quit\\
  \end{btnmap}
\end{table}
}

\opt{recorder,recorderv2fm,ondio,h1xx,h300}{% $Id$ %
%
% NOTE:
% This plugin is called "mosaique" but the tex file (and screenshot)
% is still named "mosaic". This should probably get changed sometime.
%
\subsection{Mosaique}
\screenshot{plugins/images/ss-mosaic}{Mosaique}{img:mosaique}
This simple graphics demo draws a mosaic picture on the screen of the \dap.
Press
\opt{PLAYER_PAD}{\ButtonStop}
\opt{RECORDER_PAD,ONDIO_PAD,IRIVER_H100_PAD,IRIVER_H300_PAD}{\ButtonOff}
\opt{IPOD_4G_PAD,IPOD_3G_PAD}{\ButtonMenu}
\opt{IAUDIO_X5_PAD,IRIVER_H10_PAD,SANSA_E200_PAD,SANSA_C200_PAD,GIGABEAT_PAD,MROBE100_PAD}{\ButtonPower}
to quit.
}

\opt{recorder,recorderv2fm,ondio}{\subsection{Oscillograph}
{\centering\itshape
  [Warning: Image ignored] % Unhandled or unsupported graphics:
%\includegraphics[width=3.6cm,height=2.057cm]{images/rockbox-manual-img54.png}
 \newline
Oscillograph
\par}

This demo shows the shape of the sound samples that make up the music
being played.

At faster speed rates, the Jukebox is less responsive to user input.

\subsubsection{Key controls:}

\begin{table}[h!]
\begin{center}
\begin{tabular}{|c}|c|}
\hline
KEY & ACTION \\\hline
F1 & toggles whether to scroll or not \\\hline
F2 & toggles filled / curve / plot \\\hline
F3 & reset speed to 0 \\\hline
UP & slow down scrolling \\\hline
DOWN & Speeds up scrolling \\\hline
PLAY & Pauses the demo \\\hline
OFF & Exits demo \\\hline
\end{tabular}
\end{center}
\end{table}

}

\opt{recorder,recorderv2fm,ondio,h1xx,h300}{\subsection{Snow}
\begin{figure}[h!]
\begin{center}
\includegraphics[width=4cm]{plugins/images/ss-snow-\genericimg.png}
\end{center}
\caption{Have you ever seen snow falling?}
\end{figure}
This demo replicates snow falling on your screen. If you love winter,
you will love this demo.  Or maybe not.
}

\opt{recorder,recorderv2fm,ondio}{\subsection{VU meter}
\begin{figure}[h!]
\begin{center}
\includegraphics[width=4cm]{plugins/images/ss-vumeter-\genericimg.png}
\end{center}
\end{figure}

This is a VU meter, which displays the volume of the left and right
audio channels. There are 3 types of meter selectable.  The analogue
meter is a classic needle style.  The digital meter is modelled after
LED volume displays, and the mini{}-meter option allows for the display
of small meters in addition to the main display (as above).  From the
settings menu the decay time for the meter (its memory), the meter type
and the meter scale can be changed. 

\begin{table}[h!]
\begin{tabular}{|c|c|c|}
\hline
RECORDER & ONDIO & FUNCTION \\\hline
OFF & ONOFF & Save settings and quit \\\hline
ON & MODE & Help \\\hline
F1 & HOLD MODE & Settings \\\hline
UP & UP & Raise Volume \\\hline
DOWN & DOWN & Lower Volume \\\hline
\end{tabular}
\end{table}

}

\section{\label{ref:Viewersplugins}Viewers}

Viewers are plugins which are associated with specific file extensions.
They cannot be run directly but are started by ``playing''
the associated file.  Viewers are stored in the
\textbf{/.rockbox/viewers/ }directory.

\subsection{\label{ref:Chip8emulator}Chip{}-8 Emulator (Recorder, Ondio)}
The Chip{}-8 Emulator allows you to play many old chip8 games found on
the Net. It modifies Rockbox, so file extensions .ch8 will be
recognised as chip8 games. Just press PLAY on a .ch8 file to start a
game.

There are lots of tiny Chip8 games (usually only about 256 bytes to a
couple of KB) which were made popular by the HP48
calculator's emulator for them. The original Chip8 had
64x32 pixel graphics, and the new superchip emulator supports 128x64
graphics, which almost fits on the Recorder's display.
The only problem is they are based on a 4x4 keyboard, but since most
games do not use all of the buttons, this can easily be worked around.

Some places where can you can find .ch8 files:

\begin{itemize}
\item The original chip8 patch had several attached:
\url{http://sourceforge.net/tracker/index.php?func=detail&aid=628509&group_id=44306&atid=439120}
\item Check out the HP48 chip games section:
\url{http://www.hpcalc.org/hp48/games/chip/}
\item Check out the PC emulator by the guy who wrote the HP48 emulator:
\url{http://www.pdc.kth.se/~lfo/chip8/CHIP8.htm}
\item Links to other chip8 emulators: 
\url{http://www.zophar.net/chip8.html}
\end{itemize}

\subsection{JPEG viewer (Recorder, Ondio)}
Press PLAY on a .jpg file in order to view the contents using Rockbox's greyscale library.  Use the arrow keys to move around the image, PLAY to zoom in
and ON to zoom out.  Press OFF to exit the viewer.

Note: JPEGs that use progressive scan encoding are not supported and will produce an error.

\subsection{Movie Player (Recorder, Ondio)}
Play movies on your Jukebox!  In order to do
this, movies must be in AVI format, and then converted to .RVF,
Rockbox's own video format.  For more details on how to use this plugin, please see \url{http://www.rockbox.org/twiki/bin/view/Main/VideoTutorial}.

\subsection{Rockbox\_flash (Recorder, Ondio)}
{\centering\itshape
  [Warning: Image ignored] % Unhandled or unsupported graphics:
%\includegraphics[width=4.059cm,height=2.32cm]{images/rockbox-manual-img57.png}
 \newline
Rockbox flash
\par}

For ``playing'' .UCL files on flashed Jukeboxes. Reprograms the flash memory of
the Jukebox unit (see page \pageref{ref:Rockboxinflash} for details).

\subsection{Search}
This plugin can be used on playlists.  It searches through the playlist
that it is opened on looking for any occurrences of the string entered by the
user.  The results of this search are saved to a new playlist,
\textbf{search\_results.m3u}, within the same directory as the
original playlist.

\subsection{Sort}
This plugin takes a file and sorts it in  forward alphabetical order.  Case is
ignored.  This is useful for ordering playlists generated by the ``Create Playlist'' menu option (see page \pageref{ref:Playlistsubmenu}).

\subsection{Text Viewer}
This is a Viewer for text files with word wrap. Just press PLAY on a
.txt file to display it. Has controls to handle various styles of text
formatting. Has top{}-of{}-file and bottom{}-of{}-file buttons.  You
can view files without a .txt extension by using \textbf{Open with ..}
from the Play Screen menu

\subsubsection{Controls}

\begin{itemize}
\item \textbf{F1 (Recorder) / ON{}-MINUS (Player): }
toggles Word mode between Wrap and Chop:

\begin{itemize}
\item Wrap breaks lines at white space or hyphen.
\item Chop breaks lines at the maximum column limit.
\end{itemize}

\item \textbf{F2 (Recorder) / ON{}-MENU{}-PLUS (Player): }
cycles Line mode through Normal, Join and Expand:

\begin{itemize}
\item Normal breaks lines at newline characters.
\item Join ignores unpaired newline characters  (i.e., joins lines). Useful for
adopting the orphans that occur with e{}-mail style (i.e.,pre{}-wrapped) text files.
\item Expand doubles unpaired newlines (i.e., adds a blank line). Useful
for making the paragraphs clearer in some book style text files.
\end{itemize}

\item \textbf{F3 (Recorder) / ON{}-PLUS (Player):} 
toggles View mode between Narrow and Wide:

\begin{itemize}
\item Narrow sets maximum column to the screen width.
\item Wide sets maximum column to 114. Useful for navigating large
files. (Currently, Wide and Join cannot be selected together.)
\end{itemize}

\item \textbf{ON{}-F1 (Recorder):} 
toggles Page mode between Normal and Overlap:

\begin{itemize}
\item Normal sets page{}-down/page{}-up to one full screen.
\item Overlap tells page{}-down/page{}-up to retain one line from
previous screen.
\end{itemize}

\item \textbf{ON{}-F3 (Recorder):} 
toggles Scrollbar mode, for the current View mode.

\begin{itemize}
\item Narrow mode has no scrollbar by default, until toggled.
\item Wide mode has a scrollbar by default, until toggled.
\item If file fits on one screen, there is no scrollbar and ON{}-F3 has
no effect.
\end{itemize}
\end{itemize}

Settings are not remembered after the viewer has been exited. 
Keys are as follows:

\subsubsection{Recorder}

\begin{tabular}[c]{|p{3.317cm}|p{7.104cm}|}
\hline
{\centering\bfseries\itshape
KEY
\par}
&
{\centering\bfseries\itshape
ACTION
\par}
\\\hline
{\centering
UP
\par}
&
Page{}-up (one screen up)
\\\hline
{\centering
DOWN
\par}
&
Page{}-down (one screen down)
\\\hline
{\centering
LEFT
\par}
&
Top of file (Narrow mode)\newline
One screen left (Wide mode)
\\\hline
{\centering
RIGHT
\par}
&
Bottom of file (Narrow mode)\newline
One screen right (Wide mode)
\\\hline
{\centering
ON{}-UP
\par}
&
One line up
\\\hline
{\centering
ON{}-DOWN
\par}
&
One line down
\\\hline
{\centering
ON{}-LEFT 
\par}
&
One column left
\\\hline
{\centering
ON{}-RIGHT
\par}
&
One column right
\\\hline
{\centering
OFF
\par}
&
Exit text viewer
\\\hline
\end{tabular}

\subsubsection{Player}

\begin{tabular}[c]{|p{3.291cm}|p{7.131cm}|}
\hline
{\centering\bfseries\itshape
KEY
\par}
&
{\centering\bfseries\itshape
ACTION
\par}
\\\hline
{\centering
MINUS
\par}
&
Page{}-up (one screen up)
\\\hline
{\centering
PLUS
\par}
&
Page{}-down (one screen down)
\\\hline
{\centering
MENU MINUS
\par}
&
Top of file (Narrow mode)\newline
One screen left (Wide mode)
\\\hline
{\centering
MENU PLUS
\par}
&
Bottom of file (Narrow mode)\newline
One screen right (Wide mode)
\\\hline
{\centering
STOP
\par}
&
Exit text viewer
\\\hline
\end{tabular}

\subsubsection{Compatibility}

\begin{itemize}
\item Correctly reads plain text files in Unix, Win/DOS, or Macintosh
format. Latin{}-alphabet Unicode files are  a l m o s t  r e a d a b l
e.
\item Currently prefers fixed{}-width fonts. With proportional fonts,
pretends all characters are the width of a lower{}-case 'o'.
\item Currently messages are in English 
\item Does not currently support right{}-to{}-left languages.
\end{itemize}

\subsection{VBRfix}
This function scans a VBR (Variable Bitrate)
MP3 file and updates/creates the Xing VBR header. The Xing header
contains information about the VBR stream used to calculate average bit
rate, time information and to more accurately fwd/rew in the stream.

This function is especially useful when the playback of a file skips,
fwd/rew does not work correctly or the time display is incorrect. Run
VBRfix on files you record with your Jukebox. The header is not present
in the recorded files and VBRfix adds this header.

Note: VBRfix can only run when music is
turned off (since it uses the same memory as the player) and can take a
while to complete if run on big files.

\section{Applications}

\subsection{Battery\_test}
{\centering\itshape
  [Warning: Image ignored] % Unhandled or unsupported graphics:
%\includegraphics[width=3.704cm,height=2.117cm]{images/rockbox-manual-img58.png}
 \textmd{  }  [Warning: Image ignored]
% Unhandled or unsupported graphics:
%\includegraphics[width=4.598cm,height=1.976cm]{images/rockbox-manual-img59.png}
 \newline
  Recorder battery test  Player battery test  
\par}

This plugin simulates normal power drain by spinning up the disk and
reading a big file once every 90 seconds (or thereabouts). Each
spin up also writes the battery level to a
log file. The test stops when battery level reaches 4\% in order to
avoid being unable to write to the disk.  The power usage data is saved
to a file in the root directory of the Jukebox.  This plugin can
sometimes be useful for diagnosing problems with battery charging.

\subsection{Calculator (Recorder, Ondio)}
{\centering\itshape
  [Warning: Image ignored] % Unhandled or unsupported graphics:
%\includegraphics[width=3.889cm,height=2.223cm]{images/rockbox-manual-img60.png}
 \newline
Calculator
\par}

This is a simple scientific calculator for
use on the Jukebox.  It works like a standard calculator.  Move using
the arrow keys and press PLAY to press a button.  Pressing the ``1st''
button will toggle between other available maths functions on the right
hand side.

\subsection{Calendar (Recorder, Ondio)}
{\centering\itshape
  [Warning: Image ignored] % Unhandled or unsupported graphics:
%\includegraphics[width=3.9cm,height=2.23cm]{images/rockbox-manual-img61.png}
 \newline
Calendar
\par}

This is a small and simple calendar application with memo saving function.

Dots indicate dates with memos. To add a new memo press PLAY on the
date. Includes one off, annual, monthly, and weekly memos:

\begin{tabular}[c]{|p{2.6469998cm}|p{3.95cm}|}
\hline
{\centering\bfseries\itshape
KEY
\par}
&
{\centering\bfseries\itshape
ACTION
\par}
\\\hline
{\centering
PLAY
\par}
&
monthly
\\\hline
{\centering
LEFT
\par}
&
weekly
\\\hline
{\centering
RIGHT
\par}
&
annually 
\\\hline
{\centering
ON
\par}
&
one off
\\\hline
{\centering
STOP
\par}
&
exit 
\\\hline
\end{tabular}

\subsection{Chess Clock}
{\centering\itshape
  [Warning: Image ignored] % Unhandled or unsupported graphics:
%\includegraphics[width=4.634cm,height=1.984cm]{images/rockbox-manual-img62.png}
 \newline
Chess Clock 
\par}

The chess clock plugin is designed  to
simulate a chess clock, but it can be used
in any kind of game with up to ten players.

\subsubsection{Setup}

\begin{itemize}
\item First enter the number of players (1{}-10) (press  PLAY to
continue). 
\item Then set the total game time in mm:ss (press PLAY to continue,
STOP to go back). 
\item Then the maximum round time is entered.  For example, this could
be used  to play Scrabble for a maximum of 15 minutes each, with each
round taking no longer than one minute. (press  PLAY to continue). 
\item Done. Player 1 starts in paused mode. So press PLAY to start.
\end{itemize}

\subsubsection{While playing}
The number of the current player is displayed on the top line. The time
below is the time remaining for that round (and possibly also the total
time left if different). 

Keys are as follows:

\begin{center}\begin{tabular}{|p{1.917cm}|p{2.7649999cm}|p{1.7049999cm}|p{9.433001cm}|}
\hline
{\centering\bfseries\itshape
PLAYER 
\par}
&
{\centering\bfseries\itshape
RECORDER 
\par}
&
{\centering\bfseries\itshape
ONDIO 
\par}
&
{\centering\bfseries\itshape
FUNCTION 
\par}
\\\hline
{\centering
ON 
\par}
&
{\centering
OFF 
\par}
&
{\centering
ONOFF 
\par}
&
Exit plugin 
\\\hline
{\centering
STOP 
\par}
&
{\centering
LEFT 
\par}
&
{\centering
LEFT 
\par}
&
Restart round for the player 
\\\hline
{\centering
PLAY 
\par}
&
{\centering
PLAY 
\par}
&
{\centering
RIGHT 
\par}
&
Pausing the time (press again to continue) 
\\\hline
{\centering
RIGHT 
\par}
&
{\centering
UP 
\par}
&
{\centering
UP 
\par}
&
Switch to next player 
\\\hline
{\centering
LEFT 
\par}
&
{\centering
DOWN 
\par}
&
{\centering
DOWN 
\par}
&
Switch to previous player 
\\\hline
{\centering
MENU 
\par}
&
{\centering
F1 
\par}
&
{\centering
MODE 
\par}
&
Enter a simple menu 
\\\hline
\end{tabular}\end{center}
From the menu it is possible to delete a player, modify the round time
for the current player or set the total time for the game. 

When the round time is up for a player the message ``ROUND UP!'' is shown (press  NEXT to continue). 

When the total time is up for a player the message ``TIME UP!''is shown. Then player will  then be removed from the timer. 

\subsection{Clock (Recorder)}
{\centering\itshape
  [Warning: Image ignored] % Unhandled or unsupported graphics:
%\includegraphics[width=3.528cm,height=2.016cm]{images/rockbox-manual-img63.png}
 \newline
Clock
\par}

This is a fully featured analogue and digital clock program.  

\subsubsection{Key configuration}

\begin{center}\begin{tabular}{|p{2.411cm}|p{6.012cm}|}
\hline
{\centering\bfseries\itshape
KEY
\par}
&
{\centering\bfseries\itshape
ACTION
\par}
\\\hline
{\centering
F1
\par}
&
Help
\\\hline
{\centering
F2
\par}
&
Start / Stop stopwatch
\\\hline
{\centering
F2 (Hold)
\par}
&
Reset stopwatch
\\\hline
{\centering
F3
\par}
&
Options
\\\hline
{\centering
Play
\par}
&
Select clock mode
\\\hline
{\centering
UP
\par}
&
Enable idle power off
\\\hline
{\centering
DOWN
\par}
&
Disable idle power off
\\\hline
{\centering
RIGHT
\par}
&
Enable backlight
\\\hline
{\centering
LEFT
\par}
&
Disable backlight
\\\hline
{\centering
OFF
\par}
&
Save settings to disk and exit
\\\hline
\end{tabular}\end{center}

\subsubsection{Backlight configuration}
If RIGHT or LEFT is not pressed during clock operation (with the
exception of at the Help/Options/Mode Selector/Credit screens) then the
backlight timeout will remain your Rockbox default setting (example, 15
seconds). If RIGHT or LEFT is pressed, Clock will set the backlight to
ON or OFF, respectively. When Clock is exited, your default Rockbox
setting for Backlight will be restored. 

\subsubsection{Saving Settings}
Settings are saved to disk when Clock is exited. They are saved to
\textbf{/.rockbox/rocks/.clock\_settings''}. To reset your settings
back to the defaults, simply navigate to this file using Rockbox,
highlight it, and press the ON+PLAY keys to get the Delete option. This way you can feel free to experiment with the settings {}- and you could even load
separate settings, say, one for your desk at home and one for in the car {}- by keeping two files in your \textbf{/.rockbox/rocks} folder such as
``h.clock\_settings'' and ``c.clock\_settings''. Simply remove the
``h'' for your home settings to go into effect, or add the ``h'' back and take off the ``c'' for your car settings.

In the future, loading different settings will probably be made easier
through a built{}-in settings file loader in Clock. 

\subsection{Euro Converter (Player)}
{\centering\itshape
  [Warning: Image ignored] % Unhandled or unsupported graphics:
%\includegraphics[width=4.671cm,height=2cm]{images/rockbox-manual-img64.png}
 \newline
Euro converter
\par}

This plugin converts euros back into pre{}-euro currency.  The country for which is does this is selectable by pressing the MENU key.  The MINUS and
PLUS keys move the cursor between the digits and the PLAY and STOP keys
increase and decrease the current digit.  The amount in the old
currency is displayed on the second line.

\subsection{Favorites}
{\centering\itshape
  [Warning: Image ignored] % Unhandled or unsupported graphics:
%\includegraphics[width=4.667cm,height=1.998cm]{images/rockbox-manual-img65.png}
 \newline
Favorites
\par}

When listening to any song you can open it with this plugin and it will
add the current song to a special playlist of all songs you selected in
\textbf{/favorites.m3u}.

\subsection{Firmware\_flash (Recorder, Ondio)}
{\centering\mdseries\itshape
  [Warning: Image ignored] % Unhandled or unsupported graphics:
%\includegraphics[width=3.634cm,height=2.076cm]{images/rockbox-manual-img66.png}
 \newline
Firmware\_flash
\par}

Use when flashing Rockbox (see page \pageref{ref:FlashingRockboxReal}.
In the ideal case, you'll need this tool only once. For safety reasons you may wish to delete \textbf{firmware\_flash.rock} from \textbf{/.rockbox/rocks} once flashing is complete.

\subsection{Metronome}
This plugin can be used as a metronome to keep time during music
practice.  Adjust the tempo though the interface or by tapping it out
on the appropriate button.

\begin{tabular}[c]{|p{2.587cm}|p{2.55cm}|p{2.62cm}|p{4.952cm}|}
\hline
{\centering\bfseries\itshape
PLAYER 
\par}
&
{\centering\bfseries\itshape
RECORDER 
\par}
&
{\centering\bfseries\itshape
ONDIO 
\par}
&
{\centering\bfseries\itshape
FUNCTION 
\par}
\\\hline
{\centering
STOP 
\par}
&
{\centering
OFF 
\par}
&
{\centering
ONOFF 
\par}
&
Exit plugin 
\\\hline
{\centering
PLAY 
\par}
&
{\centering
PLAY 
\par}
&
{\centering
~ 
\par}
&
Start / Stop 
\\\hline
{\centering
ON 
\par}
&
{\centering
ON 
\par}
&
{\centering
~ 
\par}
&
Tap tempo 
\\\hline
{\centering
~ 
\par}
&
{\centering
~ 
\par}
&
{\centering
MODE 
\par}
&
Start / Tap tempo
\\\hline
{\centering
~ 
\par}
&
{\centering
~ 
\par}
&
{\centering
HOLD MODE 
\par}
&
Stop 
\\\hline
{\centering
MINUS/PLUS
\par}
&
{\centering
LEFT/RIGHT 
\par}
&
{\centering
LEFT/RIGHT 
\par}
&
Adjust tempo 
\\\hline
{\centering
ON+MINUS/\newline
ON+PLUS 
\par}
&
{\centering
UP/DOWN 
\par}
&
{\centering
UP/DOWN 
\par}
&
Adjust volume
\\\hline
\end{tabular}

\subsection{Split Editor (Recorder, Ondio)}
When recording an mp3 file, it is common practice to start the recording
a little bit early and stop it a little bit late to ensure all the
desired sound is recorded. This results in recordings that contain
extra snippets of sound and the beginning and end. Unfortunately these
snippets can not be deleted easily because they are stored in the same
file as the desired recording. The purpose of the split editor is to
split a mp3 file (the input file) at a point in time (split point). Two
new files can be generated from the input file. The first file contains
the part before the split point and the second file contains the part
after the split point. Once this process has been successful the
original file can be deleted or kept as a backup. 

The whole process of splitting a mp3 file consists of three steps: 

\begin{enumerate}
\item defining the split point 
\item generating the result files. 
\item if desired delete the input file (with the browser, not the split
editor) 
\end{enumerate}

\subsubsection{How to use the Split Editor}

\begin{itemize}
\item \textbf{Pause near the split point}
When the device plays the song just hit the PAUSE button, when playback
has roughly reached the split point. This need not be very precise as
the split point can be fine tuned later.
\item \textbf{Open the split editor}

Open the plugin.  A screen similar to the one below will appear. 

{\centering\itshape
  [Warning: Image ignored] % Unhandled or unsupported graphics:
%\includegraphics[width=3.701cm,height=2.11cm]{images/rockbox-manual-img67.gif}
 \newline
The Split Editor
\par}

{\centering\upshape
Here is an explanation of the areas marked in red on the screenshot.
\par}

\begin{enumerate}
\item The waveform \newline
\newline
The waveform displays the volume of the song over time. It will appear
as the song plays and help to visually identify the point in time where
the split is desired
\item The split point indicator\newline
\newline
The split point indicator is a vertical line with a small triangle at
the top end. It is the most important control element of the split
editor. It can be moved with the LEFT and RIGHT buttons. Later, when
you have fine tuned the split point, the song will be split at this
position.
\item The split time\newline
\newline
At the top of the window a time value is displayed. This is the point in
time within the song at which the split point indicator is positioned. 
\item The locator\newline
\newline
Another vertical bar represents the position locator. It moves along as
the song plays. In contrast to the split point indicator it has no
triangles at the ends. 
\item The time bar\newline
\newline
The time bar displays the current position within the song relative to
the whole song. The entire length of the time bar represents the song
length. The length of the solid part of the time bar represents the position and length
of the displayed part of the song.
\item The scale mode\newline
\newline
Directly above the F3 button the scale mode is displayed. The waveform
can be scaled either logarithmically or linearly. In logarithmic scale
mode the letters ``dB'' are displayed, in linear mode ``\%''. Use F3 to
switch between these modes. Linear mode usually gives better optical
hints with commercially recorded music. For quiet recordings,
especially of human speech, the logarithmic scale often is preferable.
\item The loop mode \newline
\newline
Directly above the F2 button the loop mode icon is displayed. There are
4 different loop modes. Pressing F2 changes to the next loop mode. 

\begin{itemize}
\item   [Warning: Image ignored] % Unhandled or unsupported graphics:
%\includegraphics[width=0.794cm,height=0.476cm]{images/rockbox-manual-img68.gif}
  Playback loops around the split point indicator. This mode is best
used when searching and zooming for the desired point at which to split
the recording. 
\item   [Warning: Image ignored] % Unhandled or unsupported graphics:
%\includegraphics[width=0.794cm,height=0.476cm]{images/rockbox-manual-img69.gif}
  Playback loops from the split point indicator to the end of the
visible area. This mode is best used when fine tuning the split
indicator position at the beginning of a recording. 
\item   [Warning: Image ignored] % Unhandled or unsupported graphics:
%\includegraphics[width=0.794cm,height=0.476cm]{images/rockbox-manual-img70.gif}
  Playback loops from the beginning of the
visible area to the split point. This mode is best used when fine
tuning the split indicator position at the end of a recording.
\item   [Warning: Image ignored] % Unhandled or unsupported graphics:
%\includegraphics[width=0.688cm,height=0.476cm]{images/rockbox-manual-img71.gif}
  Playback doesn't loop, the borders of the visible
area as well as the split point indicator are ignored. This mode is
best used when playing the song outside of the borders of the displayed
region. 
\end{itemize}

\item Perform the split \newline
\newline
The icon directly above the F1 button indicates its function to execute
the split. When split positioning is complete open the save dialogue with F1.
\end{enumerate}

{\bfseries
Controls in the split editor }
\end{itemize}

\begin{tabular}[c]{|p{2.975cm}|p{3.047cm}|p{6.649cm}|}
\hline
{\centering\bfseries\itshape
Recorder 
\par}
&
{\centering\bfseries\itshape
Ondio 
\par}
&
{\centering\bfseries\itshape
Function 
\par}
\\\hline
{\centering
Off 
\par}
&
{\centering
On/Off 
\par}
&
Quit plugin 
\\\hline
{\centering
Left/Right 
\par}
&
{\centering
Left/Right 
\par}
&
Move the split point indicator 
\\\hline
{\centering
Up/Down 
\par}
&
{\centering
Up/Down 
\par}
&
Zoom in / out 
\\\hline
{\centering
Play 
\par}
&
{\centering
Mode 
\par}
&
Play from the split position 
\\\hline
{\centering
F1 
\par}
&
{\centering
Mode+Left 
\par}
&
Enter the save dialogue
\\\hline
{\centering
F2 
\par}
&
{\centering
Mode+Up 
\par}
&
Toggle loop modes 
\\\hline
{\centering
F3 
\par}
&
{\centering
Mode+Right 
\par}
&
Toggle logarithmic / linear scaling 
\\\hline
{\centering
On+Left 
\par}
&
{\centering
~ 
\par}
&
Play half speed 
\\\hline
{\centering
On+Right 
\par}
&
{\centering
~ 
\par}
&
Play 150\% speed 
\\\hline
{\centering
On+Play 
\par}
&
{\centering
~ 
\par}
&
Play normal speed 
\\\hline
\end{tabular}

\subsubsection{Save the files}
In the save dialogue it is possible to specify which of the files you
want to save and their names.  When finished, select
``Save'' and the files will be written to
disk. Note that files can not be overwritten, so filenames that
don't exist yet must be chosen. If unsure whether the
file already exists simply try to save it. If another file with this
name exists the dialogue will return and you can choose another
filename

{\centering\itshape
  [Warning: Image ignored] % Unhandled or unsupported graphics:
%\includegraphics[width=3.701cm,height=2.11cm]{images/rockbox-manual-img72.gif}
 \newline
Save dialogue
\par}

Controls in the save dialogue
\begin{tabular}[c]{|p{2.62cm}|p{2.266cm}|p{3.965cm}|}
\hline
{\centering\bfseries\itshape
RECORDER 
\par}
&
{\centering\bfseries\itshape
ONDIO 
\par}
&
{\centering\bfseries\itshape
FUNCTION 
\par}
\\\hline
{\centering
UP/DOWN 
\par}
&
{\centering
UP/DOWN 
\par}
&
Select item 
\\\hline
{\centering
PLAY 
\par}
&
{\centering
RIGHT 
\par}
&
Toggle / edit item 
\\\hline
\end{tabular}

\subsubsection{Scale}
The values in the waveform are scaled according to the settings of the
peak meter. These can be altered in the menu
\textbf{General Settings {}-{\textgreater} Display{}-{\textgreater} Peak Meter}. If extreme minimum /
maximum values are set the waveform might be cut off.  A minimum
setting of {}-60 dB and a maximum setting of 0 dB are recommended.
These settings should be capable of producing useful waveforms for very
soft sounds in logarithmic mode (dB). When the editor is used on loud
sounds (such as commercial rock or pop music) switching to the linear
scale may prove more effective since the logarithmic scale compresses
loud noises and makes it more difficult to identify characteristic
shapes. Note that it is always possible to toggle the scale with F3. 

\subsection{Stopwatch}
{\centering\itshape
  [Warning: Image ignored] % Unhandled or unsupported graphics:
%\includegraphics[width=3.704cm,height=2.117cm]{images/rockbox-manual-img73.png}
 \textmd{  }  [Warning: Image ignored]
% Unhandled or unsupported graphics:
%\includegraphics[width=4.667cm,height=1.998cm]{images/rockbox-manual-img74.png}
 \newline
Recorder stopwatch  Player stopwatch  
\par}

A simple stopwatch program with support for saving times.

\subsubsection{Keys are as follows:}

\begin{tabular}[c]{|p{2.9029999cm}|p{2.763cm}|p{2.199cm}|p{5.235cm}|}
\hline
{\centering\bfseries\itshape
PLAYER 
\par}
&
{\centering\bfseries\itshape
RECORDER 
\par}
&
{\centering\bfseries\itshape
ONDIO 
\par}
&
{\centering\bfseries\itshape
FUNCTION 
\par}
\\\hline
{\centering
MENU 
\par}
&
{\centering
OFF 
\par}
&
{\centering
ONOFF 
\par}
&
Quit Plugin 
\\\hline
{\centering
PLAY 
\par}
&
{\centering
PLAY 
\par}
&
{\centering
RIGHT 
\par}
&
Start / stop 
\\\hline
{\centering
STOP 
\par}
&
{\centering
LEFT
\par}
&
{\centering
LEFT 
\par}
&
Reset timer 
\\\hline
{\centering
ON 
\par}
&
{\centering
ON 
\par}
&
{\centering
MODE 
\par}
&
Take lap time 
\\\hline
{\centering
MINUS/PLUS 
\par}
&
{\centering
DOWN/UP 
\par}
&
{\centering
DOWN/UP 
\par}
&
Scroll through lap times
\\\hline
\end{tabular}

% $Id$ %
\chapter{The Main Menu}

\section{Introducing the Main Menu}
\screenshot{main_menu/images/ss-main-menu}{The main menu}{}
The \setting{Main Menu} is the screen from which the rest of the Rockbox functions can be 
accessed. It is used for a variety of functions, which are detailed below. 
All options in Rockbox can be controlled via the \setting{Main Menu}.
To enter the \setting{Main Menu}, 
  \opt{IRIVER_H100_PAD,IRIVER_H300_PAD}{press the \ButtonMode\ button.}%
  \opt{RECORDER_PAD}{press the \ButtonFOne\ button.}%
  \opt{PLAYER_PAD,IPOD_4G_PAD,IPOD_3G_PAD}{press the \ButtonMenu\
    button.}%
  \opt{ONDIO_PAD}{hold the \ButtonMenu{} button.}%
  \opt{IAUDIO_X5_PAD}{press the \ButtonRec\ button.}%

All settings are stored on the unit. However, Rockbox does not spin up 
the disk solely for the purpose of saving settings.  Instead, Rockbox will
save settings when it spins up the disk the next time, for example when 
refilling the MP3 buffer or navigating through the file browser. Changes to 
settings may therefore not be saved unless the \dap\ is shut down safely 
(see \reference{ref:Safeshutdown}).

\section{Navigating the Main Menu}
\opt{RECORDER_PAD,ONDIO_PAD,IRIVER_H100_PAD,IRIVER_H300_PAD,IAUDIO_X5_PAD,IPOD_4G_PAD,IPOD_3G_PAD}{
  \begin{table}
    \begin{btnmap}{}{}
      \opt{IPOD_4G_PAD,IPOD_3G_PAD,IPOD_VIDEO_PAD}{\ButtonScrollFwd}
      \opt{RECORDER_PAD,ONDIO_PAD,IRIVER_H100_PAD,IRIVER_H300_PAD,IAUDIO_X5_PAD}{\ButtonUp} 
      & Moves up in the menu.\\
      & Inside a setting, increases the value or
        chooses next option \\
      %
      \opt{IPOD_4G_PAD,IPOD_3G_PAD,IPOD_VIDEO_PAD}{\ButtonScrollBack}
      \opt{RECORDER_PAD,ONDIO_PAD,IRIVER_H100_PAD,IRIVER_H300_PAD,IAUDIO_X5_PAD}{\ButtonDown} 
      & Moves down in the menu.\\
      & Inside a setting, decreases the value or
        chooses previous option \\
      %
      \opt{RECORDER_PAD}{\ButtonPlay/\ButtonRight}
      \opt{IRIVER_H100_PAD,IRIVER_H300_PAD,IAUDIO_X5_PAD}{\ButtonSelect/\ButtonRight}
      \opt{ONDIO_PAD,IPOD_4G_PAD,IPOD_3G_PAD,IPOD_VIDEO_PAD}{\ButtonRight} 
      & Selects option \\
      %
      \opt{RECORDER_PAD,IRIVER_H100_PAD,IRIVER_H300_PAD}{\ButtonOff/\ButtonLeft}
      \opt{IAUDIO_X5_PAD,ONDIO_PAD,IPOD_4G_PAD,IPOD_3G_PAD,IPOD_VIDEO_PAD}{\ButtonLeft} 
      & Exits menu, setting or moves to parent menu\\
    \end{btnmap}
  \end{table}
}
\opt{PLAYER_PAD}{
  \begin{table}
    \begin{btnmap}{}{}
      %
      \ButtonLeft  & Selects previous option in the menu.\\
                   & Inside an setting, decreases the value or chooses previous option \\
      %
      \ButtonRight & Selects next option in the menu.\\
                   & Inside an setting increases the value or chooses next option \\
      %
      \ButtonPlay  & Selects item \\
      %
      \ButtonStop  & Exit menu, setting or moves to parent menu.\\
    \end{btnmap}
  \end{table}
}

\section {Recent Bookmarks}
\screenshot{main_menu/images/ss-list-bookmarks}%
  {The list bookmarks screen}{}
If the \setting{Save a list of recently created bookmarks} option is enabled 
then you can view a list of several recent bookmarks here and select one to 
jump straight to that track. See \reference{ref:Bookmarkconfigactual} 
for more details on configuring bookmarking in Rockbox.
\note{This option is off by default.}

\section{Sound Settings}
The \setting{Sound Settings} menu offers a selection of sound properties you may 
change to customize your listening experience.  The details of this menu are 
covered in detail in \reference{ref:configure_rockbox_sound}. 

\section{General Settings}
The \setting{General Settings} menu allows you to customize the way Rockbox looks 
and the way it plays music.  The details of this menu are covered in detail in 
\reference{ref:configure_rockbox_general}.

\section{Manage Settings}
The \setting{Manage Settings} option allows the saving and re-loading of user 
configuration settings, browsing the hard drive for alternate firmwares, and finally
resetting your \dap{} back to initial configuration.
%
\opt{MASCODEC}{This menu also allows the user to load different versions of the 
Rockbox firmware.} 
%
The details of this menu are covered in detail in 
\reference{ref:manage_settings}.

\section{Browse Themes}
This option will display all the currently installed themes on the \dap, 
press \ButtonRight\ to load the chosen theme and apply it.

A theme is basically a configuration file, stored in a specific directory, 
that typically changes the WPS 
\opt{h1xx,h300,x5}{and remote WPS}, font used and on some platforms
additional information such as background image and text colours.

There are a number of themes that ship with Rockbox. If none of
these suit your needs, many more can be downloaded from 
\opt{RECORDER_PAD}{\wikilink{WpsArchos}}%
\opt{h1xx}{\wikilink{WpsIriverH100}}%
\opt{h300,ipodcolor}{\wikilink{WpsIriverH300}}%
\opt{ipodvideo}{\wikilink{WpsIpod5g}}%
\opt{ipodnano}{\wikilink{WpsIpodNano}}%
\opt{ipodmini}{\wikilink{WpsIpodMini}}%
\opt{x5}{\wikilink{WpsIaudioX5}}%
.
Some of the downloads from this site will actually be standalone WPS files, 
others will be full-blown themes. 

\note{Themes do not have to be purely visual.  It is quite possible to create
a theme that switches between audio configurations for use in the car, with
headphones and when connected to an external amplifier.  See 
\reference{ref:CreateYourOwnWPS} for more details.
}

\opt{CONFIG_TUNER}{% $Id$ %
\section{\label{ref:FMradio}FM Radio}  
\opt{RECORDER_PAD}{
  \note{The early V2 models were in fact FM Recorders in disguise,
  so they had the FM radio still mounted. Rockbox enables it if present -
  in case this menu doesn't show on your unit you can skip this chapter.\\}
}
\screenshot{main_menu/images/ss-fm-radio-screen}{The FM radio screen}{}
  This menu option switches to the radio screen.
  The FM radio has the ability \opt{HAVE_RECORDING}{to record and } to
  remember station frequency settings (presets).
  \opt{MASCODEC}{\note{The radio will shorten battery life, because the
      MAS-chip is set to record mode for instant recordings.}
  }

    \begin{table}
      \begin{btnmap}{}{}
          \ButtonLeft, \ButtonRight
          & Change frequency in \setting{SCAN} mode or jump to next/previous
          station in \setting{PRESET} mode\\
          %
          Long \ButtonLeft, \ButtonRight & Seek to next station or preset in
            \setting{SCAN} mode.\\
          %
          \ButtonUp, \ButtonDown
          & Change volume.\\
          \opt{RECORDER_PAD}{
            \ButtonPlay
            & freezes all screen updates. May enhance radio reception in some
              cases.\\
          }
          %
          \opt{RECORDER_PAD}{\ButtonOn}
          \opt{ONDIO_PAD}{Long \ButtonOff}
          \opt{IRIVER_H100_PAD,IRIVER_H300_PAD}{\ButtonMode}
          \opt{IAUDIO_X5_PAD}{\ButtonRec}
          & Leave the radio screen with the radio playing.\\
          %
          \opt{RECORDER_PAD,ONDIO_PAD,IRIVER_H100_PAD,IRIVER_H300_PAD}
            {\ButtonOff}
          \opt{IAUDIO_X5_PAD}{\ButtonPower}
          & Stops the radio and returns to \setting{Main Menu}.\\
          %
          \opt{IRIVER_H100_PAD,IRIVER_H300_PAD}{\ButtonOn & Mutes radio playback.\\}
          \opt{IAUDIO_X5_PAD}{\ButtonPlay & Mutes radio playback.\\}
          %
          \opt{RECORDER_PAD,IRIVER_H100_PAD,IRIVER_H300_PAD,IAUDIO_X5_PAD}{
            \opt{RECORDER_PAD,IRIVER_H100_PAD,IRIVER_H300_PAD}{Long \ButtonOn}
            \opt{IAUDIO_X5_PAD}{Long \ButtonPlay}
            & Switches between \setting{SCAN} and \setting{PRESET} mode.\\
          }
          %
          \opt{RECORDER_PAD,IRIVER_H100_PAD,IRIVER_H300_PAD,IAUDIO_X5_PAD}{
            \opt{RECORDER_PAD}{\ButtonFTwo}
            \opt{IRIVER_H100_PAD,IRIVER_H300_PAD,IAUDIO_X5_PAD}{\ButtonSelect}
            & Opens a list of radio presets. You can view all the presets that
              you have, and switch to the station.\\
          }
          %
          \opt{RECORDER_PAD}{\ButtonFOne}
          \opt{ONDIO_PAD}{Long \ButtonMenu}
          \opt{IRIVER_H100_PAD,IRIVER_H300_PAD,IAUDIO_X5_PAD}{Long \ButtonSelect}
          & Displays the FM radio settings menu.\\
       \end{btnmap}
    \end{table}
    
  \begin{description}
    
  \item[Saving a preset:]
    Up to 64 of your favourite stations can be saved as presets.
    \opt{RECORDER_PAD}{Press \ButtonFTwo\ to go to the presets list, press
    \ButtonFOne\ to add a preset.}%
    \opt{ONDIO_PAD,IRIVER_H100_PAD,IRIVER_H300_PAD,IAUDIO_X5_PAD}{%
      \opt{ONDIO_PAD}{Hold \ButtonMenu}%
      \opt{IRIVER_H100_PAD,IRIVER_H300_PAD,IAUDIO_X5_PAD}{Hold \ButtonSelect}
      to go to the menu, then select \setting{Add preset}.%
    }
    Enter the name (maximum number of characters is 32).
    \opt{RECORDER_PAD}{Press \ButtonFTwo}%
    \opt{ONDIO_PAD}{Hold \ButtonMenu}%
    \opt{IRIVER_H100_PAD,IRIVER_H300_PAD}{Press \ButtonOn}%
    \opt{IAUDIO_X5_PAD}{Press \ButtonPlay}
    to save.

    \note{See this page for pre-made FM radio presets from all around the world.}
    \wikilink{FmPresets}

  \item[Selecting a preset:]
        \opt{ONDIO_PAD}{Hold \ButtonMenu\ to open the menu, select
          \setting{Preset}}%
        \opt{RECORDER_PAD}{Press \ButtonFTwo}%
        \opt{IRIVER_H100_PAD,IRIVER_H300_PAD,IAUDIO_X5_PAD}{Press \ButtonSelect}
        to go to the presets list.
        Use \ButtonUp\ and \ButtonDown\ to move the cursor and then press
        \opt{RECORDER_PAD}{\ButtonPlay}%
        \opt{ONDIO_PAD}{\ButtonRight}%
        \opt{IRIVER_H100_PAD,IRIVER_H300_PAD,IAUDIO_X5_PAD}{\ButtonSelect}
        to select. Use \ButtonLeft\ to leave the preset without selecting
        anything.

  \item[Removing a preset:]
        \opt{ONDIO_PAD}{Hold \ButtonMenu\ to open the menu, select \setting{Preset}}%
        \opt{RECORDER_PAD}{Press \ButtonFTwo}%
        \opt{IRIVER_H100_PAD,IRIVER_H300_PAD,IAUDIO_X5_PAD}{Press \ButtonSelect}
        to go to the presets list. Use \ButtonUp\ and \ButtonDown\ to move
        the cursor and then
        \opt{RECORDER_PAD}{press \ButtonFThree}%
        \opt{ONDIO_PAD}{Hold \ButtonRight}%
        \opt{IRIVER_H100_PAD,IRIVER_H300_PAD,IAUDIO_X5_PAD}{Hold \ButtonSelect}
        on the preset that you wish to remove, then select \setting{Remove Preset}.

      \opt{RECORDER_PAD,ONDIO_PAD}{
          \item[Recording:]
            \opt{RECORDER_PAD}{Press \ButtonFThree}%
            \opt{ONDIO_PAD}{Double press \ButtonMenu}
            to start recording the currently playing station. Press \ButtonOff\ to
              stop recording.%
            \opt{RECORDER_PAD}{ Press \ButtonPlay\ again to seamlessly start recording
              to a new file.}
            The settings for the recording can be changed in the
            \opt{RECORDER_PAD}{\ButtonFOne\ menu}%
            \opt{ONDIO_PAD}{respective menu reached through the FM radio settings menu
              (Long \ButtonMenu)}
            before starting the recording. See \reference{ref:Recordingsettings}
            for details of recording settings.
          }
  \end{description}
  \note{The radio will turn off when starting playback of an audio file.}
}

\opt{HAVE_RECORDING}{% $Id$ %
\section{\label{ref:Recording}Recording}
  \subsection{\label{ref:Whilerecordingscreen}While Recording Screen}
  \screenshot{main_menu/images/ss-while-recording-screen}{The while recording
    screen}{}

  Entering the \setting{Recording} option in the \setting{Main Menu} brings up
  a screen in wich you can choose to enter the \setting{Recording Screen} or
  the \setting{Recording Settings} (see below). The \setting{Recording Screen}
  shows the time elapsed and the size of the file being recorded. A peak meter
  is present to allow you set Gain correctly. There is also a volume setting,
  this will only affect the output level of the \dap\ and does \emph{not}
  affect the recorded sound.
  \opt{MASCODEC}{The frequency, channels and quality}
  \opt{SWCODEC}{The frequency and channels} settings are shown on the last line.
  
  The controls for this screen are:
  
    \begin{table}
      \begin{btnmap}{}{}
        
        \ButtonUp\ / \ButtonDown & Select setting.\\
        %
        \ButtonLeft\ / \ButtonRight & Adjust selected setting.\\
        %
        \opt{RECORDER_PAD,IAUDIO_X5_PAD}{\ButtonPlay}
        \opt{ONDIO_PAD}{\ButtonMenu}
        \opt{IRIVER_H100_PAD,IRIVER_H300_PAD}{\ButtonOn}
        \opt{IPOD_4G_PAD}{\fixme{FixMe}}
        & Start recording.\\
        & While recording: pause recording (press again to continue).\\
        %
        \opt{RECORDER_PAD,ONDIO_PAD,IRIVER_H100_PAD,IRIVER_H300_PAD}{\ButtonOff}
        \opt{IAUDIO_X5_PAD}{Hold \ButtonPlay}
        \opt{IPOD_4G_PAD}{\fixme{FixMe}}
        & Exit \setting{Recording Screen}.\\
        & While recording: Stop recording.\\
        %
        \opt{IRIVER_H100_PAD,IRIVER_H300_PAD,IAUDIO_X5_PAD}{
          \ButtonRec & Starts recording.\\
          & While recording: close the current file and open a new one.\\
        }
        %
        \opt{RECORDER_PAD}{\ButtonFOne}
        \opt{ONDIO_PAD}{Hold \ButtonMenu}
        \opt{IRIVER_H100_PAD,IRIVER_H300_PAD}{\ButtonMode}
        \opt{IAUDIO_X5_PAD}{Hold \ButtonRec}
        \opt{IPOD_4G_PAD}{Hold \ButtonSelect} 
        & Open \setting{Recording Settings} (see below).\\
        %
        \opt{RECORDER_PAD}{
          \ButtonFTwo & Quick menu for recording settings. A quick press will
          leave the screen up (press \ButtonFTwo\ again to exit), while holding
          it will close the screen when you release it.\\
        }
        %
        \opt{RECORDER_PAD}{
          \ButtonFThree & Quick menu for source setting.\\
          & Quick/hold works as for \ButtonFTwo.\\
          & While recording: Start a new recording file.\\
        }
      \end{btnmap}
  \end{table}
  
  \subsection{\label{ref:Recordingsettings}Recording Settings}
  \screenshot{main_menu/images/ss-recording-settings}{The recording settings screen}{}
  
  \opt{MASCODEC}{
    \begin{description}
    \item[Quality:]
      Choose the quality here (0 to 7). Default is 5, best quality is 7,
      smallest file size is 0. This setting effects how much your sound
      sample will be compressed. Higher quality settings result in larger
      MP3 files.
      
      The quality setting is just a way of selecting an average bit rate,
      or number of  bits per second, for a recording.  When  this setting
      is lowered, recordings are compressed more (meaning worse sound quality),
      and the average bitrate changes as follows.
    \end{description}
    
    \begin{table}[h!]
      \begin{center}
        \begin{tabularx}{0.75\textwidth}{lX}\toprule
          \emph{Frequency} & \emph{Bitrate} (Kbit/s) -- quality 0$\rightarrow$7 \\\midrule
          44100Hz stereo        & 75, 80, 90, 100, 120, 140, 160, 170 \\
          22050Hz stereo        & 39, 41, 45, 50,  60,  80,  110, 130 \\
          44100Hz mono          & 65, 68, 73, 80,  90,  105, 125, 140 \\
          22050Hz mono          & 35, 38, 40, 45,  50,  60,  75,  90 \\\bottomrule
        \end{tabularx}
      \end{center}
    \end{table}
  }
  
    \begin{description}
    \opt{MASCODEC,x5}{
    \item[Frequency:]
      Choose the recording frequency (sample rate) -- 48kHz, 44.1kHz, 32kHz and
      24kHz, 22.05kHz, 16kHz are available. Higher sample rates use up more disk
      space, but give better sound quality.
      This setting determines which frequency range can accurately be reproduced
      during playback, Lower frequencies produce smaller files.
      \opt{MASCODEC}{
        The frequency setting also determines which version of the MPEG standard
        the sound is recorded using:\\
        MPEG v1 for 48, 44.1 and 32\\
        MPEG v2 for 24, 22.05 and 16\\
      }
  }
  \item[Source:]
    Choose the source of the recording. This can be
    \opt{recorder,recorderv2fm,h1xx}{SPDIF (digital),} microphone or line in.
    \opt{CONFIG_TUNER}{For recording from the radio see \rockref{ref:FMradio}.}
    
    % Add h1xx here whenever it supports setting recording frequency.
    \opt{recorder,recorderv2fm}
        {\note{You cannot change the sample rate for digital recordings.}}

  \opt{MASCODEC,x5}{
    \item[Channels:]
      This allows you to select mono or stereo recording. Please note that
      for mono recording, only the left channel is recorded.  Mono recordings
      are usually somewhat smaller than stereo.
  }

  \opt{MASCODEC}{
    \item[Independent Frames:]
      The independent frames option tells the \dap\ to encode with the bit
      reservoir disabled, so the frames are independent of each other. This
      makes a file easier to edit.
  }

  \item[Time Split:]
    This option is useful when timing recordings. If set to active it stops
    a recording at a given interval and then starts recording again with a
    new file, which is useful for long term recordings.
    \newline
    The splits are seamless (frame accurate), no audio is lost at the split
    point. The break between recordings is only the time required to stop
    and restart the recording, on the order of 2 -- 4 seconds.
    \newline
    Options (hours:minutes between splits): off, 24:00, 18:00, 12:00, 10:00,
    8:00, 6:00, 4:00, 2:00, 1:20 (80 minute CD), 1:14 (74 minute  CD), 1:00,
    00:30, 00:15, 00:10, 00:05.
    
  \item[Prerecord Time:]
    This setting buffers a small amount of audio so that when the record button
    is pressed, the recording will begin from that number of seconds earlier.
    This is useful for ensuring that a recording begins before a cue that is
    being waited for.\\
    Options: \setting{Off}, \setting{1 -- 30 seconds}
    
  \item[Directory:]
    Allows changing the location where the recorded files are saved. The
    default location is \fname{/recordings}.
    
  \item[Show recording screen on startup:]
    If set to yes, the \dap\ will start up with the while recording screen
    showing.\\
    Options: \setting{Yes}, \setting{No}
    
  \item[Clipping Light:]
    Causes the backlight to flash on when clipping has been detected.\\
    Options: \setting{Off}, \setting{Remote unit only},
      \setting{Main and remote unit}, \setting{Main unit only}.
    
  \opt{MASCODEC}{
    \item[Trigger:]
    \fixme{Add description of triggered recording.}
  }

  \end{description}
}

\section{\label{ref:playlistoptions}Playlist Options}
  This menu allows you to work with playlists. Playlists can be created in 
  three ways. Playing a file in a directory causes all the files in the 
  directory to be placed in a playlist. Playlists can be created manually by
  either using the  \setting{File Menu} (see \reference{ref:Filemenu}) or using
  the \setting{Playlist Options} menu.  Both automatically and manually created
  playlists can be edited using this menu.

\begin{description}
\item[Create Playlist:]
  Rockbox will create a playlist with all tracks in the current directory 
and all subdirectories. The playlist will be created one folder level ``up'' 
from where you currently are.
  
\item[View Current Playlist:]
  Displays the contents of the playlist currently stored in memory.
  
\item[Save Current Playlist:]
  Saves the current dynamic playlist, excluding queued tracks, to the 
specified file. If no path is provided then playlist is saved to current 
directory (see \reference{ref:Playlistsubmenu}).
  
\item[Recursively Insert Directories: ]
  If set to \setting{On}, then when a directory is inserted or queued into a 
  dynamic playlist, all subdirectories will also be inserted. If set to \setting{Ask},
  Rockbox will prompt the user about whether to include subdirectories.
  Options: \setting{Off}, \setting{Ask}, \setting{On}

\item[Warn When Erasing Dynamic Playlist: ]
  If set to \setting{Yes}, Rockbox will provide a warning if the user attempts to
  take an action that will cause Rockbox to erase the current dynamic playlist.
  Options: \setting{Yes}, \setting{No}
\end{description}

\section{Browse Plugins}
  With this option you can load and run various plugins that have been
written for Rockbox. There are a wide variety of these supplied with
Rockbox, including several games, some impressive demos and a number of
utilities. A detailed description of the different plugins is to be found in 
\reference{ref:plugins}.

\section{\label{ref:Info}Info}
  This option shows RAM buffer size, battery voltage level and estimated time
remaining, disk total space and disk free space.
\opt{player}{Use the MINUS and PLUS keys to step through several 
pages of information.}

\begin{description}
\item[Rockbox Info:]
  Displays some basic system information.  This is, from top to bottom, 
the amount of memory Rockbox has available for storing music (the buffer). The
battery status.\nopt{ondio}{Hard disk size and the amount of free space on the
disk.}\opt{ondio}{Memory size divided into \emph{Int} and \emph{MMC}. Int shows
the size and state (free space) of the internal memory, and MMC shows size and
state of the plugged in Multi Media Card (in case one is plugged in).}

\opt{ondio}{memory} size and the amount 
of free space on the disk.
  
\item[Version:]
  Software version and credits display.
  
\item[Debug (Keep Out!):]
  This submenu is intended to be used \emph{only} by Rockbox developers.
  It shows hardware, disk, battery status and other technical information.  
  \warn{It is not recommended that users access this menu unless instructed to
  do so in the course of fixing a problem with Rockbox. If you think you have 
  messed up your settings by use of this menu please try to reset \emph{all} 
  settings before asking for help.}
\end{description}

\opt{player}{
  \section{Shutdown}
  This menu option saves the Rockbox configuration and turns off the hard 
  drive before shutting down the machine. For maximum safety this procedure 
  is recommended when turning off the \dap. (There is a very small risk
  of hard disk corruption otherwise.) See \reference{ref:Safeshutdown}
  for more details.
}

\opt{RECORDER_PAD,IRIVER_H100_PAD,IRIVER_H300_PAD,IPOD_4G_PAD,IPOD_3G_PAD,IAUDIO_X5_PAD}
{
\section{Quick Menu}
  Whilst not strictly part of the \setting{Main Menu}, it is worth noting that a few
  of the more commonly used settings are available from the \setting{Quick Menu}.
  The \setting{Quick Menu} screen is accessed by holding the 
    \opt{RECORDER_PAD}{\ButtonFTwo}%
    \opt{IRIVER_H100_PAD,IRIVER_H300_PAD}{\ButtonMode}%
    \opt{IPOD_4G_PAD,IPOD_3G_PAD,IPOD_VIDEO_PAD}{\ButtonMenu}%
    \opt{IAUDIO_X5_PAD}{\ButtonRec}%
  {} key, and it allows rapid access to the \setting{Shuffle} and \setting{Repeat} modes 
  (\reference{ref:PlaybackOptions})
  and the \setting{Show Files} option (\reference{ref:ShowFiles}).
}

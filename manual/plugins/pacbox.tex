\subsection{Pacbox}
Pacbox is an emulator of the Pacman arcade machine hardware. It is a port of \emph{PIE - Pacman Instructional Emulator} (\url{http://www.ascotti.org/programming/pie/pie.htm}).
\begin{figure}[h!]
  \begin{center}
    \includegraphics[width=4cm]{plugins/images/ss-pacbox-\genericimg.png}
  \end{center}
  \caption{Pacbox}
\end{figure}

\subsubsection{ROMs}
To use the emulator to play Pacman, you need a copy of ROMs for "Midway Pacman".
\begin{center}
  \begin{tabular}{ll}\toprule
    \textbf{Filename} & \textbf{MD5 checksum}\\\midrule
    pacman.5e & 2791455babaf26e0b396c78d2b45f8f6\\
    pacman.5f & 9240f35d1d2beee0ff17195653b5e405\\
    pacman.6e & 290aa5eae9e2f63587b5dd5a7da932da\\
    pacman.6f & 19a886fcd8b5e88b0ed1b97f9d8659c0\\
    pacman.6h & d7cce8bffd9563b133ec17ebbb6373d4\\
    pacman.6j & 33c0e197be4c787142af6c3be0d8f6b0\\\bottomrule
  \end{tabular}
\end{center}

These need to be stored in the \fname{/.rockbox/pacman/} directory on your player.
In the MAME ROMs collection the necessary files can be found in \fname{pacman.zip} and \fname{puckman.zip}.

\subsubsection{Keys}
\begin{center}
  \begin{tabular}{ll}\toprule
    \textbf{Key} & \textbf{Action}\\\midrule
    \opt{h1xx,h300}{RIGHT}\opt{ipodcolor,ipodnano}{NEXT} & Move Up\\
    \opt{h1xx,h300}{LEFT}\opt{ipodcolor,ipodnano}{PREV} & Move Down\\
    \opt{h1xx,h300}{UP}\opt{ipodcolor,ipodnano}{MENU} & Move Left\\
    \opt{h1xx,h300}{DOWN}\opt{ipodcolor,ipodnano}{PLAY} & Move Right\\
    \opt{h1xx,h300}{REC}\opt{ipodcolor,ipodnano}{SELECT} & Insert Coin\\
    \opt{h1xx,h300,ipodcolor,ipodnano}{SELECT} & 1-Player Start\\
    \opt{h1xx,h300}{ON}\opt{ipodcolor,ipodnano}{n/a} & 2-Player Start\\
    \opt{h1xx,h300}{MODE}\opt{ipodcolor,ipodnano}{SELECT+MENU} & Menu\\\bottomrule
  \end{tabular}
\end{center}


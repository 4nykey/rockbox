\subsection{PictureFlow}
\screenshot{plugins/images/ss-pictureflow}{PictureFlow}{img:pictureflow}
PictureFlow provides a visualisation of your albums with their associated cover
art. \opt{swcodec}{It is possible to start playback of the selected
album from PictureFlow. Playback will start from the selected track.}

\opt{hwcodec}{
\note{PictureFlow is a visualisation only. It cannot be used to select and
-play music.  Also, using this plugin will cause playback to stop.}
}

\subsubsection{Requirements}
PictureFlow uses both the album art (see \reference{ref:album_art}) and 
database (see \reference{ref:database}) features of Rockbox.
It is therefore important that these are working correctly before attempting
to use PictureFlow. In addition, there are some other points of which to be
aware:

  \begin{itemize}
    \item PictureFlow will accept album art larger than the dimensions of the
    screen, but the larger the dimensions, the longer they will take to scale.
  \end{itemize}

\subsubsection{Keys}
\begin{table}
    \begin{btnmap}{}{}
        \opt{scrollwheel,IRIVER_H10_PAD}{
          \opt{scrollwheel}{\ButtonScrollFwd{} / \ButtonScrollBack}
          \opt{IRIVER_H10_PAD}{\ButtonScrollUp{} / \ButtonScrollDown}
          & Scroll through albums / track list\\}

        \nopt{scrollwheel,IRIVER_H10_PAD}{\ButtonLeft{} / \ButtonRight

            \opt{IRIVER_RC_H100_PAD}{&}
        & Scroll through albums\\}

        \nopt{scrollwheel,IRIVER_H10_PAD}{\ButtonUp{} / \ButtonDown
           \opt{IRIVER_RC_H100_PAD}{&}
        & Scroll through track list\\}

        \opt{IRIVER_H10_PAD}{\ButtonRight}
        \opt{ONDIO_PAD}{\ButtonUp}
        \opt{RECORDER_PAD}{\ButtonOn}
        \nopt{ONDIO_PAD,IRIVER_H10_PAD,RECORDER_PAD}{\ButtonSelect}
          \opt{IRIVER_RC_H100_PAD}{&}
        & Enter track list \opt{swcodec}{/ Play album from selected
          track}\\

        \ButtonLeft
            \opt{IRIVER_RC_H100_PAD}{&}
        & Exit track list\\

        \opt{IRIVER_H100_PAD,IRIVER_H300_PAD}{\ButtonMode}
        \opt{IPOD_1G2G_PAD,IPOD_3G_PAD,IPOD_4G_PAD,GIGABEAT_PAD,%
             GIGABEAT_S_PAD,MROBE100_PAD}{\ButtonMenu}
        \opt{IAUDIO_X5_PAD}{\ButtonRec}
        \opt{SANSA_E200_PAD}{\ButtonDown}
        \opt{IRIVER_H10_PAD,SANSA_C200_PAD}{\ButtonPower}
        \opt{RECORDER_PAD}{\ButtonFOne}
        \opt{ONDIO_PAD}{Long \ButtonMenu}
           \opt{IRIVER_RC_H100_PAD}{&}
        & Enter menu\\

        \opt{IRIVER_H100_PAD,IRIVER_H300_PAD,RECORDER_PAD,ONDIO_PAD}{\ButtonOff}
        \opt{IAUDIO_X5_PAD,GIGABEAT_PAD,GIGABEAT_S_PAD,SANSA_E200_PAD,%
             MROBE100_PAD}{\ButtonPower}
        \opt{SANSA_C200_PAD,IRIVER_H10_PAD}{Long \ButtonPower}
        \opt{IPOD_1G2G_PAD,IPOD_3G_PAD,IPOD_4G_PAD}{Long \ButtonMenu}
           \opt{IRIVER_RC_H100_PAD}{&}
        & Exit PictureFlow\\
    \end{btnmap}
\end{table}

\subsubsection{Main Menu}
\begin{description}
  \item[Settings.] Enter the settings menu.
  \item[Return.] Exit menu.
  \item[Quit.] Exit PictureFlow plugin.
\end{description}

\subsubsection{Settings Menu}

\begin{description}
  \item[Show FPS.] Displays frames per second on screen.
  \item[Spacing.] The distance between the front edges of the side slides, i.e. changes
  the degree of overlap of the side slides. A larger number means less overlap. Scales with zoom.
  \item[Centre margin.] The distance, in screen pixels, with zoom at 100, between
  the centre and side slides. Scales with zoom.
  \item[Number of slides.] Sets the number of slides at each side, including the
  centre slide. Therefore if set to 4, there will be 3 slides on the left,
  the centre slide, and then 3 slides on the right.
  \item[Zoom.] Changes the distance at which slides are rendered from the "camera".
  \item[Show album title.] Allows setting the album title to be shown above or
  below the cover art, or not at all.
  \item[Resize Covers.] Set whether to automatically resize the covers or to leave
  them at their original size.
  \item[Rebuild cache.] Rebuild the PictureFlow cache. This is needed in order
  for PictureFlow to pick up new albums, and may occasionally be needed if albums
  are removed.
\end{description}

\subsection{Text Viewer}
\screenshot{plugins/images/ss-text_viewer-main}{Text Viewer}{img:text_viewer-main}
This is a Viewer for text files with word wrap. Just open a \fname{.txt} or \fname{.nfo} file to
display it. The text viewer features controls to handle various styles of text
formatting, has top{}-of{}-file and bottom{}-of{}-file buttons. You can view
files without a \fname{.txt} or \fname{.nfo} extension by using \emph{Open with} from the
\emph{Context Menu} (see \reference{ref:Contextmenu}). You can put bookmarks any pages 
and it can jump the page where the bookmark is put.

\subsubsection{Default keys}
\begin{table}
    \begin{btnmap}{}{}
    \opt{PLAYER_PAD}{\ButtonLeft}
    \opt{RECORDER_PAD,ONDIO_PAD,IRIVER_H100_PAD,IRIVER_H300_PAD,IAUDIO_X5_PAD%
        ,SANSA_E200_PAD,SANSA_FUZE_PAD,GIGABEAT_PAD,MROBE100_PAD}{\ButtonUp}
    \opt{IPOD_4G_PAD,IPOD_3G_PAD}{\ButtonScrollBack}
    \opt{IRIVER_H10_PAD}{\ButtonScrollUp}
    \opt{SANSA_C200_PAD,SANSA_CLIP_PAD}{\ButtonVolUp}
    \opt{GIGABEAT_S_PAD}{\ButtonPrev}
    \opt{COWON_D2_PAD}{\ButtonMinus{} / \TouchTopMiddle}
    \opt{HAVEREMOTEKEYMAP}{&
        \opt{IAUDIO_M3_PAD}{\ButtonRCUp}
    }
        & Scroll{}-up\\

    \opt{PLAYER_PAD}{\ButtonRight}
    \opt{RECORDER_PAD,ONDIO_PAD,IRIVER_H100_PAD,IRIVER_H300_PAD,IAUDIO_X5_PAD%
        ,SANSA_E200_PAD,SANSA_FUZE_PAD,GIGABEAT_PAD,MROBE100_PAD}{\ButtonDown}
    \opt{IPOD_4G_PAD,IPOD_3G_PAD}{\ButtonScrollFwd}
    \opt{IRIVER_H10_PAD}{\ButtonScrollDown}
    \opt{SANSA_C200_PAD,SANSA_CLIP_PAD}{\ButtonVolDown}
    \opt{GIGABEAT_S_PAD}{\ButtonNext}
    \opt{COWON_D2_PAD}{\ButtonPlus{} / \TouchBottomMiddle}
    \opt{HAVEREMOTEKEYMAP}{&
        \opt{IAUDIO_M3_PAD}{\ButtonRCDown}
    }
        & Scroll{}-down\\

    \opt{PLAYER_PAD}{\ButtonMenu+\ButtonLeft}
    \opt{GIGABEAT_S_PAD}{\ButtonPlay+\ButtonLeft}
    \opt{COWON_D2_PAD}{\TouchMidLeft}
    \nopt{PLAYER_PAD,GIGABEAT_S_PAD,COWON_D2_PAD,IAUDIO_M3_PAD}{\ButtonLeft}
    \opt{HAVEREMOTEKEYMAP}{&
        \opt{IAUDIO_M3_PAD}{\ButtonRCLeft}
    }
        & Top of file (Narrow mode) /
        One screen left (Wide mode)\\

    \opt{PLAYER_PAD}{\ButtonMenu+\ButtonRight}
    \opt{GIGABEAT_S_PAD}{\ButtonPlay+\ButtonRight}
    \opt{COWON_D2_PAD}{\TouchMidRight}
    \nopt{PLAYER_PAD,GIGABEAT_S_PAD,COWON_D2_PAD,IAUDIO_M3_PAD}{\ButtonRight}
    \opt{HAVEREMOTEKEYMAP}{&
        \opt{IAUDIO_M3_PAD}{\ButtonRCRight}
    }
        & Bottom of file (Narrow mode) /
        One screen right (Wide mode)\\

    \opt{RECORDER_PAD,IRIVER_H100_PAD,IRIVER_H300_PAD,SANSA_E200_PAD%
        ,SANSA_FUZE_PAD,SANSA_C200_PAD,SANSA_CLIP_PAD,GIGABEAT_S_PAD}{%
        \opt{RECORDER_PAD,IRIVER_H100_PAD,IRIVER_H300_PAD}{\ButtonOn+\ButtonUp}
        \opt{SANSA_E200_PAD,SANSA_FUZE_PAD}{\ButtonScrollBack}
        \opt{SANSA_C200_PAD,SANSA_CLIP_PAD,GIGABEAT_S_PAD}{\ButtonUp}
    \opt{HAVEREMOTEKEYMAP}{& }
        & One line up\\
    }

    \opt{RECORDER_PAD,IRIVER_H100_PAD,IRIVER_H300_PAD,SANSA_E200_PAD%
        ,SANSA_FUZE_PAD,SANSA_C200_PAD,SANSA_CLIP_PAD,GIGABEAT_S_PAD}{%
        \opt{RECORDER_PAD,IRIVER_H100_PAD,IRIVER_H300_PAD}{\ButtonOn+\ButtonDown}
        \opt{SANSA_E200_PAD,SANSA_FUZE_PAD}{\ButtonScrollFwd}
        \opt{SANSA_C200_PAD,SANSA_CLIP_PAD,GIGABEAT_S_PAD}{\ButtonDown}
    \opt{HAVEREMOTEKEYMAP}{& }
        & One line down\\
    }

    \opt{RECORDER_PAD,IRIVER_H100_PAD,IRIVER_H300_PAD,GIGABEAT_S_PAD}{
        \opt{RECORDER_PAD,IRIVER_H100_PAD,IRIVER_H300_PAD}{\ButtonOn+\ButtonLeft}
        \opt{GIGABEAT_S_PAD}{\ButtonLeft}
    \opt{HAVEREMOTEKEYMAP}{& }
        & One column left\\
    }

    \opt{RECORDER_PAD,IRIVER_H100_PAD,IRIVER_H300_PAD,GIGABEAT_S_PAD}{
        \opt{RECORDER_PAD,IRIVER_H100_PAD,IRIVER_H300_PAD}{\ButtonOn+\ButtonRight}
        \opt{GIGABEAT_S_PAD}{\ButtonRight}
    \opt{HAVEREMOTEKEYMAP}{& }
        & One column right\\
    }

    \opt{RECORDER_PAD,PLAYER_PAD,IPOD_4G_PAD,IPOD_3G_PAD,IAUDIO_X5_PAD%
        ,IRIVER_H10_PAD,GIGABEAT_S_PAD}{\ButtonPlay}
    \opt{IRIVER_H100_PAD,IRIVER_H300_PAD}{\ButtonSelect}
    \opt{ONDIO_PAD}{\ButtonMenu}
    \opt{GIGABEAT_PAD}{\ButtonA}
    \opt{SANSA_C200_PAD,SANSA_E200_PAD}{\ButtonRec}
    \opt{SANSA_CLIP_PAD}{\ButtonHome}
    \opt{SANSA_FUZE_PAD}{\ButtonDown+\ButtonSelect}
    \opt{MROBE100_PAD}{\ButtonDisplay}
    \opt{COWON_D2_PAD}{\TouchCenter}
    \opt{HAVEREMOTEKEYMAP}{&
        \opt{IAUDIO_M3_PAD}{\ButtonRCMode}
    }
        & Toggle autoscroll\\

    \opt{PLAYER_PAD}{\ButtonOn}
    \opt{RECORDER_PAD}{\ButtonFTwo}
    \opt{ONDIO_PAD}{\ButtonMenu+\ButtonOff}
    \opt{IRIVER_H100_PAD,IRIVER_H300_PAD}{\ButtonOn+\ButtonSelect}
    \opt{IPOD_4G_PAD,IPOD_3G_PAD,GIGABEAT_PAD,GIGABEAT_S_PAD,MROBE100_PAD}{\ButtonSelect}
    \opt{IRIVER_H10_PAD}{\ButtonFF}
    \opt{IAUDIO_X5_PAD}{\ButtonRec}
    \opt{SANSA_E200_PAD,SANSA_C200_PAD,SANSA_CLIP_PAD}{\ButtonDown+\ButtonSelect}
    \opt{COWON_D2_PAD}{\ButtonMenu+\ButtonPlus}
    \opt{SANSA_FUZE_PAD}{\ButtonUp+\ButtonSelect}
    \opt{HAVEREMOTEKEYMAP}{&
        \opt{IAUDIO_M3_PAD}{\ButtonRCPlay+\ButtonRCMode}
    }
        & Set/Reset Bookmark\\

    \opt{RECORDER_PAD}{\ButtonFOne}
    \opt{ONDIO_PAD}{Long \ButtonMenu}
    \opt{PLAYER_PAD,IPOD_4G_PAD,IPOD_3G_PAD,GIGABEAT_PAD,GIGABEAT_S_PAD%
        ,MROBE100_PAD}{\ButtonMenu}
    \opt{IRIVER_H100_PAD,IRIVER_H300_PAD}{\ButtonMode}
    \opt{IAUDIO_X5_PAD,SANSA_C200_PAD,SANSA_CLIP_PAD,SANSA_E200_PAD}{\ButtonSelect}
    \opt{SANSA_FUZE_PAD}{Long \ButtonSelect}
    \opt{IRIVER_H10_PAD}{\ButtonRew}
    \opt{COWON_D2_PAD}{\ButtonMenu{} / \TouchTopRight}
    \opt{HAVEREMOTEKEYMAP}{&
        \opt{IAUDIO_M3_PAD}{\ButtonRCPlay}
    }
        & Enter menu\\

    \opt{PLAYER_PAD}{\ButtonStop}
    \opt{RECORDER_PAD,ONDIO_PAD,IRIVER_H100_PAD,IRIVER_H300_PAD}{\ButtonOff}
    \opt{IPOD_4G_PAD,IPOD_3G_PAD}{\ButtonSelect+\ButtonMenu}
    \opt{IAUDIO_X5_PAD,IRIVER_H10_PAD,SANSA_E200_PAD,SANSA_C200_PAD,SANSA_CLIP_PAD%
        ,GIGABEAT_PAD,MROBE100_PAD}{\ButtonPower}
    \opt{SANSA_FUZE_PAD}{Long \ButtonHome}
    \opt{GIGABEAT_S_PAD}{\ButtonBack}
    \opt{COWON_D2_PAD}{\ButtonPower{} / \TouchTopLeft}
    \opt{IAUDIO_M3_PAD}{\ButtonRec}
    \opt{HAVEREMOTEKEYMAP}{&
        \opt{IAUDIO_M3_PAD}{\ButtonRCRec}
        \opt{IRIVER_RC_H100_PAD}{\ButtonRCStop}
    }
        & Exit text viewer\\

    \end{btnmap}
\end{table}

\subsubsection{The Viewer's Menu}

\begin{description}
\item[Return] to the text view.
\item[Viewer Options] set the settings for the reading file.
    \begin{description}
    \item[Encoding]
    sets the codepage in the text viewer.
% ToDo: wrap some of the following settings into a \opt{lcd_bitmap} to exlude
% ones that don't work on charcell - as soon as the plugin itself does
    Available settings:
    \setting{ISO-8859-1} (Latin 1).
    \setting{ISO-8859-7} (Greek),
    \opt{lcd_bitmap}{
        \setting{ISO-8859-8} (Hebrew),
    }
    \setting{CP1251} (Cyrillic),
    \opt{lcd_bitmap}{
        \setting{ISO-8859-11} (Thai),
        \setting{CP1256} (Arabic),
    }
    \setting{ISO-8859-9} (Turkish),
    \setting{ISO-8859-2} (Latin Extended),
    \setting{CP1250} (Central European),
    \opt{lcd_bitmap}{
        \setting{SJIS} (Japanese),
        \setting{GB-2312} (Simple Chinese),
        \setting{KSX-1001} (Korean),
        \setting{BIG5} (Traditional Chinese),
    }
    \setting{UTF-8} (Unicode),
    This setting only applies to the plugin and is independent from the
    \setting{Default Codepage} setting (see \reference{ref:Defaultcodepage}).
    \item[Word Wrap] toggles between Wrap and Chop.
        \begin{description}
            \item[On] breaks lines at the maximum column limit.
            \item[Off (Chop Words)] breaks lines at white space or hyphen.
        \end{description}
    \item[Line Mode] cycles through Normal, Join and Expand and Reflow Lines.
        \begin{description}
            \item[Normal] breaks lines at newline characters.
            \item[Join] joins lines. Useful for adopting the orphans that
            occur with e{}-mail style (i.e. pre{}-wrapped) text files.
            \item[Expand] adds a blank line. Useful for making the paragraphs
            clearer in some book style text files.
            \opt{lcd_bitmap}{
            \item[Reflow Lines] justifies the text fully.
            }
        \end{description}
    \item[Wide View] toggles between Narrow and Wide.
        \begin{description}
            \item[No (Narrow)] sets maximum column to the screen width.
            \item[Yes] sets maximum column to 114. Useful for navigating large files.
            (Currently, Wide and Join cannot be selected together.)
        \end{description}
    \opt{lcd_bitmap}{
    \item[Show Scrollbar] toggles scrollbar for the current View mode. If the
    file fits on one screen, there is no scrollbar and toggling this setting
    has no effect.
        \begin{description}
            \item[Off] has no scrollbar by default, until toggled.
            \item[On] has a scrollbar by default, until toggled.
        \end{description}
    \item[Overlap Pages] toggles between Normal and Overlap.
        \begin{description}
            \item[No] sets page{}-down/page{}-up to one full screen.
            \item[Yes] tells page{}-down/page{}-up to retain one line from previous screen.
        \end{description}
    }
    \opt{lcd_bitmap}{
    \item[Show Header] selects whether or not the status bar and the reading file path.
        \begin{description}
            \item[None] do not display the status bar and the reading file path.
            \item[File Path] displays the reading file path. (the status bar does not display.)
            \item[Status bar] displays the status bar. (the reading file path does not display.)
            \item[Both] display the status bar and the reading file path.
        \end{description}
        \note{If the position of the status bar is not top, then \setting{Status bar} and \setting{Both} are not displayed.}

    \item[Show Footer] selects whether or not the status bar and the page number.
        \begin{description}
            \item[None] do not display the status bar and the page number.
            \item[Page Num] displays the page number. (the status bar does not display.)
            \item[Status bar] s the status bar. (the page number does not display.)
            \item[Both] display the status bar and the page number.
        \end{description}
        \note{If the position of the status bar is not bottom, then \setting{Status bar} and \setting{Both} are not displayed.}

    \item[Font] selects the font which displsy the text. When the font is changed, also the font of the text viewer's menu changes. But when this plugin exit, \setting{Main Menu} is displayed with the original font (i.e., the font which selects \setting{Settings $\rightarrow$ Theme Settings $\rightarrow$ Font}. See \reference{ref:configure_rockbox_themes}).
    }
    \item[Scroll Mode] controls the function of the ``Scroll-up'' and
    ``Scroll-down'' buttons.
        \begin{description}
            \item[Scroll by Page]
            \item[Scroll by Line]
        \end{description}
    \item[Auto-scroll Speed]
    controls the speed of auto-scrolling in number of lines per scroll step,
    available options are \setting{1} to \setting{10} lines. As an example,
    a setting of \setting{4} will scroll up the text four lines per second.
    \end{description}
    \note{When the setting is changed, it takes time to calculate the number of pages again when the size of the text file is large.}

\item[Show Playback Menu] controls the playback of the currently loaded playlist
and change the volume of your \dap{} without leaving the plugin.
\item[Select Bookmark] selects the bookmark which is registered and jump to the corresponding page.
In the figure below, the line where '*' is displayed at the head shows the page which is running now.
If you do not select any bookmarks, the text viewer displays the page which is reading now.


\screenshot{plugins/images/ss-text_viewer-sel_bk_menu}{select bookmark menu}{img:text_viewer-sel_bk}

\item[Global Settings] set global settings for the text viewer. When reading a file for the first time, this setting is used. The setting items are the same as \setting{Viewer Options}. The global settings are stored in \fname{.rockbox/rocks/viewers/viewer.dat}.
\item[Quit] Exits the plugin. At this time, the text viewer automatically stores its settings, the current position and bookmarks in \fname{.rockbox/rocks/viewers/viewer\_file.dat}.
\end{description}

\subsubsection{Bookmarks}
    When you press the \opt{PLAYER_PAD}{\ButtonOn}\opt{RECORDER_PAD}{\ButtonFTwo}%
    \opt{ONDIO_PAD}{\ButtonMenu+\ButtonOff}%
    \opt{IRIVER_H100_PAD,IRIVER_H300_PAD}{\ButtonOn+\ButtonSelect}%
    \opt{IPOD_4G_PAD,IPOD_3G_PAD,GIGABEAT_PAD,GIGABEAT_S_PAD,MROBE100_PAD}{\ButtonSelect}%
    \opt{IRIVER_H10_PAD}{\ButtonFF}\opt{IAUDIO_X5_PAD}{\ButtonRec}%
    \opt{SANSA_E200_PAD,SANSA_C200_PAD,SANSA_CLIP_PAD}{\ButtonDown+\ButtonSelect}%
    \opt{COWON_D2_PAD}{\ButtonMenu+\ButtonPlus}%
    \opt{SANSA_FUZE_PAD}{\ButtonUp+\ButtonSelect}%
    \opt{IAUDIO_M3_PAD}{\ButtonRCPlay+\ButtonRCMode} button, the bookmark is added to the first 
    line on this page and 
    \opt{PLAYER_PAD}{the bookmark symbol adds the first line}%
    \nopt{PLAYER_PAD}{the first line reverses} as shown in the figure below. 
    And when you press this button again, this bookmark is deleted, and returns to the original 
    screen.

\screenshot{plugins/images/ss-text_viewer-bookmark}{bookmark}{img:text_viewer-bookmark}


\subsubsection{Limitations}

\begin{itemize}
\item The display of the menu is English regardless of the codepgae. 
\item Does not currently support right{}-to{}-left languages.
\end{itemize}

% $Id$ %
\section{\label{ref:SystemOptions}System}
\subsection{Start Screen}
    Set the screen that Rockbox will start in. Selecting
    \setting{Resume Playback} will resume playback where it was when the \dap{}
    was shut off if there is a playlist to resume and will then end up in the
    WPS. Selecting \setting{Previous Screen} will make Rockbox start in the
    screen it was when the \dap{} was shut off.

\nopt{sansa}%will probably be there on Sansas one day -  exclude it the simple way without specific option
{
  \subsection{Battery}
  Options relating to the \opt{archos}{batteries}\nopt{archos}{battery}
  in the \dap.
  \begin{description}
    \item [Battery Capacity: ]This setting can be used to tell Rockbox what
      capacity (in mAh) the battery being used has. The default is
      \opt{player,recorder}{1500mAh}%
      \opt{recorderv2fm}{2200mAh}%
      \opt{ondiosp,ondiofm}{1000mAh}%
      \opt{h1xx,h300}{1300mAh}%
      \opt{ipodmini}{400mAh (1G) or 630mAh (2G)}%
      \opt{ipodcolor}{700mAh}%
      \opt{ipodnano,ipodvideo,ipod4g,ipod3g}{\fixme{unknown}}%
      \opt{m5,x5}{950mAh}%
      , which is the capacity value for the standard
      \opt{archos}{batteries}\nopt{archos}{battery} shipped with the \dap.
        Rockbox uses this value to estimate remaining battery life for the
        info screen and WPS. Changing this setting has no effect whatsoever on
        actual battery life. This setting only affects the accuracy of the
        battery life estimation as shown on screen.
      \opt{iaudio}{Rockbox does not automatically distinguish between the
        ``L'' models and the ``simple'' models which determine the default
        value. If your dap{} is an \opt{m5}{M5L}\opt{x5}{X5L} set the value
        to 2250mAh for more accuracy in the runtime estimation.}
      \opt{ipod,sansa}{This value is fairly meaningless in the \playerman{}
        family at present, and work is on-going into finding a better way to
        determine battery life.}

      \opt{BATTERY_TYPES}{%Ondios only (currently), following the lang features.txt
      \item [Battery Type: ]This setting tells Rockbox which type of battery
        is currently used in the \dap{}. The two supported battery types are
        ``Alkaline'' or ``NiMH''.}

      \opt{HAVE_USB_CHARGING}{
%TODO: check where it is working correctly (except H300) and append the option to the respective platform file
      \item [Charge During USB Connection: ]This option lets you control whether
        the \dap{} should charge during the USB connection and hence draw the
        full 500mA. Turning it \setting{Off} is recommended if the dap{} is
        connected through an unpowered USB hub or a laptop port.
      }

     \end{description}
}
\opt{HAVE_DIRCACHE,HAVE_DISK_STORAGE}{
\subsection{Disk}
    Options relating to the hard disk.

  \begin{description}
    \opt{HAVE_DISK_STORAGE}{
    \item [Disk Spindown: ]Rockbox has a timer that makes it spin down the
      hard disk after it is idle for a certain amount of time. This setting
      controls the amount of time between the last user activity and the time
      that the disk spins down. This idle time is only affected by user
      activity, like navigating through the \setting{File Browser}. When the
      hard disk spins up to fill the audio buffer, it automatically spins down
      afterwards.
    }

    \opt{HAVE_DIRCACHE}{
    \item [Directory Cache: ]Rockbox has the ability to cache the contents of
      your drive in RAM. The \setting{Directory Cache} takes a small amount of
      memory away from Rockbox that would otherwise be used to buffer music,
      but it speeds up navigation in the file browser by eliminating
      the slight pause between the time a navigation button is pressed and the
      time Rockbox responds. Turning this setting on activates the
      directory cache, and turning it off deactivates the directory
      cache.
      \note{The first time you enable the directory cache,
      Rockbox will request a reboot of the \dap{} and upon restarting take a few
      minutes to scan the drive. After this, the directory cache will work in the
      background.}
    }
  \end{description}
} % \opt{HAVE_DIRCACHE,HAVE_DISK_STORAGE}

\opt{CONFIG_RTC}{
  \subsection{Time and Date}
  Time related menu options.
    \begin{description}
      \item [Set Time/Date: ] Set current time and date.
      \item [Time Format: ] Choose 12 or 24 hour clock.
    \end{description}
}

\subsection{Idle Poweroff}
Rockbox can be configured to turn off power after the unit has been idle for a
defined number of minutes. The \dap{} is idle when playback is stopped or
paused. It is not idle while the USB or charger is connected, or while
recording. Settings are either \setting{Off} or \setting{1} to \setting{10}
minutes in 1 minute steps. Then \setting{15,30,45} and \setting{60} minutes are
available.

\opt{HAVE_RTC_ALARM}{
\subsection{Wake-Up Alarm}
% TODO this isn't quite right for all targets, I think
    This option turns the \dap{} off and then starts it up again at the
    specified time. Use \ActionSettingInc{} and \ActionSettingDec{} to adjust
    the minutes setting, \ActionAlarmHoursDec{} and \ActionAlarmHoursInc{} to
    adjust the hours.
    \ActionAlarmSet{} confirms the alarm and shuts the \dap{} down, and
    \ActionAlarmCancel{} cancels setting an alarm. If the \dap{} is turned on
    again before the alarm occurs, the alarm will be cancelled. Using this
    option disables the \setting{Sleep Timer}.

  \opt{HAVE_RECORDING,CONFIG_TUNER}{
    \subsection{Alarm Wake up Screen}
    This option controls what the \dap{} does when it is woken up by the alarm.
  }
}

\subsection{Limits}
This sub menu relates to limits in the Rockbox operating system.
  \begin{description}
    \item [Max Entries in File Browser: ]This setting controls the limit on
    the number of files that you can put in any particular directory in the
    file browser. You can configure the size to be between 50 and
    10,000 files in steps of 50. The default is 400. Higher values will shorten
    the music buffer, so you should increase this setting \emph{only} if you have
    directories with a large number of files.

    \item [Max Playlist Size: ]This setting controls the maximum size of
    a playlist. The playlist size can be between 1,000 and 32,000 files,
    in steps of 1,000 (default is 10,000). Higher values will shorten the
    music buffer, so you should increase this setting \emph{only} if you
    have very large playlists.
  \end{description}

% TODO: this needs to be rewritten in another style, it lets you mix sound from another source into the music
\opt{player}{
  \subsection{Line In} This option activates the line in port on \dap, which is
    off by default. This is useful for such applications as:
    \begin{itemize}
    \item Game boy $\rightarrow$ \dap $\rightarrow$ human
    \item laptop $\rightarrow$ \dap $\rightarrow$ human
    \item LAN party computer $\rightarrow$ \dap $\rightarrow$ human
    \end{itemize}
}

\opt{HAVE_CAR_ADAPTER_MODE}{
\subsection{Car Adapter Mode}
      This option turns \setting{On} and \setting{Off} the car ignition
      auto stop function.
  \begin{description}
  \item [Car Adapter Mode: ]When using the \dap{} in a car,
  \setting{Car Adapter Mode} automatically stops playback on the \dap{} when
  power (i.e. from cigarette lighter power adapter) to the external DC in jack
  is turned off. If the \setting{Car Adapter Mode} is set to \setting{On},
  Rockbox will pause playback when the external power off condition is
  detected. Rockbox will then shutdown the \dap{} after the length of time set
  in the \setting{Idle Poweroff} setting (see above).
  If power to the DC in jack is turned back on before the \emph{Idle Poweroff}
  function has shut the \dap{} off, playback will be resumed 5 seconds after
  the power is applied. This delay is to allow for the time while the car
  engine is being started.
  \end{description}

  Once the \dap{} is shut off either manually, or automatically with the
  \setting{Idle Poweroff} function, it must be powered up manually to resume
  playback.
}

\opt{HAVE_BUTTON_LIGHTS}{
  \opt{e200}{
  \subsection{Wheel Light Timeout}
    This setting controls the amount of time the wheel lights shine after a
    button press or wheel turn. If set to \setting{Off}, the LEDs will not
    light when a button is pressed. If set to \setting{On}, the lights will
    never shut off. If set to a time (1 to 120 seconds), the wheel will stay
    lit for that amount of time after a button press or wheel turn.
  }
  \nopt{e200}{
  \subsection{Button Light Timeout}
    This setting controls the amount of time the button lights shine after a
    button press. If set to \setting{Off}, the LEDs will not light when a
    button is pressed. If set to \setting{On}, the lights will never shut off.
    If set to a time (1 to 120 seconds), the buttons will stay lit for
    that amount of time after a button press.
  }
  \opt{gigabeatf}{
    \subsection{Button Light Brightness}
      Changes the brightness of the button lights.
  }
}

% $Id$ %
\section{\label{ref:Bookmarkconfigactual}Bookmarking}
  Bookmarks allow you to save your current position within a track so that 
  you can return to it at a later time. Bookmarks are saved on a per folder
  basis. In other words, all of the files in the same folder have their
  bookmarks stored together in the folder where the files are located. You
  can store multiple bookmarks for the same track.
  \begin{description}
  \item [Bookmark on Stop. ]This option controls whether Rockbox writes a
    bookmark to the disk when playback is stopped. Setting this to
    \setting{No} turns automatic bookmarking completely off. In contrast
    \setting{Yes} turns automatic bookmarking on while \setting{Ask} 
    asks on stopping the track if a bookmark should be created.
    With the above options \setting{Yes} and \setting{Ask} if there is an existing 
    \fname{.bmark} file the current position information will be added to the
    front of the existing list, up to the maximum number of allowed bookmarks
    per file (currently 10). If no \fname{.bmark} file exists, one will be
    created with the new bookmark information. Finally, if the 
    \setting{Maintain a list of Recently Used Bookmarks}
     option is enabled, the bookmarking
    information will be added to recent bookmarks list.
    \begin{description}    
      \item[Yes -- Recent Only.]
            Turns on automatic bookmarking -- One bookmark only
      \item[Ask -- Recent Only.]
            Asks if a bookmark should be created when stopping track -- 
            One bookmark only
    \end{description}
    With the two \setting{Recent Only} options, nothing is written to the 
    \fname{.bmark} file. If the \setting{Maintain a list of Recently Used Bookmarks}
    option is enabled, the bookmarking information will however be added to
    recent bookmarks list. 
    \note{The \setting{Resume} function remembers your position in the most
      recently accessed track regardless of how the \setting{Bookmark on Stop}
      option is set.}
    
  \item [Load Last Bookmark. ]

  When the \setting{Load Last Bookmark} option is set to \setting{Yes}, Rockbox
  automatically returns to the position of the last bookmark within a file
  when that file is played. 

  When the \setting{Load Last Bookmark} option is set to \setting{Ask}, Rockbox will
  give the user the option of starting from the beginning of the track
  of or from the bookmark. 

  When the \setting{Load Last Bookmark} option is set to \setting{No}, playback always
  starts from the beginning of the track, and the user must play the bookmark
  or use the \setting{Load Bookmark} function in the Main Menu, while the file
  is playing, to resume at the bookmarked location.
    
  \item [Maintain a list of Recently Used Bookmarks. ]

  This list of Most Recent Bookmarks (MRB's) may be accessed through the
  \setting{Recent Bookmarks} option of the \setting{Bookmarks} sub menu of the 
  Main Menu. When set to \setting{Yes} each new bookmark will be added to the
  MRB list. Setting this to \setting{No} disables the addition of bookmarks to
  the MRB list. \setting{Unique Only} will remove an old bookmark for the current
  track from the MRB list and replace it with the new one if a bookmark in the
  MRB list already existed. Otherwise this will behave like the 
  \setting{Yes} setting.
  \fixme{The above information was obtained by reading the source code, but my C is rather rusty...}
  \end{description}

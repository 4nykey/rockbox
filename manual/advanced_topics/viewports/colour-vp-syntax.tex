\subsubsection{Viewport Declaration Syntax}

\config{\%V(x,y,[width],[height],[font]) \%Vf([fgcolour]) \%Vb([bgcolour]) %
    \%Vg(start, end [,text])}%

    \begin{itemize}
      \item \%Vf and \%Vb set the foreground and background colours respectively.
      \item `fgcolour' and `bgcolour' are 6-digit RGB888 colours, e.g. FF00FF.
      \item \%Vg defines a gradient fill that can then be used with the \%Vs tag.
      `start' and `end' set the initial and final colours, and the optional `text'
      sets the text colour. Colours are 6-digit RGB888, e.g. FF00FF.
      \item `font' is a number: 0 is the built-in system font, 1 is the
      current menu font, and 2-9 are additional skin loaded fonts (see 
      \reference{ref:multifont}).
      \item Only the coordinates \emph{have} to be specified. Leaving the other
      definitions blank will set them to their default values.
    \end{itemize}

\note{The correct number of commas with hyphens in
      blank fields are still needed.}

\begin{example}
    %V(12,20,-,-,1) %Vf(000000) %Vb(FFFFFF) %Vg(FFC0CB, FF0000, FFFF00)
    %sThis viewport is displayed permanently. It starts 12px from the left and
    %s20px from the top of the screen, and fills the rest of the screen from
    %sthat point. The lines will scroll if this text does not fit in the viewport.
    %sThe user font is used, and the foreground and background are set to black
    %sand white respectively. The line gradient is set to pink to red with yellow
    %text.
\end{example}
\begin{rbtabular}{.75\textwidth}{XX}{\textbf{Viewport definition} & \textbf{Default value}}{}{}
  width/height & remaining part of screen \\
  font & user defined \\
  foreground/background colours & defined by theme \\
\end{rbtabular}


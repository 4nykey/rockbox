% $Id$ %
\chapter{Installation}\label{sec:installation}

\section{Prerequisites}\label{sec:prerequisites}
\index{Installation!Prerequisites}
Before installing Rockbox you should make sure you meet the prerequisites.
Also you may need some tools for installation. In most cases these will be
already available on your computer but if not you need to get some additional
software.

\begin{description}
  
\item[ZIP utility.]\index{zip}
  Rockbox is distributed as an archive using the
  \fname{.zip} format. Thus you need a tool to handle that compressed 
  format. Usually your computer should have a tool installed that can 
  handle the \fname{.zip} file format. Windows XP has builtin support for 
  \fname{.zip} files and presents them to you as folders unless you have 
  installed a third party program that handles compressed files. For
  other operating systems this may vary. If the \fname{.zip} file format 
  is not recognized on your computer you can find a program to handle them 
  at \url{http://www.info-zip.org/} or \url{http://sevenzip.sf.net/} which 
  can be downloaded and used free of charge.
  
\item[USB connection.]  To transfer Rockbox to your \dap{} you need to 
  connect it to your computer. To proceed you need to know where to access the 
  \dap{}. On Windows this means you need to figure out the drive letter 
  associated with the device. On Linux you need to know the mount point of 
  your \dap{}.
  
  \opt{ipod}{
    \note{Your \dap{} should enter disk mode automatically when connected to a 
      computer via USB. If your computer does not recognize your \dap{}, you 
      may need to enter the disk mode manually. Disconnect your \dap{} from the
      computer. Reset the \dap{} by pressing and holding the \ButtonMenu{} and 
      \ButtonSelect{} buttons simultaneously. As soon as the \dap{} resets, 
      press and hold the \ButtonMenu{} and \ButtonPlay{} buttons 
      simultaneously. Your \dap{} should enter disk mode, and you can try
      reconnecting to the computer.
    }
  }

\item[Text editor.]  As you will see in the following chapters, Rockbox is 
  highly configurable. In addition to saving configurations within Rockbox, 
  Rockbox also allows you to create customized configuration files. If you 
  would like to edit custom configuration files on your computer, you will 
  need a text editor like Windows' ``Wordpad''.

\end{description}


\section{Installing Rockbox}\label{sec:installing_rockbox}
\index{Installation}
\opt{MASCODEC}{
  \subsection{Using the windows installer}
  Using the Windows self installing executable to install Rockbox is the 
  easiest method of installing the software on your \dap{}. Simply follow the
  on-screen instructions and select the appropriate drive letter and 
  \dap{}-model when prompted.  You can use ``Add / Remove Programs'' to 
  uninstall the software at a later date.
  
  \subsection{Manual installation}
  For non{}-Windows users and those wishing to install manually from the 
  archive the procedure is still fairly simple.
}

\opt{SWCODEC}{
  \subsection{Introduction}
  
  There are two separate components of Rockbox that need to be installed in 
  order to run Rockbox.
  
  \begin{description}
    
  \item[The Rockbox bootloader.] \index{Bootloader}
    The bootloader is the program that tells your 
    \dap{} how to boot and load other components of Rockbox. This is the 
    component of Rockbox that is installed to the flash memory of your 
    \playerman. 
  
  \item[The Rockbox firmware.] Unlike the \playerman\ firmware, which runs 
    entirely from flash memory, most of the Rockbox code is contained in a 
    ``build'' that resides on your \daps{} hard drive. This makes it easy to 
    update Rockbox. The build consists of a file named \firmwarefilename\ and a
    directory called \fname{.rockbox}, both of which are located in the root 
    directory of your hard drive.
    
  \end{description}
  
  \subsection{Installing the bootloader}
  \opt{h1xx,h300}{% $Id$ %
\subsection{Installing the bootloader}
  Installing the bootloader is the trickiest part of the installation.
  The Rockbox bootloader allows users to boot into either the Rockbox 
  firmware or the iriver firmware. For legal reasons, we cannot distribute 
  the bootloader. Instead, we have developed a program that will patch the 
  Iriver firmware with the Rockbox bootloader. These instructions will explain 
  how to download and patch the Iriver firmware with the Rockbox bootloader 
  and install it on your jukebox.

\begin{enumerate}
  \item Download a supported version of the Iriver firmware for your 
  \playername\ from the Iriver website or from 
  \wikilink{ManualRockboxInstall}.
  Supported Iriver firmware versions currently include 
  \opt{IRIVER_H100_PAD}{1.63US, 1.63EU, 1.63K, 1.65US, 1.65EU, 1.65K, 1.66US, 
    1.66EU and 1.66K.  Note that the H140 uses the same firmware as the H120;
    H120 and H140 owners should use the	firmware called \fname{ihp\_120.hex}.
    Likewise, the iHP110 and iHP115 use the same firmware, called 
    \fname{ihp\_100.hex}.   Be sure to use the correct firmware file for 
    your player.}
  \opt{IRIVER_H300_PAD}{1.28K, 1.28EU, 1.28J, 1.29K, 1.29J and 1.30EU.
    \note{The US H3xx firmware is not currently supported and cannot be
    patched to be used with the bootloader. If you wish to install Rockbox
    on a US \playername\, you must use an international firmware, which will
    permanently remove DRM support from the player.}
  }
  If the file that you downloaded is a \fname{.zip} file, use an unzip 
  utility such as \fname{InfoZip}, \fname{7zip}, \fname{WinRAR},	or 
  \fname{WinZip} to extract the \fname{.hex} from the \fname{.zip} file
  to your desktop. Likewise, if the file that you downloaded is an 
  \fname{.exe} file, double-click on the \fname{.exe}	file to extract 
  the \fname{.hex} file to your desktop.
  %
  \item Download the firmware patcher \fname{fwpatcher.exe} from 
  \url{http://download.rockbox.org/bootloader/iriver/} and save it to your desktop.
    \warn{The firmware patcher contains Unicode support, which is not supported by 
    all versions of Windows. If you have difficulty with the firmware patcher, try 
    downloading the alternate firmware patcher \fname{fwpatchernu.exe}, which is 
    built without Unicode support.}
  %
  \item Go to your desktop and double-click on whichever version of the firmware 
  patcher you downloaded in the prior step.
  %
  \item In the firmware patcher dialog box, click on the BROWSE button and navigate
  to the \fname{.hex} file that you previously downloaded to your desktop.
  %
  \item Click PATCH. The firmware patcher will patch the original firmware to 
  include the Rockbox bootloader. The \fname{.hex} file on your desktop is now
  a modified version of the original \fname{.hex} file.
  %
  \item Turn on your \playername\ and connect it to your computer via USB.
  %
  \item Copy or move the modified \fname{.hex} file to the ROOT directory of 
    your jukebox.
  %
  \item Disconnect the jukebox from USB. (Be sure to use Windows' ``safely remove
  hardware'' option.)
  \warn{Before proceeding further, make sure that your player has a full charge, 
    or that it is connected to the power adaptor.}
  %
  \item Update your \playername s firmware with the patched bootloader. To do this, turn 
    the jukebox on. Press and hold the 
    \opt{IRIVER_H100_PAD}{\ButtonSelect{} button }%
    \opt{IRIVER_H300_PAD}{\ButtonSelect{} button }%
    to enter the main menu, and navigate to \setting{General $\rightarrow$ Firmware 
    Upgrade}. Select \setting{Yes} when asked to confirm if you want to upgrade the 
    firmware. The \playername{} will display a message indicating that the
    firmware update 
    is in progress. Do not interrupt this process. When the firmware update is 
    complete, the player will turn itself off. (The update firmware process usually 
    takes a minute or so.)

    You have now installed the Rockbox bootloader. 

\opt{h1xx}{\note{If you install the Rockbox bootloader, but do not install the
  Rockbox firmware, the Rockbox bootloader will load the iriver firmware when the
  jukebox is turned on.}}

\end{enumerate}
}
  \opt{ipod}{% $Id$ %
\subsection{Installing the bootloader}
\warn{These instructions are preliminary and may contain errors! 
Please check the wiki for up-to-date and improved installation instructions!
If you find errors you're of course welcomed to report them so we can fix it
for the next daily builds.}

  Installing the bootloader is the trickiest part of the installation.
  The process is different depending on your operating system, but before
  starting, connect the \dap{} to the computer using either an USB \fixme{or
  Firewire?} cable. Next, create a folder on the computer's harddrive and
  download the following file to that folder:
  \opt{ipodvideo}{\wikilink{IpodInstallation/bootloader-video.bin}}
  \opt{ipodnano}{\wikilink{IpodInstallation/bootloader-nano.bin}}
  \opt{ipodmini}{\wikilink{IpodInstallation/bootloader-mini1g.bin} or 
    \wikilink{IpodInstallation/bootloader-mini1g.bin} depending on which
    generation your \dap{} is.\fixme{Describe how to identify 1/2G}}
  \opt{ipodcolor}{\wikilink{IpodInstallation/bootloader-color.bin}}
  \opt{ipod4g}{\wikilink{IpodInstallation/bootloader-4g.bin}}

  When that is done, proceed to the section below that matches the operating
  system on the computer.
  \note{These instructions all require you to have administrator rights
  on your computer, regardless of the operating system.}
  \note{Rockbox only works on FAT32 partitions (called ``Windows formatted'' by
    Apple). So if your \dap{} is Mac formatted (HFS+), you should first convert
    it to FAT32. Information on how to do this can be found on the Rockbox
    website. \fixme{Include these instructions?}}

\subsubsection{Windows users}
\begin{enumerate}
  \item Download the following two programs and save them in the folder just
    created. These programs will be used in the next steps:
    \begin{itemize}
      \item \wikilink{IpodInstallation/ipodpatcher.exe}
      \item \wikilink{IpodInstallation/ipod_fw.exe}
    \end{itemize}
  \item Locate the \dap{} by opening a command windows. You can do this by
    clicking ``Start'', ``Execute'' and typing \fname{cmd}. Press Enter to
    execute that command. Now change directory to the
    folder you created and run the following commands:
    \begin{code}
    ipodpatcher 0
    ipodpatcher 1
    ipodpatcher 2
    ipodpatcher 3
    \end{code}
    Keep increasing the number until the \dap{} is located. 

    Output for an unsuccessful attempt to contact the \dap{}...
    \begin{code}
    C:/rockbox>ipodpatcher 0
    ipodpatcher v0.3 - (C) Dave Chapman 2006
    This is free software; see the source for copying conditions.  There is NO
    warranty; not even for MERCHANTABILITY or FITNESS FOR A PARTICULAR PURPOSE.

    [INFO] Reading partition table from \textbackslash\textbackslash{}.\textbackslash{}PhysicalDrive0
    Drive is not an iPod, aborting
    \end{code}
    
    A successful connection to the \dap{} will look similar to this...
    \begin{code}
    C:\textbackslash{}rockbox>ipodpatcher 6
    ipodpatcher v0.3 - (C) Dave Chapman 2006
    This is free software; see the source for copying conditions.  There is NO
    warranty; not even for MERCHANTABILITY or FITNESS FOR A PARTICULAR PURPOSE.

    [INFO] Reading partition table from \textbackslash\textbackslash{}.\textbackslash{}PhysicalDrive6
    Part    Start Sector    End Sector    Size (MB)  Type
       0              63        160649        78.4   Empty (0x00)
       1          160650       7984304      3820.1   W95 FAT32 (0x0b)
   \end{code}
    Remember the number that corresponds to your \dap{} -- in the 
    following steps, \emph{N} should be replaced with the number just found.
  \item Now, extract the firmware partition currently on the \dap{} with the
    following command:
    \begin{code}
    ipodpatcher -r \emph{N} bootpartition.bin
    \end{code}
    \note{You should keep a safe backup of this \fname{bootpartition.bin} file
      for use if you ever wish to either upgrade the Rockbox bootloader or
      uninstall Rockbox from your Ipod}
  \item Extract the Apple firmware from the partition image image just created:
    \begin{code}
    ipod_fw -o apple_os.bin -e 0 bootpartition.bin
    \end{code}
\optv{ipodvideo}{
  \item Similarly, extract the Broadcom firmware:
    \begin{code}
    ipod_fw -o apple_sw_5g_rcsc.bin -e 1 bootpartition.bin
    \end{code}
}
  \item Merge the Rockbox bootloader you downloaded previously with the Apple
    firmware:
\optv{ipodnano}{
    \begin{code}
    ipod_fw -g nano -o rockboot.bin -i apple_os.bin bootloader-nano.bin
    \end{code}
}
\optv{ipodvideo}{
    \begin{code}
    ipod_fw -g video -o rockboot.bin -i apple_os.bin bootloader-video.bin
    \end{code}
}
\optv{ipodmini}{
    \begin{code}
    ipod_fw -g mini -o rockboot.bin -i apple_os.bin bootloader-mini1g.bin
    \end{code}
    Or, if you have a 2G mini:
    \begin{code}
    ipod_fw -g mini -o rockboot.bin -i apple_os.bin bootloader-mini2g.bin
    \end{code}
}
\optv{ipodcolor}{
    \begin{code}
    ipod_fw -g color -o rockboot.bin -i apple_os.bin bootloader-color.bin
    \end{code}
}
\optv{ipod4g}{
    \begin{code}
    ipod_fw -g 4g -o rockboot.bin -i apple_os.bin bootloader-4g.bin
    \end{code}
}
\item
    Install the Rockbox-enabled firmware:
    \begin{code}
    ipodpatcher -w \emph{N} rockboot.bin
    \end{code}
\end{enumerate}

Now you can proceed installing the firmware itself.

\subsubsection{Mac OS X users}
\begin{enumerate}
  \item Download the following two programs and save them in the folder just
    created. These programs will be used in the next steps:
    \begin{itemize}
      \item \wikilink{IpodInstallationFromMacOSX/diskdump}
      \item \wikilink{IpodInstallationFromMacOSX/ipod_fw}
    \end{itemize}
    Start a Terminal and type navigate into the folder you created. Before
    you can continue, you need to ensure that Mac OS knows that the
    \fname{ipod\_fw}
    and diskdump files you downloaded are executable programs. To do this,
    type the following command:
    \begin{code}
    chmod +x ipod_fw diskdump
    \end{code}
  \item Locate the \dap{} by running the following command:
    \begin{code}
    mount
    \end{code}
    The output will look something like this: \fixme{Add full example}
    \begin{code}
    /dev/disk1s2 on /Volumes/DAVE_S IPOD 1 (local, nodev, nosuid)
    \end{code}
    In this example, the \dap\ is located at /dev/disk1s2 Remember the 
    location of your \dap\  -- in the following steps, /dev/disk1s2 should be
    replaced with the location just found.
  \item Before continuing, the \dap\ must be ``unmounted'', which is
    done with the following command:
    \begin{code}
    diskutil unmount /dev/disk1s2
    \end{code}
  \item Now, extract the Apple firmware currently on the \dap{} with the
    following command:
    \note{The last part of the location is left out.}
    \begin{code}
    ./diskdump -r /dev/disk1 bootpartition.bin
    \end{code}
    \note{You should keep a safe backup of this \fname{bootpartition.bin} file
      for use if you ever wish to either upgrade the Rockbox bootloader or
      uninstall Rockbox from your iPod
    }
  \item Extract the Apple firmware from this partition image:
    \begin{code}
    ./ipod_fw -o apple_os.bin -e 0 bootpartition.bin
    \end{code}
\optv{ipodvideo}{
  \item Similarly, extract the Broadcom firmware:
    \begin{code}
    ./ipod_fw -o apple_sw_5g_rcsc.bin -e 1 bootpartition.bin
    \end{code}
}
  \item Merge the Rockbox bootloader you downloaded previously with the Apple
    firmware:
\optv{ipodnano}{
    \begin{code}
    ./ipod_fw -g nano -o rockboot.bin -i apple_os.bin bootloader-nano.bin
    \end{code}
}
\optv{ipodvideo}{
    \begin{code}
    ./ipod_fw -g video -o rockboot.bin -i apple_os.bin bootloader-video.bin
    \end{code}
}
\optv{ipodmini}{
    \begin{code}
    ./ipod_fw -g mini -o rockboot.bin -i apple_os.bin bootloader-mini1g.bin
    \end{code}
    Or, if you have a 2G Mini:
    \begin{code}
    ./ipod_fw -g mini -o rockboot.bin -i apple_os.bin bootloader-mini2g.bin
    \end{code}
}
\optv{ipodcolor}{
    \begin{code}
    ./ipod_fw -g color -o rockboot.bin -i apple_os.bin bootloader-color.bin
    \end{code}
}
\optv{ipod4g}{
    \begin{code}
    ./ipod_fw -g 4g -o rockboot.bin -i apple_os.bin bootloader-4g.bin
    \end{code}
}
  \item
    Install the Rockbox-enabled firmware:
    \note{The last part of the location is left out.}
    \begin{code}
    ./diskdump -w /dev/disk1 rockboot.bin
    \end{code}
\end{enumerate}

Now, proceed with installing the firmware itself.

\subsubsection{Linux users}
\begin{enumerate}
  \item Download the following and save it in the folder just
    created:
    \begin{itemize}
      \item \url{http://www.rockbox.org/viewcvs.cgi/*checkout*/tools/ipod_fw.c}
    \end{itemize}
    Now compile it to an executable by opening a command prompt and changing
    to the folder created previously. Thn run the following command:
    \begin{code}
    gcc -o ipod_fw ipod_fw.c
    \end{code}
    If you get the message that the command gcc is not found, you need to
    install gcc. How to do this depends on your Linux distribution, and
    you should consult its documentation for help on this.
  \item Locate your Ipod by running the command \verb|dmesg|. In the output
    something like the following should be seen:
\begin{code}
    usb 4-1: new high speed USB device using ehci_hcd and address 7
    scsi4 : SCSI emulation for USB Mass Storage devices
    usb-storage: device found at 7
    usb-storage: waiting for device to settle before scanning
      Vendor: Apple     Model: iPod              Rev: 1.62
      Type:   Direct-Access                      ANSI SCSI revision: 00
    SCSI device sdb: 58605120 512-byte hdwr sectors (30006 MB)
\end{code}
    You need the device name of your \dap, which you can find in the last line.
    In this example, the \dap\ is located on \fname{/dev/sdb}. In the following,
    \fname{/dev/sdb} should be replaced with the location just found.
  \item Run \verb|fdisk -l /dev/sdb|. Verify that the
    output is similar to the one below:
    \begin{code}
       Device Boot      Start         End      Blocks   Id  System
    /dev/sdb1               1          10       80293+   0  Empty
    /dev/sdb2              11        3648    29222235    b  W95 FAT32
    \end{code}
  \item Back up the partition table using the following command:
    \note{The last part of the location is left out.}
    \begin{code}
    dd if=/dev/\emph{sdb} of=mbr.bin count=1
    \end{code}

  \item Now, extract the firmware partition currently on the \dap{} with the
    following command:
    \begin{code}
    dd if=/dev/\emph{sdb1} of=bootpartition.bin
    \end{code}
    \note{You should keep a safe backup of this \fname{bootpartition.bin} file
      for use if you ever wish to either upgrade the Rockbox bootloader or
      uninstall Rockbox from your Ipod
    }
  \item Extract the Apple firmware from this partition image:
    \begin{code}
    ./ipod_fw -o apple_os.bin -e 0 bootpartition.bin
    \end{code}
\optv{ipodvideo}{
  \item Similarly, extract the Broadcom firmware: 
    \begin{code}
    ./ipod_fw -o apple_sw_5g_rcsc.bin -e 1 bootpartition.bin
    \end{code}
}

  \item Merge the Rockbox bootloader you downloaded previously with the Apple
    firmware: 
\optv{ipodnano}{
    \begin{code}
    ./ipod_fw -g nano -o rockboot.bin -i apple_os.bin bootloader-nano.bin
    \end{code}
}
\optv{ipodvideo}{
    \begin{code}
    ./ipod_fw -g video -o rockboot.bin -i apple_os.bin bootloader-video.bin
    \end{code}
}
\optv{ipodmini}{
    \begin{code}
    ./ipod_fw -g mini -o rockboot.bin -i apple_os.bin bootloader-mini1g.bin
    \end{code}
    Or, if you have a 2G Mini:
    \begin{code}
    ./ipod_fw -g mini -o rockboot.bin -i apple_os.bin bootloader-mini2g.bin
    \end{code}
}
\optv{ipodcolor}{
    \begin{code}
    ./ipod_fw -g color -o rockboot.bin -i apple_os.bin bootloader-color.bin
    \end{code}
}
\optv{ipod4g}{
    \begin{code}
    ./ipod_fw -g 4g -o rockboot.bin -i apple_os.bin bootloader-4g.bin
    \end{code}
}
  \item
    Install the Rockbox-enabled firmware:
    \begin{code}
    dd if=rockboot.bin of=/dev/\emph{sdb1}
    \end{code}
\end{enumerate}
Now you can install the firmware itself.

}
  \opt{x5}{\fixme{This is merely a copy of the wiki page IaudioBoot, so this section needs
a more natural language and also error checking by Iaudio owners.}

The \playername{} has a builtin boot loader which performs the
firmware update, and can also access the hard drive via USB. Therefore the
Rockbox bootloader can be very minimalistic, without USB mode.
This also makes it less dangerous to install the Rockbox bootloader, as you can
always restore it using the \playerman{} bootloader.

\note{The current bootloader is not prepared to coexist with the original
firmware. It replaces the original firmware.}

\subsubsection{Installation}
\begin{itemize}
\item Download the Rockbox bootloader binary from 
\url{http://download.rockbox.org/bootloader/iaudio/}.
  \opt{x5}{Use the \fname{x5v\_fw.bin} file if your \dap{} is a X5V. If it is a X5,
    use the \fname{x5\_fw.bin} file.}
  \opt{m5}{Use the \fname{m5\_fw.bin} file.}
\item Copy it to the \fname{FIRMWARE} directory on your \dap{}.
\item Turn the \dap{} off, remove the USB cable and insert the charger. The
Rockbox bootloader will automatically be flashed.
\end{itemize}
}
  \opt{h10,h10_5gb}{\fixme{To do:  Complete this section H10 platforms.}}

  \subsection{Installing the firmware} 
  After installing the bootloader, the installation becomes fairly easy. 
} 
      
There are three different types of firmware binaries from Rockbox website:
\label{Version} 
current version, daily build and CVS build. You need to decide which one
you want to install and get the version for your \dap{}.

\begin{description}

\item[Current Version.] The current version is the latest stable release, free 
  of known critical bugs.  The current stable release of Rockbox, version 2.5, 
  is available at \url{http://www.rockbox.org/download/}.
  \opt{SWCODEC}{ 
    \note{The current stable release is available only for Archos jukeboxes. 
      There has not yet been a stable release for the \playername{}.  Until 
      there is a stable release for \playername{}, use a daily build or CVS 
      build.
    }
  }
  
\item[Daily Build.] The daily build is a development version of Rockbox. It
  contains features and patches developed since last stable version.  It 
  may also contain bugs! This daily build is generated automatically every day 
  and can be found at \url{http://www.rockbox.org/daily.shtml}.
  
\item[CVS Build (formerly, ``Bleeding Edge Build.'')] CVS stands for 
  ``Concurrent Versions System.'' CVS is the system that Rockbox 
  developers use to keep track of changes to the Rockbox source code. CVS 
  builds are made automatically every time there is a change to the 
  Rockbox source. These builds are for people who want to test the code 
  that developers just checked in. 
  
\end{description}

\nopt{player}{
  \note{\index{Installation!Fonts}
    Rockbox has a fonts package that is available at 
    \url{http://www.rockbox.org/daily.shtml}.  While the daily builds and CVS
    builds change frequently, the fonts package rarely changes.  Thus, the 
    fonts package is not included in the daily builds and CVS builds. (The 
    stable release, on the other hand, does not change, so fonts are 
    included with the stable release.)  When installing Rockbox for the 
    first time, you should install the  fonts package. 
  }
}

Because daily builds and CVS builds are development versions which change 
frequently, they may behave differently than described in this manual, or 
they may introduce new (and maybe annoying) bugs. If you do not want to get 
undefined behaviour from your \dap\ you should really stick to the current 
stable release, if there is one for your \dap{}. If you want to help the 
project development, you can try development builds and help by reporting 
bugs. Just be aware that these are development builds that are  highly 
functional, but not perfect!

After downloading the Rockbox package connect your \dap{} to the computer via 
USB as described in the manual that came with your \dap{}. Take the file that 
you downloaded above, and extract its contents to your \daps{} drive.

Use the ``Extract all'' command of your unzip program to extract the files in 
the \fname{.zip} file onto your \dap{}. Note that the entire contents of the 
\fname{/zip} file should be extracted directly to the root of your \daps{} 
drive.  Do not try to create a separate directory or folder on your \dap{} for 
the Rockbox files!  The \fname{.zip} file already contains the internal 
directory structure that Rockbox needs. 

\note{
  If the contents of the \fname{.zip} file are extracted correctly, you   will 
  have a file called \fname{\firmwarefilename} in the main folder of  your 
  \daps{} drive, and also a folder called /\fname{.rockbox}, which contains a 
  number of other folders and system files needed by Rockbox. If you receive a 
  ``-1'' error when you start Rockbox, you have not extracted the contents of 
  the \fname{.zip} file to the proper location. 
}

\section{Enabling Speech Support (optional)}\label{sec:enabling_speech_support}
\index{Speech}\index{Installation!Optional Steps}
If you wish to use speech support you will also need a language file, available
from \wikilink{VoiceFiles}. For the English language, the file is called
\fname{english.voice}. When it has been downloaded, unpack this file and copy 
it into the \fname{lang} folder which is inside the \fname{/.rockbox} folder on
your \dap{}. Voice menus are turned on by default. See
\reference{ref:Voiceconfiguration} for details on voice settings.

\section{Running Rockbox} 
Remove your \dap{} from the computer's USB port.%
\nopt{ipod}{Unplug any connected power supply and turn the unit off. When
you next turn the unit on, Rockbox should load.}%
\opt{ipod}{Rebooting the Ipod by holding
  \opt{IPOD_4G_PAD}{\ButtonMenu{}+\ButtonSelect{}}%
  \opt{IPOD_3G_PAD}{\ButtonMenu{}+\ButtonPlay{}}
  for a couple of seconds until the \dap{} reboots. Now Rockbox should load.
}%
When you see the Rockbox splash screen, Rockbox is loaded and ready for
use.

\opt{ipod}{
  \note{
    Rockbox starts in the \setting{File Browser}. If you have loaded music onto
    your player using Itunes, you will not be able to see your music because
    Itunes changes your files' names and hides them in directories in the 
    \fname{Ipod\_Control} folder. You can view files placed on your \dap{} by 
    Itunes by initializing and using Rockbox's Tag Cache. See 
    \reference{ref:tagcache} for more information. 
  } 
}

\section{Updating Rockbox} Updating Rockbox is easy. Download a Rockbox build.
(The latest release of the Rockbox software will always be available from 
\url{http://www.rockbox.org/download/}). Unzip the build to the root directory 
of your \dap{} like you did in the installation step before. If your unzip
program asks you whether to overwrite files, choose the ``Yes to all'' option.
The new build will be installed over your current build.

\note{
  Settings are stored on an otherwise-unused sector of your hard disk, not in 
  any of the files contained in the Rockbox build. Therefore, generally 
  speaking, installing a new build does \emph{not} reset Rockbox to its default
  settings. Be aware, however, that from time to time, a change is made to the 
  Rockbox source code that \emph{does} cause settings to be reset to their 
  defaults when a Rockbox build is updated. Thus it is recommended to save your
  settings using the \setting{Manage Settings} $\rightarrow$ 
  \setting{Write .cfg file} function before updating your Rockbox build so that
  you can easily restore the settings if necessary. For additional information 
  on how to save, load, and reset Rockbox's settings, see 
  \reference{ref:SystemOptions}.
}

\section{Uninstalling Rockbox}\index{Installation!uninstall}

If you would like to go back to using the original \playerman{} software, then
connect the \playerman{} to your computer, and delete the
\fname{\firmwarefilename} file. 

\optv{ipod}{
  Next, open a command window (Windows) or a terminal window (Mac or Linux). 
  Navigate to the folder you created when you downloaded the 
  \fname{ipodpatcher} program you used to install the Rockbox bootloader. 
  Type the following command:
        
  \begin{code}
    ipodpatcher -w \emph{N} bootpartition.bin
  \end{code}
  
  Remember that \emph{N} is the number that you found when you installed 
  Rockbox on your \playerman{}.
}
        
If you wish to clean up your disk, you may also wish to delete the 
\fname{.rockbox} folder and its contents. Turn the \playerman{} off.

\opt{h300}{Press and hold the \ButtonRec{} button.}

Turn the \dap{} back on and the original \playerman{} software will load.

\opt{h1xx}{ 
  \note{
    There's no need to remove the installed bootloader. If you want to remove 
    it, simply flash an unpatched \playerman{} firmware. Be aware that doing so
    will also remove the bootloader USB mode. As that mode can come in quite 
    handy (especially when having disk errors) it is recommended to keep the 
    bootloader. It also gives you the possibility of trying Rockbox anytime 
    later by simply installing the distribution files. 
  }
}

\opt{h300}{ 
  \note{
    There's no need to remove the installed bootloader, although you if you 
    retain the Rockbox bootloader, you will need to hold the \ButtonRec{} 
    button each time you want to start the original firmware. If you want to 
    remove it simply flash an unpatched \playerman{} firmware. Be aware that 
    doing so will also remove the bootloader USB mode. As that mode can come in
    quite handy (especially when having disk errors), you may wish to keep the 
    bootloader. It also gives you the possibility of trying Rockbox anytime 
    later by simply installing a new build.
  }
}

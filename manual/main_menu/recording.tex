% $Id$ %
\section{\label{ref:Recording}Recording}
\subsection{\label{ref:Whilerecordingscreen}While Recording Screen}
\screenshot{main_menu/images/ss-while-recording-screen}{The while recording
  screen}{}

Entering the \setting{Recording} option in the \setting{Main Menu} brings up
a screen in wich you can choose to enter the \setting{Recording Screen} or
the \setting{Recording Settings} (see below). The \setting{Recording Screen}
shows the time elapsed and the size of the file being recorded. A peak meter
is present to allow you set Gain correctly. There is also a volume setting,
this will only affect the output level of the \dap{} and does \emph{not}
affect the recorded sound.
\note{When you start a recording, the hard disk will spin up. This will cause
the peak meters to freeze in the process. This is expected behaviour, and
nothing to worry about. Recording is continiued during the spin up.}
\opt{MASCODEC}{The frequency, channels and quality}
\opt{SWCODEC}{The frequency and channels} settings are shown on the last line.

The controls for this screen are:

\begin{table}
  \begin{btnmap}{}{}
    
    \ButtonUp\ / \ButtonDown & Select setting.\\
    %
    \ButtonLeft\ / \ButtonRight & Adjust selected setting.\\
    %
    \opt{RECORDER_PAD,IAUDIO_X5_PAD}{\ButtonPlay}
    \opt{ONDIO_PAD}{\ButtonMenu}
    \opt{IRIVER_H100_PAD,IRIVER_H300_PAD}{\ButtonOn}
    \opt{IPOD_4G_PAD,IPOD_3G_PAD}{\fixme{FixMe}}
    & Start recording.\\
    & While recording: pause recording (press again to continue).\\
    %
    \opt{RECORDER_PAD,ONDIO_PAD,IRIVER_H100_PAD,IRIVER_H300_PAD}{\ButtonOff}
    \opt{IAUDIO_X5_PAD}{Hold \ButtonPlay}
    \opt{IPOD_4G_PAD,IPOD_3G_PAD}{\fixme{FixMe}}
    & Exit \setting{Recording Screen}.\\
    & While recording: Stop recording.\\
    %
    \opt{IRIVER_H100_PAD,IRIVER_H300_PAD,IAUDIO_X5_PAD}{
      \ButtonRec & Starts recording.\\
      & While recording: close the current file and open a new one.\\
    }
    %
    \opt{RECORDER_PAD}{\ButtonFOne}
    \opt{ONDIO_PAD}{Hold \ButtonMenu}
    \opt{IRIVER_H100_PAD,IRIVER_H300_PAD}{\ButtonMode}
    \opt{IAUDIO_X5_PAD}{Hold \ButtonRec}
    \opt{IPOD_4G_PAD,IPOD_3G_PAD}{Hold \ButtonSelect} 
    & Open \setting{Recording Settings} (see below).\\
    %
    \opt{RECORDER_PAD}{
      \ButtonFTwo & Quick menu for recording settings. A quick press will
      leave the screen up (press \ButtonFTwo\ again to exit), while holding
      it will close the screen when you release it.\\
    }
    %
    \opt{RECORDER_PAD}{
      \ButtonFThree & Quick menu for source setting.\\
      & Quick/hold works as for \ButtonFTwo.\\
      & While recording: Start a new recording file.\\
    }
  \end{btnmap}
\end{table}

\subsection{\label{ref:Recordingsettings}Recording Settings}
\screenshot{main_menu/images/ss-recording-settings}{The recording settings screen}{}

\opt{MASCODEC}{
  \begin{description}
  \item[Quality:]
    Choose the quality here (0 to 7). Default is 5, best quality is 7,
    smallest file size is 0. This setting effects how much your sound
    sample will be compressed. Higher quality settings result in larger
    MP3 files.
    
    The quality setting is just a way of selecting an average bit rate,
    or number of  bits per second, for a recording.  When  this setting
    is lowered, recordings are compressed more (meaning worse sound quality),
    and the average bitrate changes as follows.
  \end{description}
  
  \begin{table}[h!]
    \begin{center}
      \begin{tabularx}{0.75\textwidth}{lX}\toprule
        \emph{Frequency} & \emph{Bitrate} (Kbit/s) -- quality 0$\rightarrow$7 \\\midrule
        44100Hz stereo        & 75, 80, 90, 100, 120, 140, 160, 170 \\
        22050Hz stereo        & 39, 41, 45, 50,  60,  80,  110, 130 \\
        44100Hz mono          & 65, 68, 73, 80,  90,  105, 125, 140 \\
        22050Hz mono          & 35, 38, 40, 45,  50,  60,  75,  90 \\\bottomrule
      \end{tabularx}
    \end{center}
  \end{table}
}

\begin{description}
  \opt{MASCODEC,x5}{
  \item[Frequency:]
    Choose the recording frequency (sample rate) -- 48kHz, 44.1kHz, 32kHz and
    24kHz, 22.05kHz, 16kHz are available. Higher sample rates use up more disk
    space, but give better sound quality.
    This setting determines which frequency range can accurately be reproduced
    during playback, Lower frequencies produce smaller files.
    \opt{MASCODEC}{
      The frequency setting also determines which version of the MPEG standard
      the sound is recorded using:\\
      MPEG v1 for 48, 44.1 and 32\\
      MPEG v2 for 24, 22.05 and 16\\
    }
  }
\item[Source:]
  Choose the source of the recording. This can be
  \opt{recorder,recorderv2fm,h1xx}{SPDIF (digital),} microphone or line in.
  \opt{CONFIG_TUNER}{For recording from the radio see \reference{ref:FMradio}.}
  
  % Add h1xx here whenever it supports setting recording frequency.
  \opt{recorder,recorderv2fm}
      {\note{You cannot change the sample rate for digital recordings.}}
      
      \opt{MASCODEC,x5}{
      \item[Channels:]
        This allows you to select mono or stereo recording. Please note that
        for mono recording, only the left channel is recorded.  Mono recordings
        are usually somewhat smaller than stereo.
      }
      
      \opt{MASCODEC}{
      \item[Independent Frames:]
        The independent frames option tells the \dap\ to encode with the bit
        reservoir disabled, so the frames are independent of each other. This
        makes a file easier to edit.
      }
      
    \item[File Split:]
      This option is useful when timing recordings. If set to active it stops
      a recording at a given interval and then starts recording again with a
      new file, which is useful for long term recordings.
      \newline
      The splits are seamless (frame accurate), no audio is lost at the split
      point. The break between recordings is only the time required to stop
      and restart the recording, on the order of 2 -- 4 seconds.
      \newline
      Options (hours:minutes between splits): off, 24:00, 18:00, 12:00, 10:00,
      8:00, 6:00, 4:00, 2:00, 1:20 (80 minute CD), 1:14 (74 minute  CD), 1:00,
      00:30, 00:15, 00:10, 00:05.
      
    \item[Prerecord Time:]
      This setting buffers a small amount of audio so that when the record button
      is pressed, the recording will begin from that number of seconds earlier.
      This is useful for ensuring that a recording begins before a cue that is
      being waited for.\\
      Options: \setting{Off}, \setting{1 -- 30 seconds}
      
    \item[Directory:]
      Allows changing the location where the recorded files are saved. The
      default location is \fname{/recordings}.
      
    \item[Show recording screen on startup:]
      If set to yes, the \dap\ will start up with the while recording screen
      showing.\\
      Options: \setting{Yes}, \setting{No}
      
    \item[Clipping Light:]
      Causes the backlight to flash on when clipping has been detected.\\
      Options: \setting{Off}, \setting{Remote unit only},
      \setting{Main and remote unit}, \setting{Main unit only}.
      
      \opt{MASCODEC}{
      \item[Trigger:]
        \fixme{Add description of triggered recording.}
      }
      
\end{description}

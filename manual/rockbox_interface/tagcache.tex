% $Id$ %
\section{Tag Cache\label{ref:tagcache}} 

\subsection{Introduction}
Tag Cache is Rockbox's music database system. Using the information 
contained in the tags (ID3v1, ID3v2, Vorbis Comments, Apev2, etc.) in your 
audio files, Rockbox builds and maintains a database of the music files on 
your player and allows you to browse them by Artist, Album and Genre.

\subsection{Using Tag Cache}
Before you use the Tag Cache for the first time, you need to tell Rockbox to
create the database from your music files. This typically takes around 
5 -- 10 minutes and you can continue listening to music normally whilst 
this is happenning. You can even shut down your player and resume the 
database building process next time you turn it on.
To force the initial database to be built, go to the main menu, then 
\setting{General Settings $\rightarrow$ Playback $\rightarrow$ Force tag cache update}.

The Tag Cache database can either be kept on disk (to save memory), or 
loaded into RAM (for fast browsing). This configuration option can be 
found in \setting{General Settings $\rightarrow$ Playback $\rightarrow$ Tag cache}.
Once the database has been built, you can browse your music using this 
database. To use the database change the show mode by changing the 
\setting{General Settings $\rightarrow$ File view $\rightarrow$ Show files} option to 
\setting{ID3 database}. This disables the File Browser and instead of the file
browser you will now access the database.
%
\begin{description}
\item[Keeping Tag Cache on disk.] This does not require extra RAM for the
Tag Cache and speeds up the boot process, but is slower when accessing.
\item[Keeping Tag Cache on disk with Dircache enabled.] With Dircache
enabled updating of the database is much faster. Also updates can be
committed in background immediately.
\item[Loading Tag Cache to RAM.] The fastest mode is when loading Tag Cache
to RAM. New files are automatically detected and Tag Cache will get updated.
Use this for the fastest browsing and searching without any waiting.
\note{To use the RAM mode, first you have to turn on 
Directory Cache in \setting{General Settings $\rightarrow$ System $\rightarrow$ 
Disk $\rightarrow$ Directory Cache}.}
\end{description}

\note{You may need to increase the value of the \setting{Max files in dir browser}
setting (\setting{General Settings $\rightarrow$ System $\rightarrow$ Limits}) 
in order to view long lists of tracks in the ID3 database browser.\\
There is no option to turn off Tag Cache completely. If you don't want
to use it just don't do the initial build of the database and don't load it
to RAM.}
%
\begin{table}
\begin{center}
  \begin{tabularx}{.75\textwidth}{XX}
  \toprule
  Supported Tags   & Unsupported Tags \\ 
  \midrule
  Artist           & Comment \\
  Album            & Performer\\
  Bitrate          & \\
  Composer         & \\
  Genre            & \\
  Length           & \\
  Title            & \\
  Track Number     & \\
  \bottomrule
  \end{tabularx}
\end{center}
\end{table}


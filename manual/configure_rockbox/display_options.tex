% $Id$ %
\section{\label{ref:Displayoptions}Display}
  
  \begin{description}

  \item[LCD Settings.]
    This sub menu contains settings that relate to the display of the \dap.
    \begin{description}
    \opt{HAVE_BACKLIGHT}{
      \item[Backlight.]
      The amount of time the backlight shines after a key press. If set to
      \setting{Off}, the backlight will not light when a button is pressed. If
      set to \setting{On}, the backlight will never shut off. If set to a time
      (1 to 90 seconds), the backlight will stay lit for that amount of time
      after a button press.
      \item[Backlight (While Plugged In).]
      This setting is equivalent to the \setting{Backlight} setting except it
      applies when the \dap{} is plugged into the charger.
      \item[Backlight on Hold.]
      This setting controls the behavior of the backlight when the Hold switch
      is toggled. If set to \setting{Normal} the backlight will behave as usual.
      If set to \setting{Off} the backlight will be turned off immediately when
      the Hold switch is engaged and if set to \setting{On} the backlight will
      be turned on and stay on while the Hold switch is engaged.
      \item[Caption Backlight.]
      This option turns on the backlight a number of seconds before the start
      of a new track, and keeps it on for the same number of seconds after the
      beginning so that the display can be read to see song information. The
      amount of time is determined by the value of the backlight timeout
      setting, but is no less than 5 seconds.
      \opt{backlight_fade_int}{
        \item[Backlight Fade In.]
        The amount of time that the backlight will take to fade from off to on
        after a button is pressed. If set to \setting{Off} the backlight will
        turn on immediately, with no fade in. Can also be set to \setting{500ms},
        \setting{1s} or \setting{2s}.
        \item[Backlight Fade Out.]
        Like Backlight fade in, this controls the amount of time that the
        backlight will take to fade from on to off after a button is pressed. If
        set to \setting{Off} the backlight will turn off immediately, with no
        fade out. Other valid values: \setting{500ms}, \setting{1s},
        \setting{2s}, \setting{3s}, \setting{4s}, \setting{5s} or \setting{10s}.
      }
      \opt{backlight_fade_bool}{
        \item[Backlight Fade In.]
        This options turns on smooth backlight fading when the backlight is
        turning on. The fading time is dependent on the brightness level you
        have chosen. If it is turned off, the backlight will turn on immediately.
        \item[Backlight Fade Out.]
        This options turns on smooth backlight fading when the backlight is
        turning off. The fading time is dependent on the brightness level you
        have chosen. If it is turned off, the backlight will turn off immediately.
      }
      \item[First Keypress Enables Backlight Only.]
      With this option enabled the first keypress while the backlight is turned
      off will only turn the backlight on without having any other effect. When
      disabled the first keypress will \emph{also} perform its appropriate action.
      
      \opt{lcd_sleep}{
        \item[Sleep (After Backlight Off).]
        This setting controls how long rockbox will wait before turning off the
        display after the backlight is turned off. Turning off the display
        saves battery power but turning on the display takes noticeably longer 
        than just turning on the backlight.
      }
      
      \opt{backlight_brightness}{
        \item[Brightness.]
        Changes the brightness of your LCD display.
      }
    } % \opt{HAVE_BACKLIGHT}
    
    \opt{archos,h100,ipodmini,ipod3g,ipod4g,x5}{
      \item[Contrast.]
      Changes the contrast of your LCD display.
      \warn{Setting the contrast too dark or too light can make it hard to
        find this menu option again!}
    } % \opt{archos,h100,ipodmini,ipod4g,x5}

    \opt{lcd_invert}{
      \item[LCD Mode.]
      This setting lets you invert the colours of the display.
    }

    \opt{lcd_flip}{
      \item[Upside Down.]
      Displays the screen so that the top of the display is nearest the buttons.
      This is sometimes useful when carrying the \dap{} in a pocket for easy
      access to the headphone socket.
    }
    \end{description}
%
  \opt{HAVE_REMOTE_LCD}{
    \item[Remote-LCD Settings.]
    This sub menu contains settings that relate to the display of the remote.
      \begin{description}
      \item[Backlight.]
        Similar to the main unit backlight this option controls the backlight
        timeout for the remote control. The remote backlight is independent
        from the main unit backlight.
      \item[Backlight on When Plugged.]
        This controls the backlight when the \dap\ is plugged into the charger.
      \item[Caption Backlight.]
        This option turns on the backlight a number of seconds before the start
        of a new track, and keeps it on for the same number of seconds after the
        beginning so that the display can be read to see song information. The
        amount of time is determined by the value of the backlight timeout
        setting, but is no less than 5 seconds.
        \item[First Keypress Enables Backlight Only.]
          This controls what happens when you press a button on your remote
          while the backlight is turned off. Like for the main unit, if this
          setting is set to \setting{Yes}, the first keypress will light up the
          remote backlight, but have no other effect. If set to \setting{No},
          the first keypress will light up the remote backlight
          \emph{and} engage the function of the key that is pressed.
       \item[Contrast.]
         Changes the contrast of your remote's LCD display.
         \warn{Setting the contrast too dark or too light can make it hard to
           find this menu option again!}
       \item[LCD Mode.]
         This setting lets you invert the whole screen, so now you get a
         black background and light text and graphics.
       \item[Upside Down.]
         Displays the screen so that the top of the display is nearest 
         the buttons. This is sometimes useful when carrying the \dap\ in a
         pocket for easy access to the headphone socket.
       \opt{h100,h300}{
         \item[Reduce Ticking.]
           Enable this option if you can hear a ticking sound in your headphones
           when using your remote.
       }
    \end{description}
  }
%
  \item[Scrolling.]
    This feature controls how text will scroll in Rockbox. You can configure
    the following parameters:
    \begin{description}
    \item[Scroll Speed.]
      Sets how many times per second the automatic horizontal scrolling text 
      will move a step.
    \item[Scroll Start Delay.]
      Controls how many milliseconds Rockbox should wait before a new
      text begins automatically scrolling.
    \opt{lcd_bitmap}{
      \item[Scroll Step Size.]
        Defines the number of pixels the text should move for each step, as used
        by the Scroll Speed setting.
    }
    \opt{HAVE_REMOTE_LCD}{
      \item[Remote Scrolling Options.]
        The options here have the same effect on the remote LCD as the options
        mentioned above have on the main LCD.
    }
    \item[Bidirectional Scroll Limit.]
      Rockbox has two different automatic horizontal scrolling methods: 1) always
      scrolling the text to the left until the line has ended and then beginning
      again at the start, and 2) moving to the left until you can read the end of
      the line and then scrolling right until you see the beginning again.
      Rockbox chooses which method it should use depending of how much it has to
      scroll to the left. This setting lets you tell Rockbox where that limit
      is, expressed in percentage of the line length.
    \opt{lcd_bitmap}{
      \item[Screen Scrolls Out of View.]
        Screens can be manually scrolled horizontally by pressing
        \opt{recorder,recorderv2fm}{\ButtonOn+\ButtonRight/\ButtonLeft}
        \opt{ondio}{\ButtonMenu+\ButtonRight/\ButtonLeft}
        \opt{c200,e200,h100,h300,ipod1g2g,ipod3g,ipod4g,ipodcolor,ipodmini,ipodnano,
        ipodnano2g,ipodvideo,x5,mrobe100,gigabeatf,gigabeats,e200v2}
        {Long \ButtonRight/\ButtonLeft}\opt{h10,h10_5gb}{Long \ButtonRew/\ButtonFF}.
        Setting this option to \setting{Yes}
        will keep the list entries at their fixed positions and allow them to be
        scrolled out of view, whereas \setting{No} will only scroll those entries
        which surpass the right margin.
      \item[Screen Scroll Step Size.]
        Defines the number of pixels the horizontal manual screen scroll should move
        for each step.
    }
    \opt{player}{
      \item[Jump Scroll.]
        This setting makes text scroll a page at a time instead of a character
        at a time. If set to \setting{One time}, \setting{2}, \setting{3} or
        \setting{4} it will scroll a line in paged mode that many times and
        then scroll it a character at a time. If set to \setting{Always} lines
        will always scroll in paged mode.
      \item[Jump Scroll Delay.]
        Controls how long the delay is before a page is scrolled.
    }
    \item[Paged Scrolling.]
      When set to \setting{Yes} scrolling vertically on pages that surpass the
      screen size will page up/down instead of simply changing lines. This can be
      useful on slow displays.
    \end{description}
%
  \opt{lcd_bitmap}{
    \item[Peak Meter.]
      The peak meter can be configured with a number of parameters. 
      \begin{description}
      \item[Peak Release.]
        This determines how fast the bar shrinks when the music becomes
        softer. Lower values make the peak meter look smoother.
        Expressed in scale units per 10ms.
      \item[Peak Hold Time.]
        Specifies the time after which the peak indicator will reset.
        For example, if you set this value to 5s, the peak indicator displays
        the loudest volume value that occurred within the last 5 seconds.
        Larger values are useful if you want to find the peak level of a song,
        which might be of interest when copying music from the \dap\ via the
        analogue output to some other recording device.
      \item[Clip Hold Time.]
        The number of seconds that the clipping indicator will be visible
        after clipping is detected.
      \opt{recording}{  
        \item[Clip Counter.]
          Show the number of times the clip indicator went active during
          recording in front of the peak meters.
      }
      \item[\label{ref:Peakmetersetting}Scale.]
        Select whether the peak meter displays linear or logarithmic values.
        The human ear perceives loudness on a logarithmic scale. If the Scale
        setting is set to \setting{Logarithmic} (dB) scale, the volume values
        are scaled logarithmically. The volume meters of digital audio
        devices usually are scaled this way. On the other hand, if you
        are interested in the power level that is applied to your headphones
        you should choose \setting{Linear} display. This setting cannot be
        displayed in units like volts or watts because such units depend
        on your headphones.
      \item[Minimum and maximum range.]
        These two options define the full value range that the peak meter
        displays. Recommended values for the \setting{Logarithmic} (dB) setting
        are {}-40 dB for minimum and 0 dB for maximum. Recommended values
        for \setting{Linear} display are 0 and 100\%. Note that {}-40 dB is
        approximately 1\% in linear value, but if you change the minimum
        setting in linear mode slightly and then change to the dB scale,
        there will be a large change. You can use these values for `zooming'
        into the peak meter.
      \end{description}
    }
    \item[\label{ref:Defaultcodepage}Default Codepage.]
      A codepage describes the way extended characters that are not available
      within the ASCII character set are encoded. ID3v1 tags do not have a
      codepage encoding contained so Rockbox needs to know what encoding has
      been used when generating these tags. This should be ``ISO-8859-1'' but
      to support languages outside Western Europe most applications use
      the setting of your operating system instead. If your operating system
      uses a different codepage and you are getting garbled extended characters
      you should adjust this settings. In most cases sticking to
      ``ISO-8859-1'' would be sufficient.
  \end{description}

% $Id$ %
\subsection{Installing the bootloader}
  Installing the bootloader is the trickiest part of the installation.
  The Rockbox bootloader allows users to boot into either the Rockbox 
  firmware or the iriver firmware. For legal reasons, we cannot distribute 
  the bootloader. Instead, we have developed a program that will patch the 
  Iriver firmware with the Rockbox bootloader. These instructions will explain 
  how to download and patch the Iriver firmware with the Rockbox bootloader 
  and install it on your jukebox.


\begin{enumerate}
  \item Download a supported version of the Iriver firmware for your 
  \playername{} from the Iriver website or from 
  \wikilink{ManualRockboxInstall}.
  Supported Iriver firmware versions currently include 
  \opt{IRIVER_H100_PAD}{1.63US, 1.63EU, 1.63K, 1.65US, 1.65EU, 1.65K, 1.66US, 
    1.66EU and 1.66K.  Note that the H140 uses the same firmware as the H120;
    H120 and H140 owners should use the	firmware called \fname{ihp\_120.hex}.
    Likewise, the iHP110 and iHP115 use the same firmware, called 
    \fname{ihp\_100.hex}.   Be sure to use the correct firmware file for 
    your player.}
  \opt{IRIVER_H300_PAD}{1.28K, 1.28EU, 1.28J, 1.29K, 1.29J and 1.30EU.
    \note{The US H3xx firmware is not currently supported and cannot be
    patched to be used with the bootloader. If you wish to install Rockbox
    on a US \playername\, you must use an international firmware, which will
    permanently remove DRM support from the player.}
  }%
  If the file that you downloaded is a \fname{.zip} file, use an unzip 
  utility like mentioned in the prerequisites section to extract
  the \fname{.hex} from the \fname{.zip} file
  to your desktop. Likewise, if the file that you downloaded is an 
  \fname{.exe} file, double-click on the \fname{.exe} file to extract 
  the \fname{.hex} file to your desktop.
  When running linux you should be able extracting \fname{.exe}
  files using \fname{unzip}.
  %
  \item Download the firmware patcher \fname{fwpatcher.exe} from 
  \url{http://download.rockbox.org/bootloader/iriver/} and save it to your desktop.
    \warn{The firmware patcher contains Unicode support, which is not supported by 
    all versions of Windows. If you have difficulty with the firmware patcher, try 
    downloading the alternate firmware patcher \fname{fwpatchernu.exe}, which is 
    built without Unicode support.}
  %
  \item Go to your desktop and double-click on whichever version of the firmware 
  patcher you downloaded in the prior step.
  %
  \item In the firmware patcher dialog box, click on the \setting{Browse}
  button and navigate
  to the \fname{.hex} file that you previously downloaded to your desktop.
  %
  \item Click \setting{Patch}. The firmware patcher will patch the
    original firmware to include the Rockbox bootloader. The \fname{.hex}
    file on your desktop is now a modified version of the original
    \fname{.hex} file.
  %
  \item Turn on your \playerman{} and connect it to your computer via USB.
  %
  \item Copy or move the modified \fname{.hex} file to the root folder of
    your jukebox.
  %
  \item Disconnect the jukebox from USB. (Be sure to use Windows' ``safely remove
  hardware'' option.)
  \warn{Before proceeding further, make sure that your player has a full charge
    or that it is connected to the power adaptor. Interrupting the next step
    due to a power failure most likely will brick your \dap{}.}
  %
  \item Update your \daps{} firmware with the patched bootloader. To do this, turn
    the jukebox on. Press and hold the
    \opt{IRIVER_H100_PAD,IRIVER_H300_PAD}{\ButtonSelect{} button }%
    to enter the main menu, and navigate to \setting{General $\rightarrow$ Firmware 
    Upgrade}. Select \setting{Yes} when asked to confirm if you want to upgrade the 
    firmware. The \playerman{} will display a message indicating that the
    firmware update 
    is in progress. Do \emph{not} interrupt this process. When the
    firmware update is complete the player will turn itself off. (The update
    firmware process usually takes a minute or so.)
    
    You have now installed the Rockbox bootloader.
\end{enumerate}

\note{If you install the Rockbox bootloader but do not install the
  Rockbox firmware the Rockbox bootloader will load the iriver firmware when the
  jukebox is turned on.
  To load the \playerman{} firmware press and hold \ButtonRec{} before
  powering up the \dap{} until the \playerman{} logo appears.
  }

\note{The bootloader has a builtin ``bootloader USB mode''. This function
  switches to USB mode when the \dap{} is connected to a computer upon
  power-up. This way you can access the \daps{} harddisk without the need
  to boot any firmware (which is also useful when your harddisk is 
  damaged). The screen will simply display the text ``bootloader USB mode''.
  After you disconnect the \dap{} from USB the bootloader will
  continue booting Rockbox. As in bootloader USB mode the firmware
  itself hasn't been loaded this is also a simple way of updating Rockbox.
  After the disconnect the bootloader will load the updated version of
  Rockbox.
}

\subsubsection{Viewport Declaration Syntax}

{\config{\%V}}{\textbar}x{\textbar}y{\textbar}[width]{\textbar}[height]{\textbar}[font]{\textbar}%

    \begin{itemize}
      \item 'font' is a number - '0' is the built-in system font, '1' is the
      user-selected font.
      \item Only the coordinates \emph{have} to be specified. Leaving the other
      definitions blank will set them to their default values.
      \note{The correct number of {\textbar}s with hyphens in blank fields
      are still needed in any case.}
    \end{itemize}
  
\begin{example}
    %V|12|20|-|-|1|
    %sThis viewport is displayed permanently. It starts 12px from the left and
    %s20px from the top of the screen, and fills the rest of the screen from
    %sthat point. The lines will scroll if this text does not fit in the viewport.
    %sThe user font is used.
\end{example}
\begin{rbtabular}{.75\textwidth}{XX}{Viewport definition & Default value}{}{}
  width/height & remaining part of screen \\
  font & user defined \\
\end{rbtabular}

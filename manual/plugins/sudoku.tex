\subsection{\label{ref:Sudoku}Sudoku}
\screenshot{plugins/images/ss-sudoku}{Sudoku}{fig:sudoku}
To play Sudoku you need sudoku games stored as .ss files (see links below).
Once you have your .ss file just open it in the normal file browser to start playing.
You can create and save your own grids under the \setting{New} menu option.
Press the menu button when you have finished and enter the full path
to save to including the .ss extension (e.g. \fname{/sudoku/new.ss}).

\textbf{The thing on the left (AKA the scratchpad)}\\
When you play sudoku on paper most people like to mark numbers in
cells that are possible candidates for the cells.
This can be done with the column on the left. Change the number
under the cursor to a number which might be valid and press the scratchpad
button, the number will then be added on the left.
The column is stored seperatly for every cell on the board.
These are \emph{NOT} saved when saving the game.

\begin{table}
    \begin{btnmap}{}{}

    \opt{RECORDER_PAD,ONDIO_PAD,h1xx,h300,x5}{Direction keys}
    \opt{IPOD_4G_PAD}{\ButtonScrollFwd/\ButtonScrollBack} 
    & Move the cursor\\
    %
    \opt{RECORDER_PAD}{\ButtonPlay}
    \opt{ONDIO_PAD}{\ButtonMenu}
    \opt{h1xx,h300}{\ButtonSelect/\ButtonOn}
    \opt{IPOD_4G_PAD}{\ButtonLeft/\ButtonRight} 
    & Change number under the cursor\\
    %
    \opt{RECORDER_PAD}{\ButtonFOne}
    \opt{ONDIO_PAD}{Long press on \ButtonMenu}
    \opt{h1xx,h300}{\ButtonMode}
    \opt{IPOD_4G_PAD}{\ButtonMenu}
    \opt{x5}{\ButtonPlay} 
    & Open Menu\\
    %
    \opt{RECORDER_PAD}{\ButtonFTwo}
    \opt{ONDIO_PAD}{\ButtonMenu+\ButtonLeft}
    \opt{h1xx,h300,x5}{\ButtonRec}
    \opt{IPOD_4G_PAD}{\ButtonPlay} 
    & Add/Remove number to scratchpad\\
    %
    \opt{RECORDER_PAD,ONDIO_PAD,h1xx,h300}{\ButtonOff}
    \opt{x5}{\ButtonPower}
    \opt{IPOD_4G_PAD}{Menu $\rightarrow$ Quit}
    & Quit\\
    %
    \end{btnmap}
\end{table}

Some places where can you can find .ss files:
\begin{itemize}
\item Simple Sudoku (Advanced Puzzle Packs 1 and 2 located near the bottom of that page):
\url{http://www.angusj.com/sudoku/}
\item Kjell's Sudoku generator/solver:
\url{http://kjell.haxx.se/sudoku/}
\end{itemize}

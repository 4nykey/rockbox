% $Id:$ %
\screenshot{main_menu/images/ss-recording-settings}{The recording settings screen}{}

\note{To change the location where recordings are stored open the
      \setting{File Menu} (see \reference{ref:Filemenu}) on the directory
      where you want to store them in the \setting{File Browser} and select
      \setting{Set As Recording Directory}.}

\opt{MASCODEC}{
  \section{Quality}
    Choose the quality here (0 to 7). Default is 5, best quality is 7,
    smallest file size is 0. This setting effects how much your sound
    sample will be compressed. Higher quality settings result in larger
    MP3 files.
    
    The quality setting is just a way of selecting an average bit rate,
    or number of  bits per second, for a recording.  When  this setting
    is lowered, recordings are compressed more (meaning worse sound quality),
    and the average bitrate changes as follows.
  
  \begin{table}[h!]
    \begin{center}
      \begin{tabularx}{0.75\textwidth}{lX}\toprule
        \emph{Frequency} & \emph{Bitrate} (Kbit/s) -- quality 0$\rightarrow$7 \\\midrule
        44100Hz stereo   & 75, 80, 90, 100, 120, 140, 160, 170 \\
        22050Hz stereo   & 39, 41, 45, 50,  60,  80,  110, 130 \\
        44100Hz mono     & 65, 68, 73, 80,  90,  105, 125, 140 \\
        22050Hz mono     & 35, 38, 40, 45,  50,  60,  75,  90 \\\bottomrule
      \end{tabularx}
    \end{center}
  \end{table}
}

\opt{SWCODEC}{
  \section{Format}
    Choose which format to save your recording in. The available choices are
    the two uncompressed formats \setting{PCM Wave} and \setting{AIFF}, the
    losslessly compressed \setting{WavPack} and the lossy
    \setting{MPEG Layer 3}.

  \section{Encoder Settings}
    This sets the bitrate when using the \setting{MPEG Layer 3} format. And has
    no settings for the other formats.
}

  \section{Frequency}
    Choose the recording frequency (sample rate).
    \opt{MASCODEC}{48kHz, 44.1kHz, 32kHz, 24kHz, 22.05kHz, 16kHz}
    \opt{h1xx,h300}{44.1kHz, 22.05kHz and 11.025kHz}
    \opt{x5}{88.2kHz, 44.1kHz, 22.05kHz and 11.025kHz}
    are available. Higher sample rates use up more disk space, but give better
    sound quality.
    \opt{SWCODEC}{\note{The 11.025kHz setting is not available when using%
      \setting{MPEG Layer 3} format.}
    }%
    \opt{MASCODEC}{
      The frequency setting also determines which version of the MPEG standard
      the sound is recorded using:\\
      MPEG v1 for 48, 44.1 and 32\\
      MPEG v2 for 24, 22.05 and 16\\
    }
    \opt{recorder,recorderv2fm,h1xx}
      {\note{You cannot change the sample rate for digital recordings.}
    }  

\section{Source}
  Choose the source of the recording. This can be
  \opt{recorder,recorderv2fm,h1xx}{\setting{SPDIF (digital)},}%
  \setting{Mic} or \setting{Line In}.
  \opt{CONFIG_TUNER}{For recording from the radio see \reference{ref:FMradio}.}

\section{Channels}
  This allows you to select mono or stereo recording. Please note that
  for mono recording, only the left channel is recorded. Mono recordings
  are usually somewhat smaller than stereo.

\opt{MASCODEC}{
  \section{Independent Frames}
    The independent frames option tells the \dap{} to encode with the bit
    reservoir disabled, so the frames are independent of each other. This
    makes a file easier to edit.
}
      
\section{File Split Options}
  This sub menu contains options for file splitting, which can be used to split
  up long recordings into manageable pieces. The splits are seamless (frame
  accurate), no audio is lost at the split point. The break between recordings
  is only the time required to stop and restart the recording, on the order of
  2 -- 4 seconds.
  \begin{description}
    \item[Split Measure:]
      This option controls wether to split the recording when the
      \setting{Split Filesize} is reached or when the
      \setting{Split Time} has elapsed.

    \item[What to do when Splitting:]
      This controls what will happend when the splitting condition is
      fullfilled the two available options here are
      \setting{Start a new file} or \setting{Stop recording}.

    \item[Split Time:]
      Set the time to record between each split, if time is used as
      \setting{Split Measure}.\\
      Options (hours:minutes between splits): Off, 00:05, 00:10, 00:15, 00:30,
      1:00, 1:14 (74 minute  CD), 1:20 (80 minute CD), 2:00, 4:00, 8:00, 10:00,
      12:00, 18:00, 24:00.

    \item[Split Filesize:]
      Set the filesize to record between each split, if filesize is used as
      \setting{Split Measure}.

  \end{description}

\section{Prerecord Time}
    This setting buffers a small amount of audio so that when the record button
    is pressed, the recording will begin from that number of seconds earlier.
    This is useful for ensuring that a recording begins before a cue that is
    being waited for.

\section{Clear Recording Directory}
    Resets the location where the recorded files are saved to the root of your
    \daps{} drive.

\nopt{ondio}{
  \section{Clipping Light}
    Causes the backlight to flash on when clipping has been detected.\\
    Options: \setting{Off}, \setting{Main unit only},
    \setting{Main and remote unit}, \setting{Remote unit only}.
}
  \section{Trigger}
    \fixme{Add description of triggered recording.}

\opt{h1xx,h300}{%
  \section{Automatic Gain Control}
    The \setting{Automatic Gain Control} has five different presets for
    automatically controlling the gain while recording.
    \begin{description}
      \item[Safety (clip):]
        This preset will lower the gain when the levels get too high (-1dB)
        and will never increase gain.
        
      \item[Live (slow):]
        This preset is designed to be used for recording of live shows and has
        quite large headroom for loud parts. It heads for a nominal target peak
        level of -9dB and will slowly increase or decrease gain to reach it.
        
      \item[DJ-Set (slow):]
        This preset heads for a nominal target peak level of -5dB and will
        slowly increase or decrease gain to reach it.
        
      \item[Medium:]
        This preset heads for a nominal target peak level of -6dB and will
        increase or decrease gain to reach it.

      \item[Voice (fast):]
        This preset is designed to be used for voice recording and heads for a
        nominal target peak level of -7dB and will quickly increase or
        decrease gain to reach it.
    \end{description}

  \section{AGC clip time}
    This setting controls how long the level is too loud or soft before the
    \setting{Automatic Gain Control} kicks in.
}%

% $Id$ %
\section{\label{ref:Autoresumeconfigactual}Automatic resume}

The automatic resume feature stores and recalls resume positions for
all tracks without user intervention.  These resume points are stored
in the database, and thus automatic resume only works when the
database has been initialized.

When automatic resume is enabled, manually selected tracks resume
playback at their last playback position.  It does not matter in which
way you start the track; tracks are resumed whether they are navigated
to through the database browser or file browser, from a playlist or
bookmark, or by skipping through tracks in a playlist.

Optionally, you can also enable automatic resume for automatic track
transitions.  In this case, the next track will be resumed as well
instead of starting playback at its beginning.  This is most useful
for podcasts, and can be enabled on a per-directory basis.

A track's resume position is updated whenever playback of that track
stops, including when explicitly stopping the track, powering off the
\dap{}, or starting playback of another track.

If you intend to start a track from its beginning but notice that it
was resumed, you can press \ActionWpsSkipPrev{} in the WPS to skip back to
its beginning.  When pressing \ActionWpsSkipPrev{} again in the first few
seconds of a track to go to the previous track, the previously (on
first button press) saved resume position is retained.  Therefore, you
can also use \ActionWpsSkipPrev{} and \ActionWpsSkipNext{} to skip
across tracks in a playlist without losing their resume position.

\begin{description}
\item[Automatic resume.] This option enables or disables automatic
  resume globally.  When Rockbox detects that the database (which is
  needed for this feature) has not been initialized yet, it asks
  whether it should be initialized right away.

\item[Resume on automatic track change.] Controls whether the next
  track in an automatic track transition should be resumed at its last
  playback position as well.
  \begin{description}
  \item[No.] Automatic resume works only for manual track selection.
  \item[Yes.] Always attempt to resume -- for both manual and
    automatic track changes.
  \item[In custom directories only.] Configure directories in which to
    enable resume on automatic track change.  Selecting this option
    starts the text editor, in which you can enter the (absolute,
    case-insensitive) directory names separated by colons (``:'').

    A typical value is ``/podcast'', which matches all files in
    directories \fname{/PODCAST}, \fname{/Podcast} or \fname{/podcast}
    and their subdirectories, but not in directories \fname{/podcasts}
    (mind the trailing ``s'') or \fname{/audio/podcast}.
  \end{description}
\end{description}

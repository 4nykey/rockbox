% $Id$ %
\section{\label{ref:WPS}While Playing Screen}
The While Playing Screen (WPS) displays various pieces of information about the
currently playing audio file.
%
\opt{lcd_bitmap}{%
  The appearance of the WPS can be configured using WPS configuration files.
  The items shown depend on your configuration -- all item can be turned on
  or off independently. Refer to \reference{ref:wps_tags} for details on how
  to change the display of the WPS.
  \begin{itemize}
    \nopt{ondio}{
    \item Status bar: The Status bar shows Battery level, charger status, 
      volume, play mode, repeat mode, shuffle mode\opt{rtc}{ and clock}.
      In contrast to all other items, the status bar is always at the top of
      the screen.
    }
    \opt{ondio}{
    \item Status bar: The Status bar shows Battery level, USB power mode, key
      lock status, memory access indicator. In contrast to all other items, the
      status bar is always at the top of the screen.
    }
  \item (Scrolling) path and filename of the current song.
  \item The ID3 track name.
  \item The ID3 album name.
  \item The ID3 artist name.
  \item Bit rate. VBR files display average bitrate and ``(avg)''
  \item Elapsed and total time.
  \item A slidebar progress meter representing where in the song you are.
  \item Peak meter.
  \end{itemize}
}
\opt{recorder,recorderv2fm,ondio}{
  \note{
  \begin{itemize}
  \item The number of lines shown depends on the size of the font used.
  \item The peak meter is only visible if you turn off the status bar or if
    using a small font that gives 8 or more display lines.
  \end{itemize}
  }
}
%
\opt{player}{
  \note{
  \begin{itemize}
  \item Playlist index/Playlist size: Artist {}- Title.
  \item Current{}-time Progress{}-indicator Left.
  \end{itemize}
  }
}

See \reference{ref:ConfiguringtheWPS} for details of customising
your WPS (While Playing Screen).


\subsection{\label{ref:WPS_Key_Controls}WPS Key Controls}

\begin{table}
  \begin{btnmap}{}{}
      \ActionWpsVolUp{} / \ActionWpsVolDown & Volume up/down.\\
      %
      \ActionWpsSkipPrev & Go to beginning of track, or if pressed while in the
        first seconds of a track, go to previous track.\\
      %
      \ActionWpsSeekBack & Rewind in track.\\
      %
      \ActionWpsSkipNext & Go to next track.\\
      %
      \ActionWpsSeekFwd & Fast forward in track.\\
      %
      \ActionWpsPlay & Toggle play/pause.\\
      %
      \ActionWpsStop & Stop playback.\\
      %
      \ActionWpsBrowse & Return to the \setting{File Browser}.\\
      %
      \ActionWpsContext & Enter \setting{WPS Context Menu}.\\
      %
      \opt{RECORDER_PAD,IRIVER_H100_PAD,IRIVER_H300_PAD,IRIVER_H10_PAD}{%
        \ActionWpsPitchScreen & Show \setting{Pitch Screen}
        (see \reference{sec:pitchscreen}).\\}%
      %
      \opt{ONDIO_PAD}{\ActionWpsContext{} twice}%
      \nopt{ONDIO_PAD}{\ActionWpsMenu}%
      & Enter \setting{Main Menu}%
      \opt{ONDIO_PAD}{ via the \setting{WPS Context Menu}}.\\%
      %
      \nopt{PLAYER_PAD,ONDIO_PAD}{%
        \ActionWpsQuickScreen & Enter \setting{Quick Screen}.\\}%
      %
      %These actions need definitions for the other targets
      \opt{RECORDER_PAD}{%
        \ButtonFThree & Toggles Display quick screen.\\%
        \ButtonFOne+\ButtonDown & Key lock on/off.\\%
        \ButtonFOne+\ButtonPlay & Mute on/off.\\%
      }%
      \opt{PLAYER_PAD}{%
        \ButtonMenu+\ButtonStop & Key lock on/off.\\%
        \ButtonMenu+\ButtonPlay & Mute on/off.\\%
      }%
      \opt{ONDIO_PAD}{%
        Long \ButtonMenu+\ButtonDown & Key lock on/off.\\%
      }%
      \opt{PLAYER_PAD,RECORDER_PAD,IRIVER_H100_PAD,IRIVER_H300_PAD,IRIVER_H10_PAD}{%
        \ActionWpsIdThreeScreen & Enter \setting{ID3 Viewer}.\\%
      }%
      \opt{h1xx,h300,e200,gigabeat,h10,h10_5gb,RECORDER_PAD,MROBE100_PAD}{%
        \ActionWpsAbSetBNextDir & Skip to the next directory.\\%
        \ActionWpsAbSetAPrevDir & Skip to the previous directory.\\%
      }%
      \opt{SANSA_E200_PAD}{
        \ActionStdRec & Switches to the Recording screen \\
      }%
    \end{btnmap}
\end{table}


\opt{lcd_bitmap}{
  \subsection{\label{ref:peak_meter}Peak Meter}
  The peak meter can be displayed on the While Playing Screen and consists of
  several indicators. 
  \opt{recording}{
    For a picture of the peak meter, please see the While
    Recording Screen in \reference{ref:while_recording_screen}.
  }
  
  \begin{description}
  \item [The bar:]
    This is the wide horizontal bar. It represents the current volume value.
  \item [The peak indicator:]
    This is a little vertical line at the right end of the bar. It indicates 
    the peak volume value that occurred recently.
  \item [The clip indicator:]
    This is a little black block that is displayed at the very right of the
    scale when an overflow occurs. It usually does not show up during normal
    playback unless you play an audio file that is distorted heavily.
    \opt{recording}{
      If you encounter clipping while recording, your recording will sound distorted.
      You should lower the gain.
    }
    \note{Note that the clip detection is not very precise.
     Clipping might occur without being indicated.}
  \item [The scale:]
    Between the indicators of the right and left channel there are little dots.
    These dots represent important volume values. In linear mode each dot is a
    10\% mark. In dbfs mode the dots represent the following values (from right
    to left): 0db, {}-3db, {}-6db, {}-9db, {}-12db, {}-18db, {}-24db, {}-30db,
    {}-40db, {}-50db, {}-60db.
  \end{description}
}
\subsection{\label{sec:contextmenu}The WPS Context Menu}
Like the context menu for the \setting{File Browser}, the \setting{WPS Context Menu} 
allows you quick access to some often used functions:

\subsubsection{Playlist}
The \setting{Playlist} submenu allows you to view, save, search and
reshuffle the current playlist. To change settings for the
\setting{Playlist Viewer} press \ActionStdMenu{} while viewing the playlist
to bring up the \setting{Playlist Viewer Menu}.
    
\subsubsection{Playlist catalog}
  \begin{description}
    \item [View catalog.] This lists all playlists that are part of the
    Playlist catalog. You can load a new playlist directly from this list.
    \item [Add to playlist.] Adds the currently playing file to a playlist.
    Select the playlist you want the file to be added to and it will get
    appended to that playlist.
    \item [Add to new playlist.] Similar to the previous entry this will
    add the currently playing track to a playlist. You need to enter a name
    for the new playlist first.
  \end{description}

\subsubsection{Sound Settings}
This is a shortcut to the \setting{Sound Settings Menu}, where you can configure volume,
bass, treble, and other settings affecting the sound of your music.  
See \reference{ref:configure_rockbox_sound} for more information.

\subsubsection{Playback Settings}
This is a shortcut to the \setting{Playback Settings Menu}, where you can configure shuffle,
repeat, party mode, study mode and other settings affecting the playback of your music.  

\subsubsection{Rating}
The menu entry is only shown if \setting{Gather Runtime Information} is
enabled. It allows the asignment of a personal rating value (0 -- 10)
to a track which can be displayed in the WPS and used in the Database
browser. Press \ButtonRight{} to increment the value. The value wraps at 10.

\subsubsection{Bookmarks}
This allows you to create a bookmark in the currently-playing track.

\subsubsection{\label{ref:trackinfoviewer}Show Track Info}
\screenshot{rockbox_interface/images/ss-id3-viewer}{The track info viewer}{}
This screen is accessible from the WPS screen, and provides a detailed view of
all the identity information about the current track. This info is known as
meta data and is stored in audio file formats to keep information on artist,
album etc. To access this screen, % 
\opt{PLAYER_PAD,RECORDER_PAD,IRIVER_H100_PAD,IRIVER_H300_PAD,IRIVER_H10_PAD}{
  press \ActionWpsIdThreeScreen. }%
\opt{ONDIO_PAD,IPOD_4G_PAD,IPOD_3G_PAD,IAUDIO_X5_PAD}{press \ActionWpsContext{}
  to access the \setting{WPS Context Menu} and select
  \setting{Show Track Info}. }%
\opt{RECORDER_PAD,PLAYER_PAD,ONDIO_PAD}{Use \ButtonLeft\ and \ButtonRight\
  to move through the information.}%
\subsubsection{Open With...}
This \setting{Open With} function is the same as the \setting{Open With} 
function in the file browser's \setting{Context Menu}.

\nopt{player}{
  \subsubsection{\label{sec:pitchscreen}Pitch}
  
  The \setting{Pitch Screen} allows you to change the pitch and (at the same
  time) the playback speed of your \dap. The pitch value can be adjusted
  between 50\% and 200\%. 50\% means half the normal playback speed and the
  pitch that is an octave lower than the normal pitch. 200\% means double
  playback speed and the pitch that is an octave higher than the normal pitch.
  It is not possible to change the pitch without changing the playback speed and
  vice versa. Changing the pitch can be done in two modes: procentual and
  semitone. Initially (after the \dap{} is switched on), procentual mode
  is active.
  
  \begin{table}
    \begin{btnmap}{}{}
      \ActionPsToggleMode
      & Toggle pitch changing mode \\
      %
      \ActionPsIncSmall{} / \ActionPsDecSmall
      & Increase / Decrease pitch by 0.1\% (in procentual mode) or a semitone
        (in semitone mode)\\
      %
      \ActionPsIncBig{} / \ActionPsDecBig
      & Increase / Decrease pitch by 1\% (in procentual mode) or a semitone
        (in semitone mode)\\
      %
      \ActionPsNudgeRight{} / \ActionPsNudgeLeft
      & Temporarily increase / decrease pitch by 2.0\% \\
      %
      \ActionPsReset
      & Reset pitch to 100\% \\
      %
      \ActionPsExit
      & Leave the Pitch Screen \\
      %
    \end{btnmap}
  \end{table}

  \opt{MASCODEC}{
    \warn{Changing the pitch can cause audible 'Artifacts' or 'Dropouts'.}
  }
}

%********************QUICKSCREENS***********************************************
\opt{RECORDER_PAD}{
  \section{\label{ref:QuickScreens}Quick Screens}
  \screenshot{rockbox_interface/images/ss-quick-screen-112x64x1.png}{The F2 quick screen}{}
  \screenshot{rockbox_interface/images/ss-quick-screen2-112x64x1.png}{The F3 quick screen}{}
  Rockbox handles function buttons in a different way to the Archos software.
  \ButtonFOne\ is always bound to the menu function, while \ButtonFTwo\ and
  \ButtonFThree\ enable two quick screens.
  
  \ButtonFTwo\ displays some browse and play settings which are likely to be
  changed frequently. This settings are Shuffle mode, Repeat mode and the Show
  files options
  
  Shuffle mode plays each track in the currently playing list in a random order
  rather than in the order shown in the browser.

  Repeat mode repeats either a single track (One) or the entire playlist (All).

  Show files determines what type files can be seen in the browser.  This can be
  just MP3 files and directories (Music), Playlists, MP3 files and directories
  (Playlists), any files that Rockbox supports (Supported) or all files on the
  disk (All).

  See \reference{ref:PlaybackOptions} for more information about these
  settings.

  \begin{table}
    \begin{btnmap}{}{}
      \ButtonLeft & Controls Shuffle mode setting \\
      \ButtonRight & Controls Repeat mode setting \\
      \ButtonDown & Controls Show file setting \\
    \end{btnmap}
  \end{table}
  
  \ButtonFThree\ controls frequently used display options.
  
  Scroll bar turns the display of the Scroll bar on the left of the screen on
  or off.
  
  Status bar turns the status display at the top of the screen on or off. 
  Upside down inverts the screen so that the top of the display appears nearest
  to the buttons. This is sometimes useful when storing the \dap\ in a pocket.
  Key assignments swap over with the display orientation where it is logical 
  for them to do so.

  See \reference{ref:Displayoptions} for more information about these
  settings.
  
  \begin{table}
    \begin{btnmap}{}{}
      \ButtonLeft & Controls scroll bar display \\
      \ButtonRight & Controls status bar display \\
      \ButtonDown & Controls upside down screen setting \\
    \end{btnmap}
  \end{table}
}

% $Id$ %
\subsection{MPEG Player}
The Mpeg Player is a video player plugin capable of playing back MPEG-1 and 
MPEG-2 video streams with MPEG audio multiplexed into \fname{.mpg} files.

To play a video file, you just select it in the Rockbox \setting{File Browser}.
If your file does not have the \fname{.mpg} extension but is encoded in the
supported format, you will need to use the \setting{Open With...} context menu
option and choose \setting{mpegplayer}.

\begin{table}
\begin{btnmap}{}{}
    \opt{GIGABEAT_S_PAD}{\ButtonSelect{} or \ButtonPlay}
    \opt{GIGABEAT_PAD}{\ButtonSelect{} or \ButtonA}
    \nopt{GIGABEAT_S_PAD,GIGABEAT_PAD}{\ActionWpsPlay}
       \opt{HAVEREMOTEKEYMAP}{& } 
    & Pause / Resume\\
    \ActionWpsStop
       \opt{HAVEREMOTEKEYMAP}{& }
    & Stop\\
    \nopt{GIGABEAT_S_PAD,GIGABEAT_PAD}{\ActionWpsVolUp{} / \ActionWpsVolDown}
    \opt{GIGABEAT_S_PAD,GIGABEAT_PAD}{\ButtonLeft{} or  \ButtonVolDown{} /
        \ButtonRight{} or \ButtonVolUp}
       \opt{HAVEREMOTEKEYMAP}{& }
    & Adjust volume up / down\\
    \nopt{GIGABEAT_S_PAD,GIGABEAT_PAD}{\ActionWpsSkipPrev{} / \ActionWpsSkipNext}
    \opt{GIGABEAT_S_PAD,GIGABEAT_PAD}{\ButtonUp{} / \ButtonDown}
       \opt{HAVEREMOTEKEYMAP}{& }
    & Rewind / Fast Forward\\
    \opt{IRIVER_H100_PAD,IRIVER_H300_PAD}{\ButtonMode}
    \opt{IPOD_4G_PAD,IPOD_3G_PAD,GIGABEAT_PAD,GIGABEAT_S_PAD,MROBE100_PAD}{\ButtonMenu}
    \opt{IAUDIO_X5_PAD}{\ButtonRec}
    \opt{IRIVER_H10_PAD}{\ButtonRew}
    \opt{SANSA_E200_PAD,SANSA_FUZE_PAD,SANSA_C200_PAD}{\ButtonSelect}
       \opt{HAVEREMOTEKEYMAP}{& }
    & Open the MPEG Player menu\\
\end{btnmap}
\end{table}

When a video file is selected, the start Menu will be displayed, unless it is 
disabled via the option "start menu" (see below). In the latter case the video 
will start playing immediately - unless a resume point is found, in which case 
the resume menu is presented.

Start Menu

\begin{description}
\item[Play from beginning] Resume information is discarded and the video plays
    from the start.
\item[Resume time (min): x.x] Resume video playback at stored resume time x.x
    (start of the video if no resume time is found).
\item[Set start time (min)] A preview screen is presented consisting of a
    thumbnail preview and a progress bar where the user can select a start time
    by 'seeking' through the video. The video playback is started by pressing
    the select button.
\item[Quit mpegplayer] Exit the plugin.
\end{description}

Resume Menu

\begin{description}
\item[Yes (min): x.x] Resume video playback at stored resume time x.x.
\item[No] Play video from the beginning.
\end{description}

Main Menu

\begin{description}
\item[Display Options] Opens "Display Options" submenu - see below.
\item[Start Menu] (default: on) Enable/disable the start menu.
\item[Clear all resumes: x] Discard all x resume points.
\item[Quit mpegplayer] Exit the plugin.
\end{description}

Display Options Menu

\begin{description}
\item[Dithering] (default: off) Prevent banding effects in gradients by blending
    of colours. (only available on Sansa e200, Sansa c200 and Gigabeat F/X)
\item[Show FPS] (default: off) This option displays (once a second - if your
    video is full-screen this means it will get overwritten by the video and
    appear to flash once per second) the average number of frames decoded per
    second, the total number of frames skipped (see the Skip Frames option),
    the current time (in 100 Hz ticks) and the time the current frame is due to
    be displayed.
\item[Limit FPS] (default: on) With this option disabled, mpegplayer will
    display the video as fast as it can. Useful for benchmarking.
\item[Skip frames] (default: on) This option causes mpegplayer to attempt to
    maintain realtime playback by skipping the display of frames - but these
    frames are still decoded. Disabling this option can cause loss of A/V sync. 
\end{description}

See this page in the Rockbox wiki for information on how to encode your videos
to the supported format. \wikilink{PluginMpegplayer}

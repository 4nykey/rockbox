\section{\label{ref:SystemOptions}System Options}

\subsection{Battery}
    Options relating to the batteries in the \dap.
  \begin{description}
    \item [Battery Capacity: ]This setting can be used to tell Rockbox what
      capacity (in mAh) of battery is being used inside it. The default is
      \opt{player,recorder}{1500mAh}
      \opt{recorderv2fm}{2200mAh}
      \opt{ondiosp,ondiofm}{1000mAh}
      \opt{h1xx,h300}{1300mAh}
      \opt{ipodmini}{400mAh (1G) or 630mAh (2G)}
      \opt{ipodcolor}{700mAh}
      \opt{ipodnano,ipodvideo,ipod4g}{\fixme{unknown}}
      \opt{x5}{950mAh}
      which is the capacity value for the standard batteries shipped with the \dap.
      Rockbox uses this value to estimate remaining battery life for the status
      bar and WPS. Changing this setting has no effect whatsoever on actual battery life.
      This setting affects only the the accuracy of the battery life display on screen.
   
      \opt{ipod}{This value is fairly meaningless in the iPod family at
        present, and work is ongoing into finding a better way to determine battery life.}
    
    \opt{ondiosp,ondiofm}{
      \item [Battery Type: ]This setting tells Rockbox wich type of battery
        that is currently used in the \dap.
    }
    
  \end{description}

\nopt{ondiosp,ondiofm}{
\subsection{Disk}
    Options relating to the hard disk.

  \begin{description}
    \item [Disk Spindown: ]Rockbox has a timer that makes it spin down the
      hard disk after it is idle for a certain amount of time. This setting
      controls the amount of time between the last user activity and the time
      that the disk spins down. This idle time is only affected by user
      activity, like navigating through file browser. When the hard disk spins
      up to fill the audio buffer, it automatically spins down afterwards.
      
    \opt{recorder,h1xx,h300,x5}{
      \item [Disk Poweroff: ]This setting controls whether the disk is powered
      off or only set to ``sleep'' when spun down. If this setting is set to
      \setting{YES}, the disk will power off. If set to \setting{NO}, the disk
      will enter ``sleep'' mode. Power off uses less power but takes slightly
      longer to spin-up.
    }
    \opt{SWCODEC}{
    \item [Directory Cache: ]Rockbox has the ability to scan the contents of
      your drive in the background and save those contents to a cache in RAM.
      The \setting{Directory Cache} takes a small amount of memory away from Rockbox
      that would otherwise be used to buffer music, but it speeds up navigation
      by eliminating the slight pause between the time a navigation button is
      pressed and the time Rockbox responds. Turning this setting \setting{On}
      activates the directory cache, and turning it \setting{Off} deactivates the 
      directory cache.
      \note{The first time you set the directory cache to \setting{On}, 
      Rockbox will request a reboot of the \dap\ and upon restarting take a few
      minutes to scan the drive. Thereafter, the directory cache will work in the 
      background.}
    }
  \end{description}
} % \nopt{ondiosp,ondiofm}

\opt{CONFIG_RTC}{
  \subsection{Time and Date}
  Time related menu options.
    \begin{description}
      \item [Set Time/Date: ]Set current time and date.
      \item [Time Format: ] Choose 12 or 24 Hour clock.
    \end{description}
}

\subsection{Power Control}
  \begin{description}
    \item [\label{ref:idlepoweroff}Idle Poweroff: ]Rockbox can be configured
    to turn off power after the unit has been idle for a defined number of
    minutes. The \dap\ is idle when playback is stopped or paused. It is not
    idle while the USB or charger is connected, or while recording.

    \item [Sleep Timer: ]This option lets you power off your \dap\ after
    playing for a given time.
    \opt{recorderv2fm}{This setting is reset on boot. Using this option
    disables the \setting{Wake up alarm} (see below).}
    \opt{recorderv2fm}{
      \item [Wake up alarm: ]This option turns the \dap\ off and then starts
      it up again at the specified time. This is most useful when combined
      with the \setting{resume} setting in the \setting{Playback Options} 
      menu is set to \setting{Yes}, so that the \dap\ wakes up and immediately
      starts playing music.  Use \ButtonLeft\ and \ButtonRight\ to adjust the 
      minutes setting, \ButtonUp\ and \ButtonDown\ to adjust the HOURS. 
      \ButtonPlay\ confirms the alarm and shuts your \dap\ down, and \ButtonOff 
      cancels setting an alarm.  If the \dap\ is turned on again before the 
      alarm occurs, the alarm will be canceled. Using this option disables 
      the \setting{Sleep Timer}(see above).
    }
  \end{description}

\subsection{Limits}
This submenu relates to limits in the Rockbox operating system.
  \begin{description}
    \item [Max files in dir browser: ]This setting controls the limit on 
    the number of files that you can put in any particular directory in the
    file browser. You can configure the size to be between 50 and 10000 files
    in steps of 50 files. The default is 400. Higher values will shorten the
    music buffer, so you should increase this setting \emph{only} if you have
    directories with a large number of files.

    \item [Max playlist size: ]This setting controls the maximum size of 
    a playlist. The playlist size can be between 1,000 and 20,000 files,
    in steps of 1000 (default is 10,000). Higher values will shorten the
    music buffer, so you should increase this setting \emph{only} if you
    have very large playlists.
  \end{description}
  
\opt{player,recorder,recorderv2fm}{
\subsection{Car Adapter Mode}
      This option turns \setting{On} and \setting{Off} the car ignition 
      auto stop function.
  \begin{description}
  \item [Car Adaptor Mode: ]When using the \dap\ in a car, car adapter mode
  automatically stops playback on the \dap\ when power (i.e. from cigarette
  lighter power adapter) to the external DC in jack is turned off. If the Car
  Adaptor Mode is set to \setting{On}, Rockbox will pause playback when the 
  external power off condition is detected. Rockbox will then shutdown the 
  \dap\ after the length of time set in the \setting{Idle Poweroff} setting 
  (see above).
  If power to the DC in jack is turned back on before the \emph{Idle Poweroff}
  function has shut the \dap\ off, playback will be resumed 5 seconds after
  the power is applied. This delay is to allow for the time while the car
  engine is being started.
  \end{description}

  Once the \dap\ is shut off either manually, or automatically with the
  \emph{Idle Poweroff} function, it must be powered up manually to resume
  playback.
}

\opt{player}{
  \begin{description}
  \item [Line In (Player only): ] This option activates the line in port
  on \dap\ Player, which is off by default. This is useful for such applications as:
    \begin{itemize}
    \item Game boy $\rightarrow$ \dap $\rightarrow$ human
    \item laptop $\rightarrow$ \dap $\rightarrow$ human
    \item LAN party computer $\rightarrow$ \dap $\rightarrow$ human
    \end{itemize}
  \end{description}
}

\subsection{\label{sec:manage_settings}Manage settings}
This submenu deals with loading and saving settings. 
\opt{MASCODEC}{This submenu also allows you to load or save different
firmware versions.}
%
  \begin{description}
    \item [Browse .cfg Files: ]Opens the file browser in the 
    \fname{/.rockbox} directory and displays all .cfg (configuration) files.
    Selecting a .cfg file will cause Rockbox to load that the settings
    contained in that file. Pressing \ButtonLeft\ will exit back to the menu. 
    See the \setting{Write .cfg files} option on the Manage Settings menu for
    details of how to save and edit a configuration file.
    %
    \item [Browse Firmwares: ]This displays a list of firmware file in
    the \fname{/.rockbox} system directory. %
    \opt{SWCODEC}{This is legacy item, and is depreciated.}
    \opt{MASCODEC}{ 
    \opt{recorder,recorderv2fm}{Firmware files have an extension of .ajz.}
    \opt{player,ondio}{  Firmware files have an extension of .mod}
    Playing a firmware file loads it into memory.  Thus, it is possible to
    run the original Archos firmware or a different version of Rockbox
    from here assuming that you have the right files installed on your
    disk.  There is no need for any other file or directory to be installed
    to use this option; the firmware is resident in that one file.
    }
    \item [Reset Settings: ]This wipes the saved settings in the \dap\ and
    resets all settings to their default values. 
    \opt{h100, h300}{\note{You can also reset all settings to their default
      values by turning off the \dap\, turning it back on, and pressing
      the \ButtonRec button immediately after the \dap\ turns on.}
    }
    \item [Write .cfg file: ]This option writes a Rockbox configuration file
    to your \daps\ hard disk. The configuration file has the \fname{.cfg}
    extension and is used to store all of the user settings that are described
    throughout this manual.
    A configuration file may reside anywhere on the hard disk. Multiple
    configuration files are permitted. So, for example, you could have 
    a \fname{car.cfg} file for the settings that you use while playing your
    jukebox in your car, and a \fname{headphones.cfg} file to store the 
    settings that you use while listening to your \dap\ through headphones.

    The Rockbox configuration file is a plain text file, so once you use the
    \setting{Write .cfg file} option to create the file, you can edit the file
    on your computer using any text editor program. Configuration files use
    the following formatting rules:
    %
      \begin{enumerate}
      \item Each setting must be on a separate line.
      \item Each line has the format ``setting: value''.
      \item Values must be within the ranges specified in this manual for each
            setting.
      \item Lines starting with \# are ignored. This lets you write comments
            into your configuration files.
      \end{enumerate}

      Configuration files may be loaded using the \setting{Browse .cfg files}
      option on the \setting{Manage Settings} menu.

    \note{Configuration files do not need to contain all of the Rockbox 
      options.  You can create configuration files that change only certain
      settings. So, for example, supppose you typically use the \dap at one
      volume in the car, and another when using headphones.  Further,
      suppose you like to use an inverse LCD when you're in the car, and
      a regular LCD setting when you're using headphones.  You could create
      configuration files that control only the volume and LCD settings.
      Create a few different files with different settings, give each file
      a different name (such as \fname{car.cfg}, \fname{headphones.cfg},
      etc.), and you can then use the \setting{Browse .cfg files} option 
      to quickly change settings.}
  \end{description}


\documentclass[a4paper,11pt]{scrreprt}
\usepackage[latin1]{inputenc}
\usepackage{palatino}
%\renewcommand{\familydefault}{\sfdefault}

\usepackage{tabularx}
\usepackage{multirow}

\usepackage{float}
\floatstyle{ruled}

\usepackage[colorlinks=true, pdfstartview=FitV, linkcolor=blue, citecolor=blue, urlcolor=blue]{hyperref}
\usepackage{xspace}
\usepackage{optional}

\input{platform/\platform.tex}

\newcommand{\playername}{\playerman\ \playertype}

\newcommand{\fname}[1]{\textbf{#1}}
\newcommand{\tabeltc}[1]{{\centering #1 \par}}
\newcommand{\tabelth}[1]{{\centering \textbf{\textit{#1}} \par}}

\newcommand{\fixme}[1]{\textbf{\textcolor{red}{#1}}}

\usepackage{fancyhdr}
\usepackage{graphicx}
\usepackage{verbatim}
\usepackage{lscape}
\usepackage{makeidx}
\usepackage{amsmath}
\usepackage{amssymb}
\usepackage{fancyvrb}
\usepackage{enumerate}
\usepackage{subfigure}
\usepackage{color}
\usepackage{booktabs}
\usepackage{longtable}
\usepackage{url}
\urlstyle{sf}
\usepackage{marvosym}

% new \ifpdf to check if running in pdf mode. Helps for html generation.
\newif\ifpdf\ifx\pdfoutput\undefined\pdffalse\else\pdfoutput=1\pdftrue\fi

% mark this ad draft version (only for pdflatex) -- comment this out at release
\ifpdf
  \usepackage{pdfdraftcopy}
  \draftstring{DRAFT VERSION}
%   \draftangle{45}
\fi

% fancy header style adjustments
%\renewcommand{\chaptermark}[1]{\markboth{#1}{}}
%\renewcommand{\sectionmark}[1]{\markright{\thesection\ #1}}
\renewcommand{\rightmark}[1]{\thechapter\ }
\fancyhead{}
\fancyfoot{}
\fancyhead[L]{{\textsc{\leftmark}}}
\fancyhead[R]{\iffloatpage{}{\thepage}}
\fancyfoot[L]{\textsc{Rockbox users manual}}
\fancyfoot[R]{\textsc{\playername}}

\renewcommand{\headrulewidth}{\iffloatpage{0pt}{0.4pt}}
\renewcommand{\footrulewidth}{\iffloatpage{0pt}{0.4pt}}
\setlength{\headheight}{18.5pt}
\newcounter{example}[chapter]

\newenvironment{example}
    {\stepcounter{example}\paragraph{Example \theexample:}}
    {\hfill$\Box$
    
    \bigskip
    \noindent}

% found on the internet, posting by Donald Arseneau
% I may as well include my robust expandable definions, which can be
% used in \edef or \write where the \def would not be executed:
%
% \if\blank --- checks if parameter is blank (Spaces count as blank)
% \if\given --- checks if parameter is not blank: like \if\blank{#1}\else
% \if\nil --- checks if parameter is null (spaces are NOT null)
% use \if\given{ } ... \else ... \fi etc.
%
{\catcode`\!=8 % funny catcode so ! will be a delimiter
\catcode`\Q=3 % funny catcode so Q will be a delimiter
\long\gdef\given#1{88\fi\Ifbl@nk#1QQQ\empty!}
\long\gdef\blank#1{88\fi\Ifbl@nk#1QQ..!}% if null or spaces
\long\gdef\nil#1{\IfN@Ught#1* {#1}!}% if null
\long\gdef\IfN@Ught#1 #2!{\blank{#2}}
\long\gdef\Ifbl@nk#1#2Q#3!{\ifx#3}% same as above
} 

% add screenshot image.
% Usage: \screenshot{filename}{caption}{label}
% Note: use this only for screenshots!
% Note: leave caption empty to supress it.
\newcommand{\screenshot}[3]{
  \begin{figure}[!ht]
    \begin{center}
      \IfFileExists{#1-\genericimg.png}
        {\includegraphics[width=4cm]{#1-\genericimg.png}}
        {\IfFileExists{#1}
          {\includegraphics[width=4cm]{#1}
           \typeout{Warning: deprecated plain image name used}}%
          {\typeout{Missing image: #1 (\genericimg)}%
           \color{red}{\textbf{WARNING!} Image not found}%
          }
        }
      \if\blank{#3}\else{\label{#3}}\fi\if\blank{#2}\else{%
        \caption{#2}}\fi
    \end{center}
  \end{figure}
}

% command to display a note.
% Usage: \note{text of your note}
% Note: do NOT use \textbf or similar to emphasize text, use \emph!
\newcommand{\note}[1]{
  \ifinner\else\par\noindent\fi
  \textbf{Note:}\ %
  \ifinner#1\else\marginpar{\raisebox{-6pt}{\Huge\Writinghand}}#1\par\fi%
}

% command to display a warning.
% Usage: \warn{text of your warning}
% Note: do NOT use \textbf or similar to emphasize text!
\newcommand{\warn}[1]{
  \ifinner\else\par\noindent\fi
  \textbf{Warning:\ }%
  \ifinner#1\else\marginpar{\raisebox{-6pt}{\Huge\Stopsign}}#1\par\fi%
}

% make table floats use "!htb" as default positioning 
\makeatletter\renewcommand{\fps@table}{!htb}\makeatother
% change defaults for floats on a page as we have a lot of small images
\setcounter{topnumber}{3}    % default: 2
\setcounter{bottomnumber}{2} % default: 1
\setcounter{totalnumber}{5}  % default: 3

% command to set the default table heading for button lists
\newcommand{\btnhead}{\textbf{Key} & \textbf{Action} \\\midrule}

% environment intended to be used with button maps
% usage: \begin{btnmap}{caption}{label} Button & ButtonAction \\ \end{btnmap}
% Note: this automatically sets the table lines.
% Note: you *need* to terminate the last line with a linebreak \\
% Note: you still need to enclose this with \begin{table} / \end{table}
% Cheers for the usenet helping me building this up :)
\newenvironment{btnmap}[2]{%
  \expandafter\let\expandafter\SavedEndTab\csname endtabular*\endcsname
  \expandafter\renewcommand\expandafter*\csname endtabular*\endcsname{%
    \bottomrule
    \SavedEndTab%
    \if\given{#1}\caption{#1}\fi%
    \if\given{#2}\label{#2}\fi%
    \endcenter%
  }
  \center
\tabularx{.75\textwidth}{lX}\toprule % here is the table width defined
  \btnhead
}{%
  \endtabularx
}


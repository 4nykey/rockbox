\subsection{Text Viewer}
This is a Viewer for text files with word wrap. Just press PLAY on a
.txt file to display it. Has controls to handle various styles of text
formatting. Has top{}-of{}-file and bottom{}-of{}-file buttons.  You
can view files without a .txt extension by using \textbf{Open with ..}
from the Play Screen menu

\subsubsection{Controls}

\begin{itemize}
\item \textbf{F1 (Recorder) / ON{}-MINUS (Player): }
toggles Word mode between Wrap and Chop:

\begin{itemize}
\item Wrap breaks lines at white space or hyphen.
\item Chop breaks lines at the maximum column limit.
\end{itemize}

\item \textbf{F2 (Recorder) / ON{}-MENU{}-PLUS (Player): }
cycles Line mode through Normal, Join and Expand:

\begin{itemize}
\item Normal breaks lines at newline characters.
\item Join ignores unpaired newline characters  (i.e., joins lines). Useful for
adopting the orphans that occur with e{}-mail style (i.e.,pre{}-wrapped) text files.
\item Expand doubles unpaired newlines (i.e., adds a blank line). Useful
for making the paragraphs clearer in some book style text files.
\end{itemize}

\item \textbf{F3 (Recorder) / ON{}-PLUS (Player):} 
toggles View mode between Narrow and Wide:

\begin{itemize}
\item Narrow sets maximum column to the screen width.
\item Wide sets maximum column to 114. Useful for navigating large
files. (Currently, Wide and Join cannot be selected together.)
\end{itemize}

\item \textbf{ON{}-F1 (Recorder):} 
toggles Page mode between Normal and Overlap:

\begin{itemize}
\item Normal sets page{}-down/page{}-up to one full screen.
\item Overlap tells page{}-down/page{}-up to retain one line from
previous screen.
\end{itemize}

\item \textbf{ON{}-F3 (Recorder):} 
toggles Scrollbar mode, for the current View mode.

\begin{itemize}
\item Narrow mode has no scrollbar by default, until toggled.
\item Wide mode has a scrollbar by default, until toggled.
\item If file fits on one screen, there is no scrollbar and ON{}-F3 has
no effect.
\end{itemize}
\end{itemize}

Settings are not remembered after the viewer has been exited. 
Keys are as follows:

\subsubsection{Recorder}

\begin{table}[h!]
\begin{tabular}{|c|c|}
\hline
KEY & ACTION \\\hline
UP & Page{}-up (one screen up) \\\hline
DOWN & Page{}-down (one screen down) \\\hline
LEFT & Top of file (Narrow mode) One screen left (Wide mode) \\\hline
RIGHT & Bottom of file (Narrow mode) One screen right (Wide mode) \\\hline
ON{}-UP & One line up \\\hline
ON{}-DOWN & One line down \\\hline
ON{}-LEFT & One column left \\\hline
ON{}-RIGHT & One column right \\\hline
OFF & Exit text viewer \\\hline
\end{tabular}
\end{table}

\subsubsection{Player}

\begin{table}[h!]
\begin{tabular}{|c|c|}
\hline
KEY & ACTION \\\hline
MINUS & Page{}-up (one screen up) \\\hline
PLUS & Page{}-down (one screen down) \\\hline
MENU MINUS & Top of file (Narrow mode) One screen left (Wide mode) \\\hline
MENU PLUS & Bottom of file (Narrow mode) One screen right (Wide mode) \\\hline
STOP & Exit text viewer \\\hline
\end{tabular}
\end{table}

\subsubsection{Compatibility}

\begin{itemize}
\item Correctly reads plain text files in Unix, Win/DOS, or Macintosh
format. Latin{}-alphabet Unicode files are  a l m o s t  r e a d a b l
e.
\item Currently prefers fixed{}-width fonts. With proportional fonts,
pretends all characters are the width of a lower{}-case 'o'.
\item Currently messages are in English 
\item Does not currently support right{}-to{}-left languages.
\end{itemize}



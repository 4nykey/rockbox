% $Id$ %
\section{\label{ref:PlaybackOptions}Playback Options}
  The \setting{Playback Options} submenu allows you to configure settings
  related to audio playback.
  \begin{description}
    \item[Shuffle: ]Alters how Rockbox will select which next song to play.\\
      Options: \setting{On}/\setting{Off}.
      %
    \item[Repeat: ]Configures settings related to repeating of directories or
        playlists.\\
      Options: \setting{Off} / \setting{All} / \setting{One} / \setting{Shuffle}
        \nopt{ondiosp,ondiofm}{\setting{/A--B}}:
      \begin{description}
        %
        \item[Off: ]The current directory or playlist will not repeat 
          when it is finished.
          \note{If you have the \setting{Auto change directory} option set to
          \setting{Yes}, Rockbox will move on to the next directory on your
          hard drive. If the \setting{Auto Change Directory} option is set to
          \setting{No}, playback will stop when the current directory or
          playlist is finished.}
        %
        \item[All: ]The current directory or playlist will repeat when it is
          finished.
          \note{This option does \emph{not} shuffle all files on your \dap.
          Rockbox is playlist oriented. When you play a song, a directory, or
          an album, Rockbox creates a playlist and plays it. Thus, to shuffle
          all songs on the \dap, you need to create a playlist of all songs on
          the player, and play that playlist with shuffle mode set to
          \setting{All}.}
        %
        \item[One: ]Repeat one track over and over.
        %
        \item[Shuffle: ]When the current directory or playlist has finished
          playing, it will be shuffled and then repeated.
        %
        \nopt{ondiosp,ondiofm}{
        \item[A--B: ]Repeats between two user defined point within a track,
          typically used by musicians when attempting to learn a piece of music.
          This option is more complicated to use that the others as the \dap\
          must first be placed into A--B repeat mode and then the start and end
          points defined.\\
              \fixme{
              Hold Play and press Left  --- Sets Start Point (A)\\
              Hold Play and press Right --- Sets End Point (B)\\
              }
        }
      \end{description}

    \item[Play Selected First: ]This setting controls what happens when you
      select a file for playback while shuffle mode is on. If the
      \setting{Play Selected First} setting is \setting{Yes}, the file you
      selected will be played first. If this setting is \setting{No}, a random
      file in the directory will be played first.
   
    \item[Resume: ]Rockbox can be configured to start playing automatically
      when you turn on the \dap. If the resume function is set to start
      automatically playing, Rockbox will start at the point where you last
      turned off the \dap. The options for the \setting{Resume} function are:
      \begin{description}
        \item[Yes: ]Rockbox will unconditionally try to resume.
        \item[No:  ]Rockbox will not resume. 
        \emph{If resume is set to \setting{No}, Rockbox will start in the
          \setting{File Browser}.}
      \end{description}
      \note{Earlier versions of Rockbox had an ``Ask'' setting, which would ask whether 
      to resume when the jukebox was turned on. This setting has been eliminated because it 
      was redundant. If resume is set to ``Yes'' pressing 
      \opt{PLAYER_PAD,RECORDER_PAD,IAUDIO_X5_PAD,IPOD_4G_PAD}{\fixme{FixMe}}
      \opt{ONDIO_PAD}{\fixme{FixMe}}
      \opt{IRIVER_H100_PAD,IRIVER_H300_PAD}{\ButtonOn}
      on the \dap\ will resume from the point where the \dap\ was stopped before shutdown.

    \item[Fast-Forward/Rewind: ]How fast you want search (fast forward or rewind) to accelerate 
      when you hold down the button. \setting{Off} means no acceleration. \setting{2x/1s} means double 
      the search speed once every second the button is held. \setting{2x/5s} means double the 
      search speed once every 5 seconds the button is held.

    \item[Anti-Skip Buffer: ]This setting allows you to control how much music is stored
      in the \dap's memory whilst playing a song, acting as a buffer against shock or
      playback problems. The \dap\ transfers the selected amount of the forthcoming song
      into its memory at high speed whilst you are playing the song. It keeps a ``rolling''
      buffer, which keeps feeding more of the forthcoming song into memory as it goes along.
      If the \dap\ is knocked, shaken or jogged heavily while Rockbox is trying to read the
      hard drive, Rockbox might not be able to read the drive. Rockbox will retry over and
      over again until it succeeds, but may eventually reach the end of the memory buffer.
      When that happens, Rockbox must stop playing and wait for more data from the disk,
      which causes your music to skip. The anti-skip setting tells Rockbox how much extra
      buffer memory to spare to handle this situation. This setting therefore allows you to
      reduce the chances of there being a gap or pause during playback of songs.

    \opt{MASCODEC}{The anti-skip buffer can be set to a value between 0 and 7 seconds.}
    \opt{SWCODEC}{The anti-skip buffer can be set to various values between
      5 seconds and 10 minutes.}

    \note{Having a large anti-skip buffer tends to use more power, and may
      reduce your battery life. It is recommended to always use the lowest
      possible setting that allows correct and continuous playback.}

    \item[Fade On Stop/Pause: ]Enables and disables a fade effect when you
      pause or stop playing a song. If the Fade on Stop/Pause option is
      set to \setting{Yes}, your music will fade out when you stop or pause playback,
      and fade in when you resume playback.

    \item[Party Mode: ]Enables unstoppable music playback.  When new songs are
      selected, they are added to the end of the current dynamic playlist
      instead of being played immediately.
      The \fixme{PLAY} and \fixme{STOP} buttons are disabled.

    \opt{SWCODEC}{
    \item[Crossfade: ]
      This setting enables a cross-fader. At the end of a song, the song will fade out as the
      next song fades in, creating a smooth transition between songs.\\
      Options:
    \begin{description}
      \item[Enable Crossfade: ]If set to \setting{Off}, crossfade is disabled. If set to \setting{Always},
        songs will always crossfade into one another. If set to \setting{Shuffle}, crossfade is 
        enabled when the shuffle feature is set to \setting{Yes}, but disabled otherwise. If set to
        track skip only, tracks will only crossfade when you manually change tracks.
        %
      \item[Fade In Delay: ]The ``fade in delay'' is the length of time between when the crossfade
        process begins and when the new track begins to fade in.
        %
      \item[Fade In Duration: ]The length of time, in seconds, that it takes your music to fade in.
        %
      \item[Fade Out Delay: ]The ``fade out delay'' is the length of time between when the crossfade
        process begins and when the old track begins to fade out.
        %
      \item[Fade Out Duration: ]The length of time, in seconds, that it takes your music to fade out.
        %
      \item[Fade Out Mode: ]If set to \setting{Crossfade}, one song will fade out and the next song will 
        simultaneously fade in. If set to \setting{Mix}, the ending song will continue to play as normal
        until its end, while the starting song will fade in from under it. \setting{Mix} mode is not
        used for manual track skips, even if it is selected here.
    \end{description}
  
    \note{The crossfade setting is particularly effective when the player is set on shuffle.}
    }
  
    \opt{SWCODEC}{
      \item[Replaygain: ]This allows you to control the replaygain function.
        The purpose of replaygain is to adjust the volume of the music played 
        so that all songs (or albums, depending on your settings) have the 
        same apparent volume.  This prevents sudden changes in volume when 
        changing between songs recorded at different volume levels.
        For replaygain to work, the songs must have been processed by a program
        that adds replaygain information to the ID3 tags (vorbis tags 
        respectively).
        \note{APEv2 tags are not currently supported.}

        Options for replaygain are:
        \begin{description}
          \item[Enable Replaygain: ]This turns on/off the replaygain function.
            %
          \item[Prevent Clipping: ]Avoid clipping of a song's waveform.
            If a song would clip during playback, the volume is lowered for 
            that song. Replaygain information is needed for this to work.
            %
          \item[Replaygain type: ]Choose the type of replaygain to apply:
            \begin{description}
              \item[Album Gain: ]Maintain a constant volume level between 
                albums, but keep any intentional volume variations between 
                songs in an album. (If album gain value is not available,
                uses track gain information).
                %
              \item[Track Gain: ]Maintain a constant volume level between 
                tracks. If track gain value is not available, no replaygain 
                is applied.
                %
              \item[Track Gain if Shuffling: ]Maintains a constant volume 
                between tracks if shuffle mode is selected. Reverts to album
                mode if shuffle is off.
            \end{description}
            %
          \item[Pre-Amp: ]This allows you to adjust the volume when replaygain
            is applied. Replaygain often lowers the volume, sometimes quite
            much, so here you can compensate for that. Please note that a 
            (large) positive pre-amp setting can cause clipping, unless 
            prevent clipping is enabled.  The pre-amp can be set to any 
            decibel (dB) value between -12dB and +12dB, in increments of 0.1{}dB.
        \end{description}
        }
    }
    \item[Auto Change Directory: ]Control what Rockbox does when it reaches the end
      of a directory. If Auto Change Directory is set to \setting{Yes}, Rockbox will 
      continue to the next directory. If \setting{Auto Change Directory} is set to \setting{No},
      playback will stop at the end of the current directory.
      \note{You must have the \setting{Repeat} option set to \setting{No} for \setting{Auto Change Directory}
      to function properly.}
      %
	\end{description}

\chapter{Advanced Topics}

\section{\label{ref:CustomisingUI}Customising the userinterface}
\subsection{\label{ref:GettingExtras}Getting Extras (Fonts, Languages)}
Rockbox supports custom fonts (for the Recorder and Ondio only) and a number of
different languages. Rockbox comes with several fonts and languages already 
included. If new fonts have been created, then they will be found in the font 
package at \url{http://www.rockbox.org/daily.shtml}. The latest \fname{.lng} 
files are always included in the daily Rockbox builds.

\opt{HAVE_LCD_BITMAP}{
  \subsection{\label{ref:Loadingfonts}Loading Fonts}
  Rockbox can load fonts dynamically. Simply copy the \fname{.fnt} file to the 
  \dap\ and ``play'' them in the directory browser or select 
  \setting{General Settings $\rightarrow$ Fonts} from the Main Menu.
  If you want a font to be loaded automatically every time you start up,
  it must be located in the \fname{/.rockbox } folder and the filename
  must be at most 24 characters long.
  \warn{Advanced Users Only: Any BDF font file up to 16 pixels high should 
    be usable with Rockbox. To convert from \fname{.bdf} to \fname{.fnt}, use 
    the \fname{convbdf} tool. This tool can be found in the \fname{tools} 
    directory of the Rockbox source code.}
}

\subsection{\label{ref:Loadinglanguages}Loading Languages}
Rockbox can load language files at runtime. Simply copy the \fname{.lng} file 
\emph{(do not use the .lang file)} to the \dap\ and ``play'' it in the 
Rockbox directory browser or select \setting{General Settings $\rightarrow$ 
Languages }from the Main Menu.

\note{If you want a language to be loaded automatically every time you start 
up, it must be located in the \fname{/.rockbox }folder and the filename must 
be a maximum of 24 characters long.}

If your language is not yet supported and you want to write your own language
file find the instructions on the Rockbox website:
\wikilink{HowtoUpdateLangfile}

\section{\label{ref:ConfiguringtheWPS}Configuring the WPS}

\subsection{WPS -- General Info}

\begin{description}
\item[Description: ] The WPS or While Playing Screen is the name used to describe 
the information displayed on the \daps\ screen whilst an audio track is
being played. The default WPS is a relatively simple screen displaying
Track name, Artist, Album etc. in the default font as a purely text based
layout. There are a number of WPS files included in Rockbox, and you can 
load one of these at anytime by selecting it in
\setting{General Settings $\rightarrow$ Display $\rightarrow$ Browse .wps files}.

\opt{h1xx,h300}{There is a related option to browse \fname{.rwps} files for 
  \daps\ with LCD remote controls installed.  This will load a similar WPS 
  screen for the remote but with usually a simpler and more concise layout.}

\note{``Playing'' a \fname{.wps} from the file browser has the same effect.}

\item [File Location: ]Custom WPS files may be located anywhere on the drive. 
The only restriction is that they must end in \fname{.wps}. When you ``play''
a \fname{.wps} file, it will be used for future WPS screens, and if the 
``played'' \fname{.wps} file is located in the \fname{/.rockbox} folder, it 
will be remembered and used after reboot. The \fname{.wps} filename must be no 
more than 24 characters long for it to be remembered.
\end{description}

\subsection{\label{ref:CreateYourOwnWPS}WPS -- Build Your Own}
Quite simply, enter the WPS code in your favourite text editor, Notepad on
Windows works fine. When you save it, instead of saving it as a \fname{.txt} 
file, save it as a \fname{.wps} file. Example: Instead of \fname{Rockbox.txt}, 
save the file as \fname{Rockbox.wps}. To make sure non english characters 
display correctly in your WPS you must save the .wps file with UTF-8 character 
encoding. This can be done in most editors, for example Notepad in Windows 2000
or XP (but not in 9x/ME) can do this. See appendix \reference{ref:wps_tags} for all
the tags that are available.

\begin{itemize}
  \item All characters not preceded by \% are displayed as typed.
  \item Lines beginning with \# are comments and will be ignored.
  \item Maximum file size used is 
      \opt{recorder,recorderv2fm,ondio,h1xx,h300,ipodcolor,ipodnano}{1600}
      \opt{player}{400} bytes.
      If you have a bigger WPS file, only the first part of it will be 
      loaded and used.
\end{itemize}

\subsubsection{Conditional Tags}

\begin{description}
\item[If/else: ]
Syntax: \config{\%?xx{\textless}true{\textbar}false{\textgreater}}

If the tag specified by ``\config{xx}'' has a value, the text between the 
``\config{{\textless}}'' and the ``\config{{\textbar}}'' is displayed (the true
part), else the text between the ``\config{{\textbar}}'' and the 
``\config{{\textgreater}}'' is displayed (the false part).
The else part is optional, so the ``\config{{\textbar}}'' does not have to be 
specified if no else part is desired. The conditionals nest, so the text in the
if and else part can contain all \config{\%} commands, including conditionals.

\item[Enumerations: ]
Syntax: \config{\%?xx{\textless}alt1{\textbar}alt2{\textbar}alt3{\textbar}\dots{\textbar}else{\textgreater}}

For tags with multiple values, like Play status, the conditional can hold a 
list of alternatives, one for each value the tag can have.
Example enumeration: 
\begin{example}
    \%?mp{\textless}Stop{\textbar}Play{\textbar}Pause{\textbar}Ffwd{\textbar}Rew{\textgreater}
\end{example}

The last else part is optional, and will be displayed if the tag has no value. 
The WPS parser will always display the last part if the tag has no value, or if
the list of alternatives is too short.
\end{description}

\subsubsection{Next Song info}
You can display information about the next song -- the song that is
about to play after the one currently playing (unless you change the
plan).

If you use the upper-case versions of the
three tags: \config{F}, \config{I} and \config{D}, they will instead refer to 
the next song instead of the current one. Example: \config{\%Ig} is the genre 
name used in the next song and \config{\%Ff} is the mp3 frequency.

\note{The next song information \emph{will not} be available at all
times, but will most likely be available at the end of a song. We
suggest you use the conditional display tag a lot when displaying
information about the next song!}

\subsubsection{Alternating sublines}

It is possible to group items on each line into 2 or more groups or 
``sublines''. Each subline will be displayed in succession on the line for a 
specified time, alternating continuously through each defined subline.

Items on a line are broken into sublines with the semicolon
'\config{;}' character. The display time for
each subline defaults to 2 seconds unless modified by using the
'\config{\%t}' tag to specify an alternate
time (in seconds and optional tenths of a second) for the subline to be
displayed. 

Subline related special characters and tags: 
\begin{description}
\item[;] Split items on a line into separate sublines
\item[\%t] Set the subline display time. The
'\config{\%t}' is followed by either integer
seconds (\config{\%t5}), or seconds and tenths of a second (\config{\%t3.5}).
\end{description}

Each alternating subline can still be optionally scrolled while it is
being displayed, and scrollable formats can be displayed on the same
line with non{}-scrollable formats (such as track elapsed time) as long
as they are separated into different sublines.
Example subline definition:
\begin{example}
     %s%t4%ia;%s%it;%t3%pc %pr : Display id3 artist for 4 seconds,
                                 Display id3 title for 2 seconds,
                                 Display current and remaining track time
                                 for 3 seconds,
                                 repeat...
\end{example}

Conditionals can be used with sublines to display a different set and/or number
of sublines on the line depending on the evaluation of the conditional.
Example subline with conditionals:
\begin{example}
    %?it{\textless}%t8%s%it{\textbar}%s%fn{\textgreater};%?ia{\textless}%t3%s%ia{\textbar}%t0{\textgreater}\\
\end{example}

The format above will do two different things depending if ID3 tags are 
present. If the ID3 artist and title are present:
\begin{itemize}
\item Display id3 title for 8 seconds,
\item Display id3 artist for 3 seconds,
\item repeat\dots
\end{itemize}
If the ID3 artist and title are not present:
\begin{itemize}
\item Display the filename continuously.
\end{itemize}
Note that by using a subline display time of 0 in one branch of a conditional,
a subline can be skipped (not displayed) when that condition is met. 

\subsubsection{Using Images}
You can have as many as 52 images in your WPS. There are various ways of 
displaying images:
\begin{enumerate}
  \item Load and always show the image, using the \config{\%x} tag
  \item Preload the image with \config{\%xl} and show it with \config{\%xd}. 
    This way you can have your images displayed conditionally.
  \opt{HAVE_LCD_COLOR}{
  \item Load an image and show as backdrop using the \config{\%X} tag. The 
    image must be of the same exact dimensions as your display.
  }
\end{enumerate}

\optv{HAVE_LCD_COLOR}{
  Example on background image use:
  \begin{example}
    %X|background.bmp|
  \end{example}
  The image with filename \fname{background.bmp} is loaded and used in the WPS.
}

Example on bitmap preloading and use:
\begin{example}
    %x|a|static_icon.bmp|50|50|
    %xl|b|rep\_off.bmp|16|64|
    %xl|c|rep\_all.bmp|16|64|
    %xl|d|rep\_one.bmp|16|64|
    %xl|e|rep\_shuffle.bmp|16|64|
    %?mm<%xdb|%xdc|%xdd|%xde>
\end{example}
Four images at the same x and y position are preloaded in the example. Which 
image to display is determined by the \config{\%mm} tag (the repeat mode).

\subsubsection{Example File}
\begin{example}
    %s%?in<%in - >%?it<%it|%fn> %?ia<[%ia%?id<, %id>]> 
    %pb%pc/%pt
\end{example}
That is, ``tracknum -- title [artist, album]'', where most fields are only
displayed if available. Could also be rendered as ``filename'' or ``tracknum --
title [artist]''.

%\opt{HAVE_LCD_BITMAP}{
%  \begin{verbatim}
%    %s%?it<%?in<%in. |>%it|%fn>
%    %s%?ia<%ia|%?d2<%d2|(root)>>
%    %s%?id<%id|%?d1<%d1|(root)>> %?iy<(%iy)|>
%  
%    %al%pc/%pt%ar[%pp:%pe]
%    %fbkBit %?fv<avg|> %?iv<(id3v%iv)|(no id3)>
%    %pb
%    %pm
% % \end{verbatim}
%}

\section{\label{ref:manage_settings}Managing Rockbox settings}

	\subsection{Introduction to \fname{.cfg} files.}
	Rockbox allows users to store and load multiple settings through the use of 
	configuration files.  A configuration file is simply a text file with the 
	extension \fname{.cfg}.  

	A configuration file may reside anywhere on the hard disk. Multiple
  configuration files are permitted. So, for example, you could have
  a \fname{car.cfg} file for the settings that you use while playing your
  jukebox in your car, and a \fname{headphones.cfg} file to store the
  settings that you use while listening to your \dap\ through headphones.

	See \reference{ref:cfg_specs} below for an explanation of the format 
	for configuration files.  See \reference{ref:manage_settings_menu} for an 
	explanation of how to create, edit and load configuration files.
	
	\subsection{\label{ref:cfg_specs}Specifications for \fname{.cfg}
	files.}
	
  The Rockbox configuration file is a plain text file, so once you use the 
  \setting{Write .cfg file} option to create the file, you can edit the file on 
  your computer using any text editor program. See 
  Appendix \reference{ref:config_file_options} for available settings. Configuration 
  files use the following formatting rules: % 
  
  \begin{enumerate} 
  	\item Each setting must be on a separate line. 
  	\item Each line has the format ``setting: value''. 
  	\item Values must be within the ranges specified in this manual for each 
  	setting. 
  	\item Lines starting with \# are ignored. This lets you write comments into 
  	your configuration files. 
  \end{enumerate}

	Example of a configuration file:
		\begin{example}
	  # Example configuration file
	  # volume: 70
	  # bass: 11
	  # treble: 12
	  # balance: 0
	  # time format: 12hour
	  # volume display: numeric
	  # show files: supported
	  # wps: /.rockbox/car.wps
	  # lang: /.rockbox/afrikaans.lng
		\end{example}

    \note{As you can see from the example, configuration files do not need to 
    contain all of the Rockbox options.  You can create configuration files 
    that change only certain settings. So, for example, supppose you 
    typically use the \dap at one volume in the car, and another when using 
    headphones. Further, suppose you like to use an inverse LCD when you are 
    in the car, and a regular LCD setting when you are using headphones. You 
    could create configuration files that control only the volume and LCD 
    settings. Create a few different files with different settings, give 
    each file a different name (such as \fname{car.cfg}, 
    \fname{headphones.cfg}, etc.), and you can then use the \setting{Browse .cfg 
    files} option to quickly change settings.} 
  
	\subsection{\label{ref:manage_settings_menu}The \setting{Manage Settings} 
	menu} The \setting{Manage Settings} menu can be found in the \setting{Main 
	Menu}.  The \setting{Manage Settings} menu allows you to save and load 
	\fname{.cfg} files.  \opt{MASCODEC}{The \setting{Manage Settings} menu also 
	allows you to load or save different firmware versions.} 

	\begin{description}
		
		\item [Browse .cfg 	Files.]Opens the file browser in the 
		\fname{/.rockbox} directory and displays all \fname{.cfg} (configuration) 
		files. Selecting a \fname{.cfg}	file will cause Rockbox to load the 
		settings contained in that file.  Pressing \ButtonLeft\ will exit back to 
		the \setting{Manage Settings} menu.  See the \setting{Write .cfg files} 
		option on the \setting{Manage Settings} menu for details of how to save 
		and edit a configuration file. 
		
		\item [Browse	Firmwares.]
			%
		 	\opt{SWCODEC}{\fixme{This is a legacy item,	and is deprecated.}} 
		 	%	
	 		\opt{MASCODEC}{
	 			This displays a list of firmware files in the \fname{/.rockbox} 
	 			system directory. 
	 			% 
				\opt{recorder,recorderv2fm}{Firmware files have an extension of 
				\fname{.ajz}. }			
				%
				\opt{player,ondio}{Firmware files have an extension	of \fname{.mod}. }
				%
				Playing a firmware file loads it into	memory.  Thus, it is possible 
				to run the original Archos	firmware or a different version of Rockbox 
				from here (assuming	that you have the right files installed on your 
				disk. There is no need for any other file or directory to be 
				installed to use this option; the firmware is resident in that one 
				file. 
				}
		
		\item [Reset Settings.]This wipes the saved settings in the \dap\ and 
		resets all settings to their default values. 
		
			\opt{h100,h300}{\note{You can also reset all settings to their default 
			values by turning	off the \dap\, turning it	back on, and pressing the 
			\ButtonRec button immediately after the \dap\ turns on.} 
			} 
			\opt{ipod}{\note{You can also reset all settings to their default values 
			by turning	off the \dap\, and turning it	back on with the hold button 
			on.}
			}
			
			\item [Write .cfg file.]This option writes a \fname{.cfg} file to 
			your \daps\ hard	disk.  The configuration file has the \fname{.cfg} 
			extension and is used to store all of the user settings that are described 
			throughout this manual.
    
			Hint: Use the \setting{Write .cfg file} feature (\setting{Main 
			Menu $\rightarrow$ General Settings}) to save the current settings, then 
			use a text editor to customize the settings file. See Appendix 
			\reference{ref:config_file_options} for the full reference of available 
			options.

	\end{description}

\section{\label{ref:PartISection1}Differences between binaries}
There are 3 different types of firmware binaries from Rockbox website: 
Current Version, Daily Builds and Bleeding Edge.

\begin{description}
\item[Current Version.] The current version is the latest stable version
developed by the Rockbox Team. It's free of known critical bugs.  It is
available from \url{http://www.rockbox.org/download/}.
\item[Daily Builds.] The Daily Build is a development version of Rockbox. It
supports all new features and patches developed since last stable version. It
may also contain bugs! This version is generated automatically every day 
and can be found at \url{http://www.rockbox.org/daily.shtml}.
\item[Bleeding Edge.] Bleeding edge builds are the same as the Daily build, 
but built from the latest development on each commit to the CVS repository.
These builds are for people who want to test the code that developers just
checked in.
\end{description}
\note{If you don't want to get undefined behaviour from your \dap\ you should
really stick to the Current Version. Development versions may have lots of
changes so they may behave completely different than described in this manual,
introduce new (and maybe annoying) bugs and similar. If you want to help the
project development you can try development builds and help by reporting bugs,
feature requests and so so. But be aware that using a development build may
eat also some more time.}

\section{\label{ref:FirmwareLoading}Firmware Loading}
\opt{player,recorder,recorderv2fm,ondio}{
  When your \dap\ powers on, it loads the Archos firmware in ROM, which
  automatically checks your Jukebox hard disk's root folder for a file named 
  \firmwarefilename. Note that Archos firmware can only read the first 
  ten characters of each filename in this process, so don't rename your old 
  firmware files with names like \firmwarefilename.\fname{old} and so on, 
  because it's possible that the \dap\ will load a file other than the one you 
  intended.
}

\subsection{\label{ref:using_rolo}Using ROLO (Rockbox loader)}
Rockbox is able to load and start another firmware file without rebooting. 
You just ``play'' a file with the extension %
\opt{recorder,recorderv2fm,ondio}{\fname{.ajz}.} %
\opt{player}{\fname{.mod}.} %
\opt{iriver}{\fname{.iriver}.} %
\opt{ipod}{\fname{.ipod}.} %
\opt{iaudio}{\fname{.iaudio}.} %
This can be used to test new firmware versions without deleting your
current version.

\opt{archos}{\section{\label{ref:Rockboxinflash}Rockbox in flash}
\textbf{FLASHING ROCKBOX IS OPTIONAL!} It is not required for using
Rockbox on your Jukebox Recorder. Please read the whole section
thoroughly before flashing.

\subsection{\label{ref:PartISection61}Introduction}
Flashing in the sense used here and elsewhere in regard to Rockbox means
reprogramming the flash memory of the Jukebox unit. Flash memory
(sometimes called ``Flash ROM'') is a type of
non{}-volatile memory that can be erased and reprogrammed in circuit. It is a variation of electrically erasable
programmable read{}-only memory (EEPROM). 

A from the factory Jukebox comes with the Archos firmware flashed. It is
possible to replace the built{}-in software with Rockbox. 

Terminology used in the following:\newline
\textbf{Firmware} means the flash ROM content as a whole.\newline
\textbf{Image} means one operating software started from there. 

By reprogramming the firmware,  the Jukebox will boot much faster. The
Archos boot loader seems to take forever compared to the Rockbox
version. In fact, the Rockbox boot loader is so fast that it has to
wait for the disk to spin up.  The flashing procedure is a bit involved
for the first time, updates are very simple later on. 

\subsection{\label{ref:Method}Method}
The replaced firmware will host a bootloader and 2 images. This is made
possible by compression. The first is the
``permanent'' backup. The second is the
default image to be started.  The former is only used when you hold the
F1 key during start, and is the original Archos firmware, the second is
a current build of Rockbox. This second image is meant to be
reprogrammed whenever a Rockbox upgrade is performed.

There are two programming tools supplied: 

\begin{itemize}
\item The first one is called \textbf{firmware\_flash.rock} and is used
to program the whole flash with new content.  It can also be used to
revert back to the original firmware that is backed up as part of this
procedure.  This tool will only be needed once, and can be viewed as
``formatting'' the flash with the desired image structure. 
\item The second one is called \textbf{rockbox\_flash.rock }and is used
to reprogram only the second image. If the resulting programmed
firmware image is not operational, it is
possible to hold down the F1 key while booting to start the Jukebox
with the Archos firmware and Rockbox booted from disk to reinstall a
working firmware image. 
\end{itemize}

\subsubsection{\label{ref:PartISection63}Risks}
Well, is it dangerous? Yes, certainly, like programming a
mainboard BIOS, CD/DVD drive firmware,
mobile phone, etc. If the power fails, the chip malfunctions while
programming or particularly if the programming software malfunctions,
your Jukebox may stop functioning. The Rockbox team take no
responsibility of any kind {}- do this at your own risk. 

However, the code has been extensively tested and is known to work well.
 The new firmware file is completely read before it starts programming,
there are a lot of sanity checks. If any fail, it will not program.
There is no reason why such low level code should behave differently on
your Jukebox. 

There's one ultimate safety net to bring back Jukeboxes
with even completely garbled flash content: the UART boot mod, which in
turn requires the serial mod. This can bring the dead back to life,
with that it's possible to reflash independently from the outside, even
if the flash is completely erased. It has been used during development,
else Rockbox in flash wouldn't have been possible.
Extensive development effort went into the development of the UART boot
mod.  Mechanically adept users with good soldering skills can easily
perform these mods. Others may feel uncomfortable using the first tool
(\textbf{firmware\_flash.rock}) for reflashing the firmware.

If you are starting with a known{}-good image, you are unlikely to
experience problems.  The flash tools have been stable for quite a
while. Several users have used them extensively, even flashing while
playing! Although it worked, it's not the recommended
method.  

The flashing software is very paranoid about making sure that the
correct flash version is being installed.  If the wrong file is used,
it will simply refuse to flash the Jukebox.

About the safety of operation: Since the Rockbox boot code gives ``dual
boot'' capability, the Archos firmware is still there when you hold F1
during startup. So even if you have problems with Rockbox from flash, you can still use
the Jukebox, reflash the second image with an updated Rockbox copy,
etc. 

The flash chip being used by Archos is specified for 100,000 cycles, so
it's very unlikely that flashing it will wear it out. 

\subsection{\label{ref:Requirements}Requirements}
You need two things: 

\begin{itemize}
\item The first is a Recorder or FM model, or an Ondio SP or FM. Be sure
you're using the correct package, they differ
depending on your precise hardware! The technology works for the Player
models, too. Players can also be flashed, but Rockbox does not run
cold{}-started on those, yet. 
\item Second, you need an in{}-circuit programmable flash. Chances are
about 85\% that you have, but Archos also used an older flash chip
which can't do the trick. You can find out via Rockbox
debug menu, entry Hardware Info. If the flash info gives you question
marks, you're out of luck. The only option for
flashing if this is the case is to solder in the right chip
(SST39VF020), preferably with the firmware already in. If the chip is
blank, you'll need the UART boot mod as well. 
\end{itemize}
\subsubsection{\label{ref:FlashingProcedure}Flashing Procedure}
Here are step{}-by{}-step instructions on how to flash and update to a
current build.  It is assumed that you can install and operate Rockbox
the usual way. The flashing procedure has a lot of failsafes, and will
check for correct model, file, etc. {}- if something is incompatible it
just won't flash, that's all. 

Now here are the steps: 

\textbf{Preparation}

Install (with all the files, not just the .ajz) and use the current
daily build you'd like to have. Enable any voice
features that are helpful throughout the process, such as menus and
filename spelling. Set the file view to show all files, with  the menu
option \textbf{General Settings {}-{\textgreater} File View
{}-{\textgreater} Show Files} set to ``all''.
Have the Jukebox nicely charged to avoid
running out of power during the flash write.  Keep the Jukebox plugged
into the charger until flashing is complete.

{\bfseries
Backup }

Backup the existing flash content.  This is not an essential part of the
procedure, but is strongly recommended since you will need these files
if you wish to reverse the flashing procedure, or if you need to update
the bootloader (as opposed to the firmware) in the future.  Keep them
safe!

Access the main menu by pressing F1 then select \textbf{Info
{}-{\textgreater} Debug}.  Select the first entry, \textbf{Dump ROM
contents}, by pressing Play one more time. The disk should start to
spin. Wait for it to settle down, then plug in the USB cable  to copy
the dump file this has just been created to your PC. The main folder of
your Jukebox now should contain two strange .bin files. Copy the larger
one named
\textbf{internal\_rom\_2000000{}-203FFFF.bin}
to a safe place, then delete them both from the box. 

{\bfseries
Copy the new flash content file to your box }

Depending on your model (recorder, FM, V2 recorder), download one of the
3 packages: 

\url{http://joerg.hohensohn.bei.t-online.de/archos/flash/flash_rec.zip}

\url{http://joerg.hohensohn.bei.t-online.de/archos/flash/flash_fm.zip}

\url{http://joerg.hohensohn.bei.t-online.de/archos/flash/flash_v2.zip}
\url{http://joerg.hohensohn.bei.t-online.de/archos/flash/flash_v2.zip}

\url{http://joerg.hohensohn.bei.t-online.de/archos/flash/flash_v2.zip}

\url{http://joerg.hohensohn.bei.t-online.de/archos/flash/flash_ondiosp.zip}

\url{http://joerg.hohensohn.bei.t-online.de/archos/flash/flash_ondiofm.zip}

The zip archives contain two .bin files each. Those firmware*.bin files
are all we want, copy them to the root directory of your box. The names
differ depending on the model, the flash
plugin will pick the right one, no way of
doing this wrong. 

{\bfseries
Install the Rockbox
Bootloader (``formatting'' the flash)}

This procedure is only necessary the first time you flash Rockbox. 
Unplug the USB cable again, then select \textbf{Browse
}\textbf{Plugins}\textbf{ } from the main menu (F1).  Locate \textbf{firmware\_flash.rock}, and start it with PLAY.  Rockbox now displays an info screen, press F1 to acknowledge it and start a file check. Again wait for the disk to
settle, then press F2 to proceed to a warning message (if the plugin
has exited, you don't have the proper file) and F3 to actually program
the file. This takes maybe 15 seconds, wait for the disk to settle
again. Then press a key to exit the plugin.

{\centering\itshape
  [Warning: Image ignored] % Unhandled or unsupported graphics:
%\includegraphics[width=3.609cm,height=2.062cm]{images/rockbox-manual-img75.png}
     [Warning: Image ignored] % Unhandled or unsupported graphics:
%\includegraphics[width=3.669cm,height=2.097cm]{images/rockbox-manual-img76.png}
 \textmd{  }  [Warning: Image ignored]
% Unhandled or unsupported graphics:
%\includegraphics[width=3.739cm,height=2.136cm]{images/rockbox-manual-img77.png}
 \newline
Flashing boot loader in 3 easy steps
\par}

{\bfseries
\label{ref:FlashingRockbox}Install the Rockbox binary in flash}

All the above was necessary only once, although there will not be any
obvious difference (other than the Archos firmware loading a bit more quickly)
after the step above is complete.  Next install the actual Rockbox firmware thatwill be used from ROM.  This is how Rockbox will be updated when
installing a new release from now on. 

\begin{itemize}
\item Unpack the whole build that you are installing onto the Jukebox,
including plugins and support files.  This can be done using the Windows setup program to install the new version onto the Jukebox.
\item Test the build you are going to flash by playing the .ajz file so
that ROLO loads it up.  This puts the firmware in memory without
changing your flash, so you can check that everything is working.  If
you have just installed the bootloader (see above) then this will happen automatically as the existing Archos firmware loads the .ajz that you have just installed.  If upgrading ROMbox, this step \textbf{must }be carried out since Rockbox cannot overwrite the ROM while it is running from it.  
\item Play the .ucl file, which is usually found in the
\textbf{/.rockbox} directory, this will kick off the
\textbf{rockbox\_flash.rock} plugin. It's a bit
similar to the other one, but it's made different to
make the user aware. It will check the file, available size, etc. With
F2 it begins programming, there is no need for warning this time. If it
goes wrong, you'll still have the permanent image. 

{\centering\itshape
  [Warning: Image ignored] % Unhandled or unsupported graphics:
%\includegraphics[width=3.53cm,height=2.016cm]{images/rockbox-manual-img78.png}
 \textmd{  }  [Warning: Image ignored]
% Unhandled or unsupported graphics:
%\includegraphics[width=3.528cm,height=2.016cm]{images/rockbox-manual-img79.png}
 \newline
Using rockbox\_flash to update your boot firmware
\par}
\item It is possible that you could get an ``Incompatible
Version'' error if the plugin interface has changed since
you last flashed Rockbox. This means you are running an
``old'' copy of Rockbox, but are trying to
execute a newer plugin, the one you just downloaded. The easiest
solution is to ROLO into this new version,
by playing the\textbf{ ajbrec.ajz }file. Then you are consistent and can play
\textbf{rockbox.ucl}. 
\item When done, you can restart the box and hopefully your new Rockbox
image. 
\end{itemize}
UCLs for the latest Recorder and FM firmware are included in Rockbox 2.4
and also the daily builds.

\subsection{\label{ref:KnownIssuesAndLimits}Known Issues and Limitations}
There are two variants as to how the Jukebox starts, which is why there
are normal and \_norom firmware files. The vast majority of Jukeboxes
all have the same boot ROM content, but some have different flash
content. Rockbox identifies this boot ROM with a CRC value of 0x222F in
the hardware info screen. Some recorders have the boot ROM disabled (it
might be unprogrammed) and start directly from a flash mirror at
address zero. They need the \_norom firmware, it has a slightly
different bootloader. Without a boot ROM there is no UART boot safety
net. To compensate for that as much as possible the MiniMon monitor is
included, and can be started by pressing  F3+ON. Using this the box can
be reprogrammed via serial if the UART mod has been applied and the
first \~{}2000 bytes of the flash are OK. 

\subsubsection{ROMbox}
ROMbox is a flashable version of Rockbox that is
uncompressed and runs directly from the flash chip rather than being
copied into memory first.  The advantage of this is that memory that
would normally be used for storing the Rockbox code can be used for
buffering MP3s instead, resulting in less disk
spin{}-ups and therefore longer battery life
 Unfortunately being uncompressed, ROMbox requires more space in flash
than Rockbox and will therefore not fit in the space that is left on an
FM recorder.  ROMbox therefore runs on the V1 and V2 recorder models
only.

The procedure for flashing ROMbox is identical to the procedure for
flashing Rockbox as laid out on page \pageref{ref:FlashingRockbox}. 
The only difference is that the file to install is called
\textbf{rombox.ucl}.  ROMbox is included automatically with rockbox 2.4
and all the current daily builds, so the procedure is identical
otherwise.}

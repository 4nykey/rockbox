\subsection{\label{ref:Sudoku}Sudoku}
\screenshot{plugins/images/ss-sudoku}{Sudoku}{fig:sudoku}
Sudoku in Rockbox is implemented as both a plugin and a viewer.
When you start Sudoku in plugin mode from the \setting{Browse Plugins} menu, a 
random game will be generated automatically, and an estimate of its difficulty
(very easy, easy, medium, hard or fiendish) will be displayed on the screen.
New games can be generated from the \setting{Generate} menu option.

When you use Sudoku as a viewer for playing pre-generated Sudoku games.
You need sudoku games stored (one game per file) in text files with the
\fname{.ss} extension (see links below). You then browse these games using the
normal \setting{File Browser}, and open the file to launch Sudoku.

You can create and save your own grids under the \setting{New} menu option.
Enter the menu (as described in the key table below) when you have finished and
 enter the full path to save to including the \fname{.ss} extension 
 (e.g. \fname{/sudoku/new.ss}).

\subsubsection{The thing on the left (AKA the scratchpad)}
When you play Sudoku on paper most people like to mark numbers in cells that 
are possible candidates for the cells.
This can be done with the column on the left. Change the number
under the cursor to a number which might be valid and press the scratchpad
button, the number will then be added on the left.
The column is stored seperatly for every cell on the board.
These are \emph{NOT} saved when saving the game.

\begin{table}
    \begin{btnmap}{}{}
    \opt{RECORDER_PAD,ONDIO_PAD,IRIVER_H100_PAD,IRIVER_H300_PAD,IAUDIO_X5_PAD,SANSA_E200_PAD,SANSA_C200_PAD,GIGABEAT_PAD,MROBE100_PAD}
        {\ButtonUp{} / \ButtonDown{} / \ButtonLeft{} / \ButtonRight}
    \opt{IPOD_4G_PAD,IPOD_3G_PAD}{\ButtonScrollFwd{} / \ButtonScrollBack}
    \opt{IRIVER_H10_PAD}{\ButtonScrollUp{} / \ButtonScrollDown{} / \ButtonLeft{} / \ButtonRight}
    & Move the cursor\\
    %
    \opt{RECORDER_PAD}{\ButtonPlay}
    \opt{ONDIO_PAD}{\ButtonMenu}
    \opt{IRIVER_H100_PAD,IRIVER_H300_PAD}{\ButtonSelect{} / \ButtonOn}
    \opt{IPOD_4G_PAD,IPOD_3G_PAD}{\ButtonLeft{} / \ButtonSelect{} / \ButtonRight}
    \opt{IAUDIO_X5_PAD,GIGABEAT_PAD,MROBE100_PAD}{\ButtonSelect}
    \opt{IRIVER_H10_PAD}{\ButtonRew}
    \opt{SANSA_E200_PAD}{\ButtonScrollBack{} / \ButtonScrollFwd}
    \opt{SANSA_C200_PAD}{\ButtonSelect{} / \ButtonVolUp{} / \ButtonVolDown}
    & Change number under the cursor\\
    %
    \opt{RECORDER_PAD}{Long \ButtonPlay}
    \opt{ONDIO_PAD}{Long \ButtonMenu+\ButtonDown}
    \opt{IRIVER_H100_PAD,IRIVER_H300_PAD}{Long \ButtonOn}
    \opt{IPOD_4G_PAD,IPOD_3G_PAD}{Long \ButtonLeft{} / \ButtonSelect{} / \ButtonRight}
    \opt{IAUDIO_X5_PAD,GIGABEAT_PAD,MROBE100_PAD}{Long \ButtonSelect}
    \opt{IRIVER_H10_PAD}{Long \ButtonRew}
    \opt{SANSA_E200_PAD}{Long \ButtonScrollBack{} / \ButtonScrollFwd}
    \opt{SANSA_C200_PAD}{Long \ButtonSelect{} / \ButtonVolUp{} / \ButtonVolDown}
    & Constantly changing the number under the cursor\\
    %
    \opt{RECORDER_PAD}{\ButtonFOne}
    \opt{ONDIO_PAD}{Long \ButtonMenu}
    \opt{IRIVER_H100_PAD,IRIVER_H300_PAD}{\ButtonMode}
    \opt{IPOD_4G_PAD,IPOD_3G_PAD,GIGABEAT_PAD,MROBE100_PAD}{\ButtonMenu}
    \opt{IAUDIO_X5_PAD,IRIVER_H10_PAD}{\ButtonPlay}
    \opt{SANSA_E200_PAD}{\ButtonSelect}
    \opt{SANSA_C200_PAD}{\ButtonPower}
    & Open Menu\\
    %
    \opt{RECORDER_PAD}{\ButtonFTwo}
    \opt{ONDIO_PAD}{\ButtonMenu+\ButtonLeft}
    \opt{IRIVER_H100_PAD,IRIVER_H300_PAD,IAUDIO_X5_PAD,SANSA_E200_PAD,SANSA_C200_PAD}{\ButtonRec}
    \opt{IPOD_4G_PAD,IPOD_3G_PAD}{\ButtonPlay}
    \opt{IRIVER_H10_PAD}{\ButtonFF}
    \opt{GIGABEAT_PAD}{\ButtonA}
    \opt{MROBE100_PAD}{\ButtonDisplay}
    & Add/Remove number to scratchpad\\
    %
    \opt{RECORDER_PAD,ONDIO_PAD,IRIVER_H100_PAD,IRIVER_H300_PAD}{\ButtonOff}
    \opt{IAUDIO_X5_PAD,IRIVER_H10_PAD,SANSA_E200_PAD,GIGABEAT_PAD,MROBE100_PAD}{\ButtonPower}
    \opt{IPOD_4G_PAD,IPOD_3G_PAD}{Menu $\rightarrow$ Quit}
    \opt{SANSA_C200_PAD}{Long \ButtonPower}
    & Quit\\
    %
    \end{btnmap}
\end{table}

Some places where can you can find \fname{.ss} files:
\begin{itemize}
\item Simple Sudoku (Advanced Puzzle Packs 1 and 2 located near the bottom of that page):
\url{http://www.angusj.com/sudoku/}
\item Kjell's Sudoku generator/solver:
\url{http://kjell.haxx.se/sudoku/}
\end{itemize}

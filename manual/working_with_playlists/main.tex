\section{Working with Playlists}
\fixme{This section is currently in a half written state, with possible errors 
  and a lot of stuff missing. Please help us fix this chapter by submitting 
  additions/corrections to the tracker}

\subsection{Playlist terminology}
Some common terms that are used in Rockbox when referring to 
playlists:

\begin{description}
\item[Directory.]  A playlist!  One of the keys to getting the most out of 
  Rockbox is understanding that Rockbox \emph{always} considers the song that 
  it is playing to be part of a playlist, and in some situations, Rockbox will 
  create a playlist automatically.  For example, if you are playing the 
  contents of a directory, Rockbox will automatically create a playlist 
  containing the songs in that directory.  This means that just about anything 
  that is described in this chapter with respect to playlists also applies to 
  directories.
  
\item[Dynamic playlist.]  A dynamic playlist is a playlist that is created 
  ``On the fly.''  Any time you insert or queue tracks using the 
  \setting{Playlist submenu} (see \reference{playlist_submenu}), you are 
  creating (or adding to) a dynamic playlist.
	
\item[Insert.] In Rockbox, to \setting{Insert} an item into a playlist means 
  putting an item into a playlist and leaving it there, even after it is 
  played. As you will see later in this chapter, Rockbox can \setting{Insert} 
  into a playlist in several places.
	
\item[Queue.]  In Rockbox, to \setting{Queue} a song means to put the song 
  into a playlist and then to remove the song from the playlist once it has 
  been played.  The only difference between \setting{Insert} and 
  \setting{Queue} is that the \setting{Queue} option removes the song from the 
  playlist once it has been played, and the \setting{Insert} option does not.
\end{description}  

\subsection{Creating playlists}

Rockbox can create playlists in four different ways. 

\subsubsection{By playing a song}

Whenever a song is selected from the \setting{File Browser} using the 
\opt{IRIVER_H100_PAD,IRIVER_H300_PAD}{\ButtonSelect} button, Rockbox will automatically create a playlist 
containing  all of the songs in the directory in which that song is located.

\note{If you already have already created a dynamic playlist, playing a new 
  song will \emph{erase} the current playlist and create a new one.  If you 
  want to add a song to the current playlist rather than erasing the current 
  playlist, see the section below on ``Adding music to playlists.''}

\subsubsection{By using Insert and Queue functions}

\subsubsection{By using the Playlist Catalog}

\subsubsection{By using the Main Menu}

\subsection{Adding music to playlists}
  
\subsubsection{\label{ref:Playlistsubmenu}Adding music to a dynamic playlist} 
\screenshot{rockbox_interface/images/ss-playlist-menu}{The Playlist Submenu}{} 
The \setting{Playlist Submenu} allows you to put tracks into a 
``dynamic playlist''. If there is no music currently playing, Rockbox will 
create a new dynamic playlist and put the selected track(s) into the 
playlist.  If there is music currently playing, Rockbox will put the 
selected track(s) into the current playlist.  The place in which the newly 
selected tracks are added to the playlist is determined by the following 
options:

\begin{description} 
\item [Insert.] Add track(s) to playlist. If no other tracks have been 
  inserted then the selected track will be added immediately after current 
  playing track, otherwise they will be added to end of insertion list. 
  
\item [Insert next.] Add track(s) immediately after current playing 
  track, no matter what else has been inserted. 
  
\item [Insert last.] Add track(s) to end of playlist. 
  
\item [Queue.] Queue is the same as Insert except queued tracks are 
  deleted immediately from the playlist after they've been played. Also, 
  queued tracks are not saved to the playlist file (see 
  \reference{ref:playlistoptions}). 
  
\item [Queue next.] Queue track(s) immediately after current playing track.
  
\item [Queue last.] Queue track(s) at end of playlist. 
\end{description}

The \setting{Playlist Submenu} can be used to add either single tracks or 
entire directories to a playlist. If the \setting{Playlist Submenu} is 
invoked on a single track, it will put only that track into the playlist.  
On the other hand, if the \setting{Playlist Submenu} is invoked on a 
directory, Rockbox adds all of the tracks in that directory to the 
playlist.

Dynamic playlists are saved so resume will restore them exactly as they 
were before shutdown.

\subsection{Saving playlists}

\subsection{Loading saved playlists}

\subsection{Helpful Hints}

\subsubsection{Including subdirectories in playlists} 
You can control whether or not Rockbox includes the contents of 
subdirectories when adding an entire directory to a playlists. 
Set the \setting{Main Menu $\rightarrow$ Playlist Options $\rightarrow$ 
  Recusively Insert Directories} setting to \setting{Yes} if you would like
Rockbox to include tracks in subdirectories as well as tracks in the 
currently selected directory.

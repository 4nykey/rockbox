% $Id$ %
\subsection{MPEG Player}
The Mpeg Player is a video player plugin capable of playing back MPEG-1 and 
MPEG-2 video streams with MPEG audio multiplexed into \fname{.mpg} files.

To play a video file, you just select it in the Rockbox \setting{File Browser}.
If your file does not have the \fname{.mpg} extension but is encoded in the
supported format, you will need to use the \setting{Open With...} context menu
option and choose \setting{mpegplayer}.

\begin{table}
\begin{btnmap}{}{}
    \ActionWpsPlay & Pause/Resume\\
    \ActionWpsStop & Stop\\
    \ActionWpsVolUp{} / \ActionWpsVolDown & Adjust volume up / down\\
    \opt{IRIVER_H300_PAD}{\ButtonMode}
    \opt{IPOD_4G_PAD,IPOD_3G_PAD,GIGABEAT_PAD}{\ButtonMenu}
    \opt{IAUDIO_X5_PAD}{\ButtonRec}
    \opt{IRIVER_H10_PAD}{\ButtonRew}
    \opt{SANSA_E200_PAD}{\ButtonSelect}
    & Open the MPEG Player menu\\
\end{btnmap}
\end{table}

The menu has the following entries.

\begin{description}
\item[Show FPS] This option displays (once a second - if your video is
    full-screen this means it will get overwritten by the video and appear to
    flash once per second) the average number of frames decoded per second,
    the total number of frames skipped (see the Skip Frames option), the
    current time (in 100Hz ticks) and the time the current frame is due to be
    displayed.
\item[Limit FPS] With this option disabled, mpegplayer will display the video
    as fast as it can. Useful for benchmarking.
\item[Skip frames] This option causes mpegplayer to attempt to maintain
    realtime playback by skipping the display of frames - but these frames are
    still decoded. Disabling this option can cause loss of A/V sync.
\end{description}

See this page in the Rockbox wiki for information on how to encode your videos
to the supported format. \wikilink{PluginMpegplayer}


% $Id$ %
\chapter{\label{ref:wps_tags}WPS Tags}
\section{Status Bar}
\begin{tagmap}{}{}
\%we & Status Bar Enabled\\
\%wd & Status Bar Disabled\\
\end{tagmap}
These tags override the player setting for the display of the status bar.
They must be noted on their own line.

\section{ID3 Info}
  \begin{tagmap}{}{}
    \%ia & ID3 Artist\\
    \%ic & ID3 Composer\\
    \%id & ID3 Album Name\\
    \%ig & ID3 Genre Name\\
    \%in & ID3 Track Number\\
    \%it & ID3 Track Title\\
    \%iv & ID3 Version (1.0, 1.1, 2.2, 2.3, 2.4 or empty if no id3 tag)\\
    \%iy & ID3 Year\\
  \end{tagmap}
Remember that this information is not always available, so use the 
conditionals to show alternate information in preference to assuming.

\section{Power Related Information}
  \begin{tagmap}{}{}
    \%bl & Show numeric battery level in percent.\\
         & Can also be used in a conditional: 
           \%?bl{\textless}0{\textbar}1{\textbar}2{\textbar}3{\textbar}4{\textgreater}\\
    \%bv & Show the battery level in volts\\
    \%bt & Show estimated battery time left\\
    \%bp & ``p'' if the charger is connected \\
         & (only on targets that can charge batteries)\\
    \%bc & ``c'' if the unit is currently charging the battery\\
         & (only on targets that have software charge control or monitoring)\\
    \%bs & Sleep timer. Shows the remaining time if the sleeptimer is set\\
  \end{tagmap}

\section{File Info}
  \begin{tagmap}{}{}
    \%fb & File Bitrate (in kbps)\\
    \%fc & File Codec (e.g. ``MP3'' or ``FLAC'')\\
         & This tag can also be used in a conditional tag,\\
         & \%?fc{\textless}mp1{\textbar}mp2{\textbar}mp3{\textbar}wav%
           {\textbar}vorbis{\textbar}flac{\textbar}mpc{\textbar}a52%
           {\textbar}wavpack{\textbar}unknown{\textgreater}\\ %
         & The codec order is as follows: MP1, MP2, MP3, WAV, Ogg Vorbis (OGG),%
           FLAC, MPC, AC3, WavPack (WV), ALAC, AAC, Shorten (SHN), AIFF\\
    \%ff & File Frequency (in Hz)\\
    \%fm & File Name\\
    \%fn & File Name (without extension)\\
    \%fp & File Path\\
    \%fs & File Size (In Kilobytes)\\
    \%fv & ``(avg)'' if variable bit rate or ``'' if constant bit rate\\
    \%d1 & First directory from end of file path.\\
    \%d2 & Second directory from end of file path.\\
    \%d3 & Third directory from end of file path.\\
  \end{tagmap}
Example for the the \%dN commands: If the path is 
``/Rock/Kent/Isola/11 - 747.mp3'', \%d1 is ``Isola'', \%d2 is ``Kent'' \dots
You get the picture.

\section{Playlist/Song Info}
  \begin{tagmap}{}{}
    \%pb & Progress Bar\\
    \opt{player}{
          & This will display a 1 character ``cup'' %
            that empties as the time progresses.}
    \opt{recorder,recorderv2fm,h1xx,h300,ipodcolor,ipodnano}{
         & This will replace the entire line with a progress bar. \\
         & You can set the height, position and width of the progressbar %
           (in pixels): \%pb{\textbar}height{\textbar}leftpos%
           {\textbar}rightpos{\textbar}} \\
    \opt{player}{
    \%pf & Full-line progress bar \& time display\\
    }
    \%pc & Current Time In Song\\
    \%pe & Total Number of Playlist Entries\\
    \%pm & Peak Meter (Recorder only) The entire line is used as volume peak meter.\\
    \%pn & Playlist Name (Without path or extension)\\
    \%pp & Playlist Position\\
    \%pr & Remaining Time In Song\\
    \%ps & Shuffle. Shows 's' if shuffle mode is enabled.\\
    \%pt & Total Track Time\\
    \%pv & Current volume. Can also be used in a conditional: \\
         & \%?pv{\textless}0{\textbar}1{\textbar}2{\textbar}3%
           {\textbar}4{\textbar}5{\textbar}6{\textbar}7{\textbar}8%
           {\textbar}9{\textbar}10{\textgreater}\\
  \end{tagmap}

\section{Runtime Database}
  \begin{tagmap}{}{}
    \%rp & Song playcount\\
    \%rr & Song rating (0-10). This tag can also be used in a conditional tag, %
           \%?rr{\textless}0{\textbar}1{\textbar}2{\textbar}3{\textbar}%
           4{\textbar}5{\textbar}6{\textbar}7{\textbar}8{\textbar}9{\textbar}%
           10{\textgreater}\\
  \end{tagmap}

\opt{SWCODEC}{
\section{Sound (DSP) settings}
  \begin{tagmap}{}{}
    \%sp & display current playback pitch \\
  \end{tagmap}
}
\opt{h1xx,h300}{
\section{Hold Switches}
  \begin{tagmap}{}{}
    \%mh & ``h'' if the main unit hold switch is on\\
    \%mr & ``r'' if the remote hold switch is on\\
  \end{tagmap}
}

\section{Virtual LED}
  \begin{tagmap}{}{}
    \%lh & ``h'' if there is hard disk activity\\
  \end{tagmap}

\section{Repeat Mode}
  \begin{tagmap}{}{}
    \%mm & Repeat mode, 0-4, in the order: Off, All, One, Shuffle
           \opt{player,recorder,recorderv2fm}{, A-B}\\
  \end{tagmap}
Example: \%?mm{\textless}Off{\textbar}All{\textbar}One{\textbar}Shuffle%
{\textbar}A-B{\textgreater}

\section{Playback Mode Tags}
  \begin{tagmap}{}{}
    \%mp & Play status, 0-4, in the order: Stop, Play, Pause, 
           Fast forward, Rewind\\
  \end{tagmap}
Example: \%?mp{\textless}Stop{\textbar}Play{\textbar}Pause{\textbar}%
Ffwd{\textbar}Rew{\textgreater}

\section{Images}
  \begin{tagmap}{}{}
    \opt{h300,x5,ipodcolor,ipodvideo}{
    \%X{\textbar}filename.bmp{\textbar} 
        & Load and set a backdrop image for the WPS. %
          This image must be exactly the same size as your LCD.\\
    }
    \%P{\textbar}filename.bmp{\textbar} 
        & Load a Progress bar image for the WPS. Use \%pb tag to show the 
          progress bar\\
    \%x{\textbar}n{\textbar}filename{\textbar}x{\textbar}y{\textbar} 
        & Load and display an image\\
        & n: image ID (a-z and A-Z)\\
        & filename: filename relative to \fname{/.rockbox/} and including .bmp\\
        & x: x coordinate\\
        & y: y coordinate.\\
    \%xl{\textbar}n{\textbar}filename{\textbar}x{\textbar}y{\textbar} 
        & Preload an image for later display\\
        & n: image ID (a-z and A-Z)\\
        & filename: filename relative to \fname{/.rockbox/} and including .bmp\\
        & x: x coordinate\\
        & y: y coordinate.\\
    \%xdn & Display a preloaded image\\
        & n: image ID (a-z and A-Z)\\
  \end{tagmap}

Example: image \fname{/.rockbox/bg.bmp} with ID ``a'' at 37, 109 would be:\\
\%x{\textbar}a{\textbar}bg.bmp{\textbar}37{\textbar}109{\textbar}

\note{
  \begin{itemize}
  \item The images must be in a rockbox compatible format (1 bit per pixel BMP)
  \item The image tag must be on its own line
  \item The ID is case sensitive, giving 52 different ID's
  \item The size of the LCD screen for each player varies. See table below 
        for appropriate sizes of each device. The x and y coordinates must 
        repect each of the players' limits.
  \end{itemize}
}

\section{Alignment}
  \begin{tagmap}{}{}
    \%al & Text is left aligned\\
    \%ac & Text is center aligned\\
    \%ar & Text is right aligned\\
  \end{tagmap}
All alignment tags may be present in one line, but they need to be in the 
order left -- center -- right. If the aligned texts overlap, they are merged.

\section{Conditional Tags}

\begin{tagmap}{}{}
\%?xx{\textless}true{\textbar}false{\textgreater}
    & If / Else: Evaluate for true or false case \\
\%?xx{\textless}alt1{\textbar}alt2{\textbar}alt3{\textbar}\dots{\textbar}else{\textgreater}
    & Enumerations: Evaluate for first / second / third / \dots / last condition \\
\end{tagmap}

\section{Other Tags}
\begin{tagmap}{}{}
  \%\%          & Display a `\%'\\
  \%{\textless} & Display a `{\textless}'\\
  \%{\textbar}  & Display a `{\textbar}'\\
  \%{\textgreater} & Display a `{\textgreater}'\\
  \%;           & Display a `;'\\
  \%s           & Indicate that the line should scroll. Can occur anywhere in 
                  a line (given that the text is displayed; see conditionals 
                  above). You can specify up to 10 scrolling lines. Scrolling
                  lines can not contain dynamic content such as timers, 
                  peak meters or progress bars.\\
\end{tagmap}


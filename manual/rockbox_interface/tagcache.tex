% $Id$ %
\section{\label{ref:database}Database}

\subsection{Introduction}
This chapter describes the Rockbox music database system. Using the information
contained in the tags (ID3v1, ID3v2%
  \opt{SWCODEC}{, Vorbis Comments, Apev2, etc.}%
) in your audio files, Rockbox builds and maintains a database of the music
files on your player and allows you to browse them by Artist, Album and Genre.

\subsection{Initializing the database}
The first time you use the database, Rockbox will scan your disk for audio files.
This can take quite a while depending on the number of files on your \dap{}.
This scan happens in the background, so you can choose to return to the
Main Menu and continue to listen to music.
If you shut down your player, the scan will continue next time you turn it on.
After the scan is finished you may be prompted to restart your \dap{} before
you can use the database.

\subsubsection{Ignoring directories during database initialization}

You may have directories on your \dap{} whose contents should not be added
to the database. Placing a file named \fname{database.ignore} in a directory
will exclude the files in that directory and all its subdirectories from
scanning their tags and adding them to the database. This will speed up the
database initialization.

If a subdirectory of an 'ignored' directory should still be scanned, place a
file named \fname{database.unignore} in it. The files in that directory and
its subdirectories will be scanned and added to the database.

\subsection{\label{ref:databasemenu}The Database Menu}

\begin{description}
  \opt{SWCODEC}{
  \item[Load To Ram.]
    The database can either be kept on disk (to save memory), or
    loaded into RAM (for fast browsing). Setting this to \setting{Yes} loads
    the database to RAM, allowing faster browsing and searching. Setting this
    option to \setting{No} keeps the database on the disk, meaning slower 
    browsing but it does not use extra RAM and saves some battery on boot up. 
    
    \note{If you browse your music frequently using the database, you should
      load to RAM, as this will reduce the overall battery consumption because
      the disk will not need to spin on each search.}
  }
  
\item[Auto Update.]
  If \setting{Auto update} is set to \setting{on}, each time the \dap{}
  boots, the database will automatically be updated.
  \opt{SWCODEC}{
    \note{The \setting{Auto Update} will only check for deleted files if the
      \setting{Directory Cache} (\setting{Settings $\rightarrow$ General
      Settings $\rightarrow$ System $\rightarrow$ Disk $\rightarrow$
      Directory Cache}) is enabled. \setting{Update now} includes that check
      whether dircache has been enabled or not.}
  }%
  \opt{MASCODEC}{\setting{Auto Update} does not detect deleted files. To remove
    deleted files from the database you need to run \setting{Update Now}.}%

\item[Initialize Now.]
  You can force Rockbox to rescan your disk for tagged files by
  using the \setting{Initialize Now} function in the \setting{Database
    Menu}.
  \warn{\setting{Initialize Now} removes all database files (removing
    runtimedb data also) and rebuilds the database from scratch.}

\item[Update Now.]
  \setting{Update now} causes the database to detect new and deleted files
  \opt{SWCODEC}{
    \note{Unlike the \setting{Auto Update} function, \setting{Update Now}
      will update the database regardless of whether the \setting{Directory Cache}
      is enabled. Thus, an update using \setting{Update now} may take a long
      time.
    }
  }
  Unlike \setting{Initialize Now}, the \setting{Update Now} function
  does not remove runtime database information.
  
\item[Gather Runtime Data.]
  When enabled, rockbox will record how often and how long a track is being played, 
  when it was last played and its rating. This information can be displayed in
  the WPS and is used in the database browser to, for example, show the most played, 
  unplayed and most recently played tracks.
  
\item[Export modifications.]
  This allows for the runtime data to be exported to the file \\
  \fname{/.rockbox/database\_changelog.txt}, which backs up the runtime data in
  ASCII format. This is needed when database structures change, because new
  code cannot read old database code. But, all modifications
  exported to ASCII format should be readable by all database versions.
  
\item[Import modifications.]
  Allows the \fname{/.rockbox/database\_changelog.txt} backup to be 
  conveniently loaded into the database. If \setting{Auto Update} is
  enabled this is performed automatically when the database is initialized.
  
\end{description}

\subsection{Using the database}
Once the database has been initialized, you can browse your music by Artist, 
Album, Genre and Song Name. To use the database, go to the \setting{Main Menu}
and select \setting{Database}.\\

\note{You may need to increase the value of the \setting{Max files in dir 
browser} setting (\setting{Settings $\rightarrow$ General Settings
$\rightarrow$ System $\rightarrow$ Limits}) in order to view long lists of
tracks in the ID3 database browser.\\

There is no option to turn off database completely. If you do not want
to use it just do not do the initial build of the database and do not load it
to RAM.}
%
\begin{table}
\begin{center}
  \begin{tabularx}{.75\textwidth}{XX}%
  \toprule%
  \textbf{Tag}   & \textbf{Type}  & \textbf{Origin} \\
  \midrule
  filename              & string    & system \\ 
  album                 & string    & id tag \\
  albumartist           & string    & id tag \\
  artist                & string    & id tag \\
  comment               & string    & id tag \\
  composer              & string    & id tag \\
  genre                 & string    & id tag \\
  grouping              & string    & id tag \\
  title                 & string    & id tag \\
  bitrate               & numeric   & id tag \\
  discnum               & numeric   & id tag \\
  year                  & numeric   & id tag \\
  tracknum              & numeric   & id tag/filename \\
  autoscore             & numeric   & runtime db \\
  lastplayed            & numeric   & runtime db \\
  playcount             & numeric   & runtime db \\
  Pm (play time - min)  & numeric   & runtime db \\
  Ps (play time - sec)  & numeric   & runtime db \\
  rating                & numeric   & runtime db \\
  commitid              & numeric   & system \\
  entryage              & numeric   & system \\
  length                & numeric   & system \\
  Lm (track len - min)  & numeric   & system \\
  Ls (track len - sec)  & numeric   & system \\
  \bottomrule
  \end{tabularx}
\end{center}
\end{table}

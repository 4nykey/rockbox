% $Id$ %
\chapter{Installation}\label{sec:installation}

\opt{ipodnano}{
  \note{Rockbox presently runs only on the original Ipod Nano. Rockbox does
  \emph{not} run on the newer, second generation Ipod Nano (the all aluminium
  version). For information on identifying which Ipod you own, see this page on
  Apple's website: \url{http://www.info.apple.com/kbnum/n61688}
  }
}
\opt{e200,c200}{
  \note{Rockbox doesn't function on the newer v2 models.  They can be identified
  by checking the Sandisk firmware version number under 
  Settings $\rightarrow$ Info.  The v1
  firmware is named 01.xx.xx, while the v2 firmware begins with 03.}
}

\section{Overview}
There are two ways of installing Rockbox: automated and manual. While the
manual way is older, more tested and proven to work correctly, the
automated installation is based on a nice graphical application that does
almost everything that is needed for you. It is still important that you have
an overview of the installation process to be able to select the correct
installation options.

\opt{MASCODEC}{Rockbox itself comes as a single package. There is no need
  to install additional software for running Rockbox.}
\opt{swcodec} {
  \opt{HAVE_RB_BL_ON_DISK}{There are three separate components,
    two of which need to be installed in order to run Rockbox.}
  \opt{HAVE_RB_BL_IN_FLASH}{There are two separate components of Rockbox
    that need to be installed in order to run Rockbox.}

\begin{description}
\opt{HAVE_RB_BL_ON_DISK}{
\item[The \playerman{} bootloader.]
  The \playerman{} bootloader is the program that tells your \dap{} how to boot
  and load the remaining firmware from disk. It is also responsible for the
  disk mode on your \dap{}.

  This bootloader is stored in special flash memory in your \playerman{}.
  It is already installed on your \dap{}, so it is never necessary to modify
  this in order to install Rockbox.}

\item[The Rockbox bootloader.] \index{Bootloader}
  \opt{HAVE_RB_BL_ON_DISK}{The Rockbox bootloader is loaded from disk by
  the \playerman{} bootloader. It is responsible for loading the Rockbox
  firmware and for providing the dual boot function. It directly replaces the
  \playerman{} firmware on the \daps{} disk.
  \opt{gigabeatf}{\note{Dual boot does not currently work on the gigabeat.}}}

  \opt{HAVE_RB_BL_IN_FLASH}{
  The bootloader is the program that tells your
  \dap{} how to boot and load other components of Rockbox. This is the
  component of Rockbox that is installed to the flash memory of your
  \playerman.}

\item[The Rockbox firmware.]
  \opt{HAVE_RB_BL_IN_FLASH}{Unlike the \playerman{} firmware, which runs
  entirely from flash memory,}
  \opt{HAVE_RB_BL_ON_DISK}{Similar to the \playerman{} firmware,}
  most of the Rockbox code is contained in a
  ``build'' that resides on your \daps{} drive. This makes it easy to
  update Rockbox. The build consists of a directory called
  \fname{.rockbox} containing all of Rockbox' files, which is located in the
  root of your \daps{} drive.

\end{description}
}

\nopt{player} {
    Apart from the required parts there are some addons you might be interested
    in installing too.
    \begin{description}
    \item[Fonts.] Rockbox can load custom fonts. The fonts itself are
        distributed as separate package and thus needs to be installed
        separately. The fonts are not required to run Rockbox itself but
        a lot of themes require the fonts package to be installed.

    \item[Themes.] The view of Rockbox can be customized by themes. Depending
        on your taste you might want to install additional themes to change
        the look of Rockbox.
    \end{description}
}

\section{Prerequisites}\label{sec:prerequisites}
\index{Installation!Prerequisites}
Before installing Rockbox you should make sure you meet the prerequisites.
Also you may need some tools for installation. In most cases these will be
already available on your computer but if not you need to get some additional
software.

\begin{description}
\item[USB connection.] To transfer Rockbox to your \dap{} you need to
  connect it to your computer. To proceed you need to know where to access the
  \dap{}. On Windows this means you need to figure out the drive letter
  associated with the device. On Linux you need to know the mount point of
  your \dap{}.

  \opt{ipod}{
    Your \dap{} should enter disk mode automatically when connected to a
    computer via USB. If your computer does not recognise your \dap{}, you may
    need to enter the disk mode manually. Disconnect your \dap{} from the
    computer. Hard reset the \dap{} by pressing and holding the \ButtonMenu{} and
    \ButtonSelect{} buttons simultaneously. As soon as the \dap{} resets, press
    and hold the \ButtonSelect{} and \ButtonPlay{} buttons simultaneously. Your
    \dap{} should enter disk mode, and you can try reconnecting to the computer.
  }

  \opt{ipod3g,ipod4g,ipodcolor,ipodmini}{
    \note{\index{Firewire}Firewire detection is not supported in Rockbox at 
    the moment. Please use USB only.} 
  }
  \opt{x5}{
    \note{When instructed to connect/disconnect the USB cable, always use
    the USB port through the subpack, not the side 'USB Host' port! The side port
    is intended to be used for USB OTG connections only (digital cameras, memory
    sticks, etc)!}
  }    

  \opt{sansa}{\warn{The following steps require you to change the setting in
    \setting{Settings $\rightarrow$ USB Mode} to \setting{MSC} from within the
    original firmware. Never extract files to your \dap{} while it is in
    recovery mode.}}
  \opt{h10,h10_5gb}{\warn{The following steps require you to use UMS mode and so
    may require use of the UMS trick as described in the bootloader installation
    section.}}
  \opt{gigabeatf}{\warn{During installation, do not connect your \dap{}
    using the cradle but plug the USB cable directly to the \dap{}.}}  
\end{description}
For manual installation and customization additional software is required.
\begin{description}
\item[ZIP utility.]\index{zip}
  Rockbox is distributed as an archive using the
  \fname{.zip} format. Thus you need a tool to handle that compressed
  format. Usually your computer should have a tool installed that can
  handle the \fname{.zip} file format. Windows XP has built-in support for
  \fname{.zip} files and presents them to you as directories unless you have
  installed a third party program that handles compressed files. For
  other operating systems this may vary. If the \fname{.zip} file format
  is not recognised on your computer you can find a program to handle them
  at \url{http://www.info-zip.org/} or \url{http://sevenzip.sf.net/} which
  can be downloaded and used free of charge.
\item[Text editor.] As you will see in the following chapters, Rockbox is
  highly configurable. In addition to saving configurations within Rockbox,
  Rockbox also allows you to create customised configuration files. If you
  would like to edit custom configuration files on your computer, you will
  need a text editor like Windows' ``Wordpad''.
\end{description}

\opt{ipod}{
  \note{In addition to the requirements described above, Rockbox only works on
  Ipods formatted with the FAT32 filesystem (i.e., Ipods initialized by iTunes
  for Windows). It does not work with the HFS+ filesystem (i.e. Ipods
  initialized by iTunes for the Mac). More information and instructions for
  converting an Ipod to FAT32 can be found on the
  \wikilink{IpodConversionToFAT32} wiki
  page on the Rockbox website.  Note that after conversion, you can still use
  a FAT32 Ipod on a Mac.
  }
}

\section{Installing Rockbox}\label{sec:installing_rockbox}\index{Installation}

\opt{ipodvideo}{\warn{There are separate versions of Rockbox for the 30GB and 
  60GB/80GB models.  You must ensure you download the correct version for your 
  \dap{}.}}

\subsection{Automated Installation}

\opt{mrobe100}{\fixme{NOTE: The m:robe 100 is only supported in rbutil version 
1.05 or greater which may not be available on the download site yet.}} 

To automatically install Rockbox download the official installer and
housekeeping tool \textsc{Rockbox Utility}. It allows you to
\begin{itemize}
\item Automatically install all needed components for using Rockbox
        (``Small Installation'')
\item Automatically install all suggested components (``Full Installation'')
\item Selectively install any of all available parts of Rockbox
\item Install additional themes interactively
\item Install voice files and generate talk clips
\item Uninstall all components you installed using Rockbox Utility
\end{itemize}
Prebuilt binaries for Windows, Linux and MacOS~X are
available from \download{rbutil/}. As Rockbox Utility is still under
development more information including up-to-date download links can be
found at \wikilink{RockboxUtility}.

\warn{Upon the first start of Rockbox Utility you \emph{need} to set at least
the correct player and mountpoint in the configuration dialog. Autodetection
can detect most player types. If autodetection failed or was unable to detect
the mountpoint make sure to enter the correct values. The mountpoint indicates
the location of the \dap{} in your filesystem. On Windows, this is the drive
letter the \dap{} gets assigned, on other systems this is a path in the
filesystem.}
\opt{ipodvideo}{\warn{Autodetection is unable to distinguish between the
        \playerman{} 30~GB and 60~GB / 80~GB models and defaults to the
        30~GB model. This will usually work but you might want to check the
        detected value, especially if you experience problems with Rockbox.}}

\note{Rockbox Utility currently lacks some guiding messages. Please have a
        look at the manual installation instructions if you are stuck
        during installation.}

\subsection{Manual Installation}

\subsubsection{Choosing a Rockbox version}\label{sec:choosing_version}

There are three different types of firmware binaries from the Rockbox website:
\label{Version}
Release version, current build and daily build. You need to decide which one
you want to install and get the version for your \dap{}.

\begin{description}

\item[Release.]
  \opt{archos}{The release version is the latest stable release, free
      of known critical bugs. The current stable release of Rockbox, version
      2.5, is available at \url{http://www.rockbox.org/download/}.
  }
  \opt{swcodec}{
      There has not yet been a stable release for the \playername{}. Until
      there is a stable release for \playername{}, use a current build.
  }

\item[Current Build.] The current build is built at each source code change to
  the Rockbox SVN repository, and represent the current state of Rockbox
  development. This means that the build could contain bugs, but is most of
  the time safe to use. You can download the current build from  
  \url{http://build.rockbox.org/}.

\item[Archived Build.] In addition to the release version and the current build,
  there is also an archive of daily builds available for download. These are
  built once a day from the latest source code in the SVN repository. You can
  download archived builds from \url{http://www.rockbox.org/daily.shtml}.

\end{description}


Because current builds and daily builds are development versions which change
frequently, they may behave differently than described in this manual, or
they may introduce new (and maybe annoying) bugs. If you do not want to get
undefined behaviour from your \dap{} you should really stick to the current
stable release, if there is one for your \dap{}. If you want to help the
project development, you can try development builds and help by reporting
bugs. Just be aware that these are development builds that are highly
functional, but not perfect!

\subsubsection{Installing the firmware}\label{sec:installing_firmware}

\begin{enumerate}

\item Download your chosen version of Rockbox from the links in the
  previous section.

\item Connect your \dap{} to the computer via USB
  \opt{ipod3g,ipod4g,ipodmini,ipodcolor}{ or Firewire} as described in
  the manual that came with your \dap{}.

\item Take the file that you downloaded above, and use the ``Extract
  all'' command of your unzip program to extract the files in the
  \fname{.zip} file onto your \dap{}.

Note that the entire contents of the \fname{.zip} file should be
extracted directly to the root of your \daps{} drive. Do not try to
create a separate directory on your \dap{} for the Rockbox
files! The \fname{.zip} file already contains the internal
structure that Rockbox needs.

\end{enumerate}

\opt{archos}{
  \note{
    If the contents of the \fname{.zip} file are extracted correctly, you will
    have a file called \fname{\firmwarefilename} in the main directory of your
    \daps{} drive, and also a directory called \fname{/.rockbox}, which contains a
    number of other directories and system files needed by Rockbox. If you receive a
     ``-1'' error when you start Rockbox, you have not extracted the contents of
    the \fname{.zip} file to the proper location.
  }
}

% This has nothing to do with swcodec, just that these players need our own
% bootloader so we can decide where we want the main binary.
\opt{swcodec}{
  \note{
    If the contents of the \fname{.zip} file are extracted correctly, you will
    have a directory called \fname{/.rockbox}, which contains all the files needed
    by Rockbox, in the main directory of your \daps{} drive. If you receive a
    ``-1'' error when you start Rockbox, you have not extracted the contents of
    the \fname{.zip} file to the proper location.
  }
}

\nopt{player}{
  \subsubsection{Installing the fonts package}{\index{Installation!Fonts}\label{sec:installing_fonts}
    Rockbox has a fonts package that is available at
    \url{http://www.rockbox.org/daily.shtml} or from the \emph{extras} link in
    the menu on the Rockbox website. While the current builds and
    daily builds change frequently, the fonts package rarely changes. Thus,
    the fonts package is not included in these builds. (The release version, on
    the other hand, does not change, so fonts are included when you download a
    release). When installing Rockbox for the first time, you should install
    the fonts package.

  \begin{enumerate}

  \item Download the fonts package from the link above.

  \item Take the file that you downloaded above, and use the ``Extract
    all'' command of your unzip program to extract the files in the
    \fname{.zip} file onto your \dap{}. As with the firmware installation, the
    entire contents of the fonts \fname{.zip} should be extracted directly to the
    root of your \daps{} drive. Do not try to create a separate directory
    on your \dap{} for the fonts! The \fname{.zip} already contains the
    correct internal structure.

  \end{enumerate}
  }
}

\opt{swcodec}{
  \subsubsection{Installing the bootloader}
  \opt{h1xx,h300}{% $Id$ %
\subsection{Installing the bootloader}
  Installing the bootloader is the trickiest part of the installation.
  The Rockbox bootloader allows users to boot into either the Rockbox 
  firmware or the iriver firmware. For legal reasons, we cannot distribute 
  the bootloader. Instead, we have developed a program that will patch the 
  Iriver firmware with the Rockbox bootloader. These instructions will explain 
  how to download and patch the Iriver firmware with the Rockbox bootloader 
  and install it on your jukebox.

\begin{enumerate}
  \item Download a supported version of the Iriver firmware for your 
  \playername\ from the Iriver website or from 
  \wikilink{ManualRockboxInstall}.
  Supported Iriver firmware versions currently include 
  \opt{IRIVER_H100_PAD}{1.63US, 1.63EU, 1.63K, 1.65US, 1.65EU, 1.65K, 1.66US, 
    1.66EU and 1.66K.  Note that the H140 uses the same firmware as the H120;
    H120 and H140 owners should use the	firmware called \fname{ihp\_120.hex}.
    Likewise, the iHP110 and iHP115 use the same firmware, called 
    \fname{ihp\_100.hex}.   Be sure to use the correct firmware file for 
    your player.}
  \opt{IRIVER_H300_PAD}{1.28K, 1.28EU, 1.28J, 1.29K, 1.29J and 1.30EU.
    \note{The US H3xx firmware is not currently supported and cannot be
    patched to be used with the bootloader. If you wish to install Rockbox
    on a US \playername\, you must use an international firmware, which will
    permanently remove DRM support from the player.}
  }
  If the file that you downloaded is a \fname{.zip} file, use an unzip 
  utility such as \fname{InfoZip}, \fname{7zip}, \fname{WinRAR},	or 
  \fname{WinZip} to extract the \fname{.hex} from the \fname{.zip} file
  to your desktop. Likewise, if the file that you downloaded is an 
  \fname{.exe} file, double-click on the \fname{.exe}	file to extract 
  the \fname{.hex} file to your desktop.
  %
  \item Download the firmware patcher \fname{fwpatcher.exe} from 
  \url{http://download.rockbox.org/bootloader/iriver/} and save it to your desktop.
    \warn{The firmware patcher contains Unicode support, which is not supported by 
    all versions of Windows. If you have difficulty with the firmware patcher, try 
    downloading the alternate firmware patcher \fname{fwpatchernu.exe}, which is 
    built without Unicode support.}
  %
  \item Go to your desktop and double-click on whichever version of the firmware 
  patcher you downloaded in the prior step.
  %
  \item In the firmware patcher dialog box, click on the BROWSE button and navigate
  to the \fname{.hex} file that you previously downloaded to your desktop.
  %
  \item Click PATCH. The firmware patcher will patch the original firmware to 
  include the Rockbox bootloader. The \fname{.hex} file on your desktop is now
  a modified version of the original \fname{.hex} file.
  %
  \item Turn on your \playername\ and connect it to your computer via USB.
  %
  \item Copy or move the modified \fname{.hex} file to the ROOT directory of 
    your jukebox.
  %
  \item Disconnect the jukebox from USB. (Be sure to use Windows' ``safely remove
  hardware'' option.)
  \warn{Before proceeding further, make sure that your player has a full charge, 
    or that it is connected to the power adaptor.}
  %
  \item Update your \playername s firmware with the patched bootloader. To do this, turn 
    the jukebox on. Press and hold the 
    \opt{IRIVER_H100_PAD}{\ButtonSelect{} button }%
    \opt{IRIVER_H300_PAD}{\ButtonSelect{} button }%
    to enter the main menu, and navigate to \setting{General $\rightarrow$ Firmware 
    Upgrade}. Select \setting{Yes} when asked to confirm if you want to upgrade the 
    firmware. The \playername{} will display a message indicating that the
    firmware update 
    is in progress. Do not interrupt this process. When the firmware update is 
    complete, the player will turn itself off. (The update firmware process usually 
    takes a minute or so.)

    You have now installed the Rockbox bootloader. 

\opt{h1xx}{\note{If you install the Rockbox bootloader, but do not install the
  Rockbox firmware, the Rockbox bootloader will load the iriver firmware when the
  jukebox is turned on.}}

\end{enumerate}
}
  \opt{ipod}{% $Id$ %
\subsection{Installing the bootloader}
\warn{These instructions are preliminary and may contain errors! 
Please check the wiki for up-to-date and improved installation instructions!
If you find errors you're of course welcomed to report them so we can fix it
for the next daily builds.}

  Installing the bootloader is the trickiest part of the installation.
  The process is different depending on your operating system, but before
  starting, connect the \dap{} to the computer using either an USB \fixme{or
  Firewire?} cable. Next, create a folder on the computer's harddrive and
  download the following file to that folder:
  \opt{ipodvideo}{\wikilink{IpodInstallation/bootloader-video.bin}}
  \opt{ipodnano}{\wikilink{IpodInstallation/bootloader-nano.bin}}
  \opt{ipodmini}{\wikilink{IpodInstallation/bootloader-mini1g.bin} or 
    \wikilink{IpodInstallation/bootloader-mini1g.bin} depending on which
    generation your \dap{} is.\fixme{Describe how to identify 1/2G}}
  \opt{ipodcolor}{\wikilink{IpodInstallation/bootloader-color.bin}}
  \opt{ipod4g}{\wikilink{IpodInstallation/bootloader-4g.bin}}

  When that is done, proceed to the section below that matches the operating
  system on the computer.
  \note{These instructions all require you to have administrator rights
  on your computer, regardless of the operating system.}
  \note{Rockbox only works on FAT32 partitions (called ``Windows formatted'' by
    Apple). So if your \dap{} is Mac formatted (HFS+), you should first convert
    it to FAT32. Information on how to do this can be found on the Rockbox
    website. \fixme{Include these instructions?}}

\subsubsection{Windows users}
\begin{enumerate}
  \item Download the following two programs and save them in the folder just
    created. These programs will be used in the next steps:
    \begin{itemize}
      \item \wikilink{IpodInstallation/ipodpatcher.exe}
      \item \wikilink{IpodInstallation/ipod_fw.exe}
    \end{itemize}
  \item Locate the \dap{} by opening a command windows. You can do this by
    clicking ``Start'', ``Execute'' and typing \fname{cmd}. Press Enter to
    execute that command. Now change directory to the
    folder you created and run the following commands:
    \begin{code}
    ipodpatcher 0
    ipodpatcher 1
    ipodpatcher 2
    ipodpatcher 3
    \end{code}
    Keep increasing the number until the \dap{} is located. 

    Output for an unsuccessful attempt to contact the \dap{}...
    \begin{code}
    C:/rockbox>ipodpatcher 0
    ipodpatcher v0.3 - (C) Dave Chapman 2006
    This is free software; see the source for copying conditions.  There is NO
    warranty; not even for MERCHANTABILITY or FITNESS FOR A PARTICULAR PURPOSE.

    [INFO] Reading partition table from \textbackslash\textbackslash{}.\textbackslash{}PhysicalDrive0
    Drive is not an iPod, aborting
    \end{code}
    
    A successful connection to the \dap{} will look similar to this...
    \begin{code}
    C:\textbackslash{}rockbox>ipodpatcher 6
    ipodpatcher v0.3 - (C) Dave Chapman 2006
    This is free software; see the source for copying conditions.  There is NO
    warranty; not even for MERCHANTABILITY or FITNESS FOR A PARTICULAR PURPOSE.

    [INFO] Reading partition table from \textbackslash\textbackslash{}.\textbackslash{}PhysicalDrive6
    Part    Start Sector    End Sector    Size (MB)  Type
       0              63        160649        78.4   Empty (0x00)
       1          160650       7984304      3820.1   W95 FAT32 (0x0b)
   \end{code}
    Remember the number that corresponds to your \dap{} -- in the 
    following steps, \emph{N} should be replaced with the number just found.
  \item Now, extract the firmware partition currently on the \dap{} with the
    following command:
    \begin{code}
    ipodpatcher -r \emph{N} bootpartition.bin
    \end{code}
    \note{You should keep a safe backup of this \fname{bootpartition.bin} file
      for use if you ever wish to either upgrade the Rockbox bootloader or
      uninstall Rockbox from your Ipod}
  \item Extract the Apple firmware from the partition image image just created:
    \begin{code}
    ipod_fw -o apple_os.bin -e 0 bootpartition.bin
    \end{code}
\optv{ipodvideo}{
  \item Similarly, extract the Broadcom firmware:
    \begin{code}
    ipod_fw -o apple_sw_5g_rcsc.bin -e 1 bootpartition.bin
    \end{code}
}
  \item Merge the Rockbox bootloader you downloaded previously with the Apple
    firmware:
\optv{ipodnano}{
    \begin{code}
    ipod_fw -g nano -o rockboot.bin -i apple_os.bin bootloader-nano.bin
    \end{code}
}
\optv{ipodvideo}{
    \begin{code}
    ipod_fw -g video -o rockboot.bin -i apple_os.bin bootloader-video.bin
    \end{code}
}
\optv{ipodmini}{
    \begin{code}
    ipod_fw -g mini -o rockboot.bin -i apple_os.bin bootloader-mini1g.bin
    \end{code}
    Or, if you have a 2G mini:
    \begin{code}
    ipod_fw -g mini -o rockboot.bin -i apple_os.bin bootloader-mini2g.bin
    \end{code}
}
\optv{ipodcolor}{
    \begin{code}
    ipod_fw -g color -o rockboot.bin -i apple_os.bin bootloader-color.bin
    \end{code}
}
\optv{ipod4g}{
    \begin{code}
    ipod_fw -g 4g -o rockboot.bin -i apple_os.bin bootloader-4g.bin
    \end{code}
}
\item
    Install the Rockbox-enabled firmware:
    \begin{code}
    ipodpatcher -w \emph{N} rockboot.bin
    \end{code}
\end{enumerate}

Now you can proceed installing the firmware itself.

\subsubsection{Mac OS X users}
\begin{enumerate}
  \item Download the following two programs and save them in the folder just
    created. These programs will be used in the next steps:
    \begin{itemize}
      \item \wikilink{IpodInstallationFromMacOSX/diskdump}
      \item \wikilink{IpodInstallationFromMacOSX/ipod_fw}
    \end{itemize}
    Start a Terminal and type navigate into the folder you created. Before
    you can continue, you need to ensure that Mac OS knows that the
    \fname{ipod\_fw}
    and diskdump files you downloaded are executable programs. To do this,
    type the following command:
    \begin{code}
    chmod +x ipod_fw diskdump
    \end{code}
  \item Locate the \dap{} by running the following command:
    \begin{code}
    mount
    \end{code}
    The output will look something like this: \fixme{Add full example}
    \begin{code}
    /dev/disk1s2 on /Volumes/DAVE_S IPOD 1 (local, nodev, nosuid)
    \end{code}
    In this example, the \dap\ is located at /dev/disk1s2 Remember the 
    location of your \dap\  -- in the following steps, /dev/disk1s2 should be
    replaced with the location just found.
  \item Before continuing, the \dap\ must be ``unmounted'', which is
    done with the following command:
    \begin{code}
    diskutil unmount /dev/disk1s2
    \end{code}
  \item Now, extract the Apple firmware currently on the \dap{} with the
    following command:
    \note{The last part of the location is left out.}
    \begin{code}
    ./diskdump -r /dev/disk1 bootpartition.bin
    \end{code}
    \note{You should keep a safe backup of this \fname{bootpartition.bin} file
      for use if you ever wish to either upgrade the Rockbox bootloader or
      uninstall Rockbox from your iPod
    }
  \item Extract the Apple firmware from this partition image:
    \begin{code}
    ./ipod_fw -o apple_os.bin -e 0 bootpartition.bin
    \end{code}
\optv{ipodvideo}{
  \item Similarly, extract the Broadcom firmware:
    \begin{code}
    ./ipod_fw -o apple_sw_5g_rcsc.bin -e 1 bootpartition.bin
    \end{code}
}
  \item Merge the Rockbox bootloader you downloaded previously with the Apple
    firmware:
\optv{ipodnano}{
    \begin{code}
    ./ipod_fw -g nano -o rockboot.bin -i apple_os.bin bootloader-nano.bin
    \end{code}
}
\optv{ipodvideo}{
    \begin{code}
    ./ipod_fw -g video -o rockboot.bin -i apple_os.bin bootloader-video.bin
    \end{code}
}
\optv{ipodmini}{
    \begin{code}
    ./ipod_fw -g mini -o rockboot.bin -i apple_os.bin bootloader-mini1g.bin
    \end{code}
    Or, if you have a 2G Mini:
    \begin{code}
    ./ipod_fw -g mini -o rockboot.bin -i apple_os.bin bootloader-mini2g.bin
    \end{code}
}
\optv{ipodcolor}{
    \begin{code}
    ./ipod_fw -g color -o rockboot.bin -i apple_os.bin bootloader-color.bin
    \end{code}
}
\optv{ipod4g}{
    \begin{code}
    ./ipod_fw -g 4g -o rockboot.bin -i apple_os.bin bootloader-4g.bin
    \end{code}
}
  \item
    Install the Rockbox-enabled firmware:
    \note{The last part of the location is left out.}
    \begin{code}
    ./diskdump -w /dev/disk1 rockboot.bin
    \end{code}
\end{enumerate}

Now, proceed with installing the firmware itself.

\subsubsection{Linux users}
\begin{enumerate}
  \item Download the following and save it in the folder just
    created:
    \begin{itemize}
      \item \url{http://www.rockbox.org/viewcvs.cgi/*checkout*/tools/ipod_fw.c}
    \end{itemize}
    Now compile it to an executable by opening a command prompt and changing
    to the folder created previously. Thn run the following command:
    \begin{code}
    gcc -o ipod_fw ipod_fw.c
    \end{code}
    If you get the message that the command gcc is not found, you need to
    install gcc. How to do this depends on your Linux distribution, and
    you should consult its documentation for help on this.
  \item Locate your Ipod by running the command \verb|dmesg|. In the output
    something like the following should be seen:
\begin{code}
    usb 4-1: new high speed USB device using ehci_hcd and address 7
    scsi4 : SCSI emulation for USB Mass Storage devices
    usb-storage: device found at 7
    usb-storage: waiting for device to settle before scanning
      Vendor: Apple     Model: iPod              Rev: 1.62
      Type:   Direct-Access                      ANSI SCSI revision: 00
    SCSI device sdb: 58605120 512-byte hdwr sectors (30006 MB)
\end{code}
    You need the device name of your \dap, which you can find in the last line.
    In this example, the \dap\ is located on \fname{/dev/sdb}. In the following,
    \fname{/dev/sdb} should be replaced with the location just found.
  \item Run \verb|fdisk -l /dev/sdb|. Verify that the
    output is similar to the one below:
    \begin{code}
       Device Boot      Start         End      Blocks   Id  System
    /dev/sdb1               1          10       80293+   0  Empty
    /dev/sdb2              11        3648    29222235    b  W95 FAT32
    \end{code}
  \item Back up the partition table using the following command:
    \note{The last part of the location is left out.}
    \begin{code}
    dd if=/dev/\emph{sdb} of=mbr.bin count=1
    \end{code}

  \item Now, extract the firmware partition currently on the \dap{} with the
    following command:
    \begin{code}
    dd if=/dev/\emph{sdb1} of=bootpartition.bin
    \end{code}
    \note{You should keep a safe backup of this \fname{bootpartition.bin} file
      for use if you ever wish to either upgrade the Rockbox bootloader or
      uninstall Rockbox from your Ipod
    }
  \item Extract the Apple firmware from this partition image:
    \begin{code}
    ./ipod_fw -o apple_os.bin -e 0 bootpartition.bin
    \end{code}
\optv{ipodvideo}{
  \item Similarly, extract the Broadcom firmware: 
    \begin{code}
    ./ipod_fw -o apple_sw_5g_rcsc.bin -e 1 bootpartition.bin
    \end{code}
}

  \item Merge the Rockbox bootloader you downloaded previously with the Apple
    firmware: 
\optv{ipodnano}{
    \begin{code}
    ./ipod_fw -g nano -o rockboot.bin -i apple_os.bin bootloader-nano.bin
    \end{code}
}
\optv{ipodvideo}{
    \begin{code}
    ./ipod_fw -g video -o rockboot.bin -i apple_os.bin bootloader-video.bin
    \end{code}
}
\optv{ipodmini}{
    \begin{code}
    ./ipod_fw -g mini -o rockboot.bin -i apple_os.bin bootloader-mini1g.bin
    \end{code}
    Or, if you have a 2G Mini:
    \begin{code}
    ./ipod_fw -g mini -o rockboot.bin -i apple_os.bin bootloader-mini2g.bin
    \end{code}
}
\optv{ipodcolor}{
    \begin{code}
    ./ipod_fw -g color -o rockboot.bin -i apple_os.bin bootloader-color.bin
    \end{code}
}
\optv{ipod4g}{
    \begin{code}
    ./ipod_fw -g 4g -o rockboot.bin -i apple_os.bin bootloader-4g.bin
    \end{code}
}
  \item
    Install the Rockbox-enabled firmware:
    \begin{code}
    dd if=rockboot.bin of=/dev/\emph{sdb1}
    \end{code}
\end{enumerate}
Now you can install the firmware itself.

}
  \opt{m5,x5}{\fixme{This is merely a copy of the wiki page IaudioBoot, so this section needs
a more natural language and also error checking by Iaudio owners.}

The \playername{} has a builtin boot loader which performs the
firmware update, and can also access the hard drive via USB. Therefore the
Rockbox bootloader can be very minimalistic, without USB mode.
This also makes it less dangerous to install the Rockbox bootloader, as you can
always restore it using the \playerman{} bootloader.

\note{The current bootloader is not prepared to coexist with the original
firmware. It replaces the original firmware.}

\subsubsection{Installation}
\begin{itemize}
\item Download the Rockbox bootloader binary from 
\url{http://download.rockbox.org/bootloader/iaudio/}.
  \opt{x5}{Use the \fname{x5v\_fw.bin} file if your \dap{} is a X5V. If it is a X5,
    use the \fname{x5\_fw.bin} file.}
  \opt{m5}{Use the \fname{m5\_fw.bin} file.}
\item Copy it to the \fname{FIRMWARE} directory on your \dap{}.
\item Turn the \dap{} off, remove the USB cable and insert the charger. The
Rockbox bootloader will automatically be flashed.
\end{itemize}
}
  \opt{h10,h10_5gb}{\fixme{To do:  Complete this section H10 platforms.}}
  \opt{gigabeatf}{% $Id$

\begin{itemize}
\item Download the Rockbox bootloader from
  \url{http://download.rockbox.org/bootloader/gigabeat/}
\item Starting at the root directory of your player browse into the directory
  \fname{GBSYSTEM} and from that into the subdirectory \fname{FWIMG}.
  These directories are hidden. Make sure that you have configured your browser
  to show hidden files or you may be unable to see \fname{FWIMG}.
\item In that directory you'll find a file called \fname{FWIMG01.DAT}. This too
  may be hidden. Rename the file to \fname{FWIMG01.DAT.ORIG}. Make sure you
  spelled that name  correctly as it is needed for booting the \archosplayerman{} firmware.
  \warn{If you do not complete this step then you will be unable to uninstall Rockbox
  without a copy of the original firmware from the original install CD.}
\item Now copy the file \fname{FWIMG01.DAT} you downloaded to that directory.
  Make sure the spelling is correct.
\end{itemize}}
  \opt{sansa}{% $Id$ %
\opt{e200}{\fixme{NOTE: These instructions will not work on the 
``Rhapsody'' version of the E200 series (also known as E200R).  Please 
follow the instructions at 
\url{http://www.rockbox.org/twiki/bin/view/Main/SansaE200RInstallation}.}}

\warn{If your \daps{} original firmware starts with a version number of 03.XX.XX, then do \emph{not} proceed with these install instructions.
If your \daps{} original firmware starts with 01.XX.XX, then you can
install Rockbox.}

In order to make your \playertype{} load and execute the Rockbox firmware you
have just installed, you will need to install the Rockbox
bootloader. Unless bugs are found in the bootloader code, or
significant new features are added, you will only have to perform this
step once.

These steps use the sansapatcher tool. Source code is available in the Rockbox
SVN repository (\url{http://svn.rockbox.org/viewvc.cgi/trunk/rbutil/sansapatcher/}).

\subsubsection{Bootloader installation from Windows}

\begin{enumerate}

\item Make sure you are logged into your computer as Administrator, or a 
user with Administrator privileges and connect your \dap{}.

\item Download sansapatcher.exe from 
\download{bootloader/sandisk-sansa/sansapatcher/win32/sansapatcher.exe} 
and run it.

\item If all has gone well, you should see some information displayed about
your \playertype{} and a message asking you if you wish to install the Rockbox
bootloader. Press i followed by ENTER, and sansapatcher will now
install the bootloader. After a short time you should see the message
``[INFO] Bootloader installed successfully.'' Press ENTER again to exit
sansapatcher.

\item Disconnect your \dap{} in the usual way. The bootloader is now installed. 

\end{enumerate}

\subsubsection{Bootloader installation from Mac OS X}

\begin{enumerate}

\item Attach your \dap{} to your Mac and wait for its icon to appear in 
Finder.

\item Download and open sansa.dmg from 
\download{bootloader/sandisk-sansa/sansapatcher/macosx/sansapatcher.dmg} 
and then double-click on the sansapatcher icon inside. 

\item If all has gone well, you should see some
information displayed about your \dap{} and a message asking you if you 
wish to install the Rockbox bootloader. Press i followed by ENTER, and 
sansapatcher will now unmount your \dap{} and install the bootloader. 
After a short time you should see the message ``[INFO] Bootloader installed successfully.'' Press ENTER again to exit sansapatcher and then quit the Terminal application.

\item Your \dap{} will now automatically reconnect itself to your Mac. 
Wait for it to connect, and then eject and unplug it in the normal way. 

\end{enumerate}

\subsubsection{Bootloader installation from Linux}

\begin{enumerate}

\item Download sansapatcher from
\download{bootloader/sandisk-sansa/sansapatcher/linux32x86/sansapatcher} (32-bit x86 
binary) or \download{bootloader/sandisk-sansa/sansapatcher/linux64amd64/sansapatcher} 
(64-bit amd64 binary). You can save this anywhere you wish, but the next 
steps will assume you have saved it in your home directory.

\item Attach your \dap{} to your computer.

\item Open up a terminal window and type the following commands:

\begin{code} 
    cd $HOME
    chmod +x sansapatcher
    ./sansapatcher
\end{code}

\warn{You may need to be the root user in order for sansapatcher to have
sufficient permission to perform raw disk access to your \dap{}.}

\item If all has gone well, you should see some information displayed about
your \playertype{} and a message asking you if you wish to install the Rockbox
bootloader. Press i followed by ENTER, and sansapatcher will now install the
bootloader. After a short time you should see the message ``[INFO] Bootloader
installed successfully.'' Press ENTER again to exit sansapatcher.

\item Disconnect your \dap{} in the usual way. The bootloader is now installed.

\end{enumerate}
}
  \opt{mrobe100}{\subsubsection{Installation}
\begin{enumerate}
  \item Download 
  \opt{mrobe100}{\url{http://download.rockbox.org/bootloader/olympus/mrobe100/pp5020.mi4}}
  \item Connect your \playertype{} to the computer.
  \item Rename the \fname{pp5020.mi4} file to \fname{OF.mi4} in the \fname{System} directory on your \playertype{}.
    \note{You should keep a safe backup of this file for use if you ever wish to switch back to the \playerman{} firmware.}
    \note{If you cannot see the \fname{System} directory, you will need to make sure your operating system is configured to show hidden files and directories.}

  \item Copy the \fname{pp5020.mi4} file you downloaded to the System directory on your \dap{}.
\end{enumerate}
}
  \opt{gigabeats}{% $Id$

\warn{Before starting this procedure, ensure that you have a copy
of the original \playerman{} firmware. Without this, it is
\emph{not} possible to uninstall Rockbox. It is also needed if you want to
install the dual-boot bootloader. The \playerman{}
firmware can be downloaded from
\url{http://www.tacp.toshiba.com/tacpassets-images/firmware/MESV12US.zip}.\\}
The single-boot bootloader can only boot Rockbox, whereas the dual-boot
bootloader can boot both Rockbox and the \playerman{} firmware.
The single-boot bootloader boots Rockbox more quickly if you no longer need
access to the \playerman{} firmware.\\

Installing the bootloader is only needed once. It involves replacing the
existing firmware file on your \dap{} with another version.
When running the original \playerman{} firmware (a version of Windows CE), it is
only possible to connect the \dap{} to a PC in ``MTP mode'', which hides
the actual content of your \daps{} disk and provides restricted access
to its contents.
In reality, the \daps{} hard disk contains two partitions, a small
(150~MB) ``firmware partition'' containing the \daps{} firmware (operating
system), and a second ``data partition'' containing your media files. The main
firmware file in the bootloader partition is called \fname{nk.bin}, and
this is the file that is loaded into RAM (by the \daps{} ROM-based
bootloader) and executed when your \dap{} is powered on.

\subsubsection{Bootloader installation from Windows}

\begin{enumerate}

\item Attach your \dap{} to your computer.

\item Download \fname{beastpatcher.exe} from
\download{bootloader/toshiba/gigabeat-s/beastpatcher/win32/beastpatcher.exe}
and then perform one of the following, depending on whether you want single
or dual-boot.

\begin{description}
\item [Single Boot.] Run \fname{beastpatcher.exe}. You should see some
information displayed about
your \dap{} and a message asking you if you wish to install the Rockbox
bootloader. Press i followed by ENTER, and beastpatcher will
install the bootloader. After a short time you should see the message
``[INFO] Bootloader installed successfully''. Press ENTER again to exit
beastpatcher.

\item [Dual Boot.] Inside the \fname{MESV12US.zip} file you downloaded earlier
you should find an \fname{.iso} file.  Using e.g. 7zip
(\url{http://www.7-zip.org}) you can extract an \fname{.exe} file from this
\fname{.iso} file.  Using 7zip again, extract the \playerman{} firmware file
\fname{nk.bin} from the \fname{.exe} file and place it in the same
directory as \fname{beastpatcher.exe}.  Open a command prompt and navigate
to this directory, and then type the following commands:

\begin{code} 
    beastpatcher -d nk.bin
\end{code}

After a short time you should see the message
``[INFO] Bootloader installed successfully''. Press ENTER again to exit
beastpatcher.
\end{description}

\item After a successful installation, you need to disconnect your \dap{} from
USB, and then immediately reconnect it. It should reboot then enter the Rockbox
bootloader ``USB Mass Storage'' mode, which exposes your \daps{} disk to your
computer as a standard USB Mass Storage device.
\end{enumerate}

\subsubsection{Bootloader installation from Mac OS X}
\begin{enumerate}
\item Attach your \dap{} to your computer.

\item Download and open beastpatcher.dmg from 
\download{bootloader/toshiba/gigabeat-s/beastpatcher/macosx/beastpatcher.dmg} 
and then perform one of the following,
depending on whether you want single or dual-boot.

\begin{description}
\item [Single Boot.] Double-click on the beastpatcher icon. You can also
drag the beastpatcher icon to a location on your hard drive and launch
it from the Terminal. If all has gone well, you should see some 
information displayed about your \dap{} and a message asking you if you 
wish to install the Rockbox bootloader. Press i followed by ENTER, and 
beastpatcher will now install the bootloader. After a short time you 
should see the message ``[INFO] Bootloader installed successfully''
followed by some error messages that you can safely ignore. Press 
ENTER again to exit beastpatcher and then quit the Terminal application.

\item [Dual Boot.] Inside the \fname{MESV12US.zip} file you downloaded earlier
you should find an \fname{.iso} file.  Using e.g. 7zip
(\url{http://www.7-zip.org}) you can extract an \fname{.exe} file from this
\fname{.iso} file.  Using 7zip again, extract the \playerman{} firmware file
\fname{nk.bin} from the \fname{.exe} file and place it in the same
directory as \fname{beastpatcher}.  Open a terminal window and type the
following command:

\begin{code} 
    ./beastpatcher -d nk.bin
\end{code}
\end{description}

\item After a successful installation, your \dap{} will immediately turn off.
Turn it on again, and (because it is still connected to your Mac)
it will enter the Rockbox bootloader's
``USB Mass Storage'' mode, which exposes your \daps{} disk to your computer
as a standard USB Mass Storage device.
\end{enumerate}

\subsubsection{Bootloader installation from Linux}

\begin{enumerate}

\item Download beastpatcher from
\download{bootloader/toshiba/gigabeat-s/beastpatcher/linux32x86/beastpatcher}
(32-bit x86 binary) or 
\download{bootloader/toshiba/gigabeat-s/beastpatcher/linux64amd64/beastpatcher}
(64-bit amd64 binary). You can save this anywhere you wish, but the next 
steps will assume you have saved it in your home directory.

\item Attach your \dap{} to your computer and then perform one of the following,
depending on whether you want single or dual-boot.

\begin{description}
\item [Single Boot.] Open up a terminal window and type the following commands:

\begin{code} 
    cd $HOME
    chmod +x beastpatcher
    ./beastpatcher
\end{code}

If all has gone well, you should see some information displayed about
your \dap{} and a message asking you if you wish to install the Rockbox
bootloader. Press i followed by ENTER, and beastpatcher will now install the
bootloader. After a short time you should see the message ``[INFO] Bootloader
installed successfully'' followed by some error
messages that you can safely ignore. Press ENTER again to exit beastpatcher.

\item [Dual Boot.] Inside the \fname{MESV12US.zip} file you downloaded earlier
you should find an \fname{.iso} file.  Using e.g. 7zip
(\url{http://www.7-zip.org}) you can extract an \fname{.exe} file from this
\fname{.iso} file.  Using 7zip again, extract the \playerman{} firmware file
\fname{nk.bin} from the \fname{.exe} file and place it in the same
directory as \fname{beastpatcher}.  Open a terminal window and type the
following commands:

\begin{code} 
    cd $HOME
    chmod +x beastpatcher
    ./beastpatcher -d nk.bin
\end{code}

After a short time you should see the message
``[INFO] Bootloader installed successfully'' followed by some error
messages that you can safely ignore. Press ENTER again to exit
beastpatcher.
\end{description}

\item After a successful installation, your \dap{} will immediately turn off.
Turn it on again, and (because it is still connected to your PC)
it will enter the Rockbox bootloader's
``USB Mass Storage'' mode, which exposes your \daps{} disk to your computer
as a standard USB Mass Storage device.

\end{enumerate}
}
}

\subsection{Enabling Speech Support (optional)}\label{sec:enabling_speech_support}
\index{Speech}\index{Installation!Optional Steps}
If you wish to use speech support you will also need a voice file, english ones
are available from \url{http://www.rockbox.org/daily.shtml}. Download the
``voice'' package for your player and unzip it directly to the root of your \dap.
You should now find an \fname{english.voice} in the \fname{/.rockbox/langs}
directory on your \dap{}. Voice menus are enabled by default and will come
into effect after a reboot. See \reference{ref:Voiceconfiguration} for details
on voice settings.

\section{Running Rockbox}
Remove your \dap{} from the computer's USB port. %
\nopt{ipod,e200}{Unplug any connected power supply and turn the unit off. When
you next turn the unit on, Rockbox should load. }%
\opt{ipod}{Hard resetting the Ipod by holding
  \opt{IPOD_4G_PAD}{\ButtonMenu{}+\ButtonSelect{}}%
  \opt{IPOD_3G_PAD}{\ButtonMenu{}+\ButtonPlay{}}
  for a couple of seconds until the \dap{} resets. Now Rockbox should load.
} %
\opt{e200}{Your e200 will automatically reboot and Rockbox should load. }%
When you see the Rockbox splash screen, Rockbox is loaded and ready for
use.

\opt{ipod}{
  \note{
    Rockbox starts in the \setting{File Browser}. If you have loaded music onto
    your player using Itunes, you will not be able to see your music because
    Itunes changes your files' names and hides them in directories in the
    \fname{Ipod\_Control} directory. You can view files placed on your \dap{} by
    Itunes by initializing and using Rockbox's database. See
    \reference{ref:database} for more information.
  }
}

\section{Updating Rockbox}
Updating Rockbox is easy even if you do not use the Rockbox Utility.
Download a Rockbox build.
(The latest release of the Rockbox software will always be available from
\url{http://www.rockbox.org/download/}). Unzip the build to the root directory
of your \dap{} like you did in the installation step before. If your unzip
program asks you whether to overwrite files, choose the ``Yes to all'' option.
The new build will be installed over your current build.

\note{If you use Rockbox Utility be aware that it can not detect manually
        installed components.}

\section{Uninstalling Rockbox}\index{Installation!uninstall}
\subsection{Automatic Uninstallation}
You can uninstall Rockbox automatically by using Rockbox Utility. If you
installed Rockbox manually you can still use Rockbox Utility for uninstallation
but will not be able to do this selectively.

\opt{h1xx,h300}{\note{Rockbox Utility cannot uninstall the bootloader due to
the fact that it requires a flashing procedure. To uninstall the bootloader
completely follow the manual uninstallation instructions below.}}

\subsection{Manual Uninstallation}
\opt{archos}{
  If you would like to go back to using the original \playerman{} software,
  connect the \dap{} to your computer, and delete the
  \fname{\firmwarefilename} file.
}
\nopt{gigabeatf,m5,x5,archos,mrobe100}
  {
  \note{The Rockbox bootloader can start the original firmware on your \dap.
  (See \reference{ref:Dualboot} for more information.)}}

\opt{h10,h10_5gb}{
  If you would like to go back to using the original \playerman{} software,
  connect the \dap{} to your computer, and delete the
  \opt{h10}{\fname{H10\_20GC.mi4}}\opt{h10_5gb}{\fname{H10.mi4}} file and rename
  \fname{OF.mi4} to \opt{h10}{\fname{H10\_20GC.mi4}}\opt{h10_5gb}{\fname{H10.mi4}}
  in the \fname{System} directory on your \playertype{}. As in the installation,
  it may be necessary to first put your device into UMS mode.
}

\opt{mrobe100}{
  If you would like to go back to using the original \playerman{} software,
  connect the \dap{} to your computer, and delete the
  \fname{pp5020.mi4} file and rename
  \fname{OF.mi4} to \fname{pp5020.mi4}
  in the \fname{System} directory on your \playertype{}.
}

\opt{e200}{
  If you would like to go back to using the original \playerman{} software,
  connect the \dap{} to your computer, and follow the instructions to install
  the bootloader, but when prompted by sansapatcher, enter 'u' for uninstall,
  instead of 'i' for install. As in the installation, it may be necessary to
  first put your device into UMS mode.
}

\optv{ipod}{
  If you would like to go back to using the original \playerman{} software,
  connect the \dap{} to your computer, and follow the instructions to install
  the bootloader, but when prompted by ipodpatcher, enter 'u' for uninstall,
  instead of 'i' for install.
}

\opt{m5,x5}{
  If you would like to go back to using the original \playerman{} software,
  connect the \dap{} to your computer, download the original \playername{}
  firmware from the \playerman{} website. And copy it to the \fname{FIRMWARE}
  directory on your \playername{}. Turn off the \dap{}, remove the USB cable
  and insert the charger. The original firmware will automatically be flashed.
}

\opt{h1xx,h300}{
  \note{
    If you want to remove the Rockbox bootloader, simply flash an unpatched
    \playerman{} firmware. Be aware that doing so will also remove the bootloader
    USB mode. As that mode can come in quite handy (especially when
    having disk errors) it is recommended to keep the bootloader. It also
    gives you the possibility of trying Rockbox anytime later by simply
    installing the distribution files.
    \opt{h1xx}{
      The Rockbox bootloader will automatically start the original firmware if
      the \fname{.rockbox} directory has been deleted.
    }%
    \opt{h300}{%
      Although if you retain the Rockbox bootloader, you will need to hold the
      \ButtonRec{} button each time you want to start the original firmware.
    }
  }
}

If you wish to clean up your disk, you may also wish to delete the
\fname{.rockbox} directory and its contents.
\nopt{m5,x5}{Turn the \playerman{} off.
  Turn the \dap{} back on and the original \playerman{} software will load.
}


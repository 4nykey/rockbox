\subsubsection{Viewport Declaration Syntax}

\config{\%V(x,y,[width],[height],[font]) \%Vf([fgshade]) \%Vb([bgshade])}%

    \begin{itemize}
      \item `fgshade' and `bgshade' are numbers in the range 0 (= black) to 3
      (= white).
      \item `font' is a number: 0 is the built-in system font, 1 is the
      current menu font, and 2-9 are additional are skin loaded fonts (see 
      \reference{ref:multifont}).
      \item Only the coordinates \emph{have} to be specified. Leaving the other
      definitions blank will set them to their default values.
    \end{itemize}

\note{The correct number of commas with hyphens in
      blank fields are still needed in any case.}

\begin{example}
    %V(12,20,-,-,1) %Vf(0) %Vb(3)
    %sThis viewport is displayed permanently. It starts 12px from the left and
    %s20px from the top of the screen, and fills the rest of the screen from
    %sthat point. The lines will scroll if this text does not fit in the viewport.
    %sThe user font is used, the foreground colour is set to black and the
    %sbackground is set to white.
\end{example}
\begin{rbtabular}{.75\textwidth}{XX}{\textbf{Viewport definition} & \textbf{Default value}}{}{}
  width/height & remaining part of screen \\
  font & user defined \\
  shade & black foreground on white background \\
\end{rbtabular}


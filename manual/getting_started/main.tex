% $Id$ %
\chapter{Getting started}
\section{Welcome}
This is the manual for Rockbox. Rockbox is an open source firmware replacement
for a growing number of MP3 players. Rockbox aims to be considerably more
functional and efficient than your device's stock firmware while remaining easy
to use and customizable. Rockbox is written by users, for users. Not only is it
free to use, it's also released under the GNU public license, which means that
it will always remain free to both use and to change.

Rockbox has been in development since 2001, and recieves new features, tweaks
and fixes each day to provide you with the best possible experience on your MP3
player. A major goal of Rockbox is to be simple and easy to use, yet remain very
customizable and configurable. We believe that you should never need to go
through a series of menus for an action you perform frequently. We also believe
that you should be able to configure almost anything about Rockbox you could
want, pertaining to functionality. Another top priority of Rockbox is audio
playback quality -- Rockbox, for most models, includes a wider range of sound
settings than that device's original firmware. A lot of work has been put into
making Rockbox sound the best it can, and improvements are constantly being made.
All models have access to a large number of plugins, including many games,
applications, and graphical ``demos''. You can load different configurations
quickly for different purposes (e.g. a large font for in your car, different
sound settings for at home). Rockbox features a very wide range of languages, and
all supported models also have the ability to talk to you -- menus can be voiced
and filenames spelled out or spoken.

\section{Getting more help}
This manual is intended to be a comprehensive introduction to the Rockbox
software. There is, however, more help available.  The Rockbox website at
\url{http://www.rockbox.org/} contains very extensive documentation and guides
written by members of the Rockbox community and this should be your first port
of call when looking for further help.

\section{Naming conventions and marks}
We have some conventions especially on naming that are intended to be
consistent throughout this manual.

Manufacturer and product names are formatted in accordance with the standard
rules of English grammar, e.g. ``\playerman{} playback is currently
unsupported''. Manufacturer and model names are proper nouns, and
thus are written beginning with a capital letter.

% write a bit more about names etc. here.
\ifpdfoutput{
This manual has some parts that are marked with icons on the margin to help
you finding important parts or parts you could skip. The following icons
are used:
\note{This indicates a note. A note starts always with the text ``Note''.
  For easier finding of notes we have put this an icon in the margin like
  here. Notes are used to mark informations that could help you
  or indicate a possible ``weirdness'' in rockbox that would be explained.
}
\warn{This is a warning. In contrast to notes as mentioned above a warning
  should be taken more seriously. While ignoring notes won't cause any serious
  damage ignoring warnings \emph{could} cause serious damage. If you're new to
  rockbox you should really read the warnings before doing anything that is
  warned about.
}
\blind{This icon marks a section that are intended especially for the blind
  and visually impaired. As they can't
  read the manual in the same way seeing people can do we've added some
  additional descriptions. If you aren't blind or visually impaired you most
  likely can completely skip these blocks. To make this easier, there is an
  icon shown in the margin like here.
}
}{}% end ifpdfoutput

\section{Installing Rockbox}\label{sec:installing_rockbox}
\opt{MASCODEC}{
  \subsection{Using the windows installer}
  Using the Windows self installing executable to install Rockbox is the easiest
  method of installing the software on your Jukebox.  Simply follow the
  on-screen instructions and select the appropriate drive letter and Jukebox
  model when prompted.  You can use ``Add / Remove Programs'' to uninstall the
  software at a later date.

  \subsection{Manual installation}
  For non{}-Windows users and those wishing to install manually from the archive
  the procedure is still fairly simple.
}
\opt{SWCODEC}{
\subsection{Introduction}
There are two separate components of Rockbox that need to be installed in order
to run Rockbox.
\begin{enumerate}
\item The Rockbox bootloader. This is the component of Rockbox that is installed
  to the flash memory of your \playerman. The bootloader is the program that tells
  your \dap\ how to boot and load other components of Rockbox.
\item The Rockbox firmware. Unlike the \playerman\ firmware, which runs entirely
  from flash memory, most of the Rockbox code is contained in the build that
  resides on your jukebox's hard drive. This makes it easy to update Rockbox. The
  build contain a file named \firmwarefilename\ and a directory called
  \fname{.rockbox} which are located in the root directory of your  hard drive.
\end{enumerate}
\opt{h1xx,h300}{% $Id$ %
\subsection{Installing the bootloader}
  Installing the bootloader is the trickiest part of the installation.
  The Rockbox bootloader allows users to boot into either the Rockbox 
  firmware or the iriver firmware. For legal reasons, we cannot distribute 
  the bootloader. Instead, we have developed a program that will patch the 
  Iriver firmware with the Rockbox bootloader. These instructions will explain 
  how to download and patch the Iriver firmware with the Rockbox bootloader 
  and install it on your jukebox.

\begin{enumerate}
  \item Download a supported version of the Iriver firmware for your 
  \playername\ from the Iriver website or from 
  \wikilink{ManualRockboxInstall}.
  Supported Iriver firmware versions currently include 
  \opt{IRIVER_H100_PAD}{1.63US, 1.63EU, 1.63K, 1.65US, 1.65EU, 1.65K, 1.66US, 
    1.66EU and 1.66K.  Note that the H140 uses the same firmware as the H120;
    H120 and H140 owners should use the	firmware called \fname{ihp\_120.hex}.
    Likewise, the iHP110 and iHP115 use the same firmware, called 
    \fname{ihp\_100.hex}.   Be sure to use the correct firmware file for 
    your player.}
  \opt{IRIVER_H300_PAD}{1.28K, 1.28EU, 1.28J, 1.29K, 1.29J and 1.30EU.
    \note{The US H3xx firmware is not currently supported and cannot be
    patched to be used with the bootloader. If you wish to install Rockbox
    on a US \playername\, you must use an international firmware, which will
    permanently remove DRM support from the player.}
  }
  If the file that you downloaded is a \fname{.zip} file, use an unzip 
  utility such as \fname{InfoZip}, \fname{7zip}, \fname{WinRAR},	or 
  \fname{WinZip} to extract the \fname{.hex} from the \fname{.zip} file
  to your desktop. Likewise, if the file that you downloaded is an 
  \fname{.exe} file, double-click on the \fname{.exe}	file to extract 
  the \fname{.hex} file to your desktop.
  %
  \item Download the firmware patcher \fname{fwpatcher.exe} from 
  \url{http://download.rockbox.org/bootloader/iriver/} and save it to your desktop.
    \warn{The firmware patcher contains Unicode support, which is not supported by 
    all versions of Windows. If you have difficulty with the firmware patcher, try 
    downloading the alternate firmware patcher \fname{fwpatchernu.exe}, which is 
    built without Unicode support.}
  %
  \item Go to your desktop and double-click on whichever version of the firmware 
  patcher you downloaded in the prior step.
  %
  \item In the firmware patcher dialog box, click on the BROWSE button and navigate
  to the \fname{.hex} file that you previously downloaded to your desktop.
  %
  \item Click PATCH. The firmware patcher will patch the original firmware to 
  include the Rockbox bootloader. The \fname{.hex} file on your desktop is now
  a modified version of the original \fname{.hex} file.
  %
  \item Turn on your \playername\ and connect it to your computer via USB.
  %
  \item Copy or move the modified \fname{.hex} file to the ROOT directory of 
    your jukebox.
  %
  \item Disconnect the jukebox from USB. (Be sure to use Windows' ``safely remove
  hardware'' option.)
  \warn{Before proceeding further, make sure that your player has a full charge, 
    or that it is connected to the power adaptor.}
  %
  \item Update your \playername s firmware with the patched bootloader. To do this, turn 
    the jukebox on. Press and hold the 
    \opt{IRIVER_H100_PAD}{\ButtonSelect{} button }%
    \opt{IRIVER_H300_PAD}{\ButtonSelect{} button }%
    to enter the main menu, and navigate to \setting{General $\rightarrow$ Firmware 
    Upgrade}. Select \setting{Yes} when asked to confirm if you want to upgrade the 
    firmware. The \playername{} will display a message indicating that the
    firmware update 
    is in progress. Do not interrupt this process. When the firmware update is 
    complete, the player will turn itself off. (The update firmware process usually 
    takes a minute or so.)

    You have now installed the Rockbox bootloader. 

\opt{h1xx}{\note{If you install the Rockbox bootloader, but do not install the
  Rockbox firmware, the Rockbox bootloader will load the iriver firmware when the
  jukebox is turned on.}}

\end{enumerate}
}
\opt{ipod4g,ipodcolor,ipodnano,ipodmini,ipodvideo}
	{% $Id$ %
\subsection{Installing the bootloader}
\warn{These instructions are preliminary and may contain errors! 
Please check the wiki for up-to-date and improved installation instructions!
If you find errors you're of course welcomed to report them so we can fix it
for the next daily builds.}

  Installing the bootloader is the trickiest part of the installation.
  The process is different depending on your operating system, but before
  starting, connect the \dap{} to the computer using either an USB \fixme{or
  Firewire?} cable. Next, create a folder on the computer's harddrive and
  download the following file to that folder:
  \opt{ipodvideo}{\wikilink{IpodInstallation/bootloader-video.bin}}
  \opt{ipodnano}{\wikilink{IpodInstallation/bootloader-nano.bin}}
  \opt{ipodmini}{\wikilink{IpodInstallation/bootloader-mini1g.bin} or 
    \wikilink{IpodInstallation/bootloader-mini1g.bin} depending on which
    generation your \dap{} is.\fixme{Describe how to identify 1/2G}}
  \opt{ipodcolor}{\wikilink{IpodInstallation/bootloader-color.bin}}
  \opt{ipod4g}{\wikilink{IpodInstallation/bootloader-4g.bin}}

  When that is done, proceed to the section below that matches the operating
  system on the computer.
  \note{These instructions all require you to have administrator rights
  on your computer, regardless of the operating system.}
  \note{Rockbox only works on FAT32 partitions (called ``Windows formatted'' by
    Apple). So if your \dap{} is Mac formatted (HFS+), you should first convert
    it to FAT32. Information on how to do this can be found on the Rockbox
    website. \fixme{Include these instructions?}}

\subsubsection{Windows users}
\begin{enumerate}
  \item Download the following two programs and save them in the folder just
    created. These programs will be used in the next steps:
    \begin{itemize}
      \item \wikilink{IpodInstallation/ipodpatcher.exe}
      \item \wikilink{IpodInstallation/ipod_fw.exe}
    \end{itemize}
  \item Locate the \dap{} by opening a command windows. You can do this by
    clicking ``Start'', ``Execute'' and typing \fname{cmd}. Press Enter to
    execute that command. Now change directory to the
    folder you created and run the following commands:
    \begin{code}
    ipodpatcher 0
    ipodpatcher 1
    ipodpatcher 2
    ipodpatcher 3
    \end{code}
    Keep increasing the number until the \dap{} is located. 

    Output for an unsuccessful attempt to contact the \dap{}...
    \begin{code}
    C:/rockbox>ipodpatcher 0
    ipodpatcher v0.3 - (C) Dave Chapman 2006
    This is free software; see the source for copying conditions.  There is NO
    warranty; not even for MERCHANTABILITY or FITNESS FOR A PARTICULAR PURPOSE.

    [INFO] Reading partition table from \textbackslash\textbackslash{}.\textbackslash{}PhysicalDrive0
    Drive is not an iPod, aborting
    \end{code}
    
    A successful connection to the \dap{} will look similar to this...
    \begin{code}
    C:\textbackslash{}rockbox>ipodpatcher 6
    ipodpatcher v0.3 - (C) Dave Chapman 2006
    This is free software; see the source for copying conditions.  There is NO
    warranty; not even for MERCHANTABILITY or FITNESS FOR A PARTICULAR PURPOSE.

    [INFO] Reading partition table from \textbackslash\textbackslash{}.\textbackslash{}PhysicalDrive6
    Part    Start Sector    End Sector    Size (MB)  Type
       0              63        160649        78.4   Empty (0x00)
       1          160650       7984304      3820.1   W95 FAT32 (0x0b)
   \end{code}
    Remember the number that corresponds to your \dap{} -- in the 
    following steps, \emph{N} should be replaced with the number just found.
  \item Now, extract the firmware partition currently on the \dap{} with the
    following command:
    \begin{code}
    ipodpatcher -r \emph{N} bootpartition.bin
    \end{code}
    \note{You should keep a safe backup of this \fname{bootpartition.bin} file
      for use if you ever wish to either upgrade the Rockbox bootloader or
      uninstall Rockbox from your Ipod}
  \item Extract the Apple firmware from the partition image image just created:
    \begin{code}
    ipod_fw -o apple_os.bin -e 0 bootpartition.bin
    \end{code}
\optv{ipodvideo}{
  \item Similarly, extract the Broadcom firmware:
    \begin{code}
    ipod_fw -o apple_sw_5g_rcsc.bin -e 1 bootpartition.bin
    \end{code}
}
  \item Merge the Rockbox bootloader you downloaded previously with the Apple
    firmware:
\optv{ipodnano}{
    \begin{code}
    ipod_fw -g nano -o rockboot.bin -i apple_os.bin bootloader-nano.bin
    \end{code}
}
\optv{ipodvideo}{
    \begin{code}
    ipod_fw -g video -o rockboot.bin -i apple_os.bin bootloader-video.bin
    \end{code}
}
\optv{ipodmini}{
    \begin{code}
    ipod_fw -g mini -o rockboot.bin -i apple_os.bin bootloader-mini1g.bin
    \end{code}
    Or, if you have a 2G mini:
    \begin{code}
    ipod_fw -g mini -o rockboot.bin -i apple_os.bin bootloader-mini2g.bin
    \end{code}
}
\optv{ipodcolor}{
    \begin{code}
    ipod_fw -g color -o rockboot.bin -i apple_os.bin bootloader-color.bin
    \end{code}
}
\optv{ipod4g}{
    \begin{code}
    ipod_fw -g 4g -o rockboot.bin -i apple_os.bin bootloader-4g.bin
    \end{code}
}
\item
    Install the Rockbox-enabled firmware:
    \begin{code}
    ipodpatcher -w \emph{N} rockboot.bin
    \end{code}
\end{enumerate}

Now you can proceed installing the firmware itself.

\subsubsection{Mac OS X users}
\begin{enumerate}
  \item Download the following two programs and save them in the folder just
    created. These programs will be used in the next steps:
    \begin{itemize}
      \item \wikilink{IpodInstallationFromMacOSX/diskdump}
      \item \wikilink{IpodInstallationFromMacOSX/ipod_fw}
    \end{itemize}
    Start a Terminal and type navigate into the folder you created. Before
    you can continue, you need to ensure that Mac OS knows that the
    \fname{ipod\_fw}
    and diskdump files you downloaded are executable programs. To do this,
    type the following command:
    \begin{code}
    chmod +x ipod_fw diskdump
    \end{code}
  \item Locate the \dap{} by running the following command:
    \begin{code}
    mount
    \end{code}
    The output will look something like this: \fixme{Add full example}
    \begin{code}
    /dev/disk1s2 on /Volumes/DAVE_S IPOD 1 (local, nodev, nosuid)
    \end{code}
    In this example, the \dap\ is located at /dev/disk1s2 Remember the 
    location of your \dap\  -- in the following steps, /dev/disk1s2 should be
    replaced with the location just found.
  \item Before continuing, the \dap\ must be ``unmounted'', which is
    done with the following command:
    \begin{code}
    diskutil unmount /dev/disk1s2
    \end{code}
  \item Now, extract the Apple firmware currently on the \dap{} with the
    following command:
    \note{The last part of the location is left out.}
    \begin{code}
    ./diskdump -r /dev/disk1 bootpartition.bin
    \end{code}
    \note{You should keep a safe backup of this \fname{bootpartition.bin} file
      for use if you ever wish to either upgrade the Rockbox bootloader or
      uninstall Rockbox from your iPod
    }
  \item Extract the Apple firmware from this partition image:
    \begin{code}
    ./ipod_fw -o apple_os.bin -e 0 bootpartition.bin
    \end{code}
\optv{ipodvideo}{
  \item Similarly, extract the Broadcom firmware:
    \begin{code}
    ./ipod_fw -o apple_sw_5g_rcsc.bin -e 1 bootpartition.bin
    \end{code}
}
  \item Merge the Rockbox bootloader you downloaded previously with the Apple
    firmware:
\optv{ipodnano}{
    \begin{code}
    ./ipod_fw -g nano -o rockboot.bin -i apple_os.bin bootloader-nano.bin
    \end{code}
}
\optv{ipodvideo}{
    \begin{code}
    ./ipod_fw -g video -o rockboot.bin -i apple_os.bin bootloader-video.bin
    \end{code}
}
\optv{ipodmini}{
    \begin{code}
    ./ipod_fw -g mini -o rockboot.bin -i apple_os.bin bootloader-mini1g.bin
    \end{code}
    Or, if you have a 2G Mini:
    \begin{code}
    ./ipod_fw -g mini -o rockboot.bin -i apple_os.bin bootloader-mini2g.bin
    \end{code}
}
\optv{ipodcolor}{
    \begin{code}
    ./ipod_fw -g color -o rockboot.bin -i apple_os.bin bootloader-color.bin
    \end{code}
}
\optv{ipod4g}{
    \begin{code}
    ./ipod_fw -g 4g -o rockboot.bin -i apple_os.bin bootloader-4g.bin
    \end{code}
}
  \item
    Install the Rockbox-enabled firmware:
    \note{The last part of the location is left out.}
    \begin{code}
    ./diskdump -w /dev/disk1 rockboot.bin
    \end{code}
\end{enumerate}

Now, proceed with installing the firmware itself.

\subsubsection{Linux users}
\begin{enumerate}
  \item Download the following and save it in the folder just
    created:
    \begin{itemize}
      \item \url{http://www.rockbox.org/viewcvs.cgi/*checkout*/tools/ipod_fw.c}
    \end{itemize}
    Now compile it to an executable by opening a command prompt and changing
    to the folder created previously. Thn run the following command:
    \begin{code}
    gcc -o ipod_fw ipod_fw.c
    \end{code}
    If you get the message that the command gcc is not found, you need to
    install gcc. How to do this depends on your Linux distribution, and
    you should consult its documentation for help on this.
  \item Locate your Ipod by running the command \verb|dmesg|. In the output
    something like the following should be seen:
\begin{code}
    usb 4-1: new high speed USB device using ehci_hcd and address 7
    scsi4 : SCSI emulation for USB Mass Storage devices
    usb-storage: device found at 7
    usb-storage: waiting for device to settle before scanning
      Vendor: Apple     Model: iPod              Rev: 1.62
      Type:   Direct-Access                      ANSI SCSI revision: 00
    SCSI device sdb: 58605120 512-byte hdwr sectors (30006 MB)
\end{code}
    You need the device name of your \dap, which you can find in the last line.
    In this example, the \dap\ is located on \fname{/dev/sdb}. In the following,
    \fname{/dev/sdb} should be replaced with the location just found.
  \item Run \verb|fdisk -l /dev/sdb|. Verify that the
    output is similar to the one below:
    \begin{code}
       Device Boot      Start         End      Blocks   Id  System
    /dev/sdb1               1          10       80293+   0  Empty
    /dev/sdb2              11        3648    29222235    b  W95 FAT32
    \end{code}
  \item Back up the partition table using the following command:
    \note{The last part of the location is left out.}
    \begin{code}
    dd if=/dev/\emph{sdb} of=mbr.bin count=1
    \end{code}

  \item Now, extract the firmware partition currently on the \dap{} with the
    following command:
    \begin{code}
    dd if=/dev/\emph{sdb1} of=bootpartition.bin
    \end{code}
    \note{You should keep a safe backup of this \fname{bootpartition.bin} file
      for use if you ever wish to either upgrade the Rockbox bootloader or
      uninstall Rockbox from your Ipod
    }
  \item Extract the Apple firmware from this partition image:
    \begin{code}
    ./ipod_fw -o apple_os.bin -e 0 bootpartition.bin
    \end{code}
\optv{ipodvideo}{
  \item Similarly, extract the Broadcom firmware: 
    \begin{code}
    ./ipod_fw -o apple_sw_5g_rcsc.bin -e 1 bootpartition.bin
    \end{code}
}

  \item Merge the Rockbox bootloader you downloaded previously with the Apple
    firmware: 
\optv{ipodnano}{
    \begin{code}
    ./ipod_fw -g nano -o rockboot.bin -i apple_os.bin bootloader-nano.bin
    \end{code}
}
\optv{ipodvideo}{
    \begin{code}
    ./ipod_fw -g video -o rockboot.bin -i apple_os.bin bootloader-video.bin
    \end{code}
}
\optv{ipodmini}{
    \begin{code}
    ./ipod_fw -g mini -o rockboot.bin -i apple_os.bin bootloader-mini1g.bin
    \end{code}
    Or, if you have a 2G Mini:
    \begin{code}
    ./ipod_fw -g mini -o rockboot.bin -i apple_os.bin bootloader-mini2g.bin
    \end{code}
}
\optv{ipodcolor}{
    \begin{code}
    ./ipod_fw -g color -o rockboot.bin -i apple_os.bin bootloader-color.bin
    \end{code}
}
\optv{ipod4g}{
    \begin{code}
    ./ipod_fw -g 4g -o rockboot.bin -i apple_os.bin bootloader-4g.bin
    \end{code}
}
  \item
    Install the Rockbox-enabled firmware:
    \begin{code}
    dd if=rockboot.bin of=/dev/\emph{sdb1}
    \end{code}
\end{enumerate}
Now you can install the firmware itself.

}
\opt{x5}{\fixme{This is merely a copy of the wiki page IaudioBoot, so this section needs
a more natural language and also error checking by Iaudio owners.}

The \playername{} has a builtin boot loader which performs the
firmware update, and can also access the hard drive via USB. Therefore the
Rockbox bootloader can be very minimalistic, without USB mode.
This also makes it less dangerous to install the Rockbox bootloader, as you can
always restore it using the \playerman{} bootloader.

\note{The current bootloader is not prepared to coexist with the original
firmware. It replaces the original firmware.}

\subsubsection{Installation}
\begin{itemize}
\item Download the Rockbox bootloader binary from 
\url{http://download.rockbox.org/bootloader/iaudio/}.
  \opt{x5}{Use the \fname{x5v\_fw.bin} file if your \dap{} is a X5V. If it is a X5,
    use the \fname{x5\_fw.bin} file.}
  \opt{m5}{Use the \fname{m5\_fw.bin} file.}
\item Copy it to the \fname{FIRMWARE} directory on your \dap{}.
\item Turn the \dap{} off, remove the USB cable and insert the charger. The
Rockbox bootloader will automatically be flashed.
\end{itemize}
}
   
  \subsection{Installing the firmware}
  After installing the bootloader the installation becomes fairly easy.
}

Connect your \playername\ to the computer via USB as described in the 
manual that came with your \playername. On Windows, the \playername\ drive 
will appear as a drive letter in your ``My Computer'' folder. Take the file 
that you downloaded above, and unpack its contents to your \playername\ drive.
You can do this using a program such as \url{http://www.info-zip.org/} or 
\url{http://www.winzip.org/}.

You will need to unpack all of the files in the archive onto your hard disk. If 
this has been done correctly, you will have a file called 
\fname{\firmwarefilename} in the main folder of your \playername\ drive, and
also a folder called /\fname{.rockbox}, which contains a number of system files
used by the software.
\note{Please note that the firmware folder starts with a leading dot. You may
experience problems when trying to create such folders when using Windows.
Directly unzipping to your \dap's drive works flawlessly; it is only Windows'
Explorer that is limited in handling such files.}

\section{Enabling Speech Support (optional)}\label{sec:enabling_speech_support}
If you wish to use speech support you will also need a language file, available
from \wikilink{VoiceFiles}.  For the
English language, the file is called \fname{english.voice}. When it has been
downloaded, unpack this file and copy it into the \fname{lang} folder which is
inside the /\fname{.rockbox} folder on your Jukebox. Voice menus are turned on
by default. See \rockref{ref:Voiceconfiguration} for details on voice
settings.

\section{Running Rockbox}
Remove your Jukebox from the computer's USB port. Unplug any connected power
supply and turn the unit off. When you next turn the unit on, the Jukebox
firmware will start to load, and then it will load Rockbox for you. When you see
the Rockbox splash screen, Rockbox is loaded and ready for use.

\section{Uninstalling Rockbox}
If you would like to go back to using the original \playername\ software, then
connect the \playername\ to your computer, and delete the
\fname{\firmwarefilename} file. If you wish to clean up your disk, you may also
wish to delete the \fname{.rockbox} folder and its contents. Turn the
\playername\ off and on and the normal \playername\ software will load.

\section{Updating Rockbox}
The latest release of the Rockbox software will always be available from
\url{http://www.rockbox.org/download/}.
\opt{MASCODEC}{
  Windows users may wish to download the self-extracting Windows installer,
  which works for all Jukebox models, but those wishing to install manually or
  using a different operating system should choose the .zip archive containing
  the firmware for their model of the Jukebox.
}

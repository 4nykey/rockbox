% $Id$ %
\chapter{\label{ref:rockbox_interface}Quick Start}
\section{Basic Overview}
\subsection{The \daps{} controls}

\begin{center}
% include the front image. Using \specimg makes this fairly easy,
% but requires to use the exact value of \specimg in the filename!
% The extension is selected in the preamble, so no further \ifpdfoutput
% is necessary.
\includegraphics[height=8cm,width=10cm,keepaspectratio=true]{rockbox_interface/images/\specimg-front}
\opt{m3}{% replace with HAVEREMOTEKEYMAP when the h100 file exists or change specimg
  \end{center}
  % spacing between the two pictures, could possibly be improved
  \begin{center}
    \includegraphics[height=5.6cm,width=10cm,keepaspectratio=true]{rockbox_interface/images/\specimg-remote}
}
\end{center}

Throughout this manual, the buttons on the \dap{} are labelled according to the
picture above.
\opt{touchscreen}{
The areas of the touchscreen in the 3x3 grid mode are in turn referred as follows:
\begin{table}
    \begin{center}
    %\begin{tabularx}{.82\textwidth}{l|c|r}
    \begin{tabularx}{.9\textwidth}{X|X|X}
    \toprule
        \TouchTopLeft & \TouchTopMiddle & \TouchTopRight \\
    \midrule
        \TouchMidLeft & \TouchCenter & \TouchMidRight \\
    \midrule
        \TouchBottomLeft & \TouchBottomMiddle & \TouchBottomRight \\
    \bottomrule
    \end{tabularx}
    \end{center}
\end{table}
}
Whenever a button name is prefixed by ``Long'', a long press of approximately
one second should be performed on that button. The buttons are described in
detail in the following paragraph.

\blind{
  Additional information for blind users is available on the Rockbox website at 
  \wikilink{BlindFAQ}.
  
  %
  \opt{h100}{
  Hold or lay the \dap{} so that the side with the joystick and LCD is facing
  towards you, and the curved side is at the top. The joystick functions as
  the \ButtonUp{}, \ButtonRight{}, \ButtonLeft{}, and \ButtonDown{} buttons when
  pressed in the appropriate direction. Pressing the joystick down functions as
  \ButtonSelect{}. 
  On the right side of the \dap{} are the \ButtonOn{}, \ButtonOff{}, 
  \ButtonMode{} buttons, and the \ButtonHold{} switch. When this switch is
  switched towards the bottom of the \dap{}, hold is on, and none of the other
  buttons have any effect.

  On the left side is the \ButtonRec{} button. Above that is the internal microphone. 

  On the top panel of the \dap{}, from left to right, you can find the
  following: headphone mini jack plug, remote port, Optical line-in, Optical Line-out.

  On the bottom panel of the \dap{}, from left to right, you can find the
  following: power jack, reset switch, and USB port. In the event that your
  \dap{} hard locks, you can reset it by inserting a paper clip into the hole
  where the reset switch is.}
  % 
  \opt{h300}{
  Hold or lay the \dap{} so that the side with the button pad and
  LCD is facing towards you.  The buttons on the button pad are as follows:  top 
  left corner: \ButtonOn{}, bottom left corner: \ButtonOff{}, top right corner: 
  \ButtonRec, bottom right corner: \ButtonMode{}.  In the center of the button pad 
  is a button labelled \ButtonSelect{}.  Surrounding the \ButtonSelect{} button are
  the \ButtonUp{}, \ButtonDown{}, \ButtonLeft{}, and \ButtonRight{} buttons.
  
  On the top panel of the \dap{}, from left to right, you can find the 
  following: headphone mini jack plug, remote port, Line-in, Line-out.

  On the left hand side of the \dap{} is the internal microphone. Just underneath
  this is a small hole, the reset switch. In the event that your \dap{} hard locks,
  you can reset it by inserting a paper clip into the hole where the reset switch
  is.

  On the right hand side of the \dap{} is the \ButtonHold{} switch. When this is 
  switched towards the bottom of the \dap{}, hold is on, and none of the other 
  buttons have any effect.
  
  On the bottom panel of the \dap{}, from left to right, you can find the 
  following:  power jack and two USB ports.  The USB port on the right is used 
  to connect your \dap{} to your computer.  The USB port on the left is not 
  used in Rockbox. 
  }
  %
  \opt{ipod4g,ipodcolor,ipodvideo,ipodmini}{ 
  The main controls on the \dap{} are a slightly indented scroll wheel 
  with a flat round button in the center. Hold the \dap{} with these controls 
  facing you. 

  The top of the player will have the following, from left to 
  right:
  \opt{ipod4g,ipodcolor}{remote connector, headphone jack, \ButtonHold{} 
    switch.}
  \opt{ipodvideo}{\ButtonHold{} switch, headphone jack.}
  \opt{ipodmini}{\ButtonHold{} switch, remote connector, headphone jack.}	

  The dock connector that is used to connect your \dap{} to your computer is on 
  the bottom panel of the \dap{}.

  The button in the middle of the wheel is called \ButtonSelect{}. You can
  operate the wheel by pressing the top, bottom, left or right sections,
  or by sliding your finger around it.  The top is \ButtonMenu{}, the bottom is
  \ButtonPlay{}, the left is \ButtonLeft{}, and the right is \ButtonRight{}.
  When the manual says to \ButtonScrollFwd{}, it means to slide your finger
  clockwise around the wheel. \ButtonScrollBack{} means to slide your finger
  counterclockwise. Note that the wheel is sensitive, so you will need to move
  slowly at first and get a feel for how it works.
  
  Note that when the \ButtonHold{} switch is pushed toward the center of the \dap{}, 
  hold is on, and none of the other controls do anything.  Be sure
  \ButtonHold{} is off before trying to use your player. 
  }
  %
  \opt{ipod3g}{ 
  The main controls on the \dap{} are a slightly indented touch wheel 
  with a flat round button in the center, and four buttons in a row above the
  touch wheel. Hold the \dap{} with these controls 
  facing you. 

  The top of the player will have the following, from left to 
  right: remote connector, headphone jack, \ButtonHold{} switch.
	
  The dock connector that is used to connect your \dap{} to your computer is on 
  the bottom panel of the \dap{}.

  The button in the middle of the wheel is called \ButtonSelect{}. You can
  operate the wheel by sliding your finger around it.  The row of
  buttons consists of, from left to right, the \ButtonLeft{},
  \ButtonMenu{}, \ButtonPlay{}, and \ButtonRight{} buttons.
  When the manual says to \ButtonScrollFwd{}, it means to slide your finger
  clockwise around the wheel. \ButtonScrollBack{} means to slide your finger
  counterclockwise. Note that the wheel is sensitive, so you will need to move
  slowly at first and get a feel for how it works.
  
  Note that when the \ButtonHold{} switch is pushed toward the center of the \dap{}, 
  hold is on, and none of the other controls do anything.  Be sure
  \ButtonHold{} is off before trying to use your player. 
  }
  %
  \opt{ipod1g2g}{ 
  The main controls on the \dap{} are a slightly indented wheel 
  with a flat round button in the center, and four buttons surrounding
  it. On the 1st generation iPod, this wheel physically turns. On the
  2nd generation iPod, this wheel is touch-sensitive. Hold the \dap{} with these controls 
  facing you. 

  The top of the player will have the following, from left to 
  right: FireWire port, headphone jack, \ButtonHold{} switch.

  The FireWire port is used to connect your \dap{} to the computer and
  to charge its battery via a wall charger.
	
  The button in the middle of the wheel is called \ButtonSelect{}. You can
  operate the wheel by turning it, or sliding your finger around
  it. The top is \ButtonMenu{}, the bottom is \ButtonPlay{}, the left
  is \ButtonLeft{}, and the right is \ButtonRight{}.
  When the manual says to \ButtonScrollFwd{}, it means to slide your finger
  clockwise around the wheel. \ButtonScrollBack{} means to slide your finger
  counterclockwise. Note that the wheel is sensitive, so you will need to move
  slowly at first and get a feel for how it works.
  
  Note that when the \ButtonHold{} switch is pushed toward the center of the \dap{}, 
  hold is on, and none of the other controls do anything.  Be sure
  \ButtonHold{} is off before trying to use your player. 
  }
  %
  \opt{ipodnano,ipodnano2g}{
  The main controls on the \dap{} are a slightly indented wheel with a
  flat round button in the center. Hold the \dap{} with these controls on the
  top surface. There is a \ButtonHold{} switch at one end, and
  headphone and dock connector at the other; be sure the end with the
  switch is facing away from you.

  The button in the middle of the wheel is called \ButtonSelect{}. You can
  operate the wheel by pressing the top, bottom, left or right sections,
  or by sliding your finger around it.  The top is \ButtonMenu{}, the bottom is
  \ButtonPlay{}, the left is \ButtonLeft{}, and the right is \ButtonRight{}.
  When the manual says to \ButtonScrollFwd{}, it means to slide your finger
  clockwise around the wheel. \ButtonScrollBack{} means to slide your finger
  counterclockwise. Note that the wheel is sensitive, so you will need to move
  slowly at first and get a feel for how it works.

  Note that when the \ButtonHold{} switch is pushed toward the center of the \dap{},
  hold is on, and none of the other controls do anything; be sure \ButtonHold{} is
  off before trying to use your player.
  }
  %
  \opt{ondio}{
  The main characteristic of the Ondio case is the indent on its lower right side, 
  which is the MMC slot. Holding the \dap{} with this slot in the described position
  you'll find the following:

  On the curved top, from left to right, are the headphone jack,
  the \ButtonOff{} button,%
  \opt{recording}{ and the line in jack}.
  Apart from the already mentioned MMC slot, you will find the USB connector on
  the \daps{} right side. Below the LCD, at approximately the center of the \dap{},
  there is the main button pad of the \dap{}. The centre of the button pad dips inward
  and helps to operate the directional keys from there. Located on a two-way button
  strip are the \ButtonLeft{} and \ButtonRight{} keys, with \ButtonUp{} above it
  and \ButtonDown{} below it. The raised button positioned in the lower left of this 
  round crosspad is labelled \ButtonMenu{}.
  }
  %
  \opt{h10,h10_5gb}{
  Hold or lay the \dap{} so that the side with the scroll pad and
  LCD is facing towards you. In the centre below the lcd is the scroll pad. It
  is oriented vertically. Touching the top and bottom half of it acts as the 
  \ButtonScrollUp{}  and \ButtonScrollDown{} buttons respectively. On the left
  of the scroll pad is the \ButtonLeft{} button and on the right is the
  \ButtonRight{} button.
  
  There are three buttons on the right hand side of the \dap{}. From top to 
  bottom, they are: \ButtonRew{}, \ButtonPlay{} and \ButtonFF{}. On the left 
  hand side is the \ButtonPower{} button.

  On the top panel of the \dap{}, from left to right, you can find the 
  following: \ButtonHold{} switch, \opt{h10}{reset pin hole, }remote port and
  headphone mini jack plug. 
  
  On the bottom panel of the \dap{} is the data cable port.}
  %
  \opt{gigabeatf}{
  \note{The following description is for the Gigabeat F, but can also apply for the
  Gigabeat X. The Gigabeat F is slightly larger and more rectangular shaped, while the
  Gigabeat X is smaller and has a slightly tapered back.}

  Hold the \dap{} with the screen on top and the controls on the right hand side.  
  Below the screen is a cross-shaped touch sensitive pad which contains the 
  \ButtonUp{}, \ButtonDown{}, \ButtonLeft{} and \ButtonRight{} controls.  On the
  Gigabeat X, this pad will feel slightly raised up, while it will feel slightly
  sunken in on the Gigabeat F. On the top of the unit, from left to right, are the 
  power socket, the \ButtonHold{} switch, and the headphone socket.  The 
  \ButtonHold{} switch puts the \dap{} into hold mode when it is switched to the 
  right of the unit. The buttons will have no effect when this is the case.  
  
  Starting from the left hand side on the bottom of the unit, nearer to the front
  than the back, is a recessed switch which 
  controls whether the battery is on or off.  When this switch is to the left,
  the battery is disconnected.  This can be used for a hard reset of the unit,
  or if the \dap{} is being placed in storage.  Next to that is a connector for
  the docking station and finally on the right hand side of the bottom of the
  unit is a mini USB socket for connecting directly to USB.
  
  Finally on the right hand side of the unit are some control buttons.  Going from
  the bottom of the unit to the top there is a small round \ButtonA{} buttton then a
  rocker volume switch with of the \ButtonVolDown{} button below the \ButtonVolUp{}
  button.  Above that is are two more small round buttons, the \ButtonMenu{} 
  button and nearest to the top of the unit the \ButtonPower{} button, which is held
  down to turn the \dap{} on or off. If you have a Gigabeat X, these buttons are small
  metallic buttons that are place further up on the right hand side, and closer
  together. The layout is still the same, however.}
  %
  \opt{gigabeats}{
  Hold the \dap{} with the screen on top and the controls on the right hand side.
  Directly below the bottom edge of the screen are two buttons, \ButtonBack{}
  on the left and \ButtonMenu{} on the right. Below them is a cross-shaped pad
  which contains the \ButtonUp{}, \ButtonDown{}, \ButtonLeft{}, \ButtonRight{}
  and \ButtonSelect{} controls.
  On the top of the unit from left to right are the headphone socket and the
  \ButtonHold{} switch.  The \ButtonHold{} switch puts the \dap{} into
  hold mode when it is switched to the right of the unit.
  The buttons will have no effect when this is the case.

  Starting from the left hand side on the bottom of the unit, nearer to the back
  than the front, is a recessed switch which controls whether the battery is on
  or off.  When this switch is to the left, the battery is disconnected.
  This can be used for a hard reset of the unit, or if the \dap{} is being placed
  in storage.  Next to that is a mini USB socket for connecting directly to USB, 
  and finally a custom connector, presumably for planned accessories which were 
  never released.

  Finally on the right hand side of the unit are some control buttons and the power 
  connector.  Going from the bottom of the unit to the top, there is the power 
  connector socket, followed by three small round buttons, the
  \ButtonNext{} buttton, \ButtonPlay{} button, and \ButtonPrev{} button (from bottom
  to top) then a rocker volume switch with of the \ButtonVolDown{} button below the
  \ButtonVolUp{} button.  Above that is one more small round button, the \ButtonPower{}
  button, which is held down to turn the \dap{} on or off.}
  %
  \opt{mrobe100}{
  Hold the \dap{} with the black front facing you such that the m:robe writing 
  is readable. Below the writing is the touch sensitive pad with the 
  \ButtonMenu{}, \ButtonPlay{}, \ButtonLeft{}, \ButtonRight{} and \ButtonDisplay 
  controls indicated by their symbols. The dotted center strip is devided in 
  three parts: \ButtonUp{}, \ButtonSelect{} and \ButtonDown. On the top of the 
  unit, on the right, is the \ButtonPower{} switch, which is held down to turn 
  the \dap{} on or off.
  
  The \ButtonHold{} switch is located on the left of the \dap{}, below the 
  headphone socket. It puts the \dap{} into hold mode when it is switched to the 
  top of the unit. The buttons will have no effect when this is the case. On the 
  bottom of the unit, there is a connector for the docking station or the 
  proprietary USB connector for connecting directly to USB.}
  %
  \opt{x5,m5}{
  The \dap{} is curved so that the end with the screen on it is thicker than the 
  other end.  Hold the \dap{} wih the thick end towards the top and the screen
  facing towards you.  Half way up the front of the unit on the right hand side
  is a four way joystick which is the \ButtonUp{}, \ButtonDown{}, 
  \ButtonLeft{}, and \ButtonRight{} buttons. When pressed it serves as \ButtonSelect{}.
  
  On the right hand side of the \dap{} from top to bottom, first there is a two 
  way switch.  the \ButtonPower{} button is activated by pushing this switch up,
  and pushing this switch down until it clicks slightly will activate the 
  \ButtonHold{} button.  When the switch is in this position, none of the other
  keys will have an effect.
  
  Below the switch is a lozenge shaped button which is the \ButtonRec{} 
  button, and below that the final button on this side of the unit, the 
  \ButtonPlay{} button.  Just below this is a small hole which is difficult to
  locate by touch which is the internal microphone.  At the very bottom of 
  this side of the unit is the reset hole, which can be used to perform a hard
  reset by inserting a paper clip.
  
  On the bottom of the unit is the connector for the 
  \playerman{} subpack or dock.  On the top of the unit is a charge 
  indicator light, which may feel a bit like a button, but is not.
  
  From the top of the \dap{} on the left hand side is the headphone socket, then the 
  remote connector.  Below this is a cover which protects the \opt{x5}{USB host
  connector.}\opt{m5}{USB and charging connector}.}
  %
  \opt{e200,e200v2}{
  Hold the \dap{} with the turning wheel at the front and bottom.  On the bottom left
  of the front of the \dap{} is a raised round button, the \ButtonPower{} button.
  Above and to the left of this, on the outside of the turning wheel are four 
  buttons.  These are the \ButtonUp{}, \ButtonDown{}, \ButtonLeft{} and 
  \ButtonRight{} buttons.  Inside the wheel is the \ButtonSelect{} button.  Turning
  the wheel to the right activates the \ButtonScrollFwd{} function, and to the
  left, the \ButtonScrollBack{} function.  
  
  On the right of the unit is a slot for inserting flash cards.  On the bottom is 
  the connector for the USB cable.  On the left is the \ButtonRec{} button, and
  on the top, there is the headphone socket to the right, and the \ButtonHold{}
  switch.  Moving this switch to the right activates hold mode in which none of the
  other buttons have any effect.  Just to the left of the \ButtonHold{} switch is a
  small hole which contains the internal microphone.}
  %
  \opt{c200}{
  Hold the \dap{} with the buttons on the right and the screen on the left. On
  the right side of the unit, there is a series of four connected buttons that
  form a square. The four sides of the square are the \ButtonUp{},
  \ButtonDown{}, \ButtonLeft{} and \ButtonRight{} buttons, respectively. Inside
  the square formed by these four buttons is the \ButtonSelect{} button. At the
  bottom right corner of the square is a small separate button, the
  \ButtonPower{} button.

  Moving clockwise around the outside of the unit, on the top are the \ButtonVolUp{}
  and \ButtonVolDown{} buttons, which control the volume of playback. The buttons can
  be distinguished by a sunken triangle on the \ButtonVolDown{} button, and a
  raised triangle on the \ButtonVolUp{} button. To the right of
  the volume buttons on the top of the unit is the slot for inserting flash
  memory cards. On the right side of the unit is the connector for the USB
  cable. At center of the bottom of the \dap{} is the \ButtonRec{} button. To
  the left of the \ButtonRec{} button is the \ButtonHold{} switch. Moving this
  switch to the right activates hold mode, in which none of the other buttons
  have any effect. On the lower left side of the unit is the headphone socket.
  Immediately above the headphone socket is a lanyard loop and the microphone.
  }
  %
  \opt{fuze}{
  Hold the \dap{} with the controls on the bottom and the screen on the top. The main
  controls are a scroll wheel with four clickable points and a button in the centre; pressing
  this centre button functions as \ButtonSelect{}. Going clockwise from the top, the clickable
  points on the wheel are the \ButtonUp{}, \ButtonRight{}, \ButtonDown{}, and \ButtonLeft{}
  buttons. Turning the wheel clockwise is \ButtonScrollFwd{}, and turning it counter-clockwise
  is \ButtonScrollBack{}. Immediately above and to the right of the wheel is the \ButtonHome{}
  button.

  On the lower left of the unit is a slot for inserting microSD cards. Immediately below that is
  the opening for the microphone.

  On the bottom of the unit is the connector for connecting a USB cable and the headphone jack.
  On the lower right hand side of the unit is a two-way switch. Pressing this switch up acts as
  \ButtonPower{}, and clicking it down until it locks acts as the \ButtonHold{} switch. When the
  \ButtonHold{} switch is on, none of the other buttons have any effect.
  }
  %
  \opt{clip}{
  Hold the \dap{} with the controls on the bottom and the screen on the top. The main
  controls are a four-way pad with a button in the centre; pressing this centre button
  functions as \ButtonSelect{}. Going clockwise from the top, the four-way pad contains
  the \ButtonUp{}, \ButtonRight{}, \ButtonDown{}, and \ButtonLeft{} buttons. 
  Immediately above and to the right of the four-way pad is the \ButtonHome{} button.

  On the left hand panel is a two way switch. Pressing this switch up acts as
  \ButtonPower{}, and clicking it down until it locks acts as the \ButtonHold{}
  switch. When the \ButtonHold{} switch is on, none of the other buttons have any
  effect. Immediately above the switch is a mini-USB port to connect the \dap{} to
  a computer.

  On the right hand panel is a two-way button that acts as \ButtonVolDown{} when
  pressed on the bottom, and \ButtonVolUp{} when pressed on the top. Immediately
  above this button is the headphone jack.
  }
  %
  \opt{player}{
  The main controls of this player are a four-way button on the right below
  the screen, and two round buttons to the left of it. Hold the \dap{} with
  these controls on the bottom and facing you.

  On the left hand side, the higher of the two small buttons is the \ButtonOn{},
  the lower of the two buttons is the \ButtonMenu{} button. The large circular
  button on the right contains, clockwise from the top, the \ButtonPlay{},
  the \ButtonRight{}, the \ButtonStop{}, and the \ButtonLeft{} buttons.

  On the top on the \dap{} is the headphone jack on the left and the Line-Out
  jack on the right. On the bottom of the \dap{} is the Line-In jack on the left,
  the DC-In jack on the right, and the USB connector in the centre.
  }
  %
  \opt{recorder}{
  Holding the Jukebox in front of you, there should be three rectangular buttons
  in a horizontal line towards the middle of the unit, and below this to the left
  there is a circular four button array with the circular \ButtonPlay{} button
  as a fifth button in the centre. These are the navigation controls. Below the
  rectangular buttons and to the right of the circular buttons are two small round
  buttons one above the other.

  The \ButtonOn{} button is the topmost of the two buttons located below and to the
  left of the navigation controls whereas the lower of these two is called \ButtonOff.
  The small round button in the middle of the large circular button array is called
  \ButtonPlay{} button. To the right of the \ButtonPlay{} button there is the
  \ButtonRight{} button, left of it is the \ButtonLeft{}, above it \ButtonUp, and
  below the \ButtonPlay{} button there is the \ButtonDown{} button placed. In the row
  of three rectangular buttons the following buttons can be found (from left to right):
  \ButtonFOne{}, \ButtonFTwo{} and \ButtonFThree{}.

  On the top of the \dap{} is the headphone jack on the left and the Line-Out jack on
  the right. On the bottom of the \dap{} is the Line-In jack on the left, the
  DC-In jack on the right, and the USB connector in the centre.
  }
  \opt{recorderv2fm}{
  Holding the Jukebox in front of you, there should be three rectangular buttons
  in a horizontal line towards the middle of the unit, and below this centred on the
  middle button there are four radial arc shaped buttons placed in a cross formation
  with the circular play button as the centre of the cross. These are the navigation
  controls. Below the cross and to the left are two other buttons.

  The \ButtonOn{} button is the leftmost of the two buttons located below and to the
  left of the navigation controls whereas the rightmost and little lower one of
  these two is called \ButtonOff{}. The round button raised slightly higher than the
  others in the centre of the navigation controls is the \ButtonPlay{} button.  To
  the right of the \ButtonPlay{} button  there is the \ButtonRight{} button, left of
  it is the \ButtonLeft{}, above it \ButtonUp{}, and below the \ButtonPlay{} button
  there is the \ButtonDown{} button  placed. In the row of three rectangular buttons
  the following buttons can be found (from left to right): \ButtonFOne{}, \ButtonFTwo{}
  and \ButtonFThree{}.
  }
}

\subsection{Turning the \dap{} on and off}
\opt{cowond2}{Rockbox has a dual-boot feature with the original firmware being
  the default.\\}
To turn on and off your Rockbox enabled \dap{} use the following keys:
  \begin{table}
    \begin{btnmap}{}{}
      \opt{IRIVER_H100_PAD,IRIVER_H300_PAD}{\ButtonOn}%
      \opt{IPOD_4G_PAD}{\ButtonMenu{} / \ButtonSelect}%
      \opt{IPOD_3G_PAD}{\ButtonMenu{} / \ButtonPlay}%
      \opt{ONDIO_PAD}{\ButtonOff}\opt{RECORDER_PAD,PLAYER_PAD}%
          {Long \ButtonOn}%
      \opt{IAUDIO_X5_PAD,IRIVER_H10_PAD,SANSA_E200_PAD,SANSA_C200_PAD%
          ,GIGABEAT_PAD,MROBE100_PAD,GIGABEAT_S_PAD,sansaAMS}{\ButtonPower}%
      \opt{COWON_D2_PAD} {\ButtonPower{}, then \ButtonHold}%
    \opt{HAVEREMOTEKEYMAP}{&
      \opt{IRIVER_RC_H100_PAD}{\ButtonRCOn}%
          }
      & Start Rockbox.\\
      \opt{IRIVER_H100_PAD,IRIVER_H300_PAD}{Long \ButtonOff}%
      \opt{IPOD_4G_PAD,IPOD_3G_PAD}{Long \ButtonPlay}%
      \opt{ONDIO_PAD,recorderv2fm}{Long \ButtonOff}%
      \opt{recorder}{Double tap \ButtonOff\ when playback is stopped}%
      \opt{PLAYER_PAD}{From the Main Menu, select \textbf{Shutdown}}%
      \opt{IAUDIO_X5_PAD,IRIVER_H10_PAD,SANSA_E200_PAD,SANSA_C200_PAD%
          ,GIGABEAT_PAD,MROBE100_PAD,GIGABEAT_S_PAD,sansaAMS,COWON_D2_PAD}%
          {Long \ButtonPower}%
    \opt{HAVEREMOTEKEYMAP}{& 
      \opt{IRIVER_RC_H100_PAD}{Long \ButtonRCStop}%
          }
      & Shutdown Rockbox.\\
    \end{btnmap}
  \end{table}

\label{ref:Safeshutdown}On shutdown, Rockbox automatically saves its settings.

\opt{ipod4g,ipodcolor,ipodvideo,ipodnano,ipodmini}{%
  A low-battery symbol may appear briefly on the screen during shutdown.  This
  is a side effect of the shutdown process and does not mean the battery is
  actually low.
}%

\opt{IRIVER_H100_PAD,IRIVER_H300_PAD,IAUDIO_X5_PAD,SANSA_E200_PAD%
  ,SANSA_C200_PAD,IRIVER_H10_PAD,IPOD_4G_PAD,GIGABEAT_PAD}{%
  If you have problems with your settings, such as accidentally having
  set the colours to black on black, they can be reset at boot time.  See
  the Reset Settings in \reference{ref:manage_settings_menu} for details.
}%

\opt{PLAYER_PAD,RECORDER_PAD,ONDIO_PAD,GIGABEAT_PAD,IPOD_4G_PAD,SANSA_E200_PAD%
,SANSA_C200_PAD,IAUDIO_X5_PAD,IAUDIO_M5_PAD,IPOD_3G_PAD}{%
  In the unlikely event of a software failure, hardware poweroff or reset can be
  performed by holding down \opt{PLAYER_PAD}{\ButtonStop}\opt{RECORDER_PAD,ONDIO_PAD}
  {\ButtonOff}\opt{GIGABEAT_PAD}{the battery switch}\opt{IPOD_4G_PAD}
  {\ButtonMenu + \ButtonSelect}\opt{IPOD_3G_PAD}{\ButtonMenu + \ButtonPlay}
  \opt{SANSA_E200_PAD,SANSA_C200_PAD,IAUDIO_X5_PAD,IAUDIO_M5_PAD}
  {\ButtonPower} until the \dap{} shuts off or reboots.
}%
\opt{IRIVER_H100_PAD,IRIVER_H300_PAD,IAUDIO_M3_PAD,IRIVER_H10_PAD,MROBE100_PAD}{%
  In the unlikely event of a software failure, a hardware reset can be
  performed by inserting a paperclip gently into the Reset hole.
}%

\nopt{gigabeatf,m3,m5,x5,archos}
  {
  \subsection{Starting the original firmware}
  \label{ref:Dualboot}
  \opt{ipod4g,ipodcolor,ipodvideo,ipodnano,ipodnano2g,ipodmini}
    {
    Rockbox has a dual-boot feature. To boot into the original firmware, shut
    down the device as described above. Turn on the \ButtonHold{} switch
    immediately after turning the player on. The Apple logo will
    display for a few seconds as Rockbox loads the original firmware.
    
    You can also load the original firmware by shutting down the device,
    then clicking the \ButtonHold{} switch on and connecting the iPod
    to your computer.
 
    Regardless of which method you use to boot to the original firmware, you can
    return to Rockbox by pressing and holding \ButtonMenu{} and \ButtonSelect{}
    simultaneously until the player hard resets.
    }

  \opt{ipod1g2g,ipod3g}
    {
    Rockbox has a dual-boot feature. To boot into the original firmware, shut
    down the device as described above. Turn on the \ButtonHold{} switch
    immediately after turning the player on. The Apple logo will
    display for a few seconds as Rockbox loads the original firmware.
    
    You can also load the original firmware by shutting down the device,
    then clicking the \ButtonHold{} switch on and connecting the iPod
    to your computer.
 
    Regardless of which method you use to boot to the original firmware, you can
    return to Rockbox by pressing and holding \ButtonMenu{} and \ButtonPlay{}
    simultaneously until the player hard resets.
    }

  \opt{h100,h300}
    {
    Rockbox has a dual-boot feature. To boot into the original firmware,
    when the \dap{} is turned off, press and hold the \ButtonRec{} button,
    and then press the \ButtonOn{} button.
    }

  \opt{h10,h10_5gb}
    {
    Rockbox has a dual-boot feature. It loads the original firmware from
    the file \fname{/System/OF.mi4}. To boot into the original firmware,
    press and hold the \ButtonLeft{} button while turning on the player.
    \note{The iriver firmware does not shut down properly when you turn it off,
    it only goes to sleep. To get back into Rockbox when exiting from the
    iriver firmware, you will need to reset the player by \opt{h10}{inserting a
    pin in the reset hole}\opt{h10_5gb}{removing and reinserting the battery}.}
    }
    
  \opt{sansa,sansaAMS}
    {
    Rockbox has a dual-boot feature. To boot into the original firmware,
    press and hold the \ButtonLeft{} button while turning on the player.
    }

  \opt{sansaAMS}
    {
    The player will always boot into the original firmware if it is powered
    by a USB connection, and additionally will do so if USB is inserted while
    rockbox is running without holding \ActionStdUsbCharge{}. This feature may
    be removed in the future when Rockbox is able to handle USB transfers 
    natively.
    }

  \opt{mrobe100}
    {
    Rockbox has a dual-boot feature. It loads the original firmware from
    the file \fname{/System/OF.mi4}. To boot into the original firmware,
    when the \dap{} is turned off, press the \ButtonPower{} button once and then 
    a second time when the m:robe bootlogo (the headphone) appears. Hold the
    \ButtonPower{} button until you see the ``Loading original firmware...'' 
    message on the screen.
    }

  \opt{gigabeats}
    {
    Rockbox has a dual-boot feature. To boot into the original firmware,
    turn the \ButtonHold{} switch on just after turning on the \dap{}.
    }

  \opt{cowond2}
    {
    Use \ButtonPower{} to boot the original \playerman{} firmware.
    }

  }
\subsection{Putting music on your \dap{}}

\opt{usb_hid}{
\note{Due to a bug in some OS X versions, the \dap{} can not be mounted, unless
    the USB HID feature is disabled. See \reference{ref:USB_HID} for more
    information.\newline
}
}

With the \dap{} connected to the computer as an MSC/UMS device (like a
USB Drive), music files can be put on the player via any standard file
transfer method that you would use to copy files between drives (e.g. Drag 'n' Drop).
The default directory structure that is assumed by some parts of Rockbox
\opt{albumart}{%
    (album art searching, and missing-tag fallback in some WPSes) uses the
    parent directory of a song as the Album name, and the parent directory of
    that folder as the Artist name. While files may be organized however you
    like, see \reference{ref:album_art} for the requirements for Album
    Art to work properly, and WPSes may display information incorrectly if your
    files are not properly tagged, and you have your music organized in a way
    different than they assume when attempting to guess the Artist and Album
    names from your filetree.
}%
\nopt{albumart}{%
    (missing-tag fallback in some WPSes) uses the parent directory of a song
    as the Album name, and the parent directory of that folder as the Artist
    name. While files may be organized however you like, WPSes may display
    information incorrectly if your files are not properly tagged, and you have
    your music organized in a way different than they assume when attempting to
    guess the Artist and Album names from your filetree.
}
\opt{swcodec}{
    See \reference{ref:Supportedaudioformats} for a list of supported audio
    formats.
}

\subsection{The first contact}

After you have first started the \dap{}, you'll be presented by the
\setting{Main Menu}. From this menu you can reach every function of Rockbox,
for more information (see \reference{ref:main_menu}). To browse the files
on you \dap{}, select \setting{Files} (see \reference{ref:file_browser}), and to
browse in a view that is based on the meta-data\footnote{ID3 Tags, Vorbis
comments, etc.} of your audio files, select \setting{Database} (see
\reference{ref:database}).

\subsection{Basic controls}
When browsing files and moving through menus you usually get a list view
presented. The navigation in these lists are usually the same and should be
pretty intuitive.
In the tree view use \ActionStdNext{} and \ActionStdPrev{} to move around
the selection. Use \ActionStdOk{} to select an item. When browsing the file
system selecting an audio file plays it. The view switches to the ``While
playing screen'', usually abbreviated as ``WPS'' (see \reference{ref:WPS}. The
dynamic playlist gets replaced with the contents of the current directory. This
way you can easily treat directories as playlists. The created dynamic playlist can
be extended or modified while playing. This is also known as
``on-the-fly playlist''.
To go back to the \setting{File Browser} stop the playback with the
\ActionWpsStop{} button or return to the file browser while keeping playback
running using \ActionWpsBrowse{}.
In list views you can go back one step with \ActionTreeParentDirectory.

\subsection{Basic concepts}
\subsubsection{Playlists}
Rockbox is playlist oriented. This means that every time you play an audio file,
a so-called ``dynamic playlist'' is generated, unless you play a saved
playlist. You can modify the dynamic playlist while playing and also save
it to a file. If you do not want to use playlists you can simply play your
files directory based.
Playlists are covered in detail in \reference{ref:working_with_playlists}.

\subsubsection{Menu}
From the menu you can customise Rockbox. Rockbox itself is very customisable.
Also there are some special menus for quick access to frequently used
functions.

\subsubsection{Context Menu}
Some views, especially the file browser and the WPS have a context menu.
From the file browser this can be accessed with \ActionStdContext{}.
The contents of the context menu vary, depending on the situation it gets
called. The context menu itself presents you with some operations you can
perform with the currently highlighted file. In the file browser this is
the file (or directory) that is highlighted by the cursor. From the WPS this is
the currently playing file. Also there are some actions that do not apply
to the current file but refer to the screen from which the context menu
gets called. One example is the playback menu, which can be called using
the context menu from within the WPS.

\section{Customising Rockbox}
Rockbox' User Interface can be customised using ``Themes''. Themes usually
only affect the visual appearance, but an advanced user can create a theme
that also changes various other settings like file view, LCD settings and
all other settings that can be modified using \fname{.cfg} files. This topic
is discussed in more detail in \reference{ref:manage_settings}.
The Rockbox distribution comes with some themes that should look nice on
your \dap{}.
\opt{lcd_bitmap}{
\note{Some of the themes shipped with Rockbox need additional
fonts from the fonts package, so make sure you installed them.
Also, if you downloaded additional themes from the Internet make sure you
have the needed fonts installed as otherwise the theme may not display
properly.}
}

\opt{usb_charging}
{
    \section{Charging}

    The \dap{} can be powered over USB without connecting to your
    computer by holding \ActionStdUsbCharge{} while plugging in. This
    allows you to continue using the \dap{} normally.
}

% $Id$ %
\chapter{Browsing and playing}
\section{\label{ref:file_browser}File Browser}
\screenshot{rockbox_interface/images/ss-file-browser}{The file browser}{}
Rockbox lets you browse your music in either of two ways. The 
\setting{File Browser} lets you navigate through the files and directories on 
your \dap, entering directories and executing the default action on each file.
To help differentiate files, each file format is displayed with an icon. 

The \setting{Database Browser}, on the other hand, allows you to navigate 
through the music on your player using categories like album, artist, genre,
etc.

You can select whether to browse using the \setting{File Browser} or the
\setting{Database Browser} by selecting either \setting{Files} or
\setting{Database} in the \setting{Main Menu}.
If you choose the \setting{File Browser}, the \setting{Show Files} setting
lets you select what types of files you wish to view. See
\reference{ref:ShowFiles} for more information on the \setting{Show Files}
setting.

\note{The \setting{File Browser} allows you to manipulate your files in ways
that are not available within the \setting{Database Browser}. Read more about
\setting{Database} in \reference{ref:database}. The remainder of this section
deals with the \setting{File Browser}.}

\opt{ondio}{
Unlike the Archos Firmware, Rockbox provides multivolume support for the
MultiMediaCard, this means the \dap{} can access both data volumes (internal
memory and the MMC), thus being able to for instance, build playlists with
files from both volumes.
In the \setting{File Browser} a new directory will appear as soon as the device
has read the content after inserting the card. This new directory's name is
generated as \fname{<MMC1>}, and will behave exactly as any other directory
on the \dap{}.
}

\opt{iriverh10,iriverh10_5gb}{\note{
If your \dap{} is a MTP model, the Music directory where all your music is stored
may be hidden in the \setting{File Browser}. This may be fixed by either
changing its properties (on a computer) to not hidden, or by changing
the \setting{Show Files} setting to all.
}}

\subsection{\label{ref:controls}File Browser Controls}
\begin{btnmap}
      \ActionStdPrev{}/\ActionStdNext{}
      \opt{HAVEREMOTEKEYMAP}{& \ActionRCStdPrev{}/\ActionRCStdNext{}}
         & Go to previous/next item in list. If you are on the first/last 
           entry, the cursor will wrap to the last/first entry.\\
      %
      \opt{IRIVER_H100_PAD,IRIVER_H300_PAD,RECORDER_PAD}
        {
          \ButtonOn+\ButtonUp{}/ \ButtonDown
          \opt{HAVEREMOTEKEYMAP}{&
            \opt{IRIVER_RC_H100_PAD}{\ButtonRCSource{}/ \ButtonRCBitrate}
          }
          & Move one page up/down in the list.\\
        }
      \opt{IRIVER_H10_PAD}
        {
          \ButtonRew{}/ \ButtonFF
          & Move one page up/down in the list.\\
        }
      %
      \ActionTreeParentDirectory
      \opt{HAVEREMOTEKEYMAP}{& \ActionRCTreeParentDirectory}
      & Go to the parent directory.\\
      %
      \ActionTreeEnter
      \opt{HAVEREMOTEKEYMAP}{& \ActionRCTreeEnter}
      & Execute the default action on the selected file or enter a
        directory.\\
      %
      \ActionTreeWps 
      \opt{HAVEREMOTEKEYMAP}{& \ActionRCTreeWps}
         & If there is an audio file playing, return to the
         \setting{While Playing Screen} (WPS) without stopping playback.\\
      %
      \nopt{player,SANSA_C200_PAD}%
        {%
          \ActionTreeStop 
          \opt{HAVEREMOTEKEYMAP}{& \ActionRCTreeStop}
          & Stop audio playback.\\%
        }%
      %
      \ActionStdContext{}
      \opt{HAVEREMOTEKEYMAP}{& \ActionRCStdContext}
      & Enter the \setting{Context Menu}.\\
      %
      \ActionStdMenu{}
      \opt{HAVEREMOTEKEYMAP}{& \ActionRCStdMenu}
      & Enter the \setting{Main Menu}.\\
      %
      \opt{quickscreen}{
        \ActionStdQuickScreen
        \opt{HAVEREMOTEKEYMAP}{& \ActionRCStdQuickScreen}
        & Switch to the \setting{Quick Screen}
        (see \reference{ref:QuickScreen}). \\
      }
      \opt{RECORDER_PAD}{
        \ButtonFThree & Switch to the \setting{Quick Screen}.\\ 
        %
      }
      %
      \opt{SANSA_E200_PAD}{
        \ActionStdRec & Switch to the \setting{Recording Screen}.\\
      %
      }
      \nopt{touchscreen}{\opt{hotkey}{
        \ActionTreeHotkey
            &
        \opt{HAVEREMOTEKEYMAP}{
            &}
        Activate the \setting{Hotkey} function
        (see \reference{ref:Hotkeys}).
            \\
      }}
\end{btnmap}

\opt{RECORDER_PAD}{
  The functions of the F keys are also summarised on the button bar at the
  bottom of the screen.
}

\subsection{\label{ref:Contextmenu}\label{ref:PartIISectionFM}Context Menu}
\screenshot{rockbox_interface/images/ss-context-menu}{The Context Menu}{}

The \setting{Context Menu} allows you to perform certain operations on files or 
directories.  To access the \setting{Context Menu}, position the selector over a file 
or directory and access the context menu with \ActionStdContext{}.\\

\note{The \setting{Context Menu} is a context sensitive menu.  If the 
\setting{Context Menu} is invoked on a file, it will display options available 
for files.  If the \setting{Context Menu} is invoked on a directory, 
it will display options for directories.\\}

The \setting{Context Menu} contains the following options (unless otherwise noted, 
each option pertains both to files and directories):

\begin{description}
\item [Playlist.]
  Enters the \setting{Playlist Submenu} (see \reference{ref:playlist_submenu}).
\item [Playlist Catalog.]
  Enters the \setting{Playlist Catalog Submenu} (see 
  \reference{ref:playlist_catalog}).
\item [Rename.]
  This function lets the user modify the name of a file or directory.
\item [Cut.]
  Copies the name of the currently selected file or directory to the clipboard
  and marks it to be `cut'.
\item [Copy.]
  Copies the name of the currently selected file or directory to the clipboard
  and marks it to be `copied'.
\item [Paste.]
  Only visible if a file or directory name is on the clipboard. When selected
  it will move or copy the clipboard to the current directory.
\item [Delete.]
  Deletes the currently selected file. This option applies only to files, and
  not to directories. Rockbox will ask for confirmation before deleting a file.
  Press \ActionYesNoAccept{}
  to confirm deletion or any other key to cancel.
\item [Delete Directory.]
  Deletes the currently selected directory and all of the files and subdirectories
  it may contain. Deleted directories cannot be recovered. Use this feature with
  caution!
\opt{lcd_non-mono}{
\item [Set As Backdrop.]
  Set the selected \fname{bmp} file as background image. The bitmaps need to meet the
  conditions explained in \reference{ref:LoadingBackdrops}.
}
\item [Open with.]
  Runs a viewer plugin on the file. Normally, when a file is selected in Rockbox,
  Rockbox automatically detects the file type and runs the appropriate plugin.
  The \setting{Open With} function can be used to override the default action and
  select a viewer by hand.
  For example, this function can be used to view a text file
  even if the file has a non-standard extension (i.e., the file has an extension
  of something other than \fname{.txt}). See \reference{ref:Viewersplugins}
  for more details on viewers.
\item [Create Directory.]
  Create a new directory in the current directory on the disk.
\item [Properties.]
  Shows properties such as size and the time and date of the last modification
  for the selected file. If used on a directory, the number of files and
  subdirectories will be shown, as well as the total size.
\opt{recording}{
  \item [Set As Recording Directory.]
    Save recordings in the selected directory.
}
\item [Add to Shortcuts.]
  Adds a link to the selected item in the \fname{shortcuts.link} file.
  If the file does not already exist it will be created in the root directory.
  Note that if you create a shortcut to a file, Rockbox will not open it upon
  selecting, but simply bring you to its location in the \setting{File Browser}.
\end{description}

\subsection{\label{sec:virtual_keyboard}Virtual Keyboard}
\screenshot{rockbox_interface/images/ss-virtual-keyboard}{The virtual keyboard}{}
This is the virtual keyboard that is used when entering text in Rockbox, for 
example when renaming a file or creating a new directory.
\nopt{player}{The virtual keyboard can be easily changed by making a text file
 with the required layout. More information on how to achieve this can be found
 on the Rockbox website at \wikilink{LoadableKeyboardLayouts}.}

\opt{morse_input}{
  Also you can switch to Morse code input mode by changing the
  \setting{Use Morse Code Input} setting%
  \opt{IRIVER_H100_PAD,IRIVER_H300_PAD,IPOD_4G_PAD,IPOD_3G_PAD,IRIVER_H10_PAD%
      ,GIGABEAT_PAD,GIGABEAT_S_PAD,MROBE100_PAD,SANSA_E200_PAD,PBELL_VIBE500_PAD}%
    { or by pressing \ActionKbdMorseInput{} in the virtual keyboard}%
  .}

\nopt{player}{% no "Actions" yet in the Player's virtual keyboard

\note{When the cursor is on the input line, \ActionKbdSelect{} deletes the preceding character}

\begin{btnmap}
    \opt{IRIVER_H100_PAD,IRIVER_H300_PAD,RECORDER_PAD,GIGABEAT_PAD,GIGABEAT_S_PAD%
        ,MROBE100_PAD,SANSA_E200_PAD,SANSA_FUZE_PAD,SANSA_C200_PAD}{
        \ActionKbdCursorLeft{} / \ActionKbdCursorRight
            &
        \opt{HAVEREMOTEKEYMAP}{\ActionRCKbdCursorLeft{} / \ActionRCKbdCursorRight
            &}
        Move the line cursor within the text line.
            \\
        %
        \ActionKbdBackSpace
            &
        \opt{HAVEREMOTEKEYMAP}{
            &}
        Delete the character before the line cursor.
            \\
    }%
    \ActionKbdLeft{} / \ActionKbdRight
        &
    \opt{HAVEREMOTEKEYMAP}{\ActionRCKbdLeft{} / \ActionRCKbdRight
        &}
    Move the cursor on the virtual keyboard.
    If you move out of the picker area, you get the previous/next page of
    characters (if there is more than one).
        \\
    %
    \ActionKbdUp{} / \ActionKbdDown
        &
    \opt{HAVEREMOTEKEYMAP}{\ActionRCKbdUp{} / \ActionRCKbdDown
        &}
    Move the cursor on the virtual keyboard.
    If you move out of the picker area you get to the line edit mode.
        \\
    %
    \nopt{IPOD_3G_PAD,IPOD_4G_PAD,IRIVER_H10_PAD,ONDIO_PAD,PBELL_VIBE500_PAD}{
        \ActionKbdPageFlip
            &
        \opt{HAVEREMOTEKEYMAP}{\ActionRCKbdPageFlip
            &}
        Flip to the next page of characters (if there is more than one).
            \\
    }
    %
    \ActionKbdSelect
        &
    \opt{HAVEREMOTEKEYMAP}{\ActionRCKbdSelect
        &}
    Insert the selected keyboard letter at the current line cursor position.
        \\
    %
    \ActionKbdDone
        &
    \opt{HAVEREMOTEKEYMAP}{\ActionRCKbdDone
        &}
    Exit the virtual keyboard and save any changes.
        \\
    %
    \ActionKbdAbort
        &
    \opt{HAVEREMOTEKEYMAP}{\ActionRCKbdAbort
        &}
    Exit the virtual keyboard without saving any changes.
        \\
% to be done - create a separate section for morse imput and update the info
      \opt{morse_input}{
        \opt{IRIVER_H100_PAD,IRIVER_H300_PAD,GIGABEAT_PAD,GIGABEAT_S_PAD,MROBE100_PADD%
            ,SANSA_E200_PA,IPOD_4G_PAD,IPOD_3G_PAD,IRIVER_H10_PAD,PBELL_VIBE500_PAD}{
          \ActionKbdMorseInput
          \opt{HAVEREMOTEKEYMAP}{& \ActionRCKbdMorseInput}
          & Toggle keyboard input mode and Morse code input mode. \\}
        %
        \ActionKbdMorseSelect
        \opt{HAVEREMOTEKEYMAP}{& \ActionRCKbdMorseSelect}
        & Tap to select a character in Morse code input mode. \\
      } 
\end{btnmap}
}% end of non-Player section

\opt{player}{
  The current text line to be entered or edited is always listed on the first
  line of the display. The second line of the display can contain the character
  selection bar, as in the screenshot above.
    \begin{btnmap}
      \ButtonOn & Toggle picker- and line edit mode. \\
      \ButtonLeft{} / \ButtonRight
        & Move back and forth in the selected line (picker of input line). \\
      \ButtonPlay
        & Pick character in character bar, or act as backspace in the text line. \\
      Long \ButtonPlay & Accept \\
      \ButtonStop & Cancel \\
      \ButtonMenu & Flip picker lines. \\
    \end{btnmap}
}

% $Id$ %
\section{\label{ref:database}Database}

\subsection{Introduction}
This chapter describes the Rockbox music database system. Using the information
contained in the tags (ID3v1, ID3v2%
  \opt{swcodec}{, Vorbis Comments, Apev2, etc.}%
) in your audio files, Rockbox builds and maintains a database of the music
files on your player and allows you to browse them by Artist, Album, Genre, 
Song Name, etc.  The criteria the database uses to sort the songs can be completely
 customised. More information on how to achieve this can be found on the Rockbox
 website at \wikilink{DataBase}. 

\subsection{Initializing the Database}
The first time you use the database, Rockbox will scan your disk for audio files.
This can take quite a while depending on the number of files on your \dap{}.
This scan happens in the background, so you can choose to return to the
Main Menu and continue to listen to music.
If you shut down your player, the scan will continue next time you turn it on.
After the scan is finished you may be prompted to restart your \dap{} before
you can use the database.

\subsubsection{Ignoring Directories During Database Initialization}

You may have directories on your \dap{} whose contents should not be added
to the database. Placing a file named \fname{database.ignore} in a directory
will exclude the files in that directory and all its subdirectories from
scanning their tags and adding them to the database. This will speed up the
database initialization.

If a subdirectory of an 'ignored' directory should still be scanned, place a
file named \fname{database.unignore} in it. The files in that directory and
its subdirectories will be scanned and added to the database.

\subsection{\label{ref:databasemenu}The Database Menu}

\begin{description}
  \opt{swcodec}{
  \item[Load To RAM]
    The database can either be kept on disk (to save memory), or
    loaded into RAM (for fast browsing). Setting this to \setting{Yes} loads
    the database to RAM, allowing faster browsing and searching. Setting this
    option to \setting{No} keeps the database on the disk, meaning slower 
    browsing but it does not use extra RAM and saves some battery on boot up. 
    
    \note{If you browse your music frequently using the database, you should
      load to RAM, as this will reduce the overall battery consumption because
      the disk will not need to spin on each search.}
  }
  
\item[Auto Update]
  If \setting{Auto update} is set to \setting{on}, each time the \dap{}
  boots, the database will automatically be updated.
  \opt{swcodec}{
    \note{The \setting{Auto Update} will only check for deleted files if the
      \setting{Directory Cache} (\setting{Settings $\rightarrow$ General
      Settings $\rightarrow$ System $\rightarrow$ Disk $\rightarrow$
      Directory Cache}) is enabled. \setting{Update now} includes that check
      whether dircache has been enabled or not.}
  }%

\item[Initialize Now]
  You can force Rockbox to rescan your disk for tagged files by
  using the \setting{Initialize Now} function in the \setting{Database
    Menu}.
  \warn{\setting{Initialize Now} removes all database files (removing
    runtimedb data also) and rebuilds the database from scratch.}

\item[Update Now]
  \setting{Update now} causes the database to detect new and deleted files
  \opt{swcodec}{
    \note{Unlike the \setting{Auto Update} function, \setting{Update Now}
      will update the database regardless of whether the \setting{Directory Cache}
      is enabled. Thus, an update using \setting{Update now} may take a long
      time.
    }
  }
  Unlike \setting{Initialize Now}, the \setting{Update Now} function
  does not remove runtime database information.
  
\item[Gather Runtime Data]
  When enabled, rockbox will record how often and how long a track is being played, 
  when it was last played and its rating. This information can be displayed in
  the WPS and is used in the database browser to, for example, show the most played, 
  unplayed and most recently played tracks.
  
\item[Export Modifications]
  This allows for the runtime data to be exported to the file \\
  \fname{/.rockbox/database\_changelog.txt}, which backs up the runtime data in
  ASCII format. This is needed when database structures change, because new
  code cannot read old database code. But, all modifications
  exported to ASCII format should be readable by all database versions.
  
\item[Import Modifications.]
  Allows the \fname{/.rockbox/database\_changelog.txt} backup to be 
  conveniently loaded into the database. If \setting{Auto Update} is
  enabled this is performed automatically when the database is initialized.
  
\end{description}

\subsection{Using the Database}
Once the database has been initialized, you can browse your music 
by Artist, Album, Genre, Song Name, etc.  To use the database, go to the
 \setting{Main Menu} and select \setting{Database}.\\

\note{You may need to increase the value of the \setting{Max files in dir 
browser} setting (\setting{Settings $\rightarrow$ General Settings
$\rightarrow$ System $\rightarrow$ Limits}) in order to view long lists of
tracks in the ID3 database browser.\\

There is no option to turn off database completely. If you do not want
to use it just do not do the initial build of the database and do not load it
to RAM.}%

\begin{table}
\begin{center}
  \begin{tabularx}{.75\textwidth}{XXX}%
  \toprule%
  \textbf{Tag}   & \textbf{Type}  & \textbf{Origin} \\
  \midrule
  filename              & string    & system \\ 
  album                 & string    & id tag \\
  albumartist           & string    & id tag \\
  artist                & string    & id tag \\
  comment               & string    & id tag \\
  composer              & string    & id tag \\
  genre                 & string    & id tag \\
  grouping              & string    & id tag \\
  title                 & string    & id tag \\
  bitrate               & numeric   & id tag \\
  discnum               & numeric   & id tag \\
  year                  & numeric   & id tag \\
  tracknum              & numeric   & id tag/filename \\
  autoscore             & numeric   & runtime db \\
  lastplayed            & numeric   & runtime db \\
  playcount             & numeric   & runtime db \\
  Pm (play time - min)  & numeric   & runtime db \\
  Ps (play time - sec)  & numeric   & runtime db \\
  rating                & numeric   & runtime db \\
  commitid              & numeric   & system \\
  entryage              & numeric   & system \\
  length                & numeric   & system \\
  Lm (track len - min)  & numeric   & system \\
  Ls (track len - sec)  & numeric   & system \\
  \bottomrule
  \end{tabularx}
\end{center}
\end{table}

% $Id$ %
\section{\label{ref:WPS}While Playing Screen}
The While Playing Screen (WPS) displays various pieces of information about the
currently playing audio file.
%
\opt{lcd_bitmap}{%
  The appearance of the WPS can be configured using WPS configuration files.
  The items shown depend on your configuration -- all items can be turned on
  or off independently. Refer to \reference{ref:wps_tags} for details on how
  to change the display of the WPS.
  \begin{itemize}
    \nopt{ondio}{
    \item Status bar: The Status bar shows Battery level, charger status, 
      volume, play mode, repeat mode, shuffle mode\opt{rtc}{ and clock}.
      In contrast to all other items, the status bar is always at the top of
      the screen.
    }
    \opt{ondio}{
    \item Status bar: The Status bar shows Battery level, USB power mode, key
      lock status, memory access indicator. In contrast to all other items, the
      status bar is always at the top of the screen.
    }
  \item (Scrolling) path and filename of the current song.
  \item The ID3 track name.
  \item The ID3 album name.
  \item The ID3 artist name.
  \item Bit rate. VBR files display average bitrate and ``(avg)''
  \item Elapsed and total time.
  \item A slidebar progress meter representing where in the song you are.
  \item Peak meter.
  \end{itemize}
}
\opt{recorder,recorderv2fm,ondio}{
  \note{
  \begin{itemize}
  \item The number of lines shown depends on the size of the font used.
  \item The peak meter is only visible if you turn off the status bar or if
    using a small font that gives 8 or more display lines.
  \end{itemize}
  }
}
%
\opt{player}{
  \note{
  \begin{itemize}
  \item Playlist index/Playlist size: Artist {}- Title.
  \item Current{}-time Progress{}-indicator Left.
  \end{itemize}
  }
}

See \reference{ref:ConfiguringtheWPS} for details of customising
your WPS (While Playing Screen).


\subsection{\label{ref:WPS_Key_Controls}WPS Key Controls}

\begin{table}
  \begin{btnmap}{}{}
      \ActionWpsVolUp{} / \ActionWpsVolDown
      \opt{HAVEREMOTEKEYMAP}{& \ActionRCWpsVolUp{} / \ActionRCWpsVolDown}
      & Volume up/down.\\
      %
      \ActionWpsSkipPrev
       \opt{HAVEREMOTEKEYMAP}{& \ActionRCWpsSkipPrev}
      & Go to beginning of track, or if pressed while in the
        first seconds of a track, go to previous track.\\
      %
      \ActionWpsSeekBack
      \opt{HAVEREMOTEKEYMAP}{& \ActionRCWpsSeekBack}
      & Rewind in track.\\
      %
      \ActionWpsSkipNext
      \opt{HAVEREMOTEKEYMAP}{& \ActionRCWpsSkipNext}
      & Go to next track.\\
      %
      \ActionWpsSeekFwd
      \opt{HAVEREMOTEKEYMAP}{& \ActionRCWpsSeekFwd}
      & Fast forward in track.\\
      %
      \ActionWpsPlay
      \opt{HAVEREMOTEKEYMAP}{& \ActionRCWpsPlay}
      & Toggle play/pause.\\
      %
      \ActionWpsStop 
      \opt{HAVEREMOTEKEYMAP}{& \ActionRCWpsStop}
      & Stop playback.\\
      %
      \ActionWpsBrowse
      \opt{HAVEREMOTEKEYMAP}{& \ActionRCWpsBrowse}
      & Return to the \setting{File Browser}.\\
      %
      \ActionWpsContext
      \opt{HAVEREMOTEKEYMAP}{& \ActionRCWpsContext}
      & Enter \setting{WPS Context Menu}.\\
      %
      \opt{ONDIO_PAD}{\ActionWpsContext{} twice}%
      \nopt{ONDIO_PAD}{\ActionWpsMenu}%
      \opt{HAVEREMOTEKEYMAP}{& \ActionRCWpsMenu}
      & Enter \setting{Main Menu}%
      \opt{ONDIO_PAD}{ via the \setting{WPS Context Menu}}.\\%
      %
      \opt{quickscreen}{%
        \ActionWpsQuickScreen
        \opt{HAVEREMOTEKEYMAP}{& \ActionRCWpsQuickScreen}
          & Switches to the \setting{Quick Screen}.
          (see \reference{ref:QuickScreen}) \\}%
      %
      % software hold targets (currently Archos only)
      \nopt{hold_button}{%
          \opt{RECORDER_PAD}{\ButtonFOne+\ButtonDown}
          \opt{PLAYER_PAD}{\ButtonMenu+\ButtonStop}
          \opt{ONDIO_PAD}{\ButtonMenu+\ButtonDown}
          & Key lock on/off.\\
      }%
      %These actions need definitions for the other targets
      \opt{RECORDER_PAD}{%
        \ButtonFThree & Toggles Display quick screen.\\%
        \ButtonFOne+\ButtonPlay & Mute on/off.\\%
      }%
      \opt{PLAYER_PAD}{%
        \ButtonMenu+\ButtonPlay & Mute on/off.\\%
      }%
      % We explicitly list all the appropriate targets here and do no condition
      % on the 'pitchscreen' feature since some players have the feature but do
      % not have the button to go from the WPS to the pitch screen.
      \opt{RECORDER_PAD,IRIVER_H100_PAD,IRIVER_H300_PAD,IRIVER_H10_PAD,MROBE100_PAD%
	           ,GIGABEAT_PAD,GIGABEAT_S_PAD,SANSA_E200_PAD,SANSA_C200_PAD}{%
        \ActionWpsPitchScreen
        \opt{HAVEREMOTEKEYMAP}{& \ActionRCWpsPitchScreen}
         & Show \setting{Pitch Screen} (see \reference{sec:pitchscreen}).\\%
      }%
      \opt{PLAYER_PAD,RECORDER_PAD,IRIVER_H100_PAD,IRIVER_H300_PAD,IRIVER_H10_PAD%
          ,MROBE100_PAD,GIGABEAT_PAD,GIGABEAT_S_PAD,SANSA_E200_PAD,SANSA_C200_PAD}{%
        \ActionWpsIdThreeScreen 
          \opt{HAVEREMOTEKEYMAP}{& \ActionRCWpsIdThreeScreen}
          & Enter \setting{ID3 Viewer}.\\%
      }%
      \opt{RECORDER_PAD,IRIVER_H100_PAD,IRIVER_H300_PAD,IRIVER_H10_PAD,MROBE100_PAD%
          ,GIGABEAT_PAD,GIGABEAT_S_PAD,SANSA_E200_PAD,SANSA_C200_PAD}{%
         \ActionWpsAbSetBNextDir{} or }%
         % not all targets have the above action defined but the one below works on all
      Short \ActionWpsSkipNext{} + Long \ActionWpsSkipNext
      \opt{HAVEREMOTEKEYMAP}{
        &
          \opt{IRIVER_RC_H100_PAD}{\ActionRCWpsAbSetBNextDir{} or}
        Short \ActionRCWpsSkipNext{} + Long \ActionRCWpsSkipNext}
      & Skip to the next directory.\\
      %
      \opt{RECORDER_PAD,IRIVER_H100_PAD,IRIVER_H300_PAD,IRIVER_H10_PAD%
         ,MROBE100_PAD,GIGABEAT_PAD,GIGABEAT_S_PAD,SANSA_E200_PAD,SANSA_C200_PAD}{%
         \ActionWpsAbSetAPrevDir{} or }%
      Short \ActionWpsSkipPrev{} + Long \ActionWpsSkipPrev
      \opt{HAVEREMOTEKEYMAP}{
        &
          \opt{IRIVER_RC_H100_PAD}{\ActionRCWpsAbSetAPrevDir{} or}
        Short \ActionRCWpsSkipPrev{} + Long \ActionRCWpsSkipPrev}
      & Skip to the previous directory.\\
      %
      \opt{SANSA_E200_PAD,SANSA_C200_PAD,IRIVER_H100_PAD,IRIVER_H300_PAD}{
        \ActionStdRec
          \opt{HAVEREMOTEKEYMAP}{&} 
          & Switches to the Recording screen \\
      }%
    \end{btnmap}
\end{table}


\opt{lcd_bitmap}{
  \subsection{\label{ref:peak_meter}Peak Meter}
  The peak meter can be displayed on the While Playing Screen and consists of
  several indicators. 
  \opt{recording}{
    For a picture of the peak meter, please see the While
    Recording Screen in \reference{ref:while_recording_screen}.
  }
  
  \begin{description}
  \item [The bar:]
    This is the wide horizontal bar. It represents the current volume value.
  \item [The peak indicator:]
    This is a little vertical line at the right end of the bar. It indicates 
    the peak volume value that occurred recently.
  \item [The clip indicator:]
    This is a little black block that is displayed at the very right of the
    scale when an overflow occurs. It usually does not show up during normal
    playback unless you play an audio file that is distorted heavily.
    \opt{recording}{
      If you encounter clipping while recording, your recording will sound distorted.
      You should lower the gain.
    }
    \note{Note that the clip detection is not very precise.
     Clipping might occur without being indicated.}
  \item [The scale:]
    Between the indicators of the right and left channel there are little dots.
    These dots represent important volume values. In linear mode each dot is a
    10\% mark. In dbfs mode the dots represent the following values (from right
    to left): 0db, {}-3db, {}-6db, {}-9db, {}-12db, {}-18db, {}-24db, {}-30db,
    {}-40db, {}-50db, {}-60db.
  \end{description}
}
\subsection{\label{sec:contextmenu}The WPS Context Menu}
Like the context menu for the \setting{File Browser}, the \setting{WPS Context Menu} 
allows you quick access to some often used functions:

\subsubsection{Playlist}
The \setting{Playlist} submenu allows you to view, save, search and
reshuffle the current playlist. To change settings for the
\setting{Playlist Viewer} press \ActionStdMenu{} while viewing the playlist
to bring up the \setting{Playlist Viewer Menu}.

\subsubsection{Playlist Viewer Menu}
  \begin{description}
    \item[Show Icons.] This toggles display of the icon for the currently 
    selected playlist entry and the icon for moving a playlist entry
    \item[Show Indicies.] This toggles display of the line numbering for
       the playlist
    \item[Track Display.] This toggles between filename only and full path
       for playlist entries
    \item[Save Current Playlist.] Allows the current playlist to be saved as
       a \fname{.m3u8} playlist file
  \end{description}

    
\subsubsection{Playlist catalog}
  \begin{description}
    \item [View catalog.] This lists all playlists that are part of the
    Playlist catalog. You can load a new playlist directly from this list.
    \item [Add to playlist.] Adds the currently playing file to a playlist.
    Select the playlist you want the file to be added to and it will get
    appended to that playlist.
    \item [Add to new playlist.] Similar to the previous entry this will
    add the currently playing track to a playlist. You need to enter a name
    for the new playlist first.
  \end{description}

\subsubsection{Sound Settings}
This is a shortcut to the \setting{Sound Settings Menu}, where you can configure volume,
bass, treble, and other settings affecting the sound of your music.  
See \reference{ref:configure_rockbox_sound} for more information.

\subsubsection{Playback Settings}
This is a shortcut to the \setting{Playback Settings Menu}, where you can configure shuffle,
repeat, party mode, study mode and other settings affecting the playback of your music.  

\subsubsection{Rating}
The menu entry is only shown if \setting{Gather Runtime Information} is
enabled. It allows the asignment of a personal rating value (0 -- 10)
to a track which can be displayed in the WPS and used in the Database
browser. The value wraps at 10.

\subsubsection{Bookmarks}
This allows you to create a bookmark in the currently-playing track.

\subsubsection{\label{ref:trackinfoviewer}Show Track Info}
\screenshot{rockbox_interface/images/ss-id3-viewer}{The track info viewer}{}
This screen is accessible from the WPS screen, and provides a detailed view of
all the identity information about the current track. This info is known as
meta data and is stored in audio file formats to keep information on artist,
album etc. To access this screen, % 
\opt{PLAYER_PAD,RECORDER_PAD,IRIVER_H100_PAD,IRIVER_H300_PAD,IRIVER_H10_PAD,%
      MROBE100_PAD,SANSA_C200_PAD,SANSA_CLIP_PAD,SANSA_E200_PAD,SANSA_FUZE_PAD,%
      GIGABEAT_PAD,GIGABEAT_S_PAD}{
  press \ActionWpsIdThreeScreen. }%
\opt{ONDIO_PAD,IPOD_4G_PAD,IPOD_3G_PAD,IAUDIO_X5_PAD,IAUDIO_M3_PAD}{press
  \ActionWpsContext{} to access the \setting{WPS Context Menu} and select
  \setting{Show Track Info}. }%
\opt{RECORDER_PAD,PLAYER_PAD,ONDIO_PAD}{Use \ButtonLeft\ and \ButtonRight\
  to move through the information.}%

\subsubsection{Open With...}
This \setting{Open With} function is the same as the \setting{Open With} 
function in the file browser's \setting{Context Menu}.

\subsubsection{Delete}
Delete the currently playing file.

\opt{pitchscreen}{
  \subsubsection{\label{sec:pitchscreen}Pitch}
  
  The \setting{Pitch Screen} allows you to change the rate of playback
  (i.e. the playback speed and at the same time the pitch) of your
  \dap. The rate value can be adjusted between 50\% and 200\%. 50\%
  means half the normal playback speed and a pitch that is an octave
  lower than the normal pitch. 200\% means double playback speed and a
  pitch that is an octave higher than the normal pitch.

  The rate can be changed in two modes: procentual and semitone.
  Initially, procentual mode is active.
  
  \opt{swcodec}{
    If you've enabled the \setting{Timestretch} option in
    \setting{Sound Settings} and have since rebooted, you can also use
    timestretch mode. This allows you to change the playback speed
    without affecting the pitch, and vice versa.
    
    In timestretch mode there are separate displays for pitch and
    speed, and each can be altered independently.  Due to the
    limitations of the algorithm, speed is limited to be between 35\%
    and 250\% of the current pitch value.  Pitch must maintain the
    same ratio as well as remain between 50\% and 200\%.
  }
  
  The value of the \opt{swcodec}{rate, pitch and speed}\nopt{swcodec}{rate}
  is not persisted, i.e. after the \dap\ is turned on it will
  always be set to 100\%.

  \nopt{swcodec}{
    \begin{table}
      \begin{btnmap}{}{}
        \ActionPsToggleMode
        & Toggle pitch changing mode \\
        %
        \ActionPsIncSmall{} / \ActionPsDecSmall
        & Increase / Decrease pitch by 0.1\% (in procentual mode) or by 0.1
          semitone (in semitone mode)\\
        %
        \ActionPsIncBig{} / \ActionPsDecBig
        & Increase / Decrease pitch by 1\% (in procentual mode) or a semitone
          (in semitone mode)\\
        %
        \ActionPsNudgeLeft{} / \ActionPsNudgeRight
        & Temporarily change pitch by 2\% (beatmatch) \\
        %
        \ActionPsReset
        & Reset rate to 100\% \\
        %
        \ActionPsExit
        & Leave the Pitch Screen \\
        %
      \end{btnmap}
    \end{table}

    \warn{Changing the pitch can cause audible 'Artifacts' or 'Dropouts'.}
  }

  \opt{swcodec}{
    \begin{table}
      \begin{btnmap}{}{}
        \ActionPsToggleMode
        \opt{HAVEREMOTEKEYMAP}{& \ActionRCPsToggleMode}
        & Toggle pitch changing mode (cycles through all available modes)\\
        %
        \ActionPsIncSmall{} / \ActionPsDecSmall
        \opt{HAVEREMOTEKEYMAP}{& \ActionRCPsIncSmall{} / \ActionRCPsDecSmall}
        & Increase / Decrease pitch by 0.1\% (in procentual mode) or 0.1
          semitone (in semitone mode)\\
        %
        \ActionPsIncBig{} / \ActionPsDecBig
        \opt{HAVEREMOTEKEYMAP}{& \ActionRCPsIncBig{} / \ActionRCPsDecBig}
        & Increase / Decrease pitch by 1\% (in procentual mode) or a semitone
          (in semitone mode)\\
        %
        \ActionPsNudgeLeft{} / \ActionPsNudgeRight
        \opt{HAVEREMOTEKEYMAP}{& \ActionRCPsNudgeLeft{} / \ActionPsNudgeRight}
        & Temporarily change pitch by 2\% (beatmatch), or modify speed (in timestretch mode) \\
        %
        \ActionPsReset
        \opt{HAVEREMOTEKEYMAP}{& \ActionRCPsReset}
        & Reset pitch and speed to 100\% \\
        %
        \ActionPsExit
        \opt{HAVEREMOTEKEYMAP}{& \ActionRCPsExit}
        & Leave the Pitch Screen \\
        %
      \end{btnmap}
    \end{table}
  }

}

%********************QUICKSCREENS***********************************************
\opt{RECORDER_PAD}{
  \section{\label{ref:QuickScreens}Quick Screens}
  \screenshot{rockbox_interface/images/ss-quick-screen-112x64x1.png}{The F2 quick screen}{}
  \screenshot{rockbox_interface/images/ss-quick-screen2-112x64x1.png}{The F3 quick screen}{}
  Rockbox handles function buttons in a different way to the Archos software.
  \ButtonFOne\ is always bound to the menu function, while \ButtonFTwo\ and
  \ButtonFThree\ enable two quick screens.
  
  \ButtonFTwo\ displays some browse and play settings which are likely to be
  changed frequently. This settings are Shuffle mode, Repeat mode and the Show
  files options
  
  Shuffle mode plays each track in the currently playing list in a random order
  rather than in the order shown in the browser.

  Repeat mode repeats either a single track (One) or the entire playlist (All).

  Show files determines what type files can be seen in the browser.  This can be
  just MP3 files and directories (Music), Playlists, MP3 files and directories
  (Playlists), any files that Rockbox supports (Supported) or all files on the
  disk (All).

  See \reference{ref:PlaybackOptions} for more information about these
  settings.

  \begin{table}
    \begin{btnmap}{}{}
      \ButtonLeft & Controls Shuffle mode setting \\
      \ButtonRight & Controls Repeat mode setting \\
      \ButtonDown & Controls Show file setting \\
    \end{btnmap}
  \end{table}
  
  \ButtonFThree\ controls frequently used display options.
  
  Scroll bar turns the display of the Scroll bar on the left of the screen on
  or off.
  
  Status bar turns the status display at the top of the screen on or off. 
  Upside down inverts the screen so that the top of the display appears nearest
  to the buttons. This is sometimes useful when storing the \dap\ in a pocket.
  Key assignments swap over with the display orientation where it is logical 
  for them to do so.

  See \reference{ref:Displayoptions} for more information about these
  settings.
  
  \begin{table}
    \begin{btnmap}{}{}
      \ButtonLeft & Controls scroll bar display \\
      \ButtonRight & Controls status bar display \\
      \ButtonDown & Controls upside down screen setting \\
    \end{btnmap}
  \end{table}
}


%Include playlist section
% $Id$ %
\chapter{\label{ref:rockbox_interface}Quick Start}
\section{Basic Overview}
\subsection{The \daps{} controls}

\begin{center}
% include the front image. Using \specimg makes this fairly easy,
% but requires to use the exact value of \specimg in the filename!
% The extension is selected in the preamble, so no further \ifpdfoutput
% is necessary.
\includegraphics[height=8cm,width=10cm,keepaspectratio=true]{rockbox_interface/images/\specimg-front}
\opt{iaudiom3}{% replace with HAVEREMOTEKEYMAP when the h100 file exists or change specimg
  \end{center}
  % spacing between the two pictures, could possibly be improved
  \begin{center}
    \includegraphics[height=5.6cm,width=10cm,keepaspectratio=true]{rockbox_interface/images/\specimg-remote}
}
\end{center}

Throughout this manual, the buttons on the \dap{} are labelled according to the
picture above.
\opt{touchscreen}{
The areas of the touchscreen in the 3$\times$3 grid mode are in turn referred as follows:
\begin{table}
    \centering
    \begin{tabular}{|c|c|c|}
	\hline
        \TouchTopLeft & \TouchTopMiddle & \TouchTopRight \\ [5ex]
	\hline        
	\TouchMidLeft & \TouchCenter & \TouchMidRight \\ [5ex]
	\hline        
	\TouchBottomLeft & \TouchBottomMiddle & \TouchBottomRight \\ [5ex]
	\hline
    \end{tabular}
\end{table}
}%
Whenever a button name is prefixed by ``Long'', a long press of approximately
one second should be performed on that button. The buttons are described in
detail in the following paragraph.
\blind{%
  Additional information for blind users is available on the Rockbox website at 
  \wikilink{BlindFAQ}.
  
  %
  \opt{h100}{
  Hold or lay the \dap{} so that the side with the joystick and LCD is facing
  towards you, and the curved side is at the top. The joystick functions as
  the \ButtonUp{}, \ButtonRight{}, \ButtonLeft{}, and \ButtonDown{} buttons when
  pressed in the appropriate direction. Pressing the joystick down functions as
  \ButtonSelect{}. 
  On the right side of the \dap{} are the \ButtonOn{}, \ButtonOff{}, 
  \ButtonMode{} buttons, and the \ButtonHold{} switch. When this switch is
  switched towards the bottom of the \dap{}, hold is on, and none of the other
  buttons have any effect.

  On the left side is the \ButtonRec{} button. Above that is the internal microphone. 

  On the top panel of the \dap{}, from left to right, you can find the
  following: headphone mini jack plug, remote port, Optical line-in, Optical line-out.

  On the bottom panel of the \dap{}, from left to right, you can find the
  following: power jack, reset switch, and USB port. In the event that your
  \dap{} hard locks, you can reset it by inserting a paper clip into the hole
  where the reset switch is.}
  % 
  \opt{h300}{
  Hold or lay the \dap{} so that the side with the button pad and
  LCD is facing towards you.  The buttons on the button pad are as follows:  top 
  left corner: \ButtonOn{}, bottom left corner: \ButtonOff{}, top right corner: 
  \ButtonRec, bottom right corner: \ButtonMode{}.  In the center of the button pad 
  is a button labelled \ButtonSelect{}.  Surrounding the \ButtonSelect{} button are
  the \ButtonUp{}, \ButtonDown{}, \ButtonLeft{}, and \ButtonRight{} buttons.
  
  On the top panel of the \dap{}, from left to right, you can find the 
  following: headphone mini jack plug, remote port, line-in, line-out.

  On the left hand side of the \dap{} is the internal microphone. Just underneath
  this is a small hole, the reset switch. In the event that your \dap{} hard locks,
  you can reset it by inserting a paper clip into the hole where the reset switch
  is.

  On the right hand side of the \dap{} is the \ButtonHold{} switch. When this is 
  switched towards the bottom of the \dap{}, hold is on, and none of the other 
  buttons have any effect.
  
  On the bottom panel of the \dap{}, from left to right, you can find the 
  following:  power jack and two USB ports.  The USB port on the right is used 
  to connect your \dap{} to your computer.  The USB port on the left is not 
  used in Rockbox. 
  }
  %
  \opt{ipod4g,ipodcolor,ipodvideo,ipodmini}{ 
  The main controls on the \dap{} are a slightly indented scroll wheel 
  with a flat round button in the center. Hold the \dap{} with these controls 
  facing you. 

  The top of the player will have the following, from left to 
  right:
  \opt{ipod4g,ipodcolor}{remote connector, headphone socket, \ButtonHold{} 
    switch.}
  \opt{ipodvideo}{\ButtonHold{} switch, headphone socket.}
  \opt{ipodmini}{\ButtonHold{} switch, remote connector, headphone socket.}	

  The dock connector that is used to connect your \dap{} to your computer is on 
  the bottom panel of the \dap{}.

  The button in the middle of the wheel is called \ButtonSelect{}. You can
  operate the wheel by pressing the top, bottom, left or right sections,
  or by sliding your finger around it.  The top is \ButtonMenu{}, the bottom is
  \ButtonPlay{}, the left is \ButtonLeft{}, and the right is \ButtonRight{}.
  When the manual says to \ButtonScrollFwd{}, it means to slide your finger
  clockwise around the wheel. \ButtonScrollBack{} means to slide your finger
  counterclockwise. Note that the wheel is sensitive, so you will need to move
  slowly at first and get a feel for how it works.
  
  Note that when the \ButtonHold{} switch is pushed toward the center of the \dap{}, 
  hold is on, and none of the other controls do anything.  Be sure
  \ButtonHold{} is off before trying to use your player. 
  }
  %
  \opt{ipod3g}{ 
  The main controls on the \dap{} are a slightly indented touch wheel 
  with a flat round button in the center, and four buttons in a row above the
  touch wheel. Hold the \dap{} with these controls 
  facing you. 

  The top of the player will have the following, from left to 
  right: remote connector, headphone socket, \ButtonHold{} switch.
	
  The dock connector that is used to connect your \dap{} to your computer is on 
  the bottom panel of the \dap{}.

  The button in the middle of the wheel is called \ButtonSelect{}. You can
  operate the wheel by sliding your finger around it.  The row of
  buttons consists of, from left to right, the \ButtonLeft{},
  \ButtonMenu{}, \ButtonPlay{}, and \ButtonRight{} buttons.
  When the manual says to \ButtonScrollFwd{}, it means to slide your finger
  clockwise around the wheel. \ButtonScrollBack{} means to slide your finger
  counterclockwise. Note that the wheel is sensitive, so you will need to move
  slowly at first and get a feel for how it works.
  
  Note that when the \ButtonHold{} switch is pushed toward the center of the \dap{}, 
  hold is on, and none of the other controls do anything.  Be sure
  \ButtonHold{} is off before trying to use your player. 
  }
  %
  \opt{ipod1g2g}{ 
  The main controls on the \dap{} are a slightly indented wheel 
  with a flat round button in the center, and four buttons surrounding
  it. On the 1st generation iPod, this wheel physically turns. On the
  2nd generation iPod, this wheel is touch-sensitive. Hold the \dap{} with these controls 
  facing you. 

  The top of the player will have the following, from left to 
  right: FireWire port, headphone socket, \ButtonHold{} switch.

  The FireWire port is used to connect your \dap{} to the computer and
  to charge its battery via a wall charger.
	
  The button in the middle of the wheel is called \ButtonSelect{}. You can
  operate the wheel by turning it, or sliding your finger around
  it. The top is \ButtonMenu{}, the bottom is \ButtonPlay{}, the left
  is \ButtonLeft{}, and the right is \ButtonRight{}.
  When the manual says to \ButtonScrollFwd{}, it means to slide your finger
  clockwise around the wheel. \ButtonScrollBack{} means to slide your finger
  counterclockwise. Note that the wheel is sensitive, so you will need to move
  slowly at first and get a feel for how it works.
  
  Note that when the \ButtonHold{} switch is pushed toward the center of the \dap{}, 
  hold is on, and none of the other controls do anything.  Be sure
  \ButtonHold{} is off before trying to use your player. 
  }
  %
  \opt{ipodnano,ipodnano2g}{
  The main controls on the \dap{} are a slightly indented wheel with a
  flat round button in the center. Hold the \dap{} with these controls on the
  top surface. There is a \ButtonHold{} switch at one end, and
  headphone and dock connector at the other; be sure the end with the
  switch is facing away from you.

  The button in the middle of the wheel is called \ButtonSelect{}. You can
  operate the wheel by pressing the top, bottom, left or right sections,
  or by sliding your finger around it.  The top is \ButtonMenu{}, the bottom is
  \ButtonPlay{}, the left is \ButtonLeft{}, and the right is \ButtonRight{}.
  When the manual says to \ButtonScrollFwd{}, it means to slide your finger
  clockwise around the wheel. \ButtonScrollBack{} means to slide your finger
  counterclockwise. Note that the wheel is sensitive, so you will need to move
  slowly at first and get a feel for how it works.

  Note that when the \ButtonHold{} switch is pushed toward the center of the \dap{},
  hold is on, and none of the other controls do anything; be sure \ButtonHold{} is
  off before trying to use your player.
  }
  %
  \opt{ondio}{
  The main characteristic of the Ondio case is the indent on its lower right side, 
  which is the MMC slot. Holding the \dap{} with this slot in the described position
  you'll find the following:

  On the curved top, from left to right, are the headphone socket,
  the \ButtonOff{} button,%
  \opt{recording}{ and the line-in jack}.
  Apart from the already mentioned MMC slot, you will find the USB connector on
  the \daps{} right side. Below the LCD, at approximately the center of the \dap{},
  there is the main button pad of the \dap{}. The centre of the button pad dips inward
  and helps to operate the directional keys from there. Located on a two-way button
  strip are the \ButtonLeft{} and \ButtonRight{} keys, with \ButtonUp{} above it
  and \ButtonDown{} below it. The raised button positioned in the lower left of this 
  round crosspad is labelled \ButtonMenu{}.
  }
  %
  \opt{h10,h10_5gb}{
  Hold or lay the \dap{} so that the side with the scroll pad and
  LCD is facing towards you. In the centre below the lcd is the scroll pad. It
  is oriented vertically. Touching the top and bottom half of it acts as the 
  \ButtonScrollUp{}  and \ButtonScrollDown{} buttons respectively. On the left
  of the scroll pad is the \ButtonLeft{} button and on the right is the
  \ButtonRight{} button.
  
  There are three buttons on the right hand side of the \dap{}. From top to 
  bottom, they are: \ButtonRew{}, \ButtonPlay{} and \ButtonFF{}. On the left 
  hand side is the \ButtonPower{} button.

  On the top panel of the \dap{}, from left to right, you can find the 
  following: \ButtonHold{} switch, \opt{h10}{reset pin hole, }remote port and
  headphone mini jack plug. 
  
  On the bottom panel of the \dap{} is the data cable port.}
  %
  \opt{gigabeatf}{
  \note{The following description is for the Gigabeat F, but can also apply for the
  Gigabeat X. The Gigabeat F is slightly larger and more rectangular shaped, while the
  Gigabeat X is smaller and has a slightly tapered back.}

  Hold the \dap{} with the screen on top and the controls on the right hand side.  
  Below the screen is a cross-shaped touch sensitive pad which contains the 
  \ButtonUp{}, \ButtonDown{}, \ButtonLeft{} and \ButtonRight{} controls.  On the
  Gigabeat X, this pad will feel slightly raised up, while it will feel slightly
  sunken in on the Gigabeat F. On the top of the unit, from left to right, are the 
  power socket, the \ButtonHold{} switch, and the headphone socket.  The 
  \ButtonHold{} switch puts the \dap{} into hold mode when it is switched to the 
  right of the unit. The buttons will have no effect when this is the case.  
  
  Starting from the left hand side on the bottom of the unit, nearer to the front
  than the back, is a recessed switch which 
  controls whether the battery is on or off.  When this switch is to the left,
  the battery is disconnected.  This can be used for a hard reset of the unit,
  or if the \dap{} is being placed in storage.  Next to that is a connector for
  the docking station and finally on the right hand side of the bottom of the
  unit is a mini USB socket for connecting directly to USB.
  
  Finally on the right hand side of the unit are some control buttons.  Going from
  the bottom of the unit to the top there is a small round \ButtonA{} buttton then a
  rocker volume switch with of the \ButtonVolDown{} button below the \ButtonVolUp{}
  button.  Above that is are two more small round buttons, the \ButtonMenu{} 
  button and nearest to the top of the unit the \ButtonPower{} button, which is held
  down to turn the \dap{} on or off. If you have a Gigabeat X, these buttons are small
  metallic buttons that are place further up on the right hand side, and closer
  together. The layout is still the same, however.}
  %
  \opt{gigabeats}{
  Hold the \dap{} with the screen on top and the controls on the right hand side.
  Directly below the bottom edge of the screen are two buttons, \ButtonBack{}
  on the left and \ButtonMenu{} on the right. Below them is a cross-shaped pad
  which contains the \ButtonUp{}, \ButtonDown{}, \ButtonLeft{}, \ButtonRight{}
  and \ButtonSelect{} controls.
  On the top of the unit from left to right are the headphone socket and the
  \ButtonHold{} switch.  The \ButtonHold{} switch puts the \dap{} into
  hold mode when it is switched to the right of the unit.
  The buttons will have no effect when this is the case.

  Starting from the left hand side on the bottom of the unit, nearer to the back
  than the front, is a recessed switch which controls whether the battery is on
  or off.  When this switch is to the left, the battery is disconnected.
  This can be used for a hard reset of the unit, or if the \dap{} is being placed
  in storage.  Next to that is a mini USB socket for connecting directly to USB, 
  and finally a custom connector, presumably for planned accessories which were 
  never released.

  Finally on the right hand side of the unit are some control buttons and the power 
  connector.  Going from the bottom of the unit to the top, there is the power 
  connector socket, followed by three small round buttons, the
  \ButtonNext{} buttton, \ButtonPlay{} button, and \ButtonPrev{} button (from bottom
  to top) then a rocker volume switch with of the \ButtonVolDown{} button below the
  \ButtonVolUp{} button.  Above that is one more small round button, the \ButtonPower{}
  button, which is held down to turn the \dap{} on or off.}
  %
  \opt{mrobe100}{
  Hold the \dap{} with the black front facing you such that the m:robe writing 
  is readable. Below the writing is the touch sensitive pad with the 
  \ButtonMenu{}, \ButtonPlay{}, \ButtonLeft{}, \ButtonRight{} and \ButtonDisplay 
  controls indicated by their symbols. The dotted center strip is devided in 
  three parts: \ButtonUp{}, \ButtonSelect{} and \ButtonDown. On the top of the 
  unit, on the right, is the \ButtonPower{} switch, which is held down to turn 
  the \dap{} on or off.
  
  The \ButtonHold{} switch is located on the left of the \dap{}, below the 
  headphone socket. It puts the \dap{} into hold mode when it is switched to the 
  top of the unit. The buttons will have no effect when this is the case. On the 
  bottom of the unit, there is a connector for the docking station or the 
  proprietary USB connector for connecting directly to USB.}
  %
  \opt{iaudiom5,iaudiox5}{
  The \dap{} is curved so that the end with the screen on it is thicker than the 
  other end.  Hold the \dap{} wih the thick end towards the top and the screen
  facing towards you.  Half way up the front of the unit on the right hand side
  is a four way joystick which is the \ButtonUp{}, \ButtonDown{}, 
  \ButtonLeft{}, and \ButtonRight{} buttons. When pressed it serves as \ButtonSelect{}.
  
  On the right hand side of the \dap{} from top to bottom, first there is a two 
  way switch.  the \ButtonPower{} button is activated by pushing this switch up,
  and pushing this switch down until it clicks slightly will activate the 
  \ButtonHold{} button.  When the switch is in this position, none of the other
  keys will have an effect.
  
  Below the switch is a lozenge shaped button which is the \ButtonRec{} 
  button, and below that the final button on this side of the unit, the 
  \ButtonPlay{} button.  Just below this is a small hole which is difficult to
  locate by touch which is the internal microphone.  At the very bottom of 
  this side of the unit is the reset hole, which can be used to perform a hard
  reset by inserting a paper clip.
  
  On the bottom of the unit is the connector for the 
  \playerman{} subpack or dock.  On the top of the unit is a charge 
  indicator light, which may feel a bit like a button, but is not.
  
  From the top of the \dap{} on the left hand side is the headphone socket, then the 
  remote connector.  Below this is a cover which protects the \opt{iaudiox5}{USB
  host connector.}\opt{iaudiom5}{USB and charging connector}.}
  %
  \opt{e200,e200v2}{
  Hold the \dap{} with the turning wheel at the front and bottom.  On the bottom left
  of the front of the \dap{} is a raised round button, the \ButtonPower{} button.
  Above and to the left of this, on the outside of the turning wheel are four 
  buttons.  These are the \ButtonUp{}, \ButtonDown{}, \ButtonLeft{} and 
  \ButtonRight{} buttons.  Inside the wheel is the \ButtonSelect{} button.  Turning
  the wheel to the right activates the \ButtonScrollFwd{} function, and to the
  left, the \ButtonScrollBack{} function.  
  
  On the right of the unit is a slot for inserting flash cards.  On the bottom is 
  the connector for the USB cable.  On the left is the \ButtonRec{} button, and
  on the top, there is the headphone socket to the right, and the \ButtonHold{}
  switch.  Moving this switch to the right activates hold mode in which none of the
  other buttons have any effect.  Just to the left of the \ButtonHold{} switch is a
  small hole which contains the internal microphone.}
  %
  \opt{c200}{
  Hold the \dap{} with the buttons on the right and the screen on the left. On
  the right side of the unit, there is a series of four connected buttons that
  form a square. The four sides of the square are the \ButtonUp{},
  \ButtonDown{}, \ButtonLeft{} and \ButtonRight{} buttons, respectively. Inside
  the square formed by these four buttons is the \ButtonSelect{} button. At the
  bottom right corner of the square is a small separate button, the
  \ButtonPower{} button.

  Moving clockwise around the outside of the unit, on the top are the \ButtonVolUp{}
  and \ButtonVolDown{} buttons, which control the volume of playback. The buttons can
  be distinguished by a sunken triangle on the \ButtonVolDown{} button, and a
  raised triangle on the \ButtonVolUp{} button. To the right of
  the volume buttons on the top of the unit is the slot for inserting flash
  memory cards. On the right side of the unit is the connector for the USB
  cable. At center of the bottom of the \dap{} is the \ButtonRec{} button. To
  the left of the \ButtonRec{} button is the \ButtonHold{} switch. Moving this
  switch to the right activates hold mode, in which none of the other buttons
  have any effect. On the lower left side of the unit is the headphone socket.
  Immediately above the headphone socket is a lanyard loop and the microphone.
  }
  %
  \opt{fuze,fuzev2}{
  Hold the \dap{} with the controls on the bottom and the screen on the top. The main
  controls are a scroll wheel with four clickable points and a button in the centre; pressing
  this centre button functions as \ButtonSelect{}. Going clockwise from the top, the clickable
  points on the wheel are the \ButtonUp{}, \ButtonRight{}, \ButtonDown{}, and \ButtonLeft{}
  buttons. Turning the wheel clockwise is \ButtonScrollFwd{}, and turning it counter-clockwise
  is \ButtonScrollBack{}. Immediately above and to the right of the wheel is the \ButtonHome{}
  button.

  On the lower left of the unit is a slot for inserting microSD cards. Immediately below that is
  the opening for the microphone.

  On the bottom of the unit is the connector for connecting a USB cable and the headphone socket.
  On the lower right hand side of the unit is a two-way switch. Pressing this switch up acts as
  \ButtonPower{}, and clicking it down until it locks acts as the \ButtonHold{} switch. When the
  \ButtonHold{} switch is on, none of the other buttons have any effect.
  }
  %
  \opt{clipplus,clipv1,clipv2}{
  Hold the \dap{} with the controls on the bottom and the screen on the top. The main
  controls are a four-way pad with a button in the centre; pressing this centre button
  functions as \ButtonSelect{}. Going clockwise from the top, the four-way pad contains
  the \ButtonUp{}, \ButtonRight{}, \ButtonDown{}, and \ButtonLeft{} buttons. 
  Immediately above and to the right of the four-way pad is the \ButtonHome{} button.
  }
  %
  \opt{clipplus}{
  The \ButtonPower{} button is on the top of the \dap{}, towards the right side.

  At the bottom of the right side of the \dap{} is a slot for microSD cards.
  Above this slot on the right side is the headphone socket.

  On the left hand panel is a two-way button that acts as \ButtonVolDown{} when
  pressed on the bottom, and \ButtonVolUp{} when pressed on the top. Immediately
  above the switch is a mini-USB port to connect the \dap{} to a computer.

  }
  %
  \opt{clipv1,clipv2}{
  On the left hand panel is a two way switch. Pressing this switch up acts as
  \ButtonPower{}, and clicking it down until it locks acts as the \ButtonHold{}
  switch. When the \ButtonHold{} switch is on, none of the other buttons have any
  effect. Immediately above the switch is a mini-USB port to connect the \dap{} to
  a computer.

  On the right hand panel is a two-way button that acts as \ButtonVolDown{} when
  pressed on the bottom, and \ButtonVolUp{} when pressed on the top. Immediately
  above this button is the headphone socket.
  }
  %
  \opt{vibe500}{
  Hold or lay the \dap{} so that the side with the controls and
  LCD is facing towards you. Below the LCD is the touch sensitive pad with the \ButtonMenu{}, 
  \ButtonPlay{}, \ButtonLeft{}, \ButtonRight{} controls and the scroll pad in the centre. The 
  scroll pad is oriented vertically between the \ButtonOK{} and \ButtonCancel{} buttons.
  Sliding a finger up or down the scroll pad acts as \ButtonUp{} and \ButtonDown{} respectively. 
  Note that the scroll pad is sensitive, so you will need to move 
  slowly at first and get a feel for how it works. 

  There are two buttons on the right hand side of the \dap{}: \ButtonPower{} on the top and 
  \ButtonRec{} underneath. Under these buttons, from top to bottom you can find: USB connector, 
  power connector and the reset hole if you need to perform a hardware reset. 

  The \ButtonHold{} switch is located on the left hand side of the \dap{}. Note that when the 
  \ButtonHold{} switch is moved towards the top of the \dap{}, hold is turned on and all the 
  other controls are disabled. Be sure \ButtonHold{} is off before trying to use your player. 

  On the top on the \dap{} is the internal microphone on the left and the line-in socket on the 
  right, near the headphone socket.}
  %
  \opt{player}{
  The main controls of this player are a four-way button on the right below
  the screen, and two round buttons to the left of it. Hold the \dap{} with
  these controls on the bottom and facing you.

  On the left hand side, the higher of the two small buttons is the \ButtonOn{},
  the lower of the two buttons is the \ButtonMenu{} button. The large circular
  button on the right contains, clockwise from the top, the \ButtonPlay{},
  the \ButtonRight{}, the \ButtonStop{}, and the \ButtonLeft{} buttons.

  On the top on the \dap{} is the headphone socket on the left and the line-out
  jack on the right. On the bottom of the \dap{} is the line-in jack on the left,
  the DC-In jack on the right, and the USB connector in the centre.
  }
  %
  \opt{recorder}{
  Holding the Jukebox in front of you, there should be three rectangular buttons
  in a horizontal line towards the middle of the unit, and below this to the left
  there is a circular four button array with the circular \ButtonPlay{} button
  as a fifth button in the centre. These are the navigation controls. Below the
  rectangular buttons and to the right of the circular buttons are two small round
  buttons one above the other.

  The \ButtonOn{} button is the topmost of the two buttons located below and to the
  left of the navigation controls whereas the lower of these two is called \ButtonOff.
  The small round button in the middle of the large circular button array is called
  \ButtonPlay{} button. To the right of the \ButtonPlay{} button there is the
  \ButtonRight{} button, left of it is the \ButtonLeft{}, above it \ButtonUp, and
  below the \ButtonPlay{} button there is the \ButtonDown{} button placed. In the row
  of three rectangular buttons the following buttons can be found (from left to right):
  \ButtonFOne{}, \ButtonFTwo{} and \ButtonFThree{}.

  On the top of the \dap{} is the headphone socket on the left and the line-out jack on
  the right. On the bottom of the \dap{} is the line-in jack on the left, the
  DC-In jack on the right, and the USB connector in the centre.
  }
  \opt{recorderv2fm}{
  Holding the Jukebox in front of you, there should be three rectangular buttons
  in a horizontal line towards the middle of the unit, and below this centred on the
  middle button there are four radial arc shaped buttons placed in a cross formation
  with the circular play button as the centre of the cross. These are the navigation
  controls. Below the cross and to the left are two other buttons.

  The \ButtonOn{} button is the leftmost of the two buttons located below and to the
  left of the navigation controls whereas the rightmost and little lower one of
  these two is called \ButtonOff{}. The round button raised slightly higher than the
  others in the centre of the navigation controls is the \ButtonPlay{} button.  To
  the right of the \ButtonPlay{} button  there is the \ButtonRight{} button, left of
  it is the \ButtonLeft{}, above it \ButtonUp{}, and below the \ButtonPlay{} button
  there is the \ButtonDown{} button  placed. In the row of three rectangular buttons
  the following buttons can be found (from left to right): \ButtonFOne{}, \ButtonFTwo{}
  and \ButtonFThree{}.
  }
}

\subsection{Turning the \dap{} on and off}
\opt{cowond2}{Rockbox has a dual-boot feature with the original firmware being
  the default.\\}
To turn on and off your Rockbox enabled \dap{} use the following keys:
    \begin{btnmap}
      \opt{IRIVER_H100_PAD,IRIVER_H300_PAD}{\ButtonOn}%
      \opt{IPOD_4G_PAD}{\ButtonMenu{} / \ButtonSelect}%
      \opt{IPOD_3G_PAD}{\ButtonMenu{} / \ButtonPlay}%
      \opt{ONDIO_PAD}{\ButtonOff}\opt{RECORDER_PAD,PLAYER_PAD}%
          {Long \ButtonOn}%
      \opt{IAUDIO_X5_PAD,IRIVER_H10_PAD,SANSA_E200_PAD,SANSA_C200_PAD,ONDA_VX777_PAD%
          ,GIGABEAT_PAD,MROBE100_PAD,GIGABEAT_S_PAD,sansaAMS,PBELL_VIBE500_PAD%
          }{\ButtonPower}%
      \opt{COWON_D2_PAD} {\ButtonPower{}, then \ButtonHold}%
          &
      \opt{HAVEREMOTEKEYMAP}{
          \opt{IRIVER_RC_H100_PAD}{\ButtonRCOn}%
          \opt{IAUDIO_RC_PAD}{\ButtonRCPlay}
          &}
      Start Rockbox
          \\

      \opt{IRIVER_H100_PAD,IRIVER_H300_PAD}{Long \ButtonOff}%
      \opt{IPOD_4G_PAD,IPOD_3G_PAD}{Long \ButtonPlay}%
      \opt{ONDIO_PAD,recorderv2fm}{Long \ButtonOff}%
      \opt{recorder}{Double tap \ButtonOff\ when playback is stopped}%
      \opt{PLAYER_PAD}{From the Main Menu, select \textbf{Shutdown}}%
      \opt{IAUDIO_X5_PAD,IRIVER_H10_PAD,SANSA_E200_PAD,SANSA_C200_PAD%
          ,GIGABEAT_PAD,MROBE100_PAD,GIGABEAT_S_PAD,sansaAMS,COWON_D2_PAD%
          ,PBELL_VIBE500_PAD,ONDA_VX777_PAD}{Long \ButtonPower}%
          &
      \opt{HAVEREMOTEKEYMAP}{ 
          \opt{IRIVER_RC_H100_PAD}{Long \ButtonRCStop}%
          \opt{IAUDIO_RC_PAD}{Long \ButtonRCPlay}
          &}
      Shutdown Rockbox
          \\
    \end{btnmap}

\label{ref:Safeshutdown}On shutdown, Rockbox automatically saves its settings.

\opt{IRIVER_H100_PAD,IRIVER_H300_PAD,IAUDIO_X5_PAD,SANSA_E200_PAD%
  ,SANSA_C200_PAD,IRIVER_H10_PAD,IPOD_4G_PAD,GIGABEAT_PAD}{%
  If you have problems with your settings, such as accidentally having
  set the colours to black on black, they can be reset at boot time.  See
  the Reset Settings in \reference{ref:manage_settings_menu} for details.
}%

\opt{PLAYER_PAD,RECORDER_PAD,ONDIO_PAD,GIGABEAT_PAD,IPOD_4G_PAD,SANSA_E200_PAD%
,SANSA_C200_PAD,IAUDIO_X5_PAD,IAUDIO_M5_PAD,IPOD_3G_PAD}{%
  In the unlikely event of a software failure, hardware poweroff or reset can be
  performed by holding down \opt{PLAYER_PAD}{\ButtonStop}\opt{RECORDER_PAD,ONDIO_PAD}
  {\ButtonOff}\opt{GIGABEAT_PAD}{the battery switch}\opt{IPOD_4G_PAD}
  {\ButtonMenu{} and \ButtonSelect{} simultaneously}%
  \opt{IPOD_3G_PAD}{\ButtonMenu{} and \ButtonPlay{} simultaneously}%
  \opt{SANSA_E200_PAD,SANSA_C200_PAD,IAUDIO_X5_PAD,IAUDIO_M5_PAD}
  {\ButtonPower} until the \dap{} shuts off or reboots.
}%
\opt{IRIVER_H100_PAD,IRIVER_H300_PAD,IAUDIO_M3_PAD,IRIVER_H10_PAD,MROBE100_PAD
,PBELL_VIBE500_PAD}{%
  In the unlikely event of a software failure, a hardware reset can be
  performed by inserting a paperclip gently into the Reset hole.
}%

\nopt{gigabeatf,iaudiom3,iaudiom5,iaudiox5,archos}
  {
  \subsection{Starting the original firmware}
  \label{ref:Dualboot}
  \opt{ipod4g,ipodcolor,ipodvideo,ipodnano,ipodnano2g,ipodmini}
    {
    Rockbox has a dual-boot feature. To boot into the original firmware, shut
    down the device as described above. Turn on the \ButtonHold{} switch
    immediately after turning the player on. The Apple logo will
    display for a few seconds as Rockbox loads the original firmware.
    
    You can also load the original firmware by shutting down the device,
    then clicking the \ButtonHold{} switch on and connecting the iPod
    to your computer.
 
    Regardless of which method you use to boot to the original firmware, you can
    return to Rockbox by pressing and holding \ButtonMenu{} and \ButtonSelect{}
    simultaneously until the player hard resets.
    }

  \opt{ipod1g2g,ipod3g}
    {
    Rockbox has a dual-boot feature. To boot into the original firmware, shut
    down the device as described above. Turn on the \ButtonHold{} switch
    immediately after turning the player on. The Apple logo will
    display for a few seconds as Rockbox loads the original firmware.
    
    You can also load the original firmware by shutting down the device,
    then clicking the \ButtonHold{} switch on and connecting the iPod
    to your computer.
 
    Regardless of which method you use to boot to the original firmware, you can
    return to Rockbox by pressing and holding \ButtonMenu{} and \ButtonPlay{}
    simultaneously until the player hard resets.
    }

  \opt{h100,h300}
    {
    Rockbox has a dual-boot feature. To boot into the original firmware,
    when the \dap{} is turned off, press and hold the \ButtonRec{} button,
    and then press the \ButtonOn{} button.
    }

  \opt{h10,h10_5gb}
    {
    Rockbox has a dual-boot feature. It loads the original firmware from
    the file \fname{/System/OF.mi4}. To boot into the original firmware,
    press and hold the \ButtonLeft{} button while turning on the player.
    \note{The iriver firmware does not shut down properly when you turn it off,
    it only goes to sleep. To get back into Rockbox when exiting from the
    iriver firmware, you will need to reset the player by \opt{h10}{inserting a
    pin in the reset hole}\opt{h10_5gb}{removing and reinserting the battery}.}
    }
    
  \opt{sansa,sansaAMS}
    {
    Rockbox has a dual-boot feature. To boot into the original firmware,
    press and hold the \ButtonLeft{} button while turning on the player.
    }

  \opt{sansaAMS}
    {
    The player will always boot into the original firmware if it is powered
    by a USB connection, and additionally will do so if USB is inserted while
    rockbox is running without holding \ActionStdUsbCharge{}. This feature may
    be removed in the future when Rockbox is able to handle USB transfers 
    natively.
    }

  \opt{mrobe100}
    {
    Rockbox has a dual-boot feature. It loads the original firmware from
    the file \fname{/System/OF.mi4}. To boot into the original firmware,
    when the \dap{} is turned off, press the \ButtonPower{} button once and then 
    a second time when the m:robe bootlogo (the headphone) appears. Hold the
    \ButtonPower{} button until you see the ``Loading original firmware...'' 
    message on the screen.
    }

  \opt{gigabeats}
    {
    Rockbox has a dual-boot feature. To boot into the original firmware,
    turn the \ButtonHold{} switch on just after turning on the \dap{}.
    To return to Rockbox, shutdown the \dap{}, then turn the battery switch
    on the bottom off then on again. Rockbox should now start.
    }

  \opt{cowond2}
    {
    Use \ButtonPower{} to boot the original \playerman{} firmware.
    }

  \opt{vibe500}
    {
    Rockbox has a dual-boot feature where it is possible to load the original firmware from
    the file \fname{/System/OF.mi4}. To boot into the original firmware press and release
    \ButtonPower{} and then immediately after the backlight turns on, press the \ButtonOK{}
    button and keep it pressed until the original firmware starts.
    }

  \opt{ondavx777}
    {
    Rockbox has a dual-boot feature where it is possible to load the original firmware from
    the file \fname{/SD/ccpmp.bin}. To boot into the original firmware press and release
    \ButtonPower{} immediately after the Rockbox Logo appear on the screen.
    }

  }
\subsection{Putting music on your \dap{}}

\opt{usb_hid}{
\note{Due to a bug in some OS X versions, the \dap{} can not be mounted, unless
    the USB HID feature is disabled. See \reference{ref:USB_HID} for more
    information.\newline
}
}

With the \dap{} connected to the computer as an MSC/UMS device (like a
USB Drive), music files can be put on the player via any standard file
transfer method that you would use to copy files between drives (e.g. Drag-and-Drop).
Files may be placed wherever you like on the \dap{}, but it is strongly suggested
not to place them in the \fname{/.rockbox} folder.
The default directory structure that is assumed by some parts of Rockbox
\opt{albumart}{%
    (album art searching, and missing-tag fallback in some WPSes) uses the
    parent directory of a song as the Album name, and the parent directory of
    that folder as the Artist name. WPSes may display information incorrectly if
    your files are not properly tagged, and you have your music organized in a
    way different than they assume when attempting to guess the Artist and Album
    names from your filetree. See \reference{ref:album_art} for the requirements
    for Album Art to work properly. 
}%
\nopt{albumart}{%
    (missing-tag fallback in some WPSes) uses the parent directory of a song
    as the Album name, and the parent directory of that folder as the Artist
    name. WPSes may display
    information incorrectly if your files are not properly tagged, and you have
    your music organized in a way different than they assume when attempting to
    guess the Artist and Album names from your filetree.
}%
\opt{swcodec}{
    See \reference{ref:Supportedaudioformats} for a list of supported audio
    formats.
}

\subsection{The first contact}

After you have first started the \dap{}, you'll be presented by the
\setting{Main Menu}. From this menu you can reach every function of Rockbox,
for more information (see \reference{ref:main_menu}). To browse the files
on your \dap{}, select \setting{Files} (see \reference{ref:file_browser}), and to
browse in a view that is based on the meta-data\footnote{ID3 Tags, Vorbis
comments, etc.} of your audio files, select \setting{Database} (see
\reference{ref:database}).

\subsection{Basic controls}
When browsing files and moving through menus you usually get a list view
presented. The navigation in these lists are usually the same and should be
pretty intuitive.
In the tree view use \ActionStdNext{} and \ActionStdPrev{} to move around
the selection. Use \ActionStdOk{} to select an item. \opt{wheel_acceleration}{
Note that the scroll speed is accelerating the faster you rotate the wheel.}
When browsing the file system selecting an audio file plays it. The view 
switches to the ``While playing screen'', usually abbreviated as ``WPS'' (see 
\reference{ref:WPS}. The dynamic playlist gets replaced with the contents of 
the current directory. This way you can easily treat directories as playlists. 
The created dynamic playlist can be extended or modified while playing. This is 
also known as ``on-the-fly playlist''.
To go back to the \setting{File Browser} stop the playback with the
\ActionWpsStop{} button or return to the file browser while keeping playback
running using \ActionWpsBrowse{}.
In list views you can go back one step with \ActionTreeParentDirectory.

\subsection{Basic concepts}
\subsubsection{Playlists}
Rockbox is playlist oriented. This means that every time you play an audio file,
a so-called ``dynamic playlist'' is generated, unless you play a saved
playlist. You can modify the dynamic playlist while playing and also save
it to a file. If you do not want to use playlists you can simply play your
files directory based.
Playlists are covered in detail in \reference{ref:working_with_playlists}.

\subsubsection{Menu}
From the menu you can customise Rockbox. Rockbox itself is very customisable.
Also there are some special menus for quick access to frequently used
functions.

\subsubsection{Context Menu}
Some views, especially the file browser and the WPS have a context menu.
From the file browser this can be accessed with \ActionStdContext{}.
The contents of the context menu vary, depending on the situation it gets
called. The context menu itself presents you with some operations you can
perform with the currently highlighted file. In the file browser this is
the file (or directory) that is highlighted by the cursor. From the WPS this is
the currently playing file. Also there are some actions that do not apply
to the current file but refer to the screen from which the context menu
gets called. One example is the playback menu, which can be called using
the context menu from within the WPS.

\section{Customising Rockbox}
Rockbox' User Interface can be customised using ``Themes''. Themes usually
only affect the visual appearance, but an advanced user can create a theme
that also changes various other settings like file view, LCD settings and
all other settings that can be modified using \fname{.cfg} files. This topic
is discussed in more detail in \reference{ref:manage_settings}.
The Rockbox distribution comes with some themes that should look nice on
your \dap{}.

\opt{lcd_bitmap}{
\note{Some of the themes shipped with Rockbox need additional
fonts from the fonts package, so make sure you installed them.
Also, if you downloaded additional themes from the Internet make sure you
have the needed fonts installed as otherwise the theme may not display
properly.}
}

\nopt{ondio}{
  \opt{usb_power}{
    \section{USB Charging}

    The \dap{} can be charged over USB without connecting to your
    computer by holding \ActionStdUsbCharge{} while plugging in. This
    allows you to continue using the \dap{} normally.
  }
}

\opt{ondio}{
  \section{USB Power}

    The \dap{} can be powered over USB without connecting to your
    computer by holding \ActionStdUsbCharge{} while plugging in. This
    allows you to continue using the \dap{} normally.
}

% $Id$ %
\chapter{Browsing and playing}
\section{\label{ref:file_browser}File Browser}
\screenshot{rockbox_interface/images/ss-file-browser}{The file browser}{}
Rockbox lets you browse your music in either of two ways. The 
\setting{File Browser} lets you navigate through the files and directories on 
your \dap, entering directories and executing the default action on each file.
To help differentiate files, each file format is displayed with an icon. 

The \setting{Database Browser}, on the other hand, allows you to navigate 
through the music on your player using categories like album, artist, genre,
etc.

You can select whether to browse using the \setting{File Browser} or the
\setting{Database Browser} by selecting either \setting{Files} or
\setting{Database} in the \setting{Main Menu}.
If you choose the \setting{File Browser}, the \setting{Show Files} setting
lets you select what types of files you wish to view. See
\reference{ref:ShowFiles} for more information on the \setting{Show Files}
setting.

\note{The \setting{File Browser} allows you to manipulate your files in ways
that are not available within the \setting{Database Browser}. Read more about
\setting{Database} in \reference{ref:database}. The remainder of this section
deals with the \setting{File Browser}.}

\opt{ondio}{
Unlike the Archos Firmware, Rockbox provides multivolume support for the
MultiMediaCard, this means the \dap{} can access both data volumes (internal
memory and the MMC), thus being able to for instance, build playlists with
files from both volumes.
In the \setting{File Browser} a new directory will appear as soon as the device
has read the content after inserting the card. This new directory's name is
generated as \fname{<MMC1>}, and will behave exactly as any other directory
on the \dap{}.
}

\opt{iriverh10,iriverh10_5gb}{\note{
If your \dap{} is a MTP model, the Music directory where all your music is stored
may be hidden in the \setting{File Browser}. This may be fixed by either
changing its properties (on a computer) to not hidden, or by changing
the \setting{Show Files} setting to all.
}}

\subsection{\label{ref:controls}File Browser Controls}
\begin{btnmap}
      \ActionStdPrev{}/\ActionStdNext{}
      \opt{HAVEREMOTEKEYMAP}{& \ActionRCStdPrev{}/\ActionRCStdNext{}}
         & Go to previous/next item in list. If you are on the first/last 
           entry, the cursor will wrap to the last/first entry.\\
      %
      \opt{IRIVER_H100_PAD,IRIVER_H300_PAD,RECORDER_PAD}
        {
          \ButtonOn+\ButtonUp{}/ \ButtonDown
          \opt{HAVEREMOTEKEYMAP}{&
            \opt{IRIVER_RC_H100_PAD}{\ButtonRCSource{}/ \ButtonRCBitrate}
          }
          & Move one page up/down in the list.\\
        }
      \opt{IRIVER_H10_PAD}
        {
          \ButtonRew{}/ \ButtonFF
          & Move one page up/down in the list.\\
        }
      %
      \ActionTreeParentDirectory
      \opt{HAVEREMOTEKEYMAP}{& \ActionRCTreeParentDirectory}
      & Go to the parent directory.\\
      %
      \ActionTreeEnter
      \opt{HAVEREMOTEKEYMAP}{& \ActionRCTreeEnter}
      & Execute the default action on the selected file or enter a
        directory.\\
      %
      \ActionTreeWps 
      \opt{HAVEREMOTEKEYMAP}{& \ActionRCTreeWps}
         & If there is an audio file playing, return to the
         \setting{While Playing Screen} (WPS) without stopping playback.\\
      %
      \nopt{player,SANSA_C200_PAD}%
        {%
          \ActionTreeStop 
          \opt{HAVEREMOTEKEYMAP}{& \ActionRCTreeStop}
          & Stop audio playback.\\%
        }%
      %
      \ActionStdContext{}
      \opt{HAVEREMOTEKEYMAP}{& \ActionRCStdContext}
      & Enter the \setting{Context Menu}.\\
      %
      \ActionStdMenu{}
      \opt{HAVEREMOTEKEYMAP}{& \ActionRCStdMenu}
      & Enter the \setting{Main Menu}.\\
      %
      \opt{quickscreen}{
        \ActionStdQuickScreen
        \opt{HAVEREMOTEKEYMAP}{& \ActionRCStdQuickScreen}
        & Switch to the \setting{Quick Screen}
        (see \reference{ref:QuickScreen}). \\
      }
      \opt{RECORDER_PAD}{
        \ButtonFThree & Switch to the \setting{Quick Screen}.\\ 
        %
      }
      %
      \opt{SANSA_E200_PAD}{
        \ActionStdRec & Switch to the \setting{Recording Screen}.\\
      %
      }
      \nopt{touchscreen}{\opt{hotkey}{
        \ActionTreeHotkey
            &
        \opt{HAVEREMOTEKEYMAP}{
            &}
        Activate the \setting{Hotkey} function
        (see \reference{ref:Hotkeys}).
            \\
      }}
\end{btnmap}

\opt{RECORDER_PAD}{
  The functions of the F keys are also summarised on the button bar at the
  bottom of the screen.
}

\subsection{\label{ref:Contextmenu}\label{ref:PartIISectionFM}Context Menu}
\screenshot{rockbox_interface/images/ss-context-menu}{The Context Menu}{}

The \setting{Context Menu} allows you to perform certain operations on files or 
directories.  To access the \setting{Context Menu}, position the selector over a file 
or directory and access the context menu with \ActionStdContext{}.\\

\note{The \setting{Context Menu} is a context sensitive menu.  If the 
\setting{Context Menu} is invoked on a file, it will display options available 
for files.  If the \setting{Context Menu} is invoked on a directory, 
it will display options for directories.\\}

The \setting{Context Menu} contains the following options (unless otherwise noted, 
each option pertains both to files and directories):

\begin{description}
\item [Playlist.]
  Enters the \setting{Playlist Submenu} (see \reference{ref:playlist_submenu}).
\item [Playlist Catalog.]
  Enters the \setting{Playlist Catalog Submenu} (see 
  \reference{ref:playlist_catalog}).
\item [Rename.]
  This function lets the user modify the name of a file or directory.
\item [Cut.]
  Copies the name of the currently selected file or directory to the clipboard
  and marks it to be `cut'.
\item [Copy.]
  Copies the name of the currently selected file or directory to the clipboard
  and marks it to be `copied'.
\item [Paste.]
  Only visible if a file or directory name is on the clipboard. When selected
  it will move or copy the clipboard to the current directory.
\item [Delete.]
  Deletes the currently selected file. This option applies only to files, and
  not to directories. Rockbox will ask for confirmation before deleting a file.
  Press \ActionYesNoAccept{}
  to confirm deletion or any other key to cancel.
\item [Delete Directory.]
  Deletes the currently selected directory and all of the files and subdirectories
  it may contain. Deleted directories cannot be recovered. Use this feature with
  caution!
\opt{lcd_non-mono}{
\item [Set As Backdrop.]
  Set the selected \fname{bmp} file as background image. The bitmaps need to meet the
  conditions explained in \reference{ref:LoadingBackdrops}.
}
\item [Open with.]
  Runs a viewer plugin on the file. Normally, when a file is selected in Rockbox,
  Rockbox automatically detects the file type and runs the appropriate plugin.
  The \setting{Open With} function can be used to override the default action and
  select a viewer by hand.
  For example, this function can be used to view a text file
  even if the file has a non-standard extension (i.e., the file has an extension
  of something other than \fname{.txt}). See \reference{ref:Viewersplugins}
  for more details on viewers.
\item [Create Directory.]
  Create a new directory in the current directory on the disk.
\item [Properties.]
  Shows properties such as size and the time and date of the last modification
  for the selected file. If used on a directory, the number of files and
  subdirectories will be shown, as well as the total size.
\opt{recording}{
  \item [Set As Recording Directory.]
    Save recordings in the selected directory.
}
\item [Add to Shortcuts.]
  Adds a link to the selected item in the \fname{shortcuts.link} file.
  If the file does not already exist it will be created in the root directory.
  Note that if you create a shortcut to a file, Rockbox will not open it upon
  selecting, but simply bring you to its location in the \setting{File Browser}.
\end{description}

\subsection{\label{sec:virtual_keyboard}Virtual Keyboard}
\screenshot{rockbox_interface/images/ss-virtual-keyboard}{The virtual keyboard}{}
This is the virtual keyboard that is used when entering text in Rockbox, for 
example when renaming a file or creating a new directory.
\nopt{player}{The virtual keyboard can be easily changed by making a text file
 with the required layout. More information on how to achieve this can be found
 on the Rockbox website at \wikilink{LoadableKeyboardLayouts}.}

\opt{morse_input}{
  Also you can switch to Morse code input mode by changing the
  \setting{Use Morse Code Input} setting%
  \opt{IRIVER_H100_PAD,IRIVER_H300_PAD,IPOD_4G_PAD,IPOD_3G_PAD,IRIVER_H10_PAD%
      ,GIGABEAT_PAD,GIGABEAT_S_PAD,MROBE100_PAD,SANSA_E200_PAD,PBELL_VIBE500_PAD}%
    { or by pressing \ActionKbdMorseInput{} in the virtual keyboard}%
  .}

\nopt{player}{% no "Actions" yet in the Player's virtual keyboard

\note{When the cursor is on the input line, \ActionKbdSelect{} deletes the preceding character}

\begin{btnmap}
    \opt{IRIVER_H100_PAD,IRIVER_H300_PAD,RECORDER_PAD,GIGABEAT_PAD,GIGABEAT_S_PAD%
        ,MROBE100_PAD,SANSA_E200_PAD,SANSA_FUZE_PAD,SANSA_C200_PAD}{
        \ActionKbdCursorLeft{} / \ActionKbdCursorRight
            &
        \opt{HAVEREMOTEKEYMAP}{\ActionRCKbdCursorLeft{} / \ActionRCKbdCursorRight
            &}
        Move the line cursor within the text line.
            \\
        %
        \ActionKbdBackSpace
            &
        \opt{HAVEREMOTEKEYMAP}{
            &}
        Delete the character before the line cursor.
            \\
    }%
    \ActionKbdLeft{} / \ActionKbdRight
        &
    \opt{HAVEREMOTEKEYMAP}{\ActionRCKbdLeft{} / \ActionRCKbdRight
        &}
    Move the cursor on the virtual keyboard.
    If you move out of the picker area, you get the previous/next page of
    characters (if there is more than one).
        \\
    %
    \ActionKbdUp{} / \ActionKbdDown
        &
    \opt{HAVEREMOTEKEYMAP}{\ActionRCKbdUp{} / \ActionRCKbdDown
        &}
    Move the cursor on the virtual keyboard.
    If you move out of the picker area you get to the line edit mode.
        \\
    %
    \nopt{IPOD_3G_PAD,IPOD_4G_PAD,IRIVER_H10_PAD,ONDIO_PAD,PBELL_VIBE500_PAD}{
        \ActionKbdPageFlip
            &
        \opt{HAVEREMOTEKEYMAP}{\ActionRCKbdPageFlip
            &}
        Flip to the next page of characters (if there is more than one).
            \\
    }
    %
    \ActionKbdSelect
        &
    \opt{HAVEREMOTEKEYMAP}{\ActionRCKbdSelect
        &}
    Insert the selected keyboard letter at the current line cursor position.
        \\
    %
    \ActionKbdDone
        &
    \opt{HAVEREMOTEKEYMAP}{\ActionRCKbdDone
        &}
    Exit the virtual keyboard and save any changes.
        \\
    %
    \ActionKbdAbort
        &
    \opt{HAVEREMOTEKEYMAP}{\ActionRCKbdAbort
        &}
    Exit the virtual keyboard without saving any changes.
        \\
% to be done - create a separate section for morse imput and update the info
      \opt{morse_input}{
        \opt{IRIVER_H100_PAD,IRIVER_H300_PAD,GIGABEAT_PAD,GIGABEAT_S_PAD,MROBE100_PADD%
            ,SANSA_E200_PA,IPOD_4G_PAD,IPOD_3G_PAD,IRIVER_H10_PAD,PBELL_VIBE500_PAD}{
          \ActionKbdMorseInput
          \opt{HAVEREMOTEKEYMAP}{& \ActionRCKbdMorseInput}
          & Toggle keyboard input mode and Morse code input mode. \\}
        %
        \ActionKbdMorseSelect
        \opt{HAVEREMOTEKEYMAP}{& \ActionRCKbdMorseSelect}
        & Tap to select a character in Morse code input mode. \\
      } 
\end{btnmap}
}% end of non-Player section

\opt{player}{
  The current text line to be entered or edited is always listed on the first
  line of the display. The second line of the display can contain the character
  selection bar, as in the screenshot above.
    \begin{btnmap}
      \ButtonOn & Toggle picker- and line edit mode. \\
      \ButtonLeft{} / \ButtonRight
        & Move back and forth in the selected line (picker of input line). \\
      \ButtonPlay
        & Pick character in character bar, or act as backspace in the text line. \\
      Long \ButtonPlay & Accept \\
      \ButtonStop & Cancel \\
      \ButtonMenu & Flip picker lines. \\
    \end{btnmap}
}

\input{rockbox_interface/tagcache.tex}
\input{rockbox_interface/wps.tex}

%Include playlist section
\input{working_with_playlists/main.tex}
\input{rockbox_interface/hotkeys.tex}


% $Id$ %
\opt{hotkey}{
    \section{\label{ref:Hotkeys}Hotkeys}
    Hotkeys are user-assignable shortcut keys to functions within the
    \setting{File Browser} and \setting{WPS}.  To use one,
    press \ActionWpsHotkey{} within the \setting{File Browser} or
    \setting{WPS} screens.  The assigned function will launch with
    reference to the current file or directory, if applicable.  Each
    screen has its own assignment.  If there is no assignment for
    a given screen, the message ``No Hotkey Set'' appears briefly.
    
    There is no default assignment for the File Browser hotkey, but the WPS
    hotkey defaults to ``View Playlist''.
    
    To change the assignment of a hotkey, go into the associated
    context menu from the File Browser or WPS screen by pressing
    \ActionWpsContext, then highlighting the function you wish to assign.
    Press the hotkey (\ActionWpsHotkey) and if the function you've chosen
    is assignable a confirmation dialog screen will appear.  Press
    \ActionYesNoAccept{} to confirm the assignment or anything else to reject
    it.  If accepted, you'll see a short message confirming your choice.
    
    You can view the current assignments and reset to the default assignments
    from the Hotkey menu under \setting{General Settings}. See
    \reference{ref:HotkeySettings} for details.
}



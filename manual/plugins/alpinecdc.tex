\subsection{Alpine CD changer emulator}
This plugin emulates an Alpine CD changer. It allows to plug the Archos to a
compatible head unit and control the playback from there, too. Currently
implemented is track change, shuffle, seek, but no disk change. The plugin is a
TSR, meaning it silently operates in the background once started. It will keep
doing so until a new plugin is started.
Alpine also did M-Bus as OEM for other brands (Honda, Acura, Volvo, BMW, etc.)
Nowadays Alpine uses a different protocol, called Ai-Net, not supported by this
plugin. (As well as all other protocols, please do not ask for such!)

\subsubsection{The cable}
Hookup to Archos works by connecting the headphone output including the remote
pin (you need a 4-ring 3.5 mm plug for that) to the changer jack of the radio.
M-Bus radios have a DIN-style circular jack with 8 pins (7 in a $\sim$ 270 degree
circle, one in the center). A standard 5-pin DIN plug is OK for this, since we
don't use the other (power) pins.

As OEM, they shuffled the pins around a bit, better check first if it's not
genuine Alpine. The bus pin is pulled high to 12 volts with a $\sim$ 2kOhm resistor,
pulses driven low. Because it's open collector, this is not harmful to the
Archos.\\
ASCII art of the 4-pin headphone plug:
\begin{verbatim}
/ \
\_/ left    -> Alpine pin 5
|_| right   -> Alpine pin 4
|_| remote  -> Alpine pin 1
|_| ground  -> Alpine pin 2 + 3
\end{verbatim}

The remote pin can be programmed bidirectional, that's the reason this works.
Very luckily the M-Bus uses a single wire communication and the two radios I
tried are happy with the 3.3 Volt level the Arcos can deliver. So the
connection is a simple cable! For all protocols requiring more lines, an
external controller would be necessary.

\note{Archos FMs don't have the remote pin internally connected, but
  one unit that was opened was internally prepared for it, a matter of closing a
  bridge.} 
